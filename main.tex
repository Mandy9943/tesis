% CREACIÓN DEL DOCUMENTO
\documentclass[
	spanish, % Idioma: spanish, english, etc.	
	letterpaper, oneside
]{book}

% INFORMACIÓN DEL DOCUMENTO
\def\documenttitle { Optimización en el sistema de tratamiento de agua de la planta de bulbos en Laboratorios AICA+ mediante electrodesionización}
\def\documentsubtitle {}
\def\degreetitle {
	Trabajo de Diploma para optar por el título académico de
Ingeniero en Automática
	 
}

\def\universityname {Universidad Tecnológica de La Habana}
\def\universityfaculty {``José Antonio Echeverría``}
\def\universitydepartment {Facultad de Ingeniería Automática y Biomédica}
\def\universitydepartmentimage {departamentos/cujae}
\def\universitydepartmentimagecfg {height=3cm}
\def\universitylocation {La Habana, Cuba}

% INTEGRANTES, PROFESORES Y FECHAS
\def\documentauthor {Armando Cesar Martin Calderón}
\def\documentdate {\the\year}
\def\portrait {

	\begin{center}
		\insertimageboxed[]{departamentos/cujae}{scale=0.5}{0}{}
		\vspace{-2cm} ~ \\
		\textbf{Universidad Tecnológica de La Habana}\\
		\vspace{-1cm} ~ \\
\textbf{José Antonio Echeverría}\\
		\vspace{-1cm} ~ \\
		\textcolor{colorCujae}{\large\textbf{cujae}}\\
		\Large{ \textbf{Facultad de Ingeniería } }\\
		\vspace{-1.5cm} ~ \\
		\Large{ \textbf{Automática y Biomédica}} \\
		\vspace{0.7cm} 
		\Large{ \textbf{\degreetitle}} ~ 
	\LARGE{	\textbf{\documenttitle}}  \\
	\vspace{0.7cm} 
	\Large	{\textbf{Autor}} \\
		\documentauthor \\
		\vspace{0.7cm} 

		\textbf{Tutores} \\
	Ing. Amanda Martí Coll \\
		Ing. Rosaine Ayala Gispert \\
		\vspace{0.7cm} 

		\universitylocation \\
		Junio, 2023
	
	\end{center}
}
\def\abstracttable {
	\begin{tabular}{l}
		
	\end{tabular}
}
 
% IMPORTACIÓN DEL TEMPLATE
% Template:     Tesis LaTeX
% Documento:    Núcleo del template
% Versión:      3.2.6 (29/04/2023)
% Codificación: UTF-8
%
% Autor: Pablo Pizarro R.
%        pablo@ppizarror.com
%
% Manual template: [https://latex.ppizarror.com/tesis]
% Licencia MIT:    [https://opensource.org/licenses/MIT]

% -----------------------------------------------------------------------------
% CONFIGURACIONES
% -----------------------------------------------------------------------------
% Definiciones previas
\def\iitembcirc {\raisebox{0.55\height}{\scriptsize$\bullet$}}
\def\iitembsquare {\raisebox{0.3\height}{\tiny$\blacksquare$}}
\def\iitemdash {\raisebox{0.35\height}{\textendash}}
\def\iitemcirc {\raisebox{0.25\height}{\small$\circ$}}
\def\iitemdiamond {\raisebox{0.25\height}{\small$\diamond$}}

% Ajustes usuario


% CONFIGURACIONES GENERALES
\def\documentfontsize {12}         % Tamaño de la fuente del documento [pt]
\def\documentinterline {1.5}       % Interlineado del documento [factor]
\def\documentparindent {0}        % Tamaño del indentado de párrafos [pt]
\def\documentparskip {15}           % Tamaño adicional entre párrafos (+/-) [pt]
\def\fontdocument {arial}        % Tipografía base, ver soportadas en manual
\def\fonttypewriter {tmodern}      % Tipografía de \texttt, ver manual
\def\fonturl {same}                % Tipo de fuente url {tt,sf,rm,same}
\def\graphicxdraft {false}         % En true no carga las imágenes (modo draft)
\def\indentfirstpar {false}        % Obliga la indentación del primer párrafo
\def\pointdecimal {true}           % N° decimales con punto en vez de coma
\def\predocpageromanupper {true}  % Páginas en número romano en mayúsculas
\def\showlayoutlines {false}       % Muestra el layout de la página
\def\showlinenumbers {false}       % Muestra los números de línea del documento
\def\twopagesclearformat {blank}   % Formato nueva página twoside {blank,empty}

% ESTILO HEADER-FOOTER 
\def\chapterstyle {style5}         % Estilo de los capítulos (12 estilos)
\def\disablehfrightmark {false}    % Desactiva el rightmark del header-footer
\def\hfstyle {style7}              % Estilo header-footer (16 estilos)
\def\hfwidthcourse {0.35}          % Tamaño máximo del curso en header-footer
\def\hfwidthtitle {0.6}            % Tamaño máximo del título en header-footer
\def\hfwidthwrap {false}           % Activa el tamaño máximo en header-footer

% CONFIGURACIÓN DE LAS LEYENDAS - CAPTION
\def\captionalignment {justified}  % Posición {centered,justified,left,right}
\def\captionfontsize {small}       % Tamaño de fuente de los caption
\def\captionlabelformat {simple}   % Formato leyenda {empty,simple,parens}
\def\captionlabelsep {colon}       % Sep. {none,colon,period,space,quad,newline}
\def\captionlessmarginimage {0.1}  % Margen sup/inf de figura sin leyenda [cm]
\def\captionlrmargin {2}           % Márgenes izq/der de la leyenda [cm]
\def\captionlrmarginmc {1}         % Margen izq/der leyenda dentro de cols. [cm]
\def\captionmarginimage {0}        % Margen vertical entre caption e imagen [cm]
\def\captionmarginimages {0}       % Margen vertical entre caption e images [cm]
\def\captionmarginimagesmc {0}     % Margen vert. entre caption e imagesmc [cm]
\def\captionmarginmultimg {0}      % Margen izq/der leyendas múltiple img [cm]
\def\captionnumcode {arabic}       % N° código {arabic,alph,Alph,roman,Roman}
\def\captionnumequation {arabic}   % N° ecuac. {arabic,alph,Alph,roman,Roman}
\def\captionnumfigure {arabic}     % N° figuras {arabic,alph,Alph,roman,Roman}
\def\captionnumsubfigure {alph}    % N° subfigs. {arabic,alph,Alph,roman,Roman}
\def\captionnumsubtable {alph}     % N° subtabla {arabic,alph,Alph,roman,Roman}
\def\captionnumtable {arabic}      % N° tabla {arabic,alph,Alph,roman,Roman}
\def\captionsubchar {.}            % Carácter entre N° objeto - subfigura/tabla
\def\captiontbmarginfigure {9.35}  % Margen sup/inf de leyenda en figuras [pt]
\def\captiontbmargintable {7}      % Margen sup/inf de la leyenda en tablas [pt]
\def\captiontextbold {false}       % Etiqueta (código,figura,tabla) en negrita
\def\captiontextsubnumbold {false} % N° subfigura/subtabla en negrita
\def\codecaptiontop {true}         % Leyenda arriba del código fuente
\def\equationcaptioncenter {true}  % Ecuaciones están centradas o justificadas
\def\figurecaptiontop {false}      % Leyenda arriba de las imágenes
\def\marginaligncaptbottom {0.1}   % Margen inferior caption en align [cm]
\def\marginaligncapttop {-0.75}    % Margen superior caption en align [cm]
\def\marginalignedcaptbottom {0.1} % Margen inferior caption en aligned [cm]
\def\marginalignedcapttop {-0.75}  % Margen superior caption en aligned [cm]
\def\margineqncaptionbottom {0}    % Margen inferior caption ecuación [cm]
\def\margineqncaptiontop {-0.7}    % Margen superior caption ecuación [cm]
\def\margingathercaptbottom {0.1}  % Margen inferior caption en gather [cm]
\def\margingathercapttop {-0.9}    % Margen superior caption en gather [cm]
\def\margingatheredcaptbottom{0.1} % Margen inferior caption en gathered [cm]
\def\margingatheredcapttop {-0.7}  % Margen superior caption en gathered [cm]
\def\sectioncaptiondelimiter {.}   % Carácter delimitador n° objeto y sección
\def\showsectioncaptioncode {chap} % N° sec código {none,chap,(s/ss/sss/ssss)ec}
\def\showsectioncaptioneqn {chap}  % N° sec ecuac. {none,chap,(s/ss/sss/ssss)ec}
\def\showsectioncaptionfig {chap}  % N° sec figs. {none,chap,(s/ss/sss/ssss)ec}
\def\showsectioncaptionmat {chap}  % N° matemático {none,chap,(s/ss/sss/ssss)ec}
\def\showsectioncaptiontab {chap}  % N° sec tablas {none,chap,(s/ss/sss/ssss)ec}
\def\subcaptionfsize{footnotesize} % Tamaño de la fuente de los subcaption
\def\subcaptionlabelformat{parens} % Formato leyenda sub. {empty,simple,parens}
\def\subcaptionlabelsep {space}    % Sep. {none,colon,period,space,quad,newline}
\def\tablecaptiontop {true}        % Leyenda arriba de las tablas

% CONFIGURACIÓN DEL ÍNDICE
\def\addabstracttobookmarks {true} % Añade el resumen a los marcadores del pdf
\def\addagradectobookmarks {true}  % Añade el agradecimiento a los marcadores
\def\addindexsubtobookmarks {true} % Agrega índice tabla,codigo,etc a marcadores
\def\addindextobookmarks {true}    % Añade el índice a los marcadores del pdf
\def\charafterobjectindex {.}      % Carácter después de n° figura,tabla,código
\def\charnumpageindex {.}          % Carácter número de página en índice
\def\indexdepth {2}                % Profundidad máxima del índice
\def\indexstyle {tf}               % Tipo {f:figura,t:tabla,c:código,e:ecuación}
\def\indextitlemargin {11.4}       % Margen título índice \insertindextitle [pt]
\def\objectchaptermargin {false}   % Activa margen de objetos entre capítulos
\def\objectindexindent {false}     % Indenta la lista de objetos
\def\showappendixsecindex {true}   % Título de la sección de anexos en el índice

% ANEXO, CITAS, REFERENCIAS
\def\apacitebothers {et al.}       % Etiqueta usada en (y otros) con \shortcite
\def\apaciterefcitecharclose {]}   % Carácter final cita apacite
\def\apaciterefcitecharopen {[}    % Carácter inicial cita apacite
\def\apaciterefnumber {false}      % Lista de referencias con números
\def\apaciterefsep {5}             % Separación entre refs. {apacite} [pt]
\def\apaciteshowurl {false}        % Muestra las url en las referencias
\def\apacitestyle {apacite}        % Formato refs. apacite {apa,ieeetr,etc..}
\def\appendixindepobjnum {true}    % Anexo usa n° objetos independientes
\def\backrefpagecite {false}       % Las citas en bibliografía poseen nº de pag.
\def\bibtexindexbibliography{true} % Función \bibliography se inserta en índice
\def\bibtexrefsep {5}              % Separación entre refs. {bibtex} [pt]
\def\bibtexstyle {vancouver}          % Formato refs. bibtex {apa,ieeetr,etc...}
\def\bibtextextalign {justify}     % Alineac. bibtex {justify,left,right,center}
\def\fontsizerefbibl {\normalsize} % Tamaño fuente al usar \bibliography{file}
\def\natbibrefcitecharclose {]}    % Carácter final cita natbib
\def\natbibrefcitecharopen {[}     % Carácter inicial cita natbib
\def\natbibrefcitecompress {true}  % Comprime refererencias al citar
\def\natbibrefcitesepcomma {true}  % Separador en coma (,) o punto y coma (;)
\def\natbibrefcitetype {numbers}   % Tipo citación {authoryear,numbers,super}
\def\natbibrefsep {5}              % Separación entre referencia {natbib} [pt]
\def\natbibrefstyle {natnumurl}    % Formato refs. natbib {apa,ieeetr,etc...}
\def\stylecitereferences {natbib}  % Estilo refs. {apacite,bibtex,natbib,custom}
\def\twocolumnreferences {false}   % Referencias en dos columnas


% CONFIGURACIONES DE OBJETOS
\def\animatedimageautoplay {true}  % Autoplay en imágenes animadas
\def\animatedimagecontrols {false} % Muestra los controles en imágenes animadas
\def\animatedimageloop {true}      % Hace loops en imágenes animadas
\def\columnsepwidth {2.1}          % Separación entre columnas [em]
\def\defaultimagefolder {img/}     % Carpeta raíz de las imágenes
\def\equationleftalign {false}     % Ecuaciones alineadas a la izquierda
\def\equationrestart {none}        % Reinicio n° {none,chap,(s/ss/sss/ssss)ec}
\def\footnotelmargin {10}          % Margen entre footnote y el número [pt]
\def\footnoterestart {none}        % N° foot. {none,chap,page,(s/ss/sss/ssss)ec}
\def\footnoterulefigure {false}    % Footnote en figuras tienen línea superior
\def\footnoterulepage {true}       % Footnote en páginas tienen línea superior
\def\footnoteruletable {false}     % Footnote en tablas tienen línea superior
\def\footnotetwocolumn {false}     % Footnote en dos columnas
\def\fpremovetopbottomcenter{true} % Elimina espacio vert. al centrar con b!,t!
\def\imagedefaultplacement {H}     % Posición por defecto de las imágenes
\def\marginalignbottom {-0.4}      % Margen inferior entorno align [cm]
\def\marginalignedbottom {-0.2}    % Margen inferior entorno aligned [cm]
\def\marginalignedtop {-0.4}       % Margen superior entorno aligned [cm]
\def\marginaligntop {-0.4}         % Margen superior entorno align [cm]
\def\margineqnindexbottom {-0.9}   % Margen inferior ecuaciones índice [cm]
\def\margineqnindextop {0}         % Margen superior ecuaciones índice [cm]
\def\marginequationbottom {-0.2}   % Margen inferior ecuaciones [cm]
\def\marginequationtop {0}         % Margen superior ecuaciones [cm]
\def\marginfloatimages {-13}       % Margen sup figs. insertimageleft/right [pt]
\def\margingatherbottom {-0.2}     % Margen inferior entorno gather [cm]
\def\margingatheredbottom {-0.1}   % Margen inferior entorno gathered [cm]
\def\margingatheredtop {-0.4}      % Margen superior entorno gathered [cm]
\def\margingathertop {-0.4}        % Margen superior entorno gather [cm]
\def\marginimagebottom {-0.15}     % Margen inferior figura [cm]
\def\marginimagemultbottom {0.25}  % Margen inferior imágenes múltiples [cm]
\def\marginimagemultright {0.5}    % Margen derecho imágenes múltiples [cm]
\def\marginimagemulttop {-0.3}     % Margen superior imágenes múltiples [cm]
\def\marginimagetop {0}            % Margen superior figuras [cm]
\def\marginlinenumbers {7.5}       % Margen izquierdo (\showlinenumbers) [pt]
\def\numberedequation {true}       % Ecuaciones con \insert... numeradas
\def\senumerti {\arabic{enumi}.}   % Estilo enumerate nivel 1
\def\senumertii {\alph{enumii})}   % Estilo enumerate nivel 2
\def\senumertiii{\roman{enumiii}.} % Estilo enumerate nivel 3
\def\senumertiv {\Alph{enumiv})}   % Estilo enumerate nivel 4
\def\sitemizei {\iitembcirc}       % Estilo itemize nivel 1
\def\sitemizeii {\iitemdash}       % Estilo itemize nivel 2
\def\sitemizeiii {\iitemcirc}      % Estilo itemize nivel 3
\def\sitemizeiv {\iitembsquare}    % Estilo itemize nivel 4
\def\sitemsmargini {25}            % Margen ítems nivel 1 [pt]
\def\sitemsmarginii {22}           % Margen ítems nivel 2 [pt]
\def\sitemsmarginiii {18.7}        % Margen ítems nivel 3 [pt]
\def\sitemsmarginiv {17}           % Margen ítems nivel 4 [pt]
\def\sourcecodebgmarginbottom {0}  % Margen inferior del bloque de color [pt]
\def\sourcecodebgmarginleft {0}    % Margen izquierdo del bloque de color [pt]
\def\sourcecodebgmarginright {0}   % Margen derecho del bloque de color [pt]
\def\sourcecodebgmargintop {0}     % Margen superior del bloque de color [pt]
\def\sourcecodefontf {\ttfamily}   % Tipo de letra código fuente
\def\sourcecodefonts {\small}      % Tamaño letra código fuente
\def\sourcecodeilfontf {\ttfamily} % Tipo de letra código fuente inline
\def\sourcecodeilfonts {\small}    % Tamaño letra código fuente inline
\def\sourcecodenumbersep {6}       % Separación entre número línea y código [pt]
\def\sourcecodenumbersize {\tiny}  % Tamaño fuente número línea
\def\sourcecodeskipabove {0.75}    % Espacio sobre recuadro código [em]
\def\sourcecodeskipbelow {0.95}    % Espacio bajo recuadro código [em]
\def\sourcecodetabsize {3}         % Tamaño tabulación código fuente
\def\tabledefaultplacement {L}     % Posición por defecto de las tablas
\def\tablenotesameline {true}      % Notas en tablas en una sola línea
\def\tablenotesfontsize {\small}   % Tamaño de fuente de las notas en tablas
\def\tablepaddingh {0.75}          % Espaciado horizontal de celda de las tablas
\def\tablepaddingv {0.5}          % Espaciado vertical de celda de las tablas
\def\tikzdefaultplacement {H}      % Posición por defecto de las figuras tikz

% CONFIGURACIÓN DE LOS TÍTULOS
\def\anumsecaddtocounter {false}   % Insertar títulos anum. aumenta n° de sec
\def\chapterfontsize {\LARGE}       % Tamaño fuente de los capítulos
\def\chapterfontstyle {\bfseries}  % Estilo fuente de los capítulos
\def\charaftersectionnum {}       % Carácter después n° (s/ss/sss/ssss)ection
\def\charappendixsection {.}       % Carácter entre n° sección anexo y título
\def\charbetwchaptersection {.}    % Carácter entre nº capítulo y sección
\def\charbetwsectionsubsection {.} % Carácter entre nº sección y subsección
\def\charbetwssectionsssect {.}    % Carácter entre nº subsubsección y sssección
\def\charbetwsubsectionssect {.}   % Carácter entre nº subsección y ssección
\def\formatnumapchapter {\Alph}    % Formato nº capítulo en appendixd
\def\formatnumapsection {\arabic}  % Formato nº sección en appendixd
\def\formatnumapssection {\arabic} % Formato nº subsección en appendixd
\def\formatnumapsssection{\arabic} % Formato nº sub-subsección en appendixd
\def\formatnumapssssection{\arabic}% Formato nº sub-sub-subsección appendixd
\def\formatnumchapter {\arabic}    % Formato número capítulo
\def\formatnumsection {\arabic}    % Formato número sección
\def\formatnumssection {\arabic}   % Formato número subsección
\def\formatnumsssection {\arabic}  % Formato número sub-subsección
\def\formatnumssssection {\arabic} % Formato número sub-sub-subsección
\def\paragfontsize {\normalsize}   % Tamaño fuente paragraph
\def\paragfontstyle {\bfseries}    % Estilo fuente paragraph
\def\paragspacingbottom {4}        % Espaciado inferior en paragraph [pt]
\def\paragspacingleft {0}          % Espaciado izq. en paragraph [pt]
\def\paragspacingtop {8}           % Espaciado superior en paragraph [pt]
\def\paragsubfontsize{\normalsize} % Tamaño fuente subparagraph
\def\paragsubfontstyle {\bfseries} % Estilo fuente subparagraph
\def\paragsubspacingbottom {4}     % Espaciado inferior en subparagraph [pt]
\def\paragsubspacingleft {0}       % Espaciado izq. en subparagraph [pt]
\def\paragsubspacingtop {8}        % Espaciado superior en subparagraph [pt]
\def\sectionfontsize {\Large}      % Tamaño fuente section
\def\sectionfontstyle {\bfseries}  % Estilo fuente section
\def\sectionspacingbottom {8}     % Espaciado inferior en section [pt]
\def\sectionspacingleft {0}        % Espaciado izq. en section [pt]
\def\sectionspacingtop {14}        % Espaciado superior en section [pt]
\def\spacingaftersection {\quad}   % Espaciador después nº sección
\def\ssectionfontsize {\large}     % Tamaño fuente subtítulos
\def\ssectionfontstyle {\bfseries} % Estilo fuente subsection
\def\ssectionspacingbottom {8}    % Espaciado inferior en subsection [pt]
\def\ssectionspacingleft {0}       % Espaciado izq. en subsection [pt]
\def\ssectionspacingtop {10}       % Espaciado superior en subsection [pt]
\def\sssectionfontsize{\normalsize}% Tamaño fuente subsubsection
\def\sssectionfontstyle{\bfseries} % Estilo fuente subsubsection
\def\sssectionspacingbottom {8}    % Espaciado inferior en subsubsection [pt]
\def\sssectionspacingleft {0}      % Espaciado izq. en subsubsection [pt]
\def\sssectionspacingtop {7}      % Espaciado superior en subsubsection [pt]
\def\ssssectionfontstyle{\bfseries}% Estilo fuente subsubsubsection
\def\ssssectionfontsz{\normalsize} % Tamaño fuente subsubsubsection
\def\ssssectionspacingbottom {6}   % Espaciado inferior en subsubsubsection [pt]
\def\ssssectionspacingleft {0}     % Espaciado izq. en subsubsubsection [pt]
\def\ssssectionspacingtop {5}      % Espaciado superior en subsubsubsection [pt]

% CONFIGURACIÓN DE LOS COLORES DEL DOCUMENTO
\def\chaptercolor {black}          % Color de los capítulos
\def\captioncolor {gray}          % Color nombre objeto (código,figura,tabla)
\def\captiontextcolor {gray}      % Color de la leyenda
\def\enumerateitemcolor {black}    % Color de los enumerate por defecto
\def\highlightcolor {yellow}       % Color del subrayado con \hl
\def\indextitlecolor {black}       % Color de los títulos del índice
\def\itemizeitemcolor {black}      % Color de los ítems por defecto
\def\linenumbercolor {gray}        % Color del n° de línea (\showlinenumbers)
\def\linkcolor {black}             % Color de los links del documento
\def\maintextcolor {black}         % Color principal del texto
\def\numcitecolor {black}          % Color del n° de las referencias o citas
\def\pagescolor {white}            % Color de la página
\def\paragcolor {black}            % Color de los paragraph
\def\paragsubcolor {black}         % Color de los subparagraph
\def\sectioncolor {black}          % Color de los section
\def\showborderonlinks {false}     % Color de un link por un recuadro de color
\def\sourcecodebgcolor {lgray}     % Color de fondo del código fuente
\def\ssectioncolor {black}         % Color de los subsection
\def\sssectioncolor {black}        % Color de los subsubsection
\def\ssssectioncolor {black}       % Color de los subsubsubsection
\def\tablelinecolor {black}        % Color de las líneas de las tablas
\def\tablerowfirstcolor {none}     % Primer color de celda de las tablas
\def\tablerowsecondcolor {gray!20} % Segundo color de celda de las tablas
\def\urlcolor {magenta}            % Color de los enlaces web (\href,\url)

% MÁRGENES DE PÁGINA
\def\pagemarginbottom {2}          % Margen inferior página [cm]
\def\pagemarginleft {2}            % Margen izquierdo página [cm]
\def\pagemarginleftportrait {2}  % Margen izquierdo página portada [cm]
\def\pagemarginright {2}           % Margen derecho página [cm]
\def\pagemargintop {2}             % Margen superior página [cm]

% OPCIONES DEL PDF COMPILADO
\def\cfgbookmarksopenlevel {0}     % Nivel marcadores en pdf (1:secciones)
\def\cfgpdfbookmarkopen {true}     % Expande marcadores del nivel configurado
\def\cfgpdfcenterwindow {true}     % Centra ventana del lector al abrir el pdf
\def\cfgpdfcopyright {}            % Establece el copyright del documento
\def\cfgpdfdisplaydoctitle {true}  % Muestra título del informe en visor
\def\cfgpdffitwindow {false}       % Ajusta la ventana del lector tamaño pdf
\def\cfgpdfkeywords {}             % Palabras clave del pdf
\def\cfgpdflayout {OneColumn}      % Modo de página {OneColumn,SinglePage}
\def\cfgpdfmenubar {true}          % Muestra el menú del lector
\def\cfgpdfpageview {FitH}         % {Fit,FitH,FitV,FitR,FitB,FitBH,FitBV}
\def\cfgpdfsecnumbookmarks {true}  % Número de la sec. en marcadores del pdf
\def\cfgpdftoolbar {true}          % Muestra barra de herramientas lector pdf
\def\cfgshowbookmarkmenu {true}    % Muestra menú marcadores al abrir el pdf
\def\pdfcompilecompression {9}     % Factor de compresión del pdf (0-9)
\def\pdfcompileobjcompression {2}  % Nivel compresión objetos del pdf (0-3)
\def\pdfcompileversion {7}         % Versión mínima del pdf compilado
\def\usepdfmetadata {true}         % Añade metadatos al pdf compilado

% NOMBRE DE OBJETOS
\def\nameabstract {Resumen}           % Nombre del resumen-abstract
\def\nameagradec {Agradecimientos}    % Nombre del cap. de agradecimientos
\def\nameappendixsection {Anexos}     % Nombre de los anexos
\def\namechapter {Capítulo}           % Nombre de los capítulos
\def\nameltappendixsection {Anexo}    % Etiqueta sección en anexo/apéndices
\def\nameltcont{Índice}   % Nombre del índice de contenidos
\def\namelteqn {Índice de Ecuaciones} % Nombre de la lista de ecuaciones
\def\nameltfigure{Índice de Ilustraciones} % Nombre de la lista de figuras
\def\nameltsrc {Índice de Códigos}    % Nombre de la lista de código
\def\namelttable {Índice de Tablas}   % Nombre de la lista de tablas
\def\nameltwfigure {Figura}           % Etiqueta leyenda de las figuras
\def\nameltwsrc {Código}              % Etiqueta leyenda del código fuente
\def\nameltwtable {Tabla}             % Etiqueta leyenda de las tablas
\def\namemathcol {Corolario}          % Nombre de los colorarios
\def\namemathdefn {Definición}        % Nombre de las definiciones
\def\namemathej {Ejemplo}             % Nombre de los ejemplos
\def\namemathlem {Lema}               % Nombre de los lemas
\def\namemathobs {Observación}        % Nombre de las observaciones
\def\namemathprp {Proposición}        % Nombre de las proposiciones
\def\namemaththeorem {Teorema}        % Nombre de los teoremas
\def\namepageof { de }                % Etiqueta página # de #
\def\nameportraitpage {Cover}         % Etiqueta página de la portada
\def\namereferences {Referencias bibliográficas}    % Nombre de la sección de referencias


% -----------------------------------------------------------------------------
% IMPORTACIÓN DE LIBRERÍAS
% -----------------------------------------------------------------------------
% Se guardan variables antes de cargar librerías
\let\RE\Re
\let\IM\Im

% Parches de librerías
\let\counterwithout\relax
\let\counterwithin\relax
\let\underbar\relax
\let\underline\relax

% Si se desactiva el idioma
\def\unaccentedoperators {}
\def\decimalpoint {}
\def\bibname {}

% Parche de sectsty.sty
\makeatletter
\def\underline#1{\relax\ifmmode\@@underline{#1}\else $\@@underline{\hbox{#1}}\m@th$\relax\fi}
\def\underbar#1{\underline{\sbox\tw@{#1}\dp\tw@\z@\box\tw@}}
\makeatother

% -----------------------------------------------------------------------------
% Librerías del núcleo
% -----------------------------------------------------------------------------
% Manejo de condicionales
\usepackage{iftex}
\usepackage{ifthen}

% Verifica el tipo de compilador
\ifPDFTeX
	\def\compilertype {pdf2latex}
\else\ifXeTeX
	\def\compilertype {xelatex}
\else\ifLuaTeX
	\def\compilertype {lualatex}
\else
	\errmessage{Compilador no soportado}
	\stop
	\fi\fi
\fi

% Carga el idioma
\usepackage{tracklang}
\IfTrackedLanguage{spanish}{
	\usepackage[es-nosectiondot,es-lcroman,es-noquoting]{babel}
}{ % english, otros
	\usepackage{babel}
}

% Cambia el estilo de los títulos
\usepackage{sectsty}

% Codificación
\ifthenelse{\equal{\compilertype}{pdf2latex}}{
	\usepackage[utf8]{inputenc}}{
}

% Lanza un mensaje de error indicando mala configuración
%	#1	Parámetros opcionales (nostop,noheader)
%	#2	Mensaje de error
% 	#3	Configuración usada
%	#4	Valores esperados
\newcommand{\throwbadconfig}[4][]{
	\ifthenelse{\equal{#1}{noheader}}{
		\errmessage{LaTeX Warning: #4}
	}{
		\ifthenelse{\equal{#1}{noheader-nostop}}{
			\errmessage{LaTeX Warning: #4}
		}{
			\errmessage{LaTeX Warning: #2 (\noexpand #3= #3). Valores esperados: #4}
		}
	}
	\ifthenelse{\equal{#1}{nostop}}{}{
		\ifthenelse{\equal{#1}{noheader-nostop}}{}{
			\stop
		}
	}
}

% Librerías matemáticas
\ifthenelse{\equal{\equationleftalign}{true}}{
	\usepackage[fleqn]{amsmath}
}{
	\usepackage{amsmath}
}

% Tamaño de la fuente del documento
\usepackage{scrextend}
\usepackage{anyfontsize}
\changefontsizes{\documentfontsize pt}

% Evita error "Too many alphabets used in version normal"
\newcommand\hmmax {0}
\newcommand\bmmax {0}

% -----------------------------------------------------------------------------
% Librerías independientes
% -----------------------------------------------------------------------------
\usepackage{amsbsy}        % Símbolos matemáticos en negrita
\usepackage{amssymb}       % Librerías matemáticas
\usepackage{amsthm}        % Definición de teoremas
\usepackage{animate}       % Imágenes animadas
\usepackage{array}         % Nuevas características a las tablas
\usepackage{bigstrut}      % Líneas horizontales en tablas
\usepackage{bm}            % Caracteres en negrita en ecuaciones
\usepackage{booktabs}      % Permite manejar elementos visuales en tablas
\usepackage{caption}       % Leyendas
\usepackage{chngcntr}      % Añade números a las leyendas
\usepackage{color}         % Colores
\usepackage{datetime}      % Fechas
\usepackage{floatpag}      % Maneja estilos de páginas introducidos por objetos flotantes
\usepackage{floatrow}      % Permite administrar posiciones en los caption
\usepackage{framed}        % Permite creación de recuadros
\usepackage{gensymb}       % Simbología común
\usepackage{graphicx}      % Propiedades extra para los gráficos
\usepackage{lipsum}        % Permite crear párrafos de prueba
\usepackage{listings}      % Permite añadir código fuente
\usepackage{longtable}     % Permite utilizar tablas en varias hojas
\usepackage{mathrsfs}      % Define más fuentes matemáticas
\usepackage{mathtools}     % Permite utilizar notaciones matemáticas
\usepackage{multicol}      % Múltiples columnas
\usepackage{needspace}     % Maneja los espacios en página
\usepackage{pdflscape}     % Modo página horizontal de página
\usepackage{pdfpages}      % Permite administrar páginas en pdf
\usepackage{physics}       % Paquete de matemáticas
\usepackage{realboxes}     % Permite inserción de recuadros
\usepackage{rotating}      % Permite rotación de objetos
\usepackage{selinput}      % Compatibilidad con acentos
\usepackage{setspace}      % Cambia el espacio entre líneas
\usepackage{soul}          % Permite subrayar texto
\usepackage{stfloats}      % Permite cambiar posición de flotantes con [b] y [t]
\usepackage{subcaption}    % Permite agrupar imágenes
\usepackage{textcomp}      % Simbología común
\usepackage{wrapfig}       % Posición de imágenes
\usepackage{xspace}        % Administra espacios en párrafos y líneas
\usepackage{xurl}          % Permite añadir enlaces

% -----------------------------------------------------------------------------
% Librerías con parámetros
% -----------------------------------------------------------------------------
\usepackage[export]{adjustbox} % Agrega nuevas etiquetas de posicionado
\usepackage[makeroom]{cancel} % Cancelar términos en fórmulas
\usepackage[inline]{enumitem} % Permite enumerar ítems
\usepackage[titles]{tocloft} % Maneja entradas en el índice
\usepackage[figure,table,lstlisting]{totalcount} % Contador de objetos
\usepackage[normalem]{ulem} % Permite tachar y subrayar
\usepackage[nointegrals]{wasysym} % Contiene caracteres misceláneos
\usepackage[dvipsnames,table,usenames]{xcolor} % Paquete de colores avanzado

% -----------------------------------------------------------------------------
% Librerías condicionales
% -----------------------------------------------------------------------------
% Imágenes en modo draft
\ifthenelse{\equal{\graphicxdraft}{true}}{
	\usepackage[
		allfiguresdraft,
		filename,
		size={scriptsize},
		style={tt}
	]{draftfigure}}{
}

% Acepta codificación UTF-8 en código fuente
\ifthenelse{\equal{\compilertype}{pdf2latex}}{
	\usepackage{listingsutf8}}{
}

% Footnotes en dos columnas
\ifthenelse{\equal{\footnotetwocolumn}{true}}{
	\usepackage{dblfnote}}{
}

% Regla superior
\ifthenelse{\equal{\footnoterulepage}{true}}{
	\usepackage[bottom,hang]{footmisc} % Estilo pie de página
}{
	\usepackage[bottom,norule,hang]{footmisc}
}

% Referencias
\ifthenelse{\equal{\backrefpagecite}{true}}{
	\usepackage[pdfencoding=auto,psdextra,backref=page]{hyperref} % Enlaces, referencias
}{
	\usepackage[pdfencoding=auto,psdextra]{hyperref} % Enlaces, referencias
}
\ifthenelse{\equal{\stylecitereferences}{natbib}}{ % Formato citas natbib
	\usepackage[nottoc,notlof,notlot]{tocbibind}
	\ifthenelse{\equal{\natbibrefcitecompress}{true}}{
		\usepackage[sort&compress]{natbib}
	}{
		\usepackage{natbib}
	}
}{
\ifthenelse{\equal{\stylecitereferences}{apacite}}{ % Formato citas apacite
	\usepackage[nottoc,notlof,notlot]{tocbibind}
	\usepackage[nosectionbib]{apacite}
}{
\ifthenelse{\equal{\stylecitereferences}{bibtex}}{ % Formato citas bibtex
}{
\ifthenelse{\equal{\stylecitereferences}{custom}}{ % Formato citas custom
}{}}}
}

% Anexos/Apéndices
\ifthenelse{\equal{\showappendixsecindex}{true}}{
	\usepackage[toc]{appendix} % Eliminado en Auxiliares/Controles, sin [toc]
}{
	\usepackage{appendix}
}

% Dimensiones y geometría del documento
\ifthenelse{\equal{\compilertype}{lualatex}}{ % En lualatex sólo se puede cambiar 1 vez el margen
	\usepackage[top=\pagemargintop cm,bottom=\pagemarginbottom cm,left=\pagemarginleft cm,right=\pagemarginright cm]{geometry}
}{ % pdf2latex, xelatex
	\usepackage{geometry}
}

% Notas en tablas
\ifthenelse{\equal{\tablenotesameline}{true}}{
	\usepackage[para]{threeparttable}
}{
	\usepackage{threeparttable}
}

% Indentación del primer párrafo
\ifthenelse{\equal{\indentfirstpar}{true}}{
	\usepackage{indentfirst}}{
}

% -----------------------------------------------------------------------------
% Librerías dependientes
% -----------------------------------------------------------------------------
\usepackage{bookmark}      % Administración de marcadores en pdf
\usepackage{fancyhdr}      % Encabezados y pie de páginas
\usepackage{float}         % Administrador de posiciones de objetos
\usepackage{hyperxmp}      % Etiquetas opcionales para el pdf compilado
\usepackage{multirow}      % Agrega nuevas opciones a las tablas
\usepackage{notoccite}     % Desactiva las citas en el índice
\usepackage{titlesec}      % Administración de títulos

% -----------------------------------------------------------------------------
% Tipografía del documento
% -----------------------------------------------------------------------------
% Tipografías clásicas
\ifthenelse{\equal{\fontdocument}{lmodern}}{
	\usepackage{lmodern}
}{
\ifthenelse{\equal{\fontdocument}{arial}}{
	\usepackage{helvet}
	\renewcommand{\familydefault}{\sfdefault}
}{
\ifthenelse{\equal{\fontdocument}{arial2}}{
	\usepackage{arial}
}{
\ifthenelse{\equal{\fontdocument}{times}}{
	\usepackage{mathptmx}
}{
\ifthenelse{\equal{\fontdocument}{mathptmx}}{
	\usepackage{mathptmx}
}{
\ifthenelse{\equal{\fontdocument}{helvet}}{
	\renewcommand{\familydefault}{\sfdefault}
	\usepackage[scaled=0.95]{helvet}
	\usepackage[helvet]{sfmath}
}{
\ifthenelse{\equal{\fontdocument}{opensans}}{
	\usepackage[default,scale=0.95]{opensans}
}{
\ifthenelse{\equal{\fontdocument}{mathpazo}}{
	\usepackage{mathpazo}
}{
\ifthenelse{\equal{\fontdocument}{cambria}}{
	\usepackage{caladea}
}{
\ifthenelse{\equal{\fontdocument}{libertine}}{
	\usepackage[libertine]{newtxmath}
	\usepackage[tt=false]{libertine}
}{
\ifthenelse{\equal{\fontdocument}{custom}}{
}{

% Otros (último: fbb el 08/08/2021 - https://tug.org/FontCatalogue/seriffonts.html)
\ifthenelse{\equal{\fontdocument}{accanthis}}{
	\usepackage{accanthis}
}{
\ifthenelse{\equal{\fontdocument}{alegreya}}{
	\usepackage{Alegreya}
	\renewcommand*\oldstylenums[1]{{\AlegreyaOsF #1}}
}{
\ifthenelse{\equal{\fontdocument}{alegreyasans}}{
	\usepackage[sfdefault]{AlegreyaSans}
	\renewcommand*\oldstylenums[1]{{\AlegreyaSansOsF #1}}
}{
\ifthenelse{\equal{\fontdocument}{algolrevived}}{
	\usepackage{algolrevived}
}{
\ifthenelse{\equal{\fontdocument}{almendra}}{
	\usepackage{almendra}
}{
\ifthenelse{\equal{\fontdocument}{antpolt}}{
	\usepackage{antpolt}
}{
\ifthenelse{\equal{\fontdocument}{antpoltlight}}{
	\usepackage[light]{antpolt}
}{
\ifthenelse{\equal{\fontdocument}{anttor}}{
	\usepackage[math]{anttor}
}{
\ifthenelse{\equal{\fontdocument}{anttorcondensed}}{
	\usepackage[condensed,math]{anttor}
}{
\ifthenelse{\equal{\fontdocument}{anttorlight}}{
	\usepackage[light,math]{anttor}
}{
\ifthenelse{\equal{\fontdocument}{anttorlightcondensed}}{
	\usepackage[light,condensed,math]{anttor}
}{
\ifthenelse{\equal{\fontdocument}{arev}}{
	\let\quarternote\relax
	\let\eighthnote\relax
	\usepackage{arev}
}{
\ifthenelse{\equal{\fontdocument}{arimo}}{
	\usepackage[sfdefault]{arimo}
	\renewcommand*\familydefault{\sfdefault}
}{
\ifthenelse{\equal{\fontdocument}{arvo}}{
	\usepackage{Arvo}
}{
\ifthenelse{\equal{\fontdocument}{baskervald}}{
	\usepackage{baskervald}
}{
\ifthenelse{\equal{\fontdocument}{baskervaldx}}{
	\usepackage[lf]{Baskervaldx}
	\usepackage[bigdelims,vvarbb]{newtxmath}
	\usepackage[cal=boondoxo]{mathalfa}
	\renewcommand*\oldstylenums[1]{\textosf{#1}}
}{
\ifthenelse{\equal{\fontdocument}{berasans}}{
	\usepackage[scaled]{berasans}
	\renewcommand*\familydefault{\sfdefault}
}{
\ifthenelse{\equal{\fontdocument}{beraserif}}{
	\usepackage{bera}
}{
\ifthenelse{\equal{\fontdocument}{biolinum}}{
	\usepackage{libertine}
	\renewcommand*\familydefault{\sfdefault}
}{
\ifthenelse{\equal{\fontdocument}{bitter}}{
	\usepackage{bitter}
}{
\ifthenelse{\equal{\fontdocument}{boisik}}{
	\let\div\relax
	\usepackage{boisik}
}{
\ifthenelse{\equal{\fontdocument}{bookman}}{
	\usepackage{bookman}
}{
\ifthenelse{\equal{\fontdocument}{cabin}}{
	\usepackage[sfdefault]{cabin}
	\renewcommand*\familydefault{\sfdefault}
}{
\ifthenelse{\equal{\fontdocument}{cabincondensed}}{
	\usepackage[sfdefault,condensed]{cabin}
	\renewcommand*\familydefault{\sfdefault}
}{
\ifthenelse{\equal{\fontdocument}{caladea}}{
	\usepackage{caladea}
}{
\ifthenelse{\equal{\fontdocument}{cantarell}}{
	\usepackage[default]{cantarell}
}{
\ifthenelse{\equal{\fontdocument}{carlito}}{
	\usepackage[sfdefault]{carlito}
	\renewcommand*\familydefault{\sfdefault}
}{
\ifthenelse{\equal{\fontdocument}{charterbt}}{
	\usepackage[bitstream-charter]{mathdesign}
}{
\ifthenelse{\equal{\fontdocument}{chivolight}}{
	\usepackage[familydefault,light]{Chivo}
}{
\ifthenelse{\equal{\fontdocument}{chivoregular}}{
	\usepackage[familydefault,regular]{Chivo}
}{
\ifthenelse{\equal{\fontdocument}{clara}}{
	\usepackage{clara}
}{
\ifthenelse{\equal{\fontdocument}{clearsans}}{
	\usepackage[sfdefault]{ClearSans}
	\renewcommand*\familydefault{\sfdefault}
}{
\ifthenelse{\equal{\fontdocument}{cochineal}}{
	\usepackage{cochineal}
}{
\ifthenelse{\equal{\fontdocument}{coelacanth}}{
	\usepackage[nf]{coelacanth}
	\let\oldnormalfont\normalfont
	\def\normalfont{\oldnormalfont\mdseries}
}{
\ifthenelse{\equal{\fontdocument}{coelacanthextralight}}{
	\usepackage[el,nf]{coelacanth}
	\let\oldnormalfont\normalfont
	\def\normalfont{\oldnormalfont\mdseries}
}{
\ifthenelse{\equal{\fontdocument}{coelacanthlight}}{
	\usepackage[l,nf]{coelacanth}
	\let\oldnormalfont\normalfont
	\def\normalfont{\oldnormalfont\mdseries}
}{
\ifthenelse{\equal{\fontdocument}{comfortaa}}{
	\usepackage[default]{comfortaa}
}{
\ifthenelse{\equal{\fontdocument}{comicneue}}{
	\usepackage[default]{comicneue}
}{
\ifthenelse{\equal{\fontdocument}{comicneueangular}}{
	\usepackage[default,angular]{comicneue}
}{
\ifthenelse{\equal{\fontdocument}{computerconcrete}}{
	\usepackage{concmath}
}{
\ifthenelse{\equal{\fontdocument}{computerconcreteeuler}}{
	\let\Re\relax
	\let\Im\relax
	\usepackage{beton}
	\usepackage{euler}
}{
\ifthenelse{\equal{\fontdocument}{computermodern}}{
}{
\ifthenelse{\equal{\fontdocument}{computermodernbright}}{
	\usepackage{cmbright}
}{
\ifthenelse{\equal{\fontdocument}{crimson}}{
	\usepackage{crimson}
}{
\ifthenelse{\equal{\fontdocument}{crimsonpro}}{
	\usepackage{CrimsonPro}
	\let\oldnormalfont\normalfont
	\def\normalfont{\oldnormalfont\mdseries}
}{
\ifthenelse{\equal{\fontdocument}{crimsonproextralight}}{
	\usepackage[extralight]{CrimsonPro}
	\let\oldnormalfont\normalfont
	\def\normalfont{\oldnormalfont\mdseries}
}{
\ifthenelse{\equal{\fontdocument}{crimsonprolight}}{
	\usepackage[light]{CrimsonPro}
	\let\oldnormalfont\normalfont
	\def\normalfont{\oldnormalfont\mdseries}
}{
\ifthenelse{\equal{\fontdocument}{crimsonpromedium}}{
	\usepackage[medium]{CrimsonPro}
	\let\oldnormalfont\normalfont
	\def\normalfont{\oldnormalfont\mdseries}
}{
\ifthenelse{\equal{\fontdocument}{cyklop}}{
	\usepackage{cyklop}
}{
\ifthenelse{\equal{\fontdocument}{dejavusans}}{
	\usepackage{DejaVuSans}
	\renewcommand*\familydefault{\sfdefault}
}{
\ifthenelse{\equal{\fontdocument}{dejavusanscondensed}}{
	\usepackage{DejaVuSansCondensed}
	\renewcommand*\familydefault{\sfdefault}
}{
\ifthenelse{\equal{\fontdocument}{domitian}}{
	\usepackage{mathpazo}
	\usepackage{domitian}
	\let\oldstylenums\oldstyle
}{
\ifthenelse{\equal{\fontdocument}{droidsans}}{
	\usepackage[defaultsans]{droidsans}
	\renewcommand*\familydefault{\sfdefault}
}{
\ifthenelse{\equal{\fontdocument}{electrum}}{
	\usepackage[lf]{electrum}
}{	
\ifthenelse{\equal{\fontdocument}{erewhon}}{
	\usepackage[proportional,scaled=1.064]{erewhon}
	\usepackage[erewhon,vvarbb,bigdelims]{newtxmath}
	\renewcommand*\oldstylenums[1]{\textosf{#1}}
}{
\ifthenelse{\equal{\fontdocument}{fbb}}{
	\usepackage{fbb}
}{
\ifthenelse{\equal{\fontdocument}{fetamont}}{
	\usepackage{fetamont}
	\renewcommand*\familydefault{\sfdefault}
}{
\ifthenelse{\equal{\fontdocument}{firasans}}{
	\usepackage[sfdefault]{FiraSans}
	\renewcommand*\familydefault{\sfdefault}
}{
\ifthenelse{\equal{\fontdocument}{firasansnewtxsf}}{
	\usepackage[sfdefault]{FiraSans}
	\usepackage{newtxsf}
}{
\ifthenelse{\equal{\fontdocument}{fourier}}{
	\usepackage{fourier}
}{
\ifthenelse{\equal{\fontdocument}{fouriernc}}{
	\usepackage{fouriernc}
}{
\ifthenelse{\equal{\fontdocument}{gfsartemisia}}{
	\let\textlozenge\relax
	\usepackage{gfsartemisia}
}{
\ifthenelse{\equal{\fontdocument}{gfsartemisiaeuler}}{
	\let\textlozenge\relax
	\let\Re\relax
	\let\Im\relax
	\usepackage{gfsartemisia-euler}
}{
\ifthenelse{\equal{\fontdocument}{heuristica}}{
	\usepackage{heuristica}
	\usepackage[heuristica,vvarbb,bigdelims]{newtxmath}
	\renewcommand*\oldstylenums[1]{\textosf{#1}}
}{
\ifthenelse{\equal{\fontdocument}{iwona}}{
	\usepackage[math]{iwona}
}{
\ifthenelse{\equal{\fontdocument}{iwonacondensed}}{
	\usepackage[condensed,math]{iwona}
}{
\ifthenelse{\equal{\fontdocument}{iwonalight}}{
	\usepackage[light,math]{iwona}
}{
\ifthenelse{\equal{\fontdocument}{iwonalightcondensed}}{
	\usepackage[light,condensed,math]{iwona}
}{
\ifthenelse{\equal{\fontdocument}{kerkis}}{
	\usepackage{kmath,kerkis}
}{
\ifthenelse{\equal{\fontdocument}{kurier}}{
	\usepackage[math]{kurier}
}{
\ifthenelse{\equal{\fontdocument}{kuriercondensed}}{
	\usepackage[condensed,math]{kurier}
}{
\ifthenelse{\equal{\fontdocument}{kurierlight}}{
	\usepackage[light,math]{kurier}
}{
\ifthenelse{\equal{\fontdocument}{kurierlightcondensed}}{
	\usepackage[light,condensed,math]{kurier}
}{
\ifthenelse{\equal{\fontdocument}{lato}}{
	\usepackage[default]{lato}
}{
\ifthenelse{\equal{\fontdocument}{libertinus}}{
	\usepackage{libertinus}
}{
\ifthenelse{\equal{\fontdocument}{librebaskerville}}{
	\usepackage{librebaskerville}
}{
\ifthenelse{\equal{\fontdocument}{librebodoni}}{
	\usepackage{LibreBodoni}
}{
\ifthenelse{\equal{\fontdocument}{librecaslon}}{
	\usepackage{librecaslon}
}{
\ifthenelse{\equal{\fontdocument}{libris}}{
	\usepackage{libris}
	\renewcommand*\familydefault{\sfdefault}
}{
\ifthenelse{\equal{\fontdocument}{lxfonts}}{
	\usepackage{lxfonts}
}{
\ifthenelse{\equal{\fontdocument}{merriweather}}{
	\usepackage[sfdefault]{merriweather}
}{
\ifthenelse{\equal{\fontdocument}{merriweatherlight}}{
	\usepackage[sfdefault,light]{merriweather}
}{
\ifthenelse{\equal{\fontdocument}{mintspirit}}{
	\usepackage[default]{mintspirit}
}{
\ifthenelse{\equal{\fontdocument}{mlmodern}}{
	\usepackage{mlmodern}
}{
\ifthenelse{\equal{\fontdocument}{montserratalternatesextralight}}{
	\usepackage[defaultfam,extralight,tabular,lining,alternates]{montserrat}
	\renewcommand*\oldstylenums[1]{{\fontfamily{Montserrat-TOsF}\selectfont #1}}
}{
\ifthenelse{\equal{\fontdocument}{montserratalternatesregular}}{
	\usepackage[defaultfam,tabular,lining,alternates]{montserrat}
	\renewcommand*\oldstylenums[1]{{\fontfamily{Montserrat-TOsF}\selectfont #1}}
}{
\ifthenelse{\equal{\fontdocument}{montserratalternatesthin}}{
	\usepackage[defaultfam,thin,tabular,lining,alternates]{montserrat}
	\renewcommand*\oldstylenums[1]{{\fontfamily{Montserrat-TOsF}\selectfont #1}}
}{
\ifthenelse{\equal{\fontdocument}{montserratextralight}}{
	\usepackage[defaultfam,extralight,tabular,lining]{montserrat}
	\renewcommand*\oldstylenums[1]{{\fontfamily{Montserrat-TOsF}\selectfont #1}}
}{
\ifthenelse{\equal{\fontdocument}{montserratlight}}{
	\usepackage[defaultfam,light,tabular,lining]{montserrat}
	\renewcommand*\oldstylenums[1]{{\fontfamily{Montserrat-TOsF}\selectfont #1}}
}{
\ifthenelse{\equal{\fontdocument}{montserratregular}}{
	\usepackage[defaultfam,tabular,lining]{montserrat}
	\renewcommand*\oldstylenums[1]{{\fontfamily{Montserrat-TOsF}\selectfont #1}}
}{
\ifthenelse{\equal{\fontdocument}{montserratthin}}{
	\usepackage[defaultfam,thin,tabular,lining]{montserrat}
	\renewcommand*\oldstylenums[1]{{\fontfamily{Montserrat-TOsF}\selectfont #1}}
}{
\ifthenelse{\equal{\fontdocument}{newpx}}{
	\usepackage{newpxtext,newpxmath}
}{
\ifthenelse{\equal{\fontdocument}{nimbussans}}{
	\usepackage{nimbussans}
	\renewcommand*\familydefault{\sfdefault}
}{
\ifthenelse{\equal{\fontdocument}{noto}}{
	\usepackage[sfdefault]{noto}
	\renewcommand*\familydefault{\sfdefault}
}{
\ifthenelse{\equal{\fontdocument}{notoserif}}{
	\usepackage{notomath}
}{
\ifthenelse{\equal{\fontdocument}{opensansserif}}{
	\usepackage[default,oldstyle,scale=0.95]{opensans}
}{
\ifthenelse{\equal{\fontdocument}{overlock}}{
	\usepackage[sfdefault]{overlock}
	\renewcommand*\familydefault{\sfdefault}
}{
\ifthenelse{\equal{\fontdocument}{paratype}}{
	\usepackage{paratype}
	\renewcommand*\familydefault{\sfdefault}
}{
\ifthenelse{\equal{\fontdocument}{paratypesanscaption}}{
	\usepackage{PTSansCaption}
	\renewcommand*\familydefault{\sfdefault}
}{
\ifthenelse{\equal{\fontdocument}{paratypesansnarrow}}{
	\usepackage{PTSansNarrow}
	\renewcommand*\familydefault{\sfdefault}
}{
\ifthenelse{\equal{\fontdocument}{pxfonts}}{
	\usepackage{pxfonts}
}{
\ifthenelse{\equal{\fontdocument}{quattrocento}}{
	\usepackage[sfdefault]{quattrocento}
}{
\ifthenelse{\equal{\fontdocument}{raleway}}{
	\usepackage[default]{raleway}
}{
\ifthenelse{\equal{\fontdocument}{ralewayblack}}{
	\usepackage[black]{raleway}
}{
\ifthenelse{\equal{\fontdocument}{ralewayextralight}}{
	\usepackage[extralight]{raleway}
}{
\ifthenelse{\equal{\fontdocument}{ralewaymedium}}{
	\usepackage[medium]{raleway}
}{
\ifthenelse{\equal{\fontdocument}{ralewaylight}}{
	\usepackage[light]{raleway}
}{
\ifthenelse{\equal{\fontdocument}{ralewaythin}}{
	\usepackage[thin]{raleway}
}{
\ifthenelse{\equal{\fontdocument}{roboto}}{
	\usepackage[sfdefault]{roboto}
}{
\ifthenelse{\equal{\fontdocument}{robotocondensed}}{
	\usepackage[sfdefault,condensed]{roboto}
}{
\ifthenelse{\equal{\fontdocument}{robotolight}}{
	\usepackage[sfdefault,light]{roboto}
}{
\ifthenelse{\equal{\fontdocument}{robotolightcondensed}}{
	\usepackage[sfdefault,light,condensed]{roboto}
}{
\ifthenelse{\equal{\fontdocument}{robotothin}}{
	\usepackage[sfdefault,thin]{roboto}
}{
\ifthenelse{\equal{\fontdocument}{rosario}}{
	\usepackage[familydefault]{Rosario}
}{
\ifthenelse{\equal{\fontdocument}{sourcesanspro}}{
	\usepackage[default]{sourcesanspro}
}{
\ifthenelse{\equal{\fontdocument}{step}}{
	\usepackage[notext]{stix}
	\usepackage{step}
}{
\ifthenelse{\equal{\fontdocument}{stickstoo}}{
	\usepackage{stickstootext}
	\usepackage[stickstoo,vvarbb]{newtxmath}
}{
\ifthenelse{\equal{\fontdocument}{texgyrebonum}}{
	\usepackage{tgbonum}
}{
\ifthenelse{\equal{\fontdocument}{txfonts}}{
	\usepackage{txfonts}
}{
\ifthenelse{\equal{\fontdocument}{uarial}}{
	\usepackage{uarial}
	\renewcommand*\familydefault{\sfdefault}
}{
\ifthenelse{\equal{\fontdocument}{ugq}}{
	\renewcommand*\sfdefault{ugq}
	\renewcommand*\familydefault{\sfdefault}
}{
\ifthenelse{\equal{\fontdocument}{universalis}}{
	\usepackage[sfdefault]{universalis}
}{
\ifthenelse{\equal{\fontdocument}{universaliscondensed}}{
	\usepackage[condensed,sfdefault]{universalis}
}{
\ifthenelse{\equal{\fontdocument}{venturis}}{
	\usepackage[lf]{venturis}
	\renewcommand*\familydefault{\sfdefault}
}{
	\throwbadconfig[nostop]{Fuente desconocida}{\fontdocument}{(Fuentes recomendadas) lmodern,carial,arial2,times,mathptmx,helvet,opensans,mathpazo,cambria,libertine,custom}
	\throwbadconfig[noheader-nostop]{Fuente desconocida}{\fontdocument}{(Fuentes adicionales) accanthis,alegreya,alegreyasans,algolrevived,almendra,antpolt,antpoltlight,anttor,anttorcondensed,anttorlight,anttorlightcondensed,arev,arimo,arvo,baskervald,baskervaldx,berasans,beraserif,biolinum,bitter,boisik,bookman,cabin,cabincondensed,cantarell,caladea,carlito,charterbt,chivolight,chivoregular,clara,clearsans,cochineal,coelacanth,coelacanthextralight,coelacanthlight,comfortaa,comicneue,comicneueangular,computerconcrete,computerconcreteeuler,computermodern,computermodernbright,crimson,crimsonpro,crimsonproextralight,crimsonprolight,crimsonpromedium,cyklop}
	\throwbadconfig[noheader-nostop]{Fuente desconocida}{\fontdocument}{dejavusans,dejavusanscondensed,domitian,droidsans,electrum,erewhon,fbb,fetamont,firasans,firasansnewtxsf,fourier,fouriernc,gfsartemisia,gfsartemisiaeuler,heuristica,iwona,iwonacondensed,iwonalight,iwonalightcondensed,kerkis,kurier,kuriercondensed,kurierlight,kurierlightcondensed,lato,libertinus,librebaskerville,librebodoni,librecaslon,libris,lxfonts}
	\throwbadconfig[noheader]{Fuente desconocida}{\fontdocument}{merriweather,merriweatherlight,mintspirit,mlmodern,montserratalternatesextralight,montserratalternatesregular,montserratalternatesthin,montserratextralight,montserratlight,montserratregular,montserratthin,newpx,nimbussans,noto,notoserif,opensansserif,overlock,paratype,paratypesanscaption,paratypesansnarrow,pxfonts,quattrocento,raleway,ralewayblack,ralewayextralight,ralewaymedium,ralewaylight,ralewaythin,roboto,robotolight,robotolightcondensed,robotothin,rosario,sourcesanspro,step,stickstoo,uarial,texgyrebonum,txfonts,ugq,universalis,universaliscondensed,venturis}
	}}}}}}}}}}}}}}}}}}}}}}}}}}}}}}}}}}}}}}}}}}}}}}}}}}}}}}}}}}}}}}}}}}}}}}}}}}}}}}}}}}}}}}}}}}}}}}}}}}}}}}}}}}}}}}}}}}}}}}}}}}}}}}}}}}}}}}
}

% -----------------------------------------------------------------------------
% Tipografía typewriter
% -----------------------------------------------------------------------------
% https://tug.org/FontCatalogue/typewriterfonts.html
\ifthenelse{\equal{\fonttypewriter}{custom}}{
}{
\ifthenelse{\equal{\fonttypewriter}{tmodern}}{
	\renewcommand*\ttdefault{lmvtt}
}{
\ifthenelse{\equal{\fonttypewriter}{anonymouspro}}{
	\usepackage[ttdefault=true]{AnonymousPro}
}{
\ifthenelse{\equal{\fonttypewriter}{ascii}}{
	\usepackage{ascii}
	\let\SI\relax
}{
\ifthenelse{\equal{\fonttypewriter}{beramono}}{
	\usepackage[scaled]{beramono}
}{
\ifthenelse{\equal{\fonttypewriter}{cascadiacode}}{
	\usepackage{cascadia-code}
}{
\ifthenelse{\equal{\fonttypewriter}{cmpica}}{
	\usepackage{addfont}
	\addfont{OT1}{cmpica}{\pica}
	\addfont{OT1}{cmpicab}{\picab}
	\addfont{OT1}{cmpicati}{\picati}
	\renewcommand*\ttdefault{pica}
}{
\ifthenelse{\equal{\fonttypewriter}{cmodern}}{
}{
\ifthenelse{\equal{\fonttypewriter}{courier}}{
	\usepackage{courier}
}{
\ifthenelse{\equal{\fonttypewriter}{courier10}}{
	\usepackage{courierten}
}{
\ifthenelse{\equal{\fonttypewriter}{cmvtt}}{
	\renewcommand*\ttdefault{cmvtt}
}{
\ifthenelse{\equal{\fonttypewriter}{dejavusansmono}}{
	\usepackage[scaled]{DejaVuSansMono}
}{
\ifthenelse{\equal{\fonttypewriter}{droidsansmono}}{
	\usepackage[defaultmono]{droidsansmono}
}{
\ifthenelse{\equal{\fonttypewriter}{firamono}}{
	\usepackage[scale=0.85]{FiraMono}
}{
\ifthenelse{\equal{\fonttypewriter}{gomono}}{
	\usepackage[scale=0.85]{GoMono}
}{
\ifthenelse{\equal{\fonttypewriter}{inconsolata}}{
	\usepackage{inconsolata}
}{
\ifthenelse{\equal{\fonttypewriter}{nimbusmono}}{
	\usepackage{nimbusmono}
}{
\ifthenelse{\equal{\fonttypewriter}{newtxtt}}{
	\usepackage[zerostyle=d]{newtxtt}
}{
\ifthenelse{\equal{\fonttypewriter}{nimbusmono}}{
	\usepackage{nimbusmono}
}{
\ifthenelse{\equal{\fonttypewriter}{nimbusmononarrow}}{
	\usepackage{nimbusmononarrow}
}{
\ifthenelse{\equal{\fonttypewriter}{lcmtt}}{
	\renewcommand*\ttdefault{lcmtt}
}{
\ifthenelse{\equal{\fonttypewriter}{sourcecodepro}}{
	\usepackage[ttdefault=true,scale=0.85]{sourcecodepro}
}{
\ifthenelse{\equal{\fonttypewriter}{texgyrecursor}}{
	\usepackage{tgcursor}
}{
\ifthenelse{\equal{\fonttypewriter}{txtt}}{
	\renewcommand*\ttdefault{txtt}
}{
	\throwbadconfig{Fuente desconocida}{\fonttypewriter}{custom,anonymouspro,ascii,beramono,cascadiacode,cmpica,cmodern,courier,courier10,cvmtt,dejavusansmono,droidsansmono,firamono,gomono,inconsolata,kpmonospaced,lcmtt,newtxtt,nimbusmono,nimbusmononarrow,texgyrecursor,tmodern,txtt}
	}}}}}}}}}}}}}}}}}}}}}}}
}

% -----------------------------------------------------------------------------
% Finales
% -----------------------------------------------------------------------------
\usepackage[T1]{fontenc} % Caracteres acentuados
\ifthenelse{\equal{\showlayoutlines}{true}}{ % Muestra las líneas del layout
	\usepackage{showframe}}{
}
\ifthenelse{\equal{\showlinenumbers}{true}}{ % Muestra los números de línea
	\usepackage[switch,columnwise,running]{lineno}
	\newcommand*\linenomathpatch[1]{% Parcha entornos
		\cspreto{#1}{\linenomath}%
		\cspreto{#1*}{\linenomath}%
		\csappto{end#1}{\endlinenomath}%
		\csappto{end#1*}{\endlinenomath}}
	\linenomathpatch{equation}
	\linenomathpatch{gather}
	\linenomathpatch{multline}
	\linenomathpatch{align}
	\linenomathpatch{alignat}
	\linenomathpatch{flalign}}{
}
\usepackage{csquotes} % Citas y comillas
\ifthenelse{\equal{\compilertype}{pdf2latex}}{
	\inputencoding{utf8}}{
}

% -----------------------------------------------------------------------------
% IMPORTACIÓN DE FUNCIONES Y ENTORNOS
% -----------------------------------------------------------------------------
% Definición de variables globales
\global\def\GLOBALemptyvar {template:empty:var}   % Usado para indicar que una variable está vacía

\global\def\GLOBALcaptiondefn {\GLOBALemptyvar}   % Definición del caption
\global\def\GLOBALchapternumenabled {false}       % Numeración de capítulos empezó
\global\def\GLOBALenvappendix {false}             % Indica que el entorno anexo está activo
\global\def\GLOBALenvimageadded {false}           % Indica que una imagen ha sido añadida
\global\def\GLOBALenvimagecf {false}              % Indica que una imagen usa ContinuedFloat
\global\def\GLOBALenvimageinitialized {false}     % Entorno images activo
\global\def\GLOBALenvmulticol {false}             % Indica que el entorno multicol está activo
\global\def\GLOBALsectionanumenabled {false}      % Sección sin numeración
\global\def\GLOBALsubsectionanumenabled {false}   % Subsección sin numeración
\global\def\GLOBALsubsubsectionanumenabled{false} % Sub-subsección sin numeración
\global\def\GLOBALtablerowcolorindex {2}          % Índice tabla colores
\global\def\GLOBALtablerowcolorswitch {false}     % Tabla con colores cambiados
\global\def\GLOBALtwoside {false}                 % Indica que el documento es twoside

% Definición de formato de secciones
\global\def\GLOBALformatnumchapter {\formatnumchapter}
\global\def\GLOBALformatnumsection {\formatnumsection}
\global\def\GLOBALformatnumssection {\formatnumssection}
\global\def\GLOBALformatnumsssection {\formatnumsssection}
\global\def\GLOBALformatnumssssection {\formatnumssssection}

% Configura si el documento es twoside
\makeatletter
\if@twoside
\global\def\GLOBALtwoside {true}
\else
\fi 
\makeatother

% Signo porcentaje para archivos
\def\LOCALpercentchar#1{}
\edef\LOCALpercentchar{\expandafter\LOCALpercentchar\string\%}

% Contador global de objetos
\newcounter{templateEquations}      % Ecuaciones
\newcounter{templateFigures}        % Figuras
\newcounter{templateIndexEquations} % Ecuaciones en el índice
\newcounter{templateListings}       % Códigos fuente
\newcounter{templatePageCounter}    % Administra números de páginas
\newcounter{templateTables}         % Tablas

% Contador nivel de bookmarks marcadores
\newcounter{templateBookmarksLevelPrev}
\setcounter{templateBookmarksLevelPrev}{\cfgbookmarksopenlevel}
\addtocounter{templateBookmarksLevelPrev}{-1}

% Aumenta contador de páginas
\stepcounter{templatePageCounter}
\AtBeginShipout{\stepcounter{templatePageCounter}}

% Define latex para uso en referencias
\let\latex\LaTeX

% Nuevas dimensiones
\newlength{\coregluevarcm}
\setlength{\coregluevarcm}{0.25 cm}
\newlength{\corefontwidth}
\settowidth{\corefontwidth}{template}

% Lanza un mensaje de error
% 	#1	Función del error
%	#2	Mensaje
\newcommand{\throwerror}[2]{%
	\errmessage{LaTeX Error: \noexpand#1 #2 (linea \the\inputlineno)}%
	\stop
}

% Lanza un mensaje de advertencia
%	#1	Mensaje
\newcommand{\throwwarning}[1]{%
	\errmessage{LaTeX Warning: #1 (linea \the\inputlineno)}%
}

% Lanza un mensaje de error indicando mala configuración dentro de begin{document}
%	#1	Mensaje de error
% 	#2	Configuración usada
%	#3	Valores esperados
\newcommand{\throwbadconfigondoc}[3]{%
	\errmessage{#1 \noexpand #2=#2. Valores esperados: #3}%
	\stop%
}

% Chequea que un módulo no haya sido cargado antes de terminar el template
%	#1	Nombre del módulo
\makeatletter%
\newcommand{\checkmodulenotloaded}[1]{%
	\@ifpackageloaded{#1}{%
		\throwwarning{Template Error: No se pueden cargar paquetes (#1) antes de importar template.tex}%
		\stop%
	}{}%
}
\makeatother%

% Comprueba si una variable está definida
%	#1	Variable
\newcommand{\checkvardefined}[1]{%
	\ifthenelse{\isundefined{#1}}{%
		\errmessage{LaTeX Warning: Variable \noexpand#1 no definida}%
		\stop}{%
	}%
}

% Escribe un mensaje en la consola
%	#1	Mensaje
\newcommand{\coretemplatemessage}[1]{%
	\message{Template: #1}%
}

% Comprueba si una variable está definida
%	#1	Variable
%	#2	Mensaje
\newcommand{\checkextravarexist}[2]{%
	\ifthenelse{\isundefined{#1}}{%
		\errmessage{LaTeX Warning: Variable \noexpand#1 no definida}%
		\ifx\hfuzz#2\hfuzz%
			\errmessage{LaTeX Warning: Defina la variable en el bloque de INFORMACION DEL DOCUMENTO al comienzo del archivo principal del template}%
		\else%
			\errmessage{LaTeX Warning: #2}%
		\fi}{%
	}%
}

% Lanza un mensaje de error si una variable no ha sido definida
% 	#1	Función del error
%	#2	Variable
%	#3	Mensaje
\newcommand{\emptyvarerr}[3]{%
	\ifx\hfuzz#2\hfuzz%
		\errmessage{LaTeX Warning: \noexpand#1 #3 (linea \the\inputlineno)}%
	\fi
}

% Cambiar el margen de los caption
% 	#1	Margen en centímetros
\newcommand{\setcaptionmargincm}[1]{
	\captionsetup{margin=#1cm}
}

% Cambia márgenes de las páginas [cm]
% 	#1	Margen izquierdo
%	#2	Margen superior
%	#3	Margen derecho
%	#4	Margen inferior
\newcommand{\setpagemargincm}[4]{
	\ifthenelse{\equal{\compilertype}{lualatex}}{
		% Geometry no válido en lualatex
	}{
		\newgeometry{left=#1cm, top=#2cm, right=#3cm, bottom=#4cm}
	}
}

% Define el caption del índice
% 	#1	Título del caption
\newcommand{\setindexcaption}[1]{%
	\global\def\GLOBALcaptiondefn {#1}%
}

% Resetea los caption
\newcommand{\resetindexcaption}{%
	\global\def\GLOBALcaptiondefn {\GLOBALemptyvar}%
	\hbadness=10000%
}

% Cambia los márgenes del documento
%	#1	Margen izquierdo
%	#2	Margen derecho
\newcommand{\changemargin}[2]{%
	\emptyvarerr{\changemargin}{#1}{Margen izquierdo no definido}%
	\emptyvarerr{\changemargin}{#2}{Margen derecho no definido}%
	\list{}{\rightmargin#2\leftmargin#1}\item[]%
}
\let\endchangemargin=\endlist

% Chequea que las funciones sólo puedan usarse en el entorno images
\newcommand{\checkonlyonenvimage}{%
	\ifthenelse{\equal{\GLOBALenvimageinitialized}{true}}{}{%
		\throwwarning{Funciones \noexpand\addimage o \noexpand\addimageboxed no pueden usarse fuera del entorno \noexpand\images}\stop%
	}%
}

% Chequea que las funciones sólo puedan usarse fuera del entorno images
\newcommand{\checkoutsideenvimage}{%
	\ifthenelse{\equal{\GLOBALenvimageinitialized}{true}}{%
		\throwwarning{Esta funcion solo puede usarse fuera del entorno \noexpand\images}%
		\stop}{%
	}%
}

% Chequea que las funciones puedan usarse solo en el entorno multicol
\newcommand{\checkinsidemulticol}{%
	\ifthenelse{\equal{\GLOBALenvmulticol}{false}}{%
		\throwwarning{Esta funcion solo puede usarse dentro de multicols}%
		\stop}{%
	}%
}

% Chequea que las funciones puedan usarse fuera del entorno anexo
\newcommand{\checkoutsideappendix}{%
	\ifthenelse{\equal{\GLOBALenvappendix}{true}}{%
		\throwwarning{Esta funcion solo puede usarse fuera de anexo}%
		\stop}{%
	}%
}

% Verifica que una variable sea del estilo "true" o "false"
\newcommand{\corecheckbooleanvar}[1]{%
	\emptyvarerr{\corecheckbooleanvar}{#1}{Variable no definida}%
	\ifthenelse{\equal{#1}{true}}{}{%
	\ifthenelse{\equal{#1}{false}}{}{%
		\throwwarning{Variable debe ser true o false}\stop%
	}}%
}

% Centra verticalmente un texto
%	#1	Texto a centrar
\newcommand{\verticallycentertext}[1]{%
	\emptyvarerr{\verticallycentertext}{#1}{Texto no definido}%
	\topskip0pt%
	\vspace*{\fill}%
	#1%
	\vspace*{\fill}%
}

% Inserta un espacio vertical en cm con una variación +/-
%	#1	Espacio (en cm)
\newcommand{\corevspacevarcm}[1]{%
	\ifthenelse{\equal{#1}{0}}{}{%
	\ifthenelse{\equal{#1}{0.0}}{}{%
		\vspace{\dimexpr#1 cm plus #1\coregluevarcm minus #1\coregluevarcm}%
	}}%
}

% Agrega una carpeta al path de imágenes
%	#1	Carpeta
\makeatletter
\newcommand\addpathimage[1]{%
	\gappto\Ginput@path{{#1}}%
}
\makeatother

% Verifica que un tamaño de fuente sea correcto
%	#1	Tamaño de fuente
\newcommand{\corecheckfontsize}[1]{%
	\ifthenelse{\equal{#1}{normalsize}}{}{%
	\ifthenelse{\equal{#1}{small}}{}{%
	\ifthenelse{\equal{#1}{large}}{}{%
	\ifthenelse{\equal{#1}{Large}}{}{%
	\ifthenelse{\equal{#1}{LARGE}}{}{%
	\ifthenelse{\equal{#1}{huge}}{}{%
	\ifthenelse{\equal{#1}{Huge}}{}{%
	\ifthenelse{\equal{#1}{HUGE}}{}{%
	\ifthenelse{\equal{#1}{footnotesize}}{}{%
	\ifthenelse{\equal{#1}{scriptsize}}{}{%
	\ifthenelse{\equal{#1}{tiny}}{}{%
		\errmessage{LaTeX Warning: Tamano de fuente incorrecto (\noexpand #1= #1). Valores esperados: tiny,scriptsize,footnotesize,small,normalisize,large,Large,LARGE,huge,Huge,HUGE}%
		\stop%
		}}}}}}}}}}%
	}%
}

% Insertar sub-índice, a_b
% 	#1	Elemento inferior (a)
%	#2	Elemento superior (b)
\newcommand{\lpow}[2]{%
	\ensuremath{{#1}_{#2}}
}

% Insertar elevado, a^b
% 	#1	Elemento inferior (a)
%	#2	Elemento superior (b)
\newcommand{\pow}[2]{%
	\ensuremath{{#1}^{#2}}
}

% Inserta inverso función seno, sin^-1
%	#1	Elemento
\newcommand{\aasin}[1][]{%
	\ifx\hfuzz#1\hfuzz%
		\ensuremath{\sin^{-1}#1}
	\else%
		\ensuremath{{\sin}^{-1}}
	\fi%
}

% Inserta inverso función coseno, cos^-1
%	#1	Elemento
\newcommand{\aacos}[1][]{%
	\ifx\hfuzz#1\hfuzz%
		\ensuremath{\cos^{-1}#1}
	\else%
		\ensuremath{\cos^{-1}}
	\fi%
}

% Inserta inverso función tangente, tan^-1
%	#1	Elemento
\newcommand{\aatan}[1][]{%
	\ifx\hfuzz#1\hfuzz%
		\ensuremath{\tan^{-1}#1}
	\else%
		\ensuremath{\tan^{-1}}
	\fi%
}

% Inserta inverso función cosecante, csc^-1
%	#1	Elemento
\newcommand{\aacsc}[1][]{%
	\ifx\hfuzz#1\hfuzz%
		\ensuremath{\csc^{-1}#1}
	\else%
		\ensuremath{\csc^{-1}}
	\fi%
}

% Inserta inverso función secante, sec^-1
%	#1	Elemento
\newcommand{\aasec}[1][]{%
	\ifx\hfuzz#1\hfuzz%
		\ensuremath{\sec^{-1}#1}
	\else%
		\ensuremath{\sec^{-1}}
	\fi%
}

% Inserta inverso función cotangente, cot^-1
%	#1	Elemento
\newcommand{\aacot}[1][]{%
	\ifx\hfuzz#1\hfuzz%
		\ensuremath{\cot^{-1}#1}
	\else%
		\ensuremath{\cot^{-1}}
	\fi%
}

% Fracción de derivadas parciales af/ax
% 	#1	Función a derivar (f)
%	#2	Variable a derivar (x)
\newcommand{\fracpartial}[2]{%
	\ensuremath{\pdv{#1}{#2}}
}

% Fracción de derivadas parciales dobles a^2f/ax^2
% 	#1	Función a derivar (f)
%	#2	Variable a derivar (x)
\newcommand{\fracdpartial}[2]{%
	\ensuremath{\pdv[2]{#1}{#2}}
}

% Fracción de derivadas parciales en n, a^nf/ax^n
% 	#1	Función a derivar (f)
%	#2	Variable a derivar (x)
%	#3	Orden (n)
\newcommand{\fracnpartial}[3]{%
	\ensuremath{\pdv[#3]{#1}{#2}}
}

% Fracción de derivadas df/dx
% 	#1	Función a derivar (f)
%	#2	Variable a derivar (x)
\newcommand{\fracderivat}[2]{%
	\ensuremath{\dv{#1}{#2}}
}

% Fracción de derivadas dobles d^2/dx^2
% 	#1	Función a derivar (f)
%	#2	Variable a derivar (x)
\newcommand{\fracdderivat}[2]{%
	\ensuremath{\dv[2]{#1}{#2}}
}

% Fracción de derivadas en n d^nf/dx^n
% 	#1	Función a derivar (f)
%	#2	Variable a derivar (x)
%	#3	Orden de la derivada (n)
\newcommand{\fracnderivat}[3]{%
	\ensuremath{\dv[#3]{#1}{#2}}
}

% Llave superior de equivalencia
% 	#1	Elemento a igualar
%	#2	Igualdad
\newcommand{\topequal}[2]{%
	\ensuremath{\overbrace{#1}^{\mathclap{#2}}}
}
\newcommand{\topequaltext}[2]{%
	\topequal{#1}{\text{#2}}
}

% Llave inferior de equivalencia
% 	#1	Elemento a igualar
%	#2	Igualdad
\newcommand{\underequal}[2]{%
	\ensuremath{\underbrace{#1}_{\mathclap{#2}}}
}
\newcommand{\underequaltext}[2]{%
	\underequal{#1}{\text{#2}}
}

% Rectángulo superior de equivalencia
% 	#1	Elemento a igualar
%	#2	Igualdad
\newcommand{\topsequal}[2]{%
	\ensuremath{\overbracket{#1}^{\mathclap{#2}}}
}
\newcommand{\topsequaltext}[2]{%
	\topsequal{#1}{\text{#2}}
}

% Rectángulo inferior de equivalencia
% 	#1	Elemento a igualar
%	#2	Igualdad
\newcommand{\undersequal}[2]{%
	\ensuremath{\underbracket{#1}_{\mathclap{#2}}}
}
\newcommand{\undersequaltext}[2]{%
	\undersequal{#1}{\text{#2}}
}

% Función piso
% 	#1	Elemento
\newcommand{\floorexp}[1]{%
	\ensuremath{\left\lfloor{#1}\right\rfloor}
}

% Función techo
% 	#1	Elemento
\newcommand{\ceilexp}[1]{%
	\ensuremath{\left\lceil{#1}\right\rceil}
}

% Función mod
%	#1	Elemento tal que (mod #1)
\newcommand{\Mod}[1]{%
	\ensuremath{\ (\mathrm{mod}\ #1)}
}

% Paréntesis grande
% 	#1	Expresión
\newcommand{\bigp}[1]{%
	\ensuremath{\big(#1\big)}
}

% Paréntesis g+grande
% 	#1	Expresión
\newcommand{\biggp}[1]{%
	\ensuremath{\bigg(#1\bigg)}
}

% Cajón grande
% 	#1	Expresión
\newcommand{\bigc}[1]{%
	\ensuremath{\big[#1\big]}
}

% Cajón g+grande
% 	#1	Expresión
\newcommand{\biggc}[1]{%
	\ensuremath{\bigg[#1\bigg]}
}

% Llave grande
% 	#1	Expresión
\newcommand{\bigb}[1]{%
	\ensuremath{\big\{#1\big\}}
}

% Llave g+grande
% 	#1	Expresión
\newcommand{\biggb}[1]{%
	\ensuremath{\bigg\{#1\bigg\}}
}

% Expresión divergencia
\newcommand{\divexp}{%
	\ensuremath{\rm{div}\ }
}

% Expresión automorfismo
\newcommand{\Autexp}{%
	\ensuremath{\rm{Aut}}
}

% Negrita introducida por word
% 	#1	Expresión
\newcommand{\mathbit}[1]{%
	\bm{#1}
}

% Expresión diff
\newcommand{\Diffexp}{%
	\ensuremath{\rm{Diff}}
}

% Expresión imaginario
\newcommand{\Imexp}{%
	\ensuremath{\rm{Im}}
}

% Expresión imaginario en z
\newcommand{\Imzexp}{%
	\ensuremath{\rm{Im}(z)}
}

% Expresión real
\newcommand{\Reexp}{%
	\ensuremath{\rm{Re}}
}

% Expresión real en z
\newcommand{\Rezexp}{%
	\ensuremath{\rm{Re}(z)}
}

% Barra superior en elemento
%	#1 	Elemento
\newcommand{\overbar}[1]{%
	\mkern 1.5mu\overline{\mkern-1.5mu#1\mkern-1.5mu}\mkern 1.5mu
}

% Función \tilde{} pero que encierra todo el texto
%	#1 	Elemento
\makeatletter
\def\longtilde#1{%
	\mathop{\vbox{\m@th\ialign{##\crcr\noalign{\kern3\p@}%
	\sortoftildefill\crcr\noalign{\kern3\p@\nointerlineskip}%
	$\hfil\displaystyle{#1}\hfil$\crcr}}}\limits%
}
\def\sortoftildefill {%
	$\m@th \setbox\z@\hbox{$\braceld$}%
	\braceld\leaders\vrule \@height\ht\z@ \@depth\z@\hfill\braceru$%
}
\makeatother

% Definición de letras
\newcommand{\A}{\ensuremath{\mathcal{A}}}

\newcommand{\B}{\ensuremath{\mathcal{B}}}

\ifthenelse{\isundefined{\C}}{\newcommand{\C}{C}}{\let\oldC=\C}
\renewcommand{\C}{\ensuremath{\mathbb{C}}}

\newcommand{\D}{\ensuremath{\mathbb{D}}}

\newcommand{\E}{\ensuremath{\mathbb{E}}}

\newcommand{\F}{\ensuremath{\mathcal{F}}}

\ifthenelse{\isundefined{\G}}{\newcommand{\G}{G}}{\let\oldG=\G}
\renewcommand{\G}{\ensuremath{\mathcal{G}}}

\ifthenelse{\isundefined{\H}}{\newcommand{\H}{H}}{\let\oldH=\H}
\renewcommand{\H}{\ensuremath{\mathcal{H}}}

\newcommand{\I}{\ensuremath{\mathbb{I}}}

\newcommand{\J}{\ensuremath{\mathcal{J}}}

\newcommand{\K}{\ensuremath{\mathcal{K}}}

\let\oldL=\L % L con una raya
\renewcommand{\L}{\ensuremath{\mathcal{L}}}

\newcommand{\M}{\ensuremath{\mathcal{M}}}

\newcommand{\N}{\ensuremath{\mathbb{N}}}

% \renewcommand{\O}{\ensuremath{\mathbb{O}}} % O equivale a o/oo

\let\oldP=\P % P negra
\renewcommand{\P}{\ensuremath{\mathbb{P}}}

\newcommand{\Q}{\ensuremath{\mathbb{Q}}}

\newcommand{\R}{\ensuremath{\mathbb{R}}}

\let\oldS=\S % Serpiente
\renewcommand{\S}{\ensuremath{\mathcal{S}}}

\newcommand{\T}{\ensuremath{\mathcal{T}}}

\ifthenelse{\isundefined{\U}}{\newcommand{\U}{U}}{\let\oldU=\U}
\renewcommand{\U}{\ensuremath{\mathcal{U}}}

\newcommand{\V}{\ensuremath{\mathcal{V}}}

\newcommand{\W}{\ensuremath{\mathcal{W}}}

\newcommand{\X}{\ensuremath{\mathcal{X}}}

\newcommand{\Y}{\ensuremath{\mathcal{Y}}}

\newcommand{\Z}{\ensuremath{\mathbb{Z}}}

% Definición de operadores matemáticos de asignación (Typeset assigments)
\ifthenelse{\equal{\fontdocument}{step}}{}{ % Ya definidos en STEP
	\newcommand{\asteq}{\ensuremath{\mathrel{{*}{=}}}}
	\newcommand{\eqeq}{\ensuremath{\mathrel{{=}{=}}}}
}
\newcommand{\cdoteq}{\ensuremath{\mathrel{{\cdot}{=}}}}
\newcommand{\diveq}{\ensuremath{\mathrel{{/}{=}}}}
\newcommand{\eqast}{\ensuremath{\mathrel{{=}{*}}}}
\newcommand{\eqcdot}{\ensuremath{\mathrel{{=}{\cdot}}}}
\newcommand{\eqdiv}{\ensuremath{\mathrel{{=}{/}}}}
\newcommand{\eqminus}{\ensuremath{\mathrel{{=}{-}}}}
\newcommand{\eqnot}{\ensuremath{\mathrel{{=}{!}}}}
\newcommand{\eqplus}{\ensuremath{\mathrel{{=}{+}}}}
\newcommand{\eqtimes}{\ensuremath{\mathrel{{=}{\times}}}}
\newcommand{\minuseq}{\ensuremath{\mathrel{{-}{=}}}}
\newcommand{\minusminus}{\ensuremath{\mathrel{{-}{-}}}}
\newcommand{\noteq}{\ensuremath{\mathrel{{!}{=}}}}
\newcommand{\pluseq}{\ensuremath{\mathrel{{+}{=}}}}
\newcommand{\plusplus}{\ensuremath{\mathrel{{+}{+}}}}
\newcommand{\timeseq}{\ensuremath{\mathrel{{\times}{=}}}}

% Definición de teoremas y lemas
\makeatletter
	\renewenvironment{proof}[1][\proofname]{%
		\par\pushQED{\qed}%
		\normalfont\topsep6\p@\@plus6\p@\relax\trivlist%
		\item[\hskip\labelsep\scshape\footnotesize#1\@addpunct{.}]%
		\ignorespaces%
	}{%
		\popQED\endtrivlist\@endpefalse%
	}%
\makeatother

% Función que se ejecuta tras un equation
\newcommand{\coreafterequationfn}{%
	\hbadness=10000%
}

% Redimensiona una ecuación en linewidth
% 	#1	Tamaño del nuevo objeto (En linewidth)
%	#2	Ecuación a redimensionar
\newcommand{\equationresize}[2]{%
	\emptyvarerr{\equationresize}{#1}{Dimension no definida}%
	\emptyvarerr{\equationresize}{#2}{Ecuacion a redimensionar no definida}%
	\resizebox{#1\linewidth}{!}{$#2$}%
}

% Inserta el caption de un objeto tipo ecuación
%	#1	Texto del caption
\newcommand{\coreinsertequationcaption}[1]{%
	\begin{changemargin}{\captionlrmargin cm}{\captionlrmargin cm}%
		\ifthenelse{\equal{\equationcaptioncenter}{true}}{%
			\centering%
		}{%
			\justifying%
		}%
		\textcolor{\captiontextcolor}{%
			\linespread{0.5}\selectfont{%
				\begin{\captionfontsize}#1\end{\captionfontsize}%
			}%
		}%
	\end{changemargin}
}

% Insertar una ecuación
% 	#1	Label (opcional)
%	#2	Ecuación
\newcommand{\insertequation}[2][]{%
	\emptyvarerr{\insertequation}{#2}{Ecuacion no definida}%
	\ifthenelse{\equal{\numberedequation}{true}}{%
		\corevspacevarcm{\marginequationtop}%
		\begin{samepage}%
		\begin{equation}%
			\text{#1} #2
		\end{equation}
		\corevspacevarcm{\marginequationbottom}%
		\end{samepage}
		\coreafterequationfn%
	}{%
		\ifx\hfuzz#1\hfuzz%
		\else%
			\throwwarning{Label invalido en ecuacion sin numero}%
		\fi%
		\insertequationanum{#2}%
	}%
}

% Insertar una ecuación sin número
%	#1	Ecuación
\newcommand{\insertequationanum}[1]{%
	\emptyvarerr{\insertequationanum}{#1}{Ecuacion no definida}%
	\corevspacevarcm{\marginequationtop}%
	\begin{samepage}%
	\begin{equation*}%
		\ensuremath{#1}
	\end{equation*}
	\corevspacevarcm{\marginequationbottom}%
	\end{samepage}
	\coreafterequationfn%
}

% Insertar una ecuación en el índice
% 	#1	Label (opcional)
%	#2	Ecuación
%	#3	Título de la ecuación
\newcommand{\insertindexequation}[3][]{%
	\emptyvarerr{\insertindexequation}{#2}{Ecuacion no definida}%
	\emptyvarerr{\insertindexequation}{#3}{Leyenda no definida}%
	\corevspacevarcm{\margineqnindextop}%
	\begin{samepage}%
	\begin{align}%
		\text{#1} \ensuremath{#2}
	\end{align}
	\myindexequations{#3}%
	\corevspacevarcm{\margineqnindexbottom}%
	\end{samepage}
	\coreinsertequationcaption{\textit{#3}}%
	\addtocounter{templateIndexEquations}{1}%
	\coreafterequationfn%
}

% Insertar una ecuación alineada a la izquierda
% 	#1	Label (opcional)
%	#2	Ecuación
\newcommand{\insertequationleft}[2][]{%
	\emptyvarerr{\insertequationleft}{#2}{Ecuacion no definida}%
	\ifthenelse{\equal{\numberedequation}{true}}{%
		\vspace{\dimexpr\marginequationtop cm - \baselineskip}%
		\begin{samepage}%
		\begin{equation}
			\hfilneg \text{#1} #2 \hspace{10000pt minus 1fil}
		\end{equation}
		\vspace{\dimexpr-0.2\baselineskip + \marginequationbottom cm}%
		\end{samepage}
		\coreafterequationfn%
	}{%
		\ifx\hfuzz#1\hfuzz%
		\else%
			\throwwarning{Label invalido en ecuacion sin numero}%
		\fi%
		\insertequationleftanum{#2}%
	}%
}

% Insertar una ecuación sin número alineada a la izquierda
%	#1	Ecuación
\newcommand{\insertequationleftanum}[1]{%
	\emptyvarerr{\insertequationleftanum}{#1}{Ecuacion no definida}%
	\vspace{\dimexpr\marginequationtop cm - \baselineskip}%
	\begin{samepage}%
	\begin{equation*}
		\hfilneg \ensuremath{#1} \hspace{10000pt minus 1fil}
	\end{equation*}
	\vspace{\dimexpr-0.2\baselineskip + \marginequationbottom cm}%
	\end{samepage}
	\coreafterequationfn%
}

% Insertar una ecuación alineada a la derecha
% 	#1	Label (opcional)
%	#2	Ecuación
\newcommand{\insertequationright}[2][]{%
	\emptyvarerr{\insertequationright}{#2}{Ecuacion no definida}%
	\ifthenelse{\equal{\numberedequation}{true}}{%
		\vspace{\dimexpr\marginequationtop cm - \baselineskip}%
		\begin{samepage}%
		\begin{equation}
			\hspace{10000pt minus 1fil} \text{#1} #2 \hfilneg
		\end{equation}
		\vspace{\dimexpr-0.2\baselineskip + \marginequationbottom cm}%
		\end{samepage}
		\coreafterequationfn%
	}{%
		\ifx\hfuzz#1\hfuzz%
		\else%
			\throwwarning{Label invalido en ecuacion sin numero}%
		\fi%
		\insertequationrightanum{#2}%
	}%
}

% Insertar una ecuación sin número alineada a la derecha
%	#1	Ecuación
\newcommand{\insertequationrightanum}[1]{%
	\emptyvarerr{\insertequationrightanum}{#1}{Ecuacion no definida}%
	\vspace{\dimexpr\marginequationtop cm - \baselineskip}%
	\begin{samepage}%
	\begin{equation*}
		\hspace{10000pt minus 1fil} \ensuremath{#1} \hfilneg
	\end{equation*}
	\vspace{\dimexpr-0.2\baselineskip + \marginequationbottom cm}%
	\end{samepage}
	\coreafterequationfn%
}

% Insertar una ecuación con leyenda
% 	#1	Label (opcional)
%	#2	Ecuación
%	#3	Leyenda
\newcommand{\insertequationcaptioned}[3][]{%
	\emptyvarerr{\insertequationcaptioned}{#2}{Ecuacion no definida}%
	\ifx\hfuzz#3\hfuzz%
		\insertequation[#1]{#2}%
	\else%
		\ifthenelse{\equal{\numberedequation}{true}}{%
			\corevspacevarcm{\marginequationtop}%
			\begin{samepage}%
			\begin{equation}
				\text{#1} #2
			\end{equation}
			\corevspacevarcm{\margineqncaptiontop}%
			\coreinsertequationcaption{#3}%
			\corevspacevarcm{\margineqncaptionbottom}%
			\end{samepage}
			\coreafterequationfn%
		}{%
			\ifx\hfuzz#1\hfuzz%
			\else%
				\throwwarning{Label invalido en ecuacion sin numero}%
			\fi%
			\insertequationcaptionedanum{#2}{#3}%
		}%
	\fi%
}

% Insertar una ecuación con leyenda sin número
%	#1	Ecuación
%	#2	Leyenda
\newcommand{\insertequationcaptionedanum}[2]{%
	\emptyvarerr{\insertequationcaptionedanum}{#1}{Ecuacion no definida}%
	\ifx\hfuzz#2\hfuzz%
		\insertequationanum{#1}%
	\else%
		\corevspacevarcm{\marginequationtop}%
		\begin{samepage}%
		\begin{equation*}
			\ensuremath{#1}%
		\end{equation*}
		\corevspacevarcm{\margineqncaptiontop}%
		\coreinsertequationcaption{#2}%
		\corevspacevarcm{\margineqncaptionbottom}%
		\end{samepage}
		\coreafterequationfn%
	\fi%
}

% Insertar una ecuación con el ambiente gather
%	#1	Ecuación
\newcommand{\insertgather}[1]{%
	\emptyvarerr{\insertgather}{#1}{Ecuacion no definida}%
	\ifthenelse{\equal{\numberedequation}{true}}{%
		\corevspacevarcm{\margingathertop}%
		\begin{samepage}%
		\begin{gather}%
			\ensuremath{#1}
		\end{gather}
		\corevspacevarcm{\margingatherbottom}%
		\end{samepage}
		\coreafterequationfn%
	}{%
		\insertgatheranum{#1}%
	}%
}

% Insertar una ecuación con el ambiente gather sin número
%	#1	Ecuación
\newcommand{\insertgatheranum}[1]{%
	\emptyvarerr{\insertgatheranum}{#1}{Ecuacion no definida}%
	\corevspacevarcm{\margingathertop}%
	\begin{samepage}%
	\begin{gather*}%
		\ensuremath{#1}
	\end{gather*}
	\corevspacevarcm{\margingatherbottom}%
	\end{samepage}
	\coreafterequationfn%
}

% Insertar una ecuación (gather) con leyenda
%	#1	Ecuación
%	#2	Leyenda
\newcommand{\insertgathercaptioned}[2]{%
	\emptyvarerr{\insertgathercaptioned}{#1}{Ecuacion no definida}%
	\ifx\hfuzz#2\hfuzz%
		\insertgather{#1}%
	\else%
		\ifthenelse{\equal{\numberedequation}{true}}{%
			\corevspacevarcm{\margingathertop}%
			\begin{samepage}%
			\begin{gather}%
				\ensuremath{#1}
			\end{gather}
			\corevspacevarcm{\margingathercapttop}%
			\coreinsertequationcaption{#2}%
			\corevspacevarcm{\margingathercaptbottom}%
			\end{samepage}
			\coreafterequationfn%
		}{%
			\insertgathercaptionedanum{#1}{#2}%
		}%
	\fi%
}

% Insertar una ecuación (gather) con leyenda sin número
%	#1	Ecuación
%	#2	Leyenda
\newcommand{\insertgathercaptionedanum}[2]{%
	\emptyvarerr{\insertgathercaptionedanum}{#1}{Ecuacion no definida}%
	\ifx\hfuzz#2\hfuzz%
		\insertgatheranum{#1}%
	\else%
		\corevspacevarcm{\margingathertop}%
		\begin{samepage}%
		\begin{gather*}%
			\ensuremath{#1}
		\end{gather*}
		\corevspacevarcm{\margingathercapttop}%
		\coreinsertequationcaption{#2}%
		\corevspacevarcm{\margingathercaptbottom}%
		\end{samepage}
		\coreafterequationfn%
	\fi%
}

% Insertar una ecuación con el ambiente gathered
% 	#1	Label (opcional)
%	#2	Ecuación
\newcommand{\insertgathered}[2][]{%
	\emptyvarerr{\insertgathered}{#2}{Ecuacion no definida}%
	\ifthenelse{\equal{\numberedequation}{true}}{%
		\corevspacevarcm{\marginequationtop}%
		\begin{samepage}%
		\begin{equation}
			\begin{gathered}
				\text{#1} \ensuremath{#2}
			\end{gathered}
		\end{equation}
		\corevspacevarcm{\margingatheredbottom}%
		\end{samepage}
	}{%
		\ifx\hfuzz#1\hfuzz%
		\else%
			\throwwarning{Label invalido en ecuacion (gathered) sin numero}%
		\fi%
		\corevspacevarcm{\margingatheredtop}%
		\begin{samepage}%
		\begin{gather*}%
			\ensuremath{#2}
		\end{gather*}
		\corevspacevarcm{\margingatheredbottom}%
		\end{samepage}
	}%
	\coreafterequationfn%
}

% Insertar una ecuación con el ambiente gathered sin número
%	#1	Ecuación
\newcommand{\insertgatheredanum}[1]{%
	\emptyvarerr{\insertgatheredanum}{#1}{Ecuacion no definida}%
	\corevspacevarcm{\margingatheredtop}%
	\begin{samepage}%
	\begin{gather*}
		\ensuremath{#1}
	\end{gather*}
	\vspace{\dimexpr-0.15cm + \margingatheredbottom cm}%
	\end{samepage}
	\coreafterequationfn%
}

% Insertar una ecuación (gathered) con leyenda
% 	#1	Label (opcional)
%	#2	Ecuación
%	#3	Leyenda
\newcommand{\insertgatheredcaptioned}[3][]{%
	\emptyvarerr{\insertgatheredcaptioned}{#2}{Ecuacion no definida}%
	\ifx\hfuzz#3\hfuzz%
		\insertgathered[#1]{#2}%
	\else%
		\ifthenelse{\equal{\numberedequation}{true}}{%
			\corevspacevarcm{\marginequationtop}%
			\begin{samepage}%
			\begin{equation}
				\begin{gathered}
					\text{#1} \ensuremath{#2}
				\end{gathered}
			\end{equation}
			\corevspacevarcm{\margingatheredcapttop}%
			\coreinsertequationcaption{#3}%
			\corevspacevarcm{\margingatheredcaptbottom}%
			\end{samepage}
			\coreafterequationfn%
		}{%
			\ifx\hfuzz#1\hfuzz%
			\else%
				\throwwarning{Label invalido en ecuacion (gathered) sin numero}
			\fi%
			\insertgatheredcaptionedanum{#2}{#3}%
		}%
	\fi%
}

% Insertar una ecuación (gathered) con leyenda sin número
%	#1	Ecuación
%	#2	Leyenda
\newcommand{\insertgatheredcaptionedanum}[2]{
	\emptyvarerr{\insertgatheredcaptionedanum}{#1}{Ecuacion no definida}%
	\ifx\hfuzz#2\hfuzz%
		\insertgatheredanum{#1}%
	\else%
		\corevspacevarcm{\margingatheredtop}%
		\begin{samepage}%
		\begin{gather*}
			\ensuremath{#1}
		\end{gather*}
		\vspace{\dimexpr-0.2cm + \margingatheredcapttop cm}%
		\coreinsertequationcaption{#2}%
		\vspace{\dimexpr-0.05cm + \margingatheredcaptbottom cm}%
		\end{samepage}
		\coreafterequationfn%
	\fi%
}

% Insertar una ecuación con el ambiente align
%	#1	Ecuación
\newcommand{\insertalign}[1]{%
	\emptyvarerr{\insertalign}{#1}{Ecuacion no definida}%
	\ifthenelse{\equal{\numberedequation}{true}}{%
		\corevspacevarcm{\marginaligntop}%
		\begin{samepage}%
		\begin{align}
			\ensuremath{#1}
		\end{align}
		\corevspacevarcm{\marginalignbottom}%
		\end{samepage}
		\coreafterequationfn%
	}{%
		\insertalignanum{#1}%
	}%
}

% Insertar una ecuación con el ambiente align sin número
%	#1	Ecuación
\newcommand{\insertalignanum}[1]{%
	\emptyvarerr{\insertalignanum}{#1}{Ecuacion no definida}%
	\corevspacevarcm{\marginaligntop}%
	\begin{samepage}%
	\begin{align*}
		\ensuremath{#1}
	\end{align*}
	\corevspacevarcm{\marginalignbottom}%
	\end{samepage}
	\coreafterequationfn%
}

% Insertar una ecuación (align) con leyenda
%	#1	Ecuación
%	#2	Leyenda
\newcommand{\insertaligncaptioned}[2]{%
	\emptyvarerr{\insertaligncaptioned}{#1}{Ecuacion no definida}%
	\ifx\hfuzz#2\hfuzz%
		\insertalign{#1}%
	\else%
		\ifthenelse{\equal{\numberedequation}{true}}{%
			\corevspacevarcm{\marginaligntop}%
			\begin{samepage}%
			\begin{align}
				\ensuremath{#1}
			\end{align}
			\corevspacevarcm{\marginaligncapttop}%
			\coreinsertequationcaption{#2}%
			\corevspacevarcm{\marginaligncaptbottom}%
			\end{samepage}
			\coreafterequationfn%
		}{%
			\insertaligncaptionedanum{#1}{#2}%
		}%
	\fi%
}

% Insertar una ecuación (align) con leyenda sin número
%	#1	Ecuación
%	#2	Leyenda
\newcommand{\insertaligncaptionedanum}[2]{%
	\emptyvarerr{\insertaligncaptionedanum}{#1}{Ecuacion no definida}%
	\ifx\hfuzz#2\hfuzz%
		\insertalignanum{#1}%
	\else%
		\corevspacevarcm{\marginaligntop}%
		\begin{samepage}%
		\begin{align*}
			\ensuremath{#1}
		\end{align*}
		\corevspacevarcm{\marginaligncapttop}%
		\coreinsertequationcaption{#2}%
		\corevspacevarcm{\marginaligncaptbottom}%
		\end{samepage}
		\coreafterequationfn%
	\fi%
}

% Insertar una ecuación con el ambiente aligned
% 	#1	Label (opcional)
%	#2	Ecuación
\newcommand{\insertaligned}[2][]{%
	\emptyvarerr{\insertaligned}{#2}{Ecuacion no definida}%
	\ifthenelse{\equal{\numberedequation}{true}}{%
		\corevspacevarcm{\marginequationtop}%
		\begin{samepage}%
		\begin{equation}
			\begin{aligned}
				\text{#1} \ensuremath{#2}
			\end{aligned}
		\end{equation}
		\corevspacevarcm{\marginalignedbottom}%
		\end{samepage}
		\coreafterequationfn%
	}{%
		\ifx\hfuzz#1\hfuzz%
		\else%
			\throwwarning{Label invalido en ecuacion (aligned) sin numero}%
		\fi%
		\insertalignedanum{#2}%
	}%
}

% Insertar una ecuación con el ambiente aligned sin número
%	#1	Ecuación
\newcommand{\insertalignedanum}[1]{%
	\emptyvarerr{\insertalignedanum}{#1}{Ecuacion no definida}%
	\corevspacevarcm{\marginalignedtop}%
	\begin{samepage}%
	\begin{align*}
		\ensuremath{#1}
	\end{align*}
	\vspace{\dimexpr-0.2cm + \marginalignedbottom cm}%
	\end{samepage}
	\coreafterequationfn%
}

% Insertar una ecuación (aligned) con leyenda
% 	#1	Label (opcional)
%	#2	Ecuación
%	#3	Leyenda
\newcommand{\insertalignedcaptioned}[3][]{%
	\emptyvarerr{\insertalignedcaptioned}{#2}{Ecuacion no definida}%
	\ifx\hfuzz#3\hfuzz%
		\insertaligned[#1]{#2}%
	\else%
		\ifthenelse{\equal{\numberedequation}{true}}{%
			\corevspacevarcm{\marginequationtop}%
			\begin{samepage}%
			\begin{equation}
				\begin{aligned}
					\text{#1} \ensuremath{#2}
				\end{aligned}
			\end{equation}
			\corevspacevarcm{\marginalignedcapttop}%
			\coreinsertequationcaption{#3}%
			\corevspacevarcm{\marginalignedcaptbottom}%
			\end{samepage}
			\coreafterequationfn%
		}{%
			\ifx\hfuzz#1\hfuzz%
			\else%
				\throwwarning{Label invalido en ecuacion (aligned) sin numero}%
			\fi%
			\insertalignedcaptionedanum{#2}{#3}%
		}%
	\fi%
}

% Insertar una ecuación (aligned) con leyenda sin número
%	#1	Ecuación
%	#2	Leyenda
\newcommand{\insertalignedcaptionedanum}[2]{%
	\emptyvarerr{\insertalignedcaptionedanum}{#1}{Ecuacion no definida}%
	\ifx\hfuzz#2\hfuzz%
		\insertalignedanum{#1}%
	\else%
		\corevspacevarcm{\marginequationtop}%
		\begin{samepage}%
		\begin{equation}
			\begin{aligned}
				\ensuremath{#1}
			\end{aligned}
		\end{equation}
		\corevspacevarcm{\marginalignedcapttop}%
		\coreinsertequationcaption{#2}%
		\corevspacevarcm{\marginalignedcaptbottom}%
		\end{samepage}
		\coreafterequationfn%
	\fi%
}

\global\def\GLOBALimagelink {\GLOBALemptyvar} % Almacena el link de la imagen
\global\def\GLOBALimagenextmarginv {0 cm} % Almacena el margen vertical de las imágenes

% Calcula largo hspace
% Regresión entre 35,46446->9 y 52,68402->13,5
\newlength{\coreimageshspace}
\setlength{\coreimageshspace}{\dimexpr 9pt + 0.261330719\corefontwidth - 9.26795284pt}

% Añade una imagen en el entorno "images" con borde
% 	#1	Label (opcional)
%	#2	Dirección de la imagen
%	#3	Parámetros de la imagen
%	#4	Leyenda de la imagen (opcional)
\newcommand{\addimage}[4][]{%
	\addimageboxed[#1]{#2}{#3}{0}{#4}%
}

% Añade una imagen en el entorno "images" con borde
% 	#1	Label (opcional)
%	#2	Dirección de la imagen
%	#3	Parámetros de la imagen
%	#4	Ancho de la línea (en pt)
%	#5	Leyenda de la imagen (opcional)
\newcommand{\addimageboxed}[5][]{%
	\checkonlyonenvimage%
	\begingroup%
	\setlength{\fboxsep}{0 pt}%
	\setlength{\fboxrule}{#4 pt}%
	\ifthenelse{\equal{\GLOBALenvimageadded}{true}}{%
		\hspace{\dimexpr \marginimagemultright cm -\coreimageshspace}%
	}{}%
	\ifthenelse{\equal{#5}{\GLOBALemptyvar}}{ % Sin label
		\ifthenelse{\equal{\GLOBALimagelink}{\GLOBALemptyvar}}{ % Sin link
			\raisebox{\GLOBALimagenextmarginv}{%
				\fbox{\includegraphics[#3]{#2}}%
			}%
		}{ % Con link
			\raisebox{\GLOBALimagenextmarginv}{%
				\fbox{\href{\GLOBALimagelink}{\includegraphics[#3]{#2}}}%
			}%
		}%
	}{ % Con label
		\ifthenelse{\equal{\GLOBALimagelink}{\GLOBALemptyvar}}{ % Sin link
			\subfloat[#5#1]{%
				\raisebox{\GLOBALimagenextmarginv}{%
					\fbox{\includegraphics[#3]{#2}}%
				}%
			}%
		}{ % Con link
			\subfloat[#5#1]{%
				\raisebox{\GLOBALimagenextmarginv}{%
					\fbox{\href{\GLOBALimagelink}{\includegraphics[#3]{#2}}}%
				}%
			}%
		}%
	}%
	\endgroup%
	\global\def\GLOBALenvimageadded {true}%
	\global\def\GLOBALimagenextmarginv {0 cm}%
}

% Añade una imagen en el entorno "images" con borde sin leyenda
%	#1	Dirección de la imagen
%	#2	Parámetros de la imagen
\newcommand{\addimageanum}[2]{%
	\addimageboxed{#1}{#2}{0}{\GLOBALemptyvar}%
}

% Añade una imagen en el entorno "images" con borde sin leyenda
%	#1	Dirección de la imagen
%	#2	Parámetros de la imagen
%	#3	Ancho de la línea (en pt)
\newcommand{\addimageanumboxed}[3]{%
	\addimageboxed{#1}{#2}{#3}{\GLOBALemptyvar}%
}

% Añade una imagen en el entorno "images" con borde animada
% 	#1	Label (opcional)
%	#2	Dirección de la imagen animada
%	#3	Parámetros de la imagen
%	#4	FPS de la imagen
%	#5	Total imágenes no definido
%	#6	Leyenda de la imagen (opcional)
\newcommand{\addimageanimated}[6][]{%
	\addimageanimatedboxed[#1]{#2}{#3}{#4}{#5}{0}{#6}%
}

% Añade una imagen en el entorno "images" con borde animada
% 	#1	Label (opcional)
%	#2	Dirección de la imagen animada
%	#3	Parámetros de la imagen
%	#4	FPS de la imagen
%	#5	Total imágenes no definido
%	#6	Ancho de la línea (en pt)
%	#7	Leyenda de la imagen (opcional)
\newcommand{\addimageanimatedboxed}[7][]{%
	\checkonlyonenvimage%
	\begingroup%
	\setlength{\fboxsep}{0 pt}%
	\setlength{\fboxrule}{#6 pt}%
	\ifthenelse{\equal{\GLOBALenvimageadded}{true}}{%
		\hspace{\dimexpr \marginimagemultright cm - \coreimageshspace}%
	}{}%
	\ifthenelse{\equal{#7}{\GLOBALemptyvar}}{ % Sin label
		\ifthenelse{\equal{\animatedimageloop}{true}}{ % Con loop
			\ifthenelse{\equal{\animatedimageautoplay}{true}}{ % Con autoplay
				\raisebox{\GLOBALimagenextmarginv}{%
					\fbox{\animategraphics[loop,autoplay,#3]{#4}{#2-}{0}{#5}}%
				}%
			}{ % Sin autoplay
				\raisebox{\GLOBALimagenextmarginv}{%
					\fbox{\animategraphics[loop,#3]{#4}{#2-}{0}{#5}}%
				}%
			}%
		}{ % Sin loop
			\ifthenelse{\equal{\animatedimageautoplay}{true}}{ % Con autoplay
				\raisebox{\GLOBALimagenextmarginv}{%
					\fbox{\animategraphics[autoplay,#3]{#4}{#2-}{0}{#5}}%
				}%
			}{ % Sin autoplay
				\raisebox{\GLOBALimagenextmarginv}{%
					\fbox{\animategraphics[#3]{#4}{#2-}{0}{#5}}%
				}%
			}%
		}%
	}{ % Con label
		\subfloat[#7#1]{%
			\ifthenelse{\equal{\animatedimageloop}{true}}{ % Con loop
				\ifthenelse{\equal{\animatedimageautoplay}{true}}{ % Con autoplay
					\raisebox{\GLOBALimagenextmarginv}{%
						\fbox{\animategraphics[loop,autoplay,#3]{#4}{#2-}{0}{#5}}%
					}%
				}{ % Sin autoplay
					\raisebox{\GLOBALimagenextmarginv}{%
						\fbox{\animategraphics[loop,#3]{#4}{#2-}{0}{#5}}%
					}%
				}%
			}{ % Sin loop
				\ifthenelse{\equal{\animatedimageautoplay}{true}}{ % Con autoplay
					\raisebox{\GLOBALimagenextmarginv}{%
						\fbox{\animategraphics[autoplay,#3]{#4}{#2-}{0}{#5}}%
					}%
				}{ % Sin autoplay
					\raisebox{\GLOBALimagenextmarginv}{%
						\fbox{\animategraphics[#3]{#4}{#2-}{0}{#5}}%
					}%
				}%
			}%
		}%
	}%
	\endgroup%
	\global\def\GLOBALenvimageadded {true}%
	\global\def\GLOBALimagenextmarginv {0 cm}%
}

% Añade una imagen en el entorno "images" con borde sin leyenda animada
%	#1	Dirección de la imagen animada
%	#2	Parámetros de la imagen
%	#3	FPS de la imagen
%	#4	Total imágenes no definido
\newcommand{\addimageanimatedanum}[4]{%
	\addimageanimatedboxed{#1}{#2}{#3}{#4}{0}{\GLOBALemptyvar}%
}

% Añade una imagen en el entorno "images" con borde sin leyenda animada
%	#1	Dirección de la imagen animada
%	#2	Parámetros de la imagen
%	#3	FPS de la imagen
%	#4	Total imágenes no definido
%	#5	Ancho de la línea (en pt)
\newcommand{\addimageanimatedanumboxed}[5]{%
	\addimageanimatedboxed{#1}{#2}{#3}{#4}{#5}{\GLOBALemptyvar}%
}

% Añade una imagen en el entorno "images" con borde y un link
% 	#1	Label (opcional)
%	#2	Dirección de la imagen
%	#3	Parámetros de la imagen
%	#4	Link de la imagen
%	#5	Leyenda de la imagen (opcional)
\newcommand{\addimagelink}[5][]{%
	\addimagelinkboxed[#1]{#2}{#3}{0}{#4}{#5}%
}

% Añade una imagen en el entorno "images" con borde
% 	#1	Label (opcional)
%	#2	Dirección de la imagen
%	#3	Parámetros de la imagen
%	#4	Ancho de la línea (en pt)
%	#5	Link de la imagen
%	#6	Leyenda de la imagen (opcional)
\newcommand{\addimagelinkboxed}[6][]{%
	\global\def\GLOBALimagelink {#5}%
	\addimageboxed[#1]{#2}{#3}{#4}{#6}%
	\global\def\GLOBALimagelink {\GLOBALemptyvar}%
}

% Añade una imagen en el entorno "images" con borde sin leyenda
%	#1	Dirección de la imagen
%	#2	Parámetros de la imagen
%	#3	Link de la imagen
\newcommand{\addimageanumlink}[3]{%
	\addimagelink{#1}{#2}{#3}{\GLOBALemptyvar}%
}

% Añade una imagen en el entorno "images" con borde sin leyenda
%	#1	Dirección de la imagen
%	#2	Parámetros de la imagen
%	#3	Ancho de la línea (en pt)
%	#4	Link de la imagen
\newcommand{\addimageanumlinkboxed}[4]{%
	\addimagelinkboxed{#1}{#2}{#3}{#4}{\GLOBALemptyvar}%
}

% Permite continuar la numeración en el entorno "images"
\newcommand{\imagescontinuenumbering}{%
	\checkonlyonenvimage%
	\global\def\GLOBALenvimagecf {true}%
}

% Agrega un espacio horizontal en el entorno "images"
% 	#1 Tamaño del espacio
\newcommand{\imageshspace}[1]{%
	\checkonlyonenvimage%
	\global\def\GLOBALenvimageadded {false}%
	\hspace{#1}%
}

% Añade un salto de línea en el entorno "images"
\newcommand{\imagesnewline}{%
	\checkonlyonenvimage%
	\global\def\GLOBALenvimageadded {false}%
	\corevspacevarcm{\marginimagemultbottom}%
	~\linebreak\noindent%
}

% Agrega un espacio vertical en el entorno "images"
% 	#1 Tamaño del espacio
\newcommand{\imagesvspace}[1]{%
	\checkonlyonenvimage%
	\global\def\GLOBALenvimageadded {false}%
	~ \\ \vspace*{#1}%
}

% Establece el margen vertical de la siguiente imagen en el entorno "images"
%	#1	Margen vertical
\newcommand{\setnextimagevmargin}[1]{%
	\checkonlyonenvimage%
	\emptyvarerr{\setimagesvmargin}{#1}{Tamaño del margen}%
	\global\def\GLOBALimagenextmarginv {#1}%
}

% Insertar una imagen
% 	#1	Label (opcional)
%	#2	Dirección de la imagen
%	#3	Parámetros de la imagen
%	#4	Leyenda de la imagen (opcional)
\newcommand{\insertimage}[4][]{%
	\insertimageboxed[#1]{#2}{#3}{0}{#4}%
}

% Insertar una imagen con recuadro
% 	#1	Label (opcional)
%	#2	Dirección de la imagen
%	#3	Parámetros de la imagen
%	#4	Ancho de la línea (en pt)
%	#5	Leyenda de la imagen (opcional)
\newcommand{\insertimageboxed}[5][]{%
	\emptyvarerr{\insertimageboxed}{#2}{Direccion de la imagen no definida}%
	\emptyvarerr{\insertimageboxed}{#3}{Parametros de la imagen no definidos}%
	\emptyvarerr{\insertimageboxed}{#4}{Ancho de la linea no definido}%
	\checkoutsideenvimage%
	\corevspacevarcm{\marginimagetop}%
	\begin{samepage}%
	\begin{figure}[H]%
		\begingroup%
			\setlength{\fboxsep}{0 pt}%
			\setlength{\fboxrule}{#4 pt}%
			\centering%
			\ifthenelse{\equal{\GLOBALimagelink}{\GLOBALemptyvar}}{ % Sin link
				\fbox{\includegraphics[#3]{#2}}%
			}{ % Con link
				\fbox{\href{\GLOBALimagelink}{\includegraphics[#3]{#2}}}%
			}%
		\endgroup%
		\ifx\hfuzz#5\hfuzz%
			\corevspacevarcm{\captionlessmarginimage}%
		\else%
			\hspace{0cm}%
			\corevspacevarcm{\captionmarginimage}%
			\ifthenelse{\equal{\GLOBALcaptiondefn}{\GLOBALemptyvar}}{\caption{#5 #1}}{\caption[\GLOBALcaptiondefn]{#5 #1}}%
		\fi%
	\end{figure}
	\corevspacevarcm{\marginimagebottom}%
	\end{samepage}
	\resetindexcaption%
}

% Insertar una imagen animada
% 	#1	Label (opcional)
%	#2	Dirección de la imagen animada
%	#3	Parámetros de la imagen
%	#4	FPS de la imagen
%	#5	Total imágenes no definido
%	#6	Leyenda de la imagen (opcional)
\newcommand{\insertanimatedimage}[6][]{%
	\insertanimatedimageboxed[#1]{#2}{#3}{#4}{#5}{0}{#6}%
}

% Insertar una imagen animada
% 	#1	Label (opcional)
%	#2	Dirección de la imagen animada
%	#3	Parámetros de la imagen
%	#4	FPS de la imagen
%	#5	Total imágenes no definido
%	#6	Ancho de la línea (en pt)
%	#7	Leyenda de la imagen (opcional)
\newcommand{\insertanimatedimageboxed}[7][]{%
	\emptyvarerr{\insertanimatedimage}{#2}{Direccion de la imagen no definida}%
	\emptyvarerr{\insertanimatedimage}{#3}{Parametros de la imagen no definidos}%
	\emptyvarerr{\insertanimatedimage}{#4}{FPS no definido}%
	\emptyvarerr{\insertanimatedimage}{#5}{Total imagenes no definido}%
	\emptyvarerr{\insertanimatedimage}{#6}{Ancho de la línea no definido}%
	\checkoutsideenvimage%
	\corevspacevarcm{\marginimagetop}%
	\begin{samepage}%
	\begin{figure}[H]%
		\begingroup%
			\setlength{\fboxsep}{0 pt}%
			\setlength{\fboxrule}{#6 pt}%
			\centering%
			\ifthenelse{\equal{\animatedimagecontrols}{true}}{ % Muestra los controles
				\ifthenelse{\equal{\animatedimageloop}{true}}{ % Con loop
					\ifthenelse{\equal{\animatedimageautoplay}{true}}{ % Con autoplay
						\fbox{\animategraphics[controls,loop,autoplay,#3]{#4}{#2-}{0}{#5}}%	
					}{ % Sin autoplay
						\fbox{\animategraphics[controls,loop,#3]{#4}{#2-}{0}{#5}}%
					}%
				}{ % Sin loop
					\ifthenelse{\equal{\animatedimageautoplay}{true}}{ % Con autoplay
						\fbox{\animategraphics[controls,autoplay,#3]{#4}{#2-}{0}{#5}}%
					}{ % Sin autoplay
						\fbox{\animategraphics[controls,#3]{#4}{#2-}{0}{#5}}%
					}%
				}%
			}{ % Sin controles
				\ifthenelse{\equal{\animatedimageloop}{true}}{ % Con loop
					\ifthenelse{\equal{\animatedimageautoplay}{true}}{ % Con autoplay
						\fbox{\animategraphics[loop,autoplay,#3]{#4}{#2-}{0}{#5}}%
					}{ % Sin autoplay
						\fbox{\animategraphics[loop,#3]{#4}{#2-}{0}{#5}}%
					}%
				}{ % Sin loop
					\ifthenelse{\equal{\animatedimageautoplay}{true}}{ % Con autoplay
						\fbox{\animategraphics[autoplay,#3]{#4}{#2-}{0}{#5}}%
					}{ % Sin autoplay
						\fbox{\animategraphics[#3]{#4}{#2-}{0}{#5}}%
					}%
				}%
			}%
		\endgroup%
		\ifx\hfuzz#7\hfuzz%
			\corevspacevarcm{\captionlessmarginimage}%
		\else%
			\hspace{0cm}%
			\corevspacevarcm{\captionmarginimage}%
			\ifthenelse{\equal{\GLOBALcaptiondefn}{\GLOBALemptyvar}}{\caption{#7 #1}}{\caption[\GLOBALcaptiondefn]{#7 #1}}%
		\fi%
	\end{figure}
	\corevspacevarcm{\marginimagebottom}%
	\end{samepage}
	\resetindexcaption%
}

% Insertar una imagen con link
% 	#1	Label (opcional)
%	#2	Dirección de la imagen
%	#3	Parámetros de la imagen
%	#4	Link de la imagen
%	#5	Leyenda de la imagen (opcional)
\newcommand{\insertimagelink}[5][]{%
	\insertimagelinkboxed[#1]{#2}{#3}{0}{#4}{#5}%
}

% Insertar una imagen con recuadro con link
% 	#1	Label (opcional)
%	#2	Dirección de la imagen
%	#3	Parámetros de la imagen
%	#4	Ancho de la línea (en pt)
%	#5	Link de la imagen
%	#6	Leyenda de la imagen (opcional)
\newcommand{\insertimagelinkboxed}[6][]{%
	\global\def\GLOBALimagelink {#5}%
	\insertimageboxed[#1]{#2}{#3}{#4}{#6}%
	\global\def\GLOBALimagelink {\GLOBALemptyvar}%
}

% Insertar una imagen completa en un entorno multicol
% 	#1	Label (opcional)
%	#2	Dirección de la imagen
%	#3	Parámetros de la imagen
%	#4	Posición, "bottom" o "top"
%	#5	Leyenda de la imagen (opcional)
\newcommand{\insertimagemc}[5][]{%
	\insertimageboxedmc[#1]{#2}{#3}{0}{#4}{#5}%
}

% Insertar una imagen completa con recuadro en un entorno multicol
% 	#1	Label (opcional)
%	#2	Dirección de la imagen
%	#3	Parámetros de la imagen
%	#4	Ancho de la línea (en pt)
%	#5	Posición, "bottom", "top", "fixed2", "fixed3", "fixed4"
%	#6	Leyenda de la imagen (opcional)
\newcommand{\insertimageboxedmc}[6][]{%
	\emptyvarerr{\insertimageboxedmc}{#2}{Direccion de la imagen no definida}%
	\emptyvarerr{\insertimageboxedmc}{#3}{Parametros de la imagen no definidos}%
	\emptyvarerr{\insertimageboxedmc}{#4}{Ancho de la linea no definido}%
	\emptyvarerr{\insertimageboxedmc}{#5}{Posicion de la imagen no definida}%
	\checkoutsideenvimage%
	\checkinsidemulticol%
	\checkoutsideappendix%
	\setcaptionmargincm{\captionlrmarginmc}%
	\ifthenelse{\equal{#5}{bottom}}{%
		\begin{samepage}%
		\begin{figure*}[!b]
	}{%
	\ifthenelse{\equal{#5}{top}}{%
		\begin{samepage}%
		\begin{figure*}[!t]
	}{%
	\ifthenelse{\equal{#5}{fixed2}}{%
		\end{multicols}
		\begin{samepage}%
		\begin{figure*}[!h]
	}{%
	\ifthenelse{\equal{#5}{fixed2b}}{%
		\end{multicols}
		\begin{samepage}%
		\begin{figure*}[!b]
	}{%
	\ifthenelse{\equal{#5}{fixed2t}}{%
		\end{multicols}
		\begin{samepage}%
		\begin{figure*}[!t]
	}{%
	\ifthenelse{\equal{#5}{fixed3}}{%
		\end{multicols}
		\begin{samepage}%
		\begin{figure*}[!h]
	}{%
	\ifthenelse{\equal{#5}{fixed3b}}{%
		\end{multicols}
		\begin{samepage}%
		\begin{figure*}[!b]
	}{%
	\ifthenelse{\equal{#5}{fixed3t}}{%
		\end{multicols}
		\begin{samepage}%
		\begin{figure*}[!t]
	}{%
	\ifthenelse{\equal{#5}{fixed4}}{%
		\end{multicols}
		\begin{samepage}%
		\begin{figure*}[!h]
	}{%
	\ifthenelse{\equal{#5}{fixed4b}}{%
		\end{multicols}
		\begin{samepage}%
		\begin{figure*}[!h]
	}{%
	\ifthenelse{\equal{#5}{fixed4t}}{%
		\end{multicols}
		\begin{samepage}%
		\begin{figure*}[!h]
	}{%
		\errmessage{LaTeX Warning: Posicion de imagen invalida, valores esperados: bottom,top,fixed2,fixed2b,fixed2t,fixed3,fixed3b,fixed3t,fixed4,fixed4b,fixed4t}
		\stop}}}}}}}}}}
	}%
		\begingroup%
			\setlength{\fboxsep}{0 pt}%
			\setlength{\fboxrule}{#4 pt}%
			\centering%
			\fbox{\includegraphics[#3]{#2}}%
		\endgroup%
		\ifx\hfuzz#6\hfuzz%
			\corevspacevarcm{\captionlessmarginimage}%
		\else%
			\hspace{0cm}%
			\corevspacevarcm{\captionmarginimage}%
			\ifthenelse{\equal{\GLOBALcaptiondefn}{\GLOBALemptyvar}}{\caption{#6 #1}}{\caption[\GLOBALcaptiondefn]{#6 #1}}%
		\fi%
	\end{figure*}
	\end{samepage}
	\ifthenelse{\equal{#5}{fixed2}}{%
		\begin{multicols}{2}%
	}{%
	\ifthenelse{\equal{#5}{fixed2b}}{%
		\begin{multicols}{2}%
	}{%
	\ifthenelse{\equal{#5}{fixed2t}}{%
		\begin{multicols}{2}%
	}{%
	\ifthenelse{\equal{#5}{fixed3}}{%
		\begin{multicols}{3}%
	}{%
	\ifthenelse{\equal{#5}{fixed3b}}{%
		\begin{multicols}{3}%
	}{%
	\ifthenelse{\equal{#5}{fixed3t}}{%
		\begin{multicols}{3}%
	}{%
	\ifthenelse{\equal{#5}{fixed4}}{%
		\begin{multicols}{4}%
	}{%
	\ifthenelse{\equal{#5}{fixed4b}}{%
		\begin{multicols}{4}%
	}{%
	\ifthenelse{\equal{#5}{fixed4t}}{%
		\begin{multicols}{4}%
	}{%
	}}}}}}}}}%
	\setcaptionmargincm{\captionlrmargin}%
	\resetindexcaption%
}

% Insertar una imagen dentro de una tabla
%	#1	Dirección de la imagen
%	#2	Parámetros de la imagen
\newcommand{\inserttableimage}[2]{%
	\inserttableimageboxed{#1}{#2}{0}%
}

% Insertar una imagen dentro de una tabla con recuadro
%	#1	Dirección de la imagen
%	#2	Parámetros de la imagen
%	#3	Ancho de la línea (en pt)
\newcommand{\inserttableimageboxed}[3]{%
	\emptyvarerr{\inserttableimageboxed}{#1}{Direccion de la imagen no definida}%
	\emptyvarerr{\inserttableimageboxed}{#2}{Parametros de la imagen no definidos}%
	\emptyvarerr{\inserttableimageboxed}{#3}{Ancho de la linea no definido}%
	\checkoutsideenvimage%
	\begingroup%
	\setlength{\fboxsep}{0 pt}%
	\setlength{\fboxrule}{#3 pt}%
	\raisebox{-1\totalheight}{\fbox{\includegraphics[#2]{#1}}}%
	\endgroup%
	\resetindexcaption%
}

% Insertar una imagen a la izquierda, escalada, ancho fijo
% 	#1	Label (opcional)
%	#2	Dirección de la imagen
%	#3	Ancho de la imagen (en linewidth)
%	#4	Leyenda de la imagen (opcional)
\newcommand{\insertimageleft}[4][]{%
	\insertimageleftboxed[#1]{#2}{#3}{0}{#4}%
}

% Insertar una imagen a la izquierda, escalada, ancho fijo
% 	#1	Label (opcional)
%	#2	Dirección de la imagen
%	#3	Ancho de la imagen (en linewidth)
%	#4	Ancho de la línea (en pt)
%	#5	Leyenda de la imagen (opcional)
\newcommand{\insertimageleftboxed}[5][]{%
	\emptyvarerr{\insertimageleftboxed}{#2}{Direccion de la imagen no definida}%
	\emptyvarerr{\insertimageleftboxed}{#3}{Ancho de la imagen no definido}%
	\emptyvarerr{\insertimageleftboxed}{#4}{Ancho de la linea no definido}%
	\checkoutsideenvimage%
	~%
	\vspace{-\baselineskip}%
	\par%
	\begin{wrapfigure}{l}{#3\linewidth}%
		\setcaptionmargincm{0}%
		\ifthenelse{\equal{\figurecaptiontop}{true}}{}{%
			\vspace{\marginfloatimages pt}%
		}%
		\begingroup%
			\setlength{\fboxsep}{0 pt}%
			\setlength{\fboxrule}{#4 pt}%
			\centering%
			\fbox{\includegraphics[width=\linewidth]{#2}}%
		\endgroup%
		\ifx\hfuzz#5\hfuzz%
			\corevspacevarcm{\captionlessmarginimage}%
		\else%
			\corevspacevarcm{\captionmarginimage}%
			\ifthenelse{\equal{\GLOBALcaptiondefn}{\GLOBALemptyvar}}{\caption{#5 #1}}{\caption[\GLOBALcaptiondefn]{#5 #1}}%
		\fi%
	\end{wrapfigure}
	\setcaptionmargincm{\captionlrmargin}%
	\resetindexcaption%
}

% Insertar una imagen a la izquierda, ajustada en un número de líneas, escalada, ancho fijo
% 	#1	Label (opcional)
%	#2	Dirección de la imagen
%	#3	Ancho de la imagen (en linewidth)
%	#4	Altura en líneas de la imagen
%	#5	Leyenda de la imagen (opcional)
\newcommand{\insertimageleftline}[5][]{%
	\insertimageleftlineboxed[#1]{#2}{#3}{0}{#4}{#5}%
}

% Insertar una imagen recuadrada a la izquierda, ajustada en un número de líneas, escalada, ancho fijo
% 	#1	Label (opcional)
%	#2	Dirección de la imagen
%	#3	Ancho de la imagen (en linewidth)
%	#4	Ancho de la línea (en pt)
%	#5	Altura en líneas de la imagen
%	#6	Leyenda de la imagen (opcional)
\newcommand{\insertimageleftlineboxed}[6][]{%
	\emptyvarerr{\insertimageleftlineboxed}{#2}{Direccion de la imagen no definida}%
	\emptyvarerr{\insertimageleftlineboxed}{#3}{Ancho de la imagen no definido}%
	\emptyvarerr{\insertimageleftlineboxed}{#4}{Ancho de la linea no definido}%
	\emptyvarerr{\insertimageleftlineboxed}{#5}{Altura en lineas de la imagen flotante izquierda no definida}
	\checkoutsideenvimage%
	~%
	\vspace{-\baselineskip}%
	\par%
	\begin{wrapfigure}[#5]{l}{#3\linewidth}%
		\setcaptionmargincm{0}%
		\ifthenelse{\equal{\figurecaptiontop}{true}}{}{%
			\vspace{\marginfloatimages pt}}%
		\begingroup%
			\setlength{\fboxsep}{0 pt}%
			\setlength{\fboxrule}{#4 pt}%
			\centering%
			\fbox{\includegraphics[width=\linewidth]{#2}}%
		\endgroup%
		\ifx\hfuzz#6\hfuzz%
			\corevspacevarcm{\captionlessmarginimage}%
		\else%
			\corevspacevarcm{\captionmarginimage}%
			\ifthenelse{\equal{\GLOBALcaptiondefn}{\GLOBALemptyvar}}{\caption{#6 #1}}{\caption[\GLOBALcaptiondefn]{#6 #1}}%
		\fi%
	\end{wrapfigure}
	\setcaptionmargincm{\captionlrmargin}%
	\resetindexcaption%
}

% Insertar una imagen a la derecha, escalada, ancho fijo
% 	#1	Label (opcional)
%	#2	Dirección de la imagen
%	#3	Ancho de la imagen (en linewidth)
%	#4	Leyenda de la imagen (opcional)
\newcommand{\insertimageright}[4][]{%
	\insertimagerightboxed[#1]{#2}{#3}{0}{#4}%
}

% Insertar una imagen recuadrada a la derecha, escalada, ancho fijo
% 	#1	Label (opcional)
%	#2	Dirección de la imagen
%	#3	Ancho de la imagen (en linewidth)
%	#4	Ancho de la línea (en pt)
%	#5	Leyenda de la imagen (opcional)
\newcommand{\insertimagerightboxed}[5][]{%
	\emptyvarerr{\insertimagerightboxed}{#2}{Direccion de la imagen no definida}%
	\emptyvarerr{\insertimagerightboxed}{#3}{Ancho de la imagen no defindo}%
	\emptyvarerr{\insertimagerightboxed}{#4}{Ancho de la linea no definido}%
	\checkoutsideenvimage%
	~%
	\vspace{-\baselineskip}%
	\par%
	\begin{wrapfigure}{r}{#3\linewidth}%
		\setcaptionmargincm{0}%
		\ifthenelse{\equal{\figurecaptiontop}{true}}{}{%
			\vspace{\marginfloatimages pt}%
		}%
		\begingroup%
			\setlength{\fboxsep}{0 pt}%
			\setlength{\fboxrule}{#4 pt}%
			\centering%
			\fbox{\includegraphics[width=\linewidth]{#2}}%
		\endgroup%
		\ifx\hfuzz#5\hfuzz%
			\corevspacevarcm{\captionlessmarginimage}%
		\else%
			\corevspacevarcm{\captionmarginimage}%
			\ifthenelse{\equal{\GLOBALcaptiondefn}{\GLOBALemptyvar}}{\caption{#5 #1}}{\caption[\GLOBALcaptiondefn]{#5 #1}}%
		\fi%
	\end{wrapfigure}
	\setcaptionmargincm{\captionlrmargin}%
	\resetindexcaption%
}

% Insertar una imagen a la derecha, ajustada en un número de líneas, escalada, ancho fijo
% 	#1	Label (opcional)
%	#2	Dirección de la imagen
%	#3	Ancho de la imagen (en linewidth)
%	#4	Altura en líneas de la imagen
%	#5	Leyenda de la imagen (opcional)
\newcommand{\insertimagerightline}[5][]{%
	\insertimagerightlineboxed[#1]{#2}{#3}{0}{#4}{#5}%
}

% Insertar una imagen recuadrada a la derecha, ajustada en un número de líneas, escalada, ancho fijo
% 	#1	Label (opcional)
%	#2	Dirección de la imagen
%	#3	Ancho de la imagen (en linewidth)
%	#4	Ancho de la línea (en pt)
%	#5	Altura en líneas de la imagen
%	#6	Leyenda de la imagen (opcional)
\newcommand{\insertimagerightlineboxed}[6][]{%
	\emptyvarerr{\insertimagerightlineboxed}{#2}{Direccion de la imagen no definida}%
	\emptyvarerr{\insertimagerightlineboxed}{#3}{Ancho de la imagen no defindo}%
	\emptyvarerr{\insertimagerightlineboxed}{#4}{Ancho de la linea no definido}%
	\emptyvarerr{\insertimagerightlineboxed}{#5}{Altura en lineas de la imagen flotante derecha no definida}%
	\checkoutsideenvimage%
	~%
	\vspace{-\baselineskip}%
	\par%
	\begin{wrapfigure}[#5]{r}{#3\linewidth}%
		\setcaptionmargincm{0}%
		\ifthenelse{\equal{\figurecaptiontop}{true}}{}{%
			\vspace{\marginfloatimages pt}%
		}%
		\begingroup%
			\setlength{\fboxsep}{0 pt}%
			\setlength{\fboxrule}{#4 pt}%
			\centering%
			\fbox{\includegraphics[width=\linewidth]{#2}}%
		\endgroup%
		\ifx\hfuzz#6\hfuzz%
			\corevspacevarcm{\captionlessmarginimage}%
		\else%
			\corevspacevarcm{\captionmarginimage}%
			\ifthenelse{\equal{\GLOBALcaptiondefn}{\GLOBALemptyvar}}{\caption{#6 #1}}{\caption[\GLOBALcaptiondefn]{#6 #1}}%
		\fi%
	\end{wrapfigure}
	\setcaptionmargincm{\captionlrmargin}%
	\resetindexcaption%
}

% Insertar una imagen a la izquierda, propiedades variables
% 	#1	Label (opcional)
%	#2	Dirección de la imagen
%	#3	Ancho del objeto
%	#4	Propiedades de la imagen
%	#5	Leyenda de la imagen (opcional)
\newcommand{\insertimageleftp}[5][]{%
	\xspace ~ \\%
	\vspace{-2\baselineskip}%
	\par%
	\insertimageleftboxedp[#1]{#2}{#3}{#4}{0}{#5}%
}

% Insertar una imagen a la izquierda, propiedades variables
% 	#1	Label (opcional)
%	#2	Dirección de la imagen
%	#3	Ancho del objeto
%	#4	Propiedades de la imagen
%	#5	Ancho de la línea (en pt)
%	#6	Leyenda de la imagen (opcional)
\newcommand{\insertimageleftboxedp}[6][]{%
	\emptyvarerr{\insertimageleftboxedp}{#2}{Direccion de la imagen no definida}%
	\emptyvarerr{\insertimageleftboxedp}{#3}{Ancho del objeto no definido}%
	\emptyvarerr{\insertimageleftboxedp}{#4}{Propiedades de la imagen no defindos}%
	\emptyvarerr{\insertimageleftboxedp}{#5}{Ancho de la linea no definido}%
	\checkoutsideenvimage%
	~%
	\vspace{-\baselineskip}%
	\par%
	\begin{wrapfigure}{l}{#3}%
		\setcaptionmargincm{0}%
		\ifthenelse{\equal{\figurecaptiontop}{true}}{}{%
			\vspace{\marginfloatimages pt}%
		}%
		\begingroup%
			\setlength{\fboxsep}{0 pt}%
			\setlength{\fboxrule}{#5 pt}%
			\centering%
			\fbox{\includegraphics[#4]{#2}}%
		\endgroup%
		\ifx\hfuzz#6\hfuzz%
			\corevspacevarcm{\captionlessmarginimage}%
		\else%
			\corevspacevarcm{\captionmarginimage}%
			\ifthenelse{\equal{\GLOBALcaptiondefn}{\GLOBALemptyvar}}{\caption{#6 #1}}{\caption[\GLOBALcaptiondefn]{#6 #1}}%
		\fi%
	\end{wrapfigure}
	\setcaptionmargincm{\captionlrmargin}%
	\resetindexcaption%
}

% Insertar una imagen a la izquierda, ajustada en un número de líneas, propiedades variables
% 	#1	Label (opcional)
%	#2	Dirección de la imagen
%	#3	Ancho del objeto
%	#4	Propiedades de la imagen
%	#5	Altura en líneas de la imagen
%	#6	Leyenda de la imagen (opcional)
\newcommand{\insertimageleftlinep}[6][]{%
	\insertimageleftlineboxedp[#1]{#2}{#3}{#4}{0}{#5}{#6}%
}

% Insertar una imagen recuadrada a la izquierda, ajustada en un número de líneas, propiedades variables
% 	#1	Label (opcional)
%	#2	Dirección de la imagen
%	#3	Ancho del objeto
%	#4	Propiedades de la imagen
%	#5	Ancho de la línea (en pt)
%	#6	Altura en líneas de la imagen
%	#7	Leyenda de la imagen (opcional)
\newcommand{\insertimageleftlineboxedp}[7][]{%
	\emptyvarerr{\insertimageleftlineboxedp}{#2}{Direccion de la imagen no definida}%
	\emptyvarerr{\insertimageleftlineboxedp}{#3}{Ancho del objeto no definido}%
	\emptyvarerr{\insertimageleftlineboxedp}{#4}{Propiedades de la imagen no definidos}%
	\emptyvarerr{\insertimageleftlineboxedp}{#5}{Ancho de la linea no definido}%
	\emptyvarerr{\insertimageleftlineboxedp}{#6}{Altura en lineas de la imagen flotante izquierda no definida}%
	\checkoutsideenvimage%
	~%
	\vspace{-\baselineskip}%
	\par%
	\begin{wrapfigure}[#6]{l}{#3}%
		\setcaptionmargincm{0}%
		\ifthenelse{\equal{\figurecaptiontop}{true}}{}{%
			\vspace{\marginfloatimages pt}%
		}%
		\begingroup%
			\setlength{\fboxsep}{0 pt}%
			\setlength{\fboxrule}{#5 pt}%
			\centering%
			\fbox{\includegraphics[#4]{#2}}%
		\endgroup%
		\ifx\hfuzz#7\hfuzz%
			\corevspacevarcm{\captionlessmarginimage}%
		\else%
			\corevspacevarcm{\captionmarginimage}%
			\ifthenelse{\equal{\GLOBALcaptiondefn}{\GLOBALemptyvar}}{\caption{#7 #1}}{\caption[\GLOBALcaptiondefn]{#7 #1}}%
		\fi%
	\end{wrapfigure}
	\setcaptionmargincm{\captionlrmargin}%
	\resetindexcaption%
}

% Insertar una imagen a la derecha, propiedades variables
% 	#1	Label (opcional)
%	#2	Dirección de la imagen
%	#3	Ancho del objeto (en cm)
%	#4	Propiedades de la imagen
%	#5	Leyenda de la imagen (opcional)
\newcommand{\insertimagerightp}[5][]{%
	\xspace ~ \\%
	\vspace{-2\baselineskip}%
	\par%
	\insertimagerightboxedp[#1]{#2}{#3}{#4}{0}{#5}%
}

% Insertar una imagen recuadrada a la derecha, propiedades variables
% 	#1	Label (opcional)
%	#2	Dirección de la imagen
%	#3	Ancho del objeto
%	#4	Propiedades de la imagen
%	#5	Ancho de la línea (en pt)
%	#6	Leyenda de la imagen (opcional)
\newcommand{\insertimagerightboxedp}[6][]{%
	\emptyvarerr{\insertimagerightboxedp}{#2}{Direccion de la imagen no definida}%
	\emptyvarerr{\insertimagerightboxedp}{#3}{Ancho del objeto no definido}%
	\emptyvarerr{\insertimagerightboxedp}{#4}{Propiedades de la imagen no definidos}%
	\emptyvarerr{\insertimagerightboxedp}{#5}{Ancho de la linea no definido}%
	\checkoutsideenvimage%
	~%
	\vspace{-\baselineskip}%
	\par%
	\begin{wrapfigure}{r}{#3}%
		\setcaptionmargincm{0}%
		\ifthenelse{\equal{\figurecaptiontop}{true}}{}{%
			\vspace{\marginfloatimages pt}%
		}%
		\begingroup%
			\setlength{\fboxsep}{0 pt}%
			\setlength{\fboxrule}{#5 pt}%
			\centering%
			\fbox{\includegraphics[#4]{#2}}%
		\endgroup%
		\ifx\hfuzz#6\hfuzz%
			\corevspacevarcm{\captionlessmarginimage}%
		\else%
			\corevspacevarcm{\captionmarginimage}%
			\ifthenelse{\equal{\GLOBALcaptiondefn}{\GLOBALemptyvar}}{\caption{#6 #1}}{\caption[\GLOBALcaptiondefn]{#6 #1}}%
		\fi%
	\end{wrapfigure}
	\setcaptionmargincm{\captionlrmargin}%
	\resetindexcaption%
}

% Insertar una imagen a la derecha, ajustada en un número de líneas, propiedades variables
% 	#1	Label (opcional)
%	#2	Dirección de la imagen
%	#3	Ancho del objeto (en cm)
%	#4	Propiedades de la imagen
%	#5	Altura en líneas de la imagen
%	#6	Leyenda de la imagen (opcional)
\newcommand{\insertimagerightlinep}[6][]{%
	\insertimagerightlineboxedp[#1]{#2}{#3}{#4}{0}{#5}{#6}%
}

% Insertar una imagen recuadrada a la derecha, ajustada en un número de líneas, propiedades variables
% 	#1	Label (opcional)
%	#2	Dirección de la imagen
%	#3	Ancho del objeto
%	#4	Propiedades de la imagen
%	#5	Ancho de la línea (en pt)
%	#6	Altura en líneas de la imagen
%	#7	Leyenda de la imagen (opcional)
\newcommand{\insertimagerightlineboxedp}[7][]{%
	\emptyvarerr{\insertimagerightlineboxedp}{#2}{Direccion de la imagen no definida}%
	\emptyvarerr{\insertimagerightlineboxedp}{#3}{Ancho del objeto no definido}%
	\emptyvarerr{\insertimagerightlineboxedp}{#4}{Propiedades de la imagen no definidos}%
	\emptyvarerr{\insertimagerightlineboxedp}{#5}{Ancho de la linea no definido}%
	\emptyvarerr{\insertimagerightlineboxedp}{#6}{Altura en lineas de la imagen flotante derecha no definida}%
	\checkoutsideenvimage%
	~%
	\vspace{-\baselineskip}%
	\par%
	\begin{wrapfigure}[#6]{r}{#3}%
		\setcaptionmargincm{0}%
		\ifthenelse{\equal{\figurecaptiontop}{true}}{}{%
			\vspace{\marginfloatimages pt}%
		}%
		\begingroup%
			\setlength{\fboxsep}{0 pt}%
			\setlength{\fboxrule}{#5 pt}%
			\centering%
			\fbox{\includegraphics[#4]{#2}}%
		\endgroup%
		\ifx\hfuzz#7\hfuzz%
			\corevspacevarcm{\captionlessmarginimage}%
		\else%
			\corevspacevarcm{\captionmarginimage}%
			\ifthenelse{\equal{\GLOBALcaptiondefn}{\GLOBALemptyvar}}{\caption{#7 #1}}{\caption[\GLOBALcaptiondefn]{#7 #1}}%
		\fi%
	\end{wrapfigure}
	\setcaptionmargincm{\captionlrmargin}%
	\resetindexcaption%
}

% Inserta una imagen con parametros keyvals almacenados en una variable
% 	#1	Parámetros (keyvals)
%	#2	Dirección de la imagen
\newcommand{\coreinsertkeyimage}[2]{%
	\expandafter\includegraphics\expandafter[#1]{\expandafter#2}%
}

% Define la clave resolution al insertar imágenes
\makeatletter
\define@key{Gin}{resolution}{\pdfimageresolution=#1\relax}
\makeatother

\global\def\GLOBALtitlerequirechapter {false}
\global\def\GLOBALtitleinitchapter {false}
\global\def\GLOBALtitleinitsection {false}
\global\def\GLOBALtitleinitsubsection {false}
\global\def\GLOBALtitleinitsubsubsection {false}
\global\def\GLOBALtitleinitsubsubsubsection {false}

% Configura textos a añadir antes de secciones
\global\def\GLOBALtitleprechapterstr {}
\global\def\GLOBALtitlepresectionstr {}
\global\def\GLOBALtitlepresubsectionstr {}
\global\def\GLOBALtitlepresubsubsectionstr {}
\global\def\GLOBALtitlepresubsubsubsectionstr {}

% Configura que entornos pueden funcionar
\global\def\GLOBALtitlechapterenabled {true}

% Activa la numeración en las secciones
\def\coreintializetitlenumbering {%
	% Capítulo
	\renewcommand{\thechapter}{\GLOBALformatnumchapter{chapter}}%
	% Section
	\ifthenelse{\equal{\GLOBALchapternumenabled}{false}}{%
		\renewcommand{\thesection}{%
			\GLOBALformatnumsection{section}%
		}%
	}{%
		\renewcommand{\thesection}{%
			\thechapter\charbetwchaptersection\GLOBALformatnumsection{section}%
		}%
	}%
	% Subsection
	\ifthenelse{\equal{\GLOBALsectionanumenabled}{true}}{%
		\renewcommand{\thesubsection}{%
			\GLOBALformatnumssection{subsection}%
		}%
	}{%
		\renewcommand{\thesubsection}{%
			\thesection\charbetwsectionsubsection\GLOBALformatnumssection{subsection}%
		}%
	}%
	% Subsubsection
	\ifthenelse{\equal{\GLOBALsubsectionanumenabled}{true}}{%
		\renewcommand{\thesubsubsection}{%
			\GLOBALformatnumsssection{subsubsection}%
		}%
	}{%
		\renewcommand{\thesubsubsection}{%
			\thesubsection\charbetwsubsectionssect\GLOBALformatnumsssection{subsubsection}%
		}%
	}%
	% Subsubsubsection
	\ifthenelse{\equal{\GLOBALsubsubsectionanumenabled}{true}}{%
		\renewcommand{\thesubsubsubsection}{%
			\GLOBALformatnumssssection{subsubsubsection}%
		}%
	}{%
		\renewcommand{\thesubsubsubsection}{%
			\thesubsubsection\charbetwssectionsssect\GLOBALformatnumssssection{subsubsubsection}%
		}%
	}%
	\hbadness=10000%
}

% Chequea si los capítulos están activados
\def\corecheckchapterenabled {%
	\ifthenelse{\equal{\GLOBALtitlechapterenabled}{false}}{ % Verifica que el entorno esté activo
		\throwwarning{La insercion de capitulos esta desactivada}%
	}{}%
}

% Chequea si los capítulos han sido iniciados
\def\corecheckchapterinitialized {%
	\ifthenelse{\equal{\GLOBALtitlerequirechapter}{true}}{%
		\ifthenelse{\equal{\GLOBALtitleinitchapter}{false}}{%
			\throwwarning{Se requiere un nuevo capitulo}%
		}{}%
	}{}%
}

% Chequea si una sección han sido iniciada
\def\corechecksectioninitialized {%
	\ifthenelse{\equal{\GLOBALtitleinitsection}{false}}{%
		\throwwarning{Se requiere una nueva seccion}%
	}{}%
}

% Chequea si una subsección han sido iniciada
\def\corechecksubsectioninitialized {%
	\ifthenelse{\equal{\GLOBALtitleinitsubsection}{false}}{%
		\throwwarning{Se requiere una nueva subseccion}%
	}{}%
}

% Chequea si una subsubsección han sido iniciada
\def\corechecksubsubsectioninitialized {%
	\ifthenelse{\equal{\GLOBALtitleinitsubsubsection}{false}}{%
		\throwwarning{Se requiere una nueva subsubseccion}%
	}{}%
}

% Parcha el formato de capítulos
\pretocmd{\chapter}{%
	\corecheckchapterenabled%
	\ifthenelse{\equal{\showsectioncaptioncode}{chap}}{ % Reinicia código fuente
		\addtocounter{templateListings}{\value{lstlisting}}%
		\setcounter{lstlisting}{0}%
	}{}%
	\ifthenelse{\equal{\showsectioncaptioneqn}{chap}}{ % Reinicia ecuaciones
		\addtocounter{templateEquations}{\value{equation}}%
		\setcounter{equation}{0}%
	}{}%
	\ifthenelse{\equal{\equationrestart}{chap}}{ % Reinicia ecuaciones
		\addtocounter{templateEquations}{\value{equation}}%
		\setcounter{equation}{0}%
	}{}%
	\ifthenelse{\equal{\showsectioncaptionfig}{chap}}{ % Reinicia figuras
		\addtocounter{templateFigures}{\value{figure}}%
		\setcounter{figure}{0}%
	}{}%
	\ifthenelse{\equal{\showsectioncaptiontab}{chap}}{ % Reinicia tablas
		\addtocounter{templateTables}{\value{table}}%
		\setcounter{table}{0}%
	}{}%
	\global\def\GLOBALchapternumenabled {true}%
	\global\def\GLOBALsectionanumenabled {false}%
	\global\def\GLOBALsubsectionanumenabled {false}%
	\global\def\GLOBALsubsubsectionanumenabled {false}%
	\global\def\GLOBALtitleinitchapter {true}%
	\global\def\GLOBALtitleinitsection {false}%
	\global\def\GLOBALtitleinitsubsection {false}%
	\global\def\GLOBALtitleinitsubsubsection {false}%
	\global\def\GLOBALtitleinitsubsubsubsection {false}%
	\coreintializetitlenumbering%
}{}{}

% Parcha el formato de secciones al pasar desde una anum, vuelve a activar número
% de la sección
\pretocmd{\section}{%
	\ifthenelse{\equal{\showsectioncaptioncode}{sec}}{ % Reinicia código fuente
		\addtocounter{templateListings}{\value{lstlisting}}%
		\setcounter{lstlisting}{0}%
	}{}%
	\ifthenelse{\equal{\showsectioncaptioneqn}{sec}}{ % Reinicia ecuaciones
		\addtocounter{templateEquations}{\value{equation}}%
		\setcounter{equation}{0}%
	}{}%
	\ifthenelse{\equal{\equationrestart}{sec}}{ % Reinicia ecuaciones
		\addtocounter{templateEquations}{\value{equation}}%
		\setcounter{equation}{0}%
	}{}%
	\ifthenelse{\equal{\showsectioncaptionfig}{sec}}{ % Reinicia figuras
		\addtocounter{templateFigures}{\value{figure}}%
		\setcounter{figure}{0}%
	}{}%
	\ifthenelse{\equal{\showsectioncaptiontab}{sec}}{ % Reinicia tablas
		\addtocounter{templateTables}{\value{table}}%
		\setcounter{table}{0}%
	}{}%
	\global\def\GLOBALsectionanumenabled {false}%
	\global\def\GLOBALsubsectionanumenabled {false}%
	\global\def\GLOBALsubsubsectionanumenabled {false}%
	\global\def\GLOBALtitleinitsection {true}%
	\global\def\GLOBALtitleinitsubsection {false}%
	\global\def\GLOBALtitleinitsubsubsection {false}%
	\global\def\GLOBALtitleinitsubsubsubsection {false}%
	\corecheckchapterinitialized%
	\coreintializetitlenumbering%
}{}{}

% Comienza nueva subsección, si está dentro de una sectionanum entonces no dibuja el
% número de sección, si no entonces dibuja el número de forma normal
\pretocmd{\subsection}{%
	\ifthenelse{\equal{\showsectioncaptioncode}{ssec}}{ % Reinicia código fuente
		\addtocounter{templateListings}{\value{lstlisting}}%
		\setcounter{lstlisting}{0}%
	}{}%
	\ifthenelse{\equal{\showsectioncaptioneqn}{ssec}}{ % Reinicia ecuaciones
		\addtocounter{templateEquations}{\value{equation}}%
		\setcounter{equation}{0}%
	}{}%
	\ifthenelse{\equal{\equationrestart}{ssec}}{ % Reinicia ecuaciones
		\addtocounter{templateEquations}{\value{equation}}%
		\setcounter{equation}{0}%
	}{}%
	\ifthenelse{\equal{\showsectioncaptionfig}{ssec}}{ % Reinicia figuras
		\addtocounter{templateFigures}{\value{figure}}%
		\setcounter{figure}{0}%
	}{}%
	\ifthenelse{\equal{\showsectioncaptiontab}{ssec}}{ % Reinicia tablas
		\addtocounter{templateTables}{\value{table}}%
		\setcounter{table}{0}%
	}{}%
	\global\def\GLOBALsubsectionanumenabled {false}%
	\global\def\GLOBALsubsubsectionanumenabled {false}%
	\global\def\GLOBALtitleinitsubsection {true}%
	\global\def\GLOBALtitleinitsubsubsection {false}%
	\global\def\GLOBALtitleinitsubsubsubsection {false}%
	\corecheckchapterinitialized%
	\corechecksectioninitialized%
	\coreintializetitlenumbering%
}{}{}

% Comienza nueva subsubsección, aquí hay varios casos:
%	- si está dentro de una subsección sin número ignora la sección
%	- si no, entonces puede estar dentro de una sección sin número o no, en ese caso
%	  debe evaluar ambas posibilidades
\pretocmd{\subsubsection}{%
	\ifthenelse{\equal{\showsectioncaptioncode}{sssec}}{ % Reinicia código fuente
		\addtocounter{templateListings}{\value{lstlisting}}%
		\setcounter{lstlisting}{0}%
	}{}%
	\ifthenelse{\equal{\showsectioncaptioneqn}{sssec}}{ % Reinicia ecuaciones
		\addtocounter{templateEquations}{\value{equation}}%
		\setcounter{equation}{0}%
	}{}%
	\ifthenelse{\equal{\equationrestart}{sssec}}{ % Reinicia ecuaciones
		\addtocounter{templateEquations}{\value{equation}}%
		\setcounter{equation}{0}%
	}{}%
	\ifthenelse{\equal{\showsectioncaptionfig}{sssec}}{ % Reinicia figuras
		\addtocounter{templateFigures}{\value{figure}}%
		\setcounter{figure}{0}%
	}{}%
	\ifthenelse{\equal{\showsectioncaptiontab}{sssec}}{ % Reinicia tablas
		\addtocounter{templateTables}{\value{table}}%
		\setcounter{table}{0}%
	}{}%
	\global\def\GLOBALsubsubsectionanumenabled {false}%
	\global\def\GLOBALtitleinitsubsubsection {true}%
	\global\def\GLOBALtitleinitsubsubsubsection {false}%
	\corecheckchapterinitialized%
	\corechecksectioninitialized%
	\corechecksubsectioninitialized%
	\coreintializetitlenumbering%
}{}{}

% Parcha sub-sub-subsecciones
\def\corepatchaftersubsubsubsection {%
	\ifthenelse{\equal{\showsectioncaptioncode}{ssssec}}{ % Reinicia código fuente
		\addtocounter{templateListings}{\value{lstlisting}}%
		\setcounter{lstlisting}{0}%
	}{}%
	\ifthenelse{\equal{\showsectioncaptioneqn}{ssssec}}{ % Reinicia ecuaciones
		\addtocounter{templateEquations}{\value{equation}}%
		\setcounter{equation}{0}%
	}{}%
	\ifthenelse{\equal{\equationrestart}{ssssec}}{ % Reinicia ecuaciones
		\addtocounter{templateEquations}{\value{equation}}%
		\setcounter{equation}{0}%
	}{}%
	\ifthenelse{\equal{\showsectioncaptionfig}{ssssec}}{ % Reinicia figuras
		\addtocounter{templateFigures}{\value{figure}}%
		\setcounter{figure}{0}%
	}{}%
	\ifthenelse{\equal{\showsectioncaptiontab}{ssssec}}{ % Reinicia tablas
		\addtocounter{templateTables}{\value{table}}%
		\setcounter{table}{0}%
	}{}%
	\global\def\GLOBALtitleinitsubsubsubsection {true}%
	\corecheckchapterinitialized%
	\corechecksectioninitialized%
	\corechecksubsectioninitialized%
	\corechecksubsubsectioninitialized%
}

% Entorno que permite desactivar los capítulos
\makeatletter
\newcommand*\coredisabledchapter{%
	\@ifstar{\coredisabledchapterstar}{\@dblarg\coredisabledchapternostar}}
\newcommand*\coredisabledchapterstar[1]{%
	\noindent\textcolor{red}{Error (chapter):} \newline#1%
	\throwwarning{La insercion de capitulos esta desactivada}%
}
\def\coredisabledchapternostar[#1]#2{%
	\noindent\textcolor{red}{Error (chapter):} #1%
	\throwwarning{La insercion de capitulos esta desactivada}%
}
\makeatother
\let\oldchapter\chapter

% Desactiva los capítulos
\newcommand{\disablechapter}{%
	\let\chapter\coredisabledchapter%
	\global\def\GLOBALtitlechapterenabled {false}%
}

% Activa los capítulos
\newcommand{\enablechapter}{%
	\let\chapter\oldchapter%
	\global\def\GLOBALtitlechapterenabled {true}%
}

% Insertar un capítulo sin número
% 	#1	Título
\newcommand{\chapteranum}[1]{%
	\corecheckchapterenabled%
	\emptyvarerr{\chapteranum}{#1}{Titulo no definido}%
	\phantomsection%
	\needspace{3\baselineskip}%
	\chapter*{#1}%
	\addcontentsline{toc}{chapter}{#1}%
	\ifthenelse{\equal{\anumsecaddtocounter}{true}}{\stepcounter{chapter}}{}%
	\changeheadertitle{#1}%
	\setcounter{section}{0}%
	\global\def\GLOBALchapternumenabled {false}%
	\coreintializetitlenumbering%
}

% Insertar un título sin número
% 	#1	Título
\newcommand{\sectionanum}[1]{%
	\emptyvarerr{\sectionanum}{#1}{Titulo no definido}%
	\phantomsection%
	\needspace{3\baselineskip}%
	\section*{#1}%
	\addcontentsline{toc}{section}{#1}%
	\ifthenelse{\equal{\anumsecaddtocounter}{true}}{\stepcounter{section}}{}%
	\changeheadertitle{#1}%
	\setcounter{subsection}{0}%
	\global\def\GLOBALsectionanumenabled {true}%
	\coreintializetitlenumbering%
}

% Insertar un título sin número y sin indexar
% 	#1	Título
\newcommand{\sectionanumnoi}[1]{%
	\emptyvarerr{\sectionanumnoi}{#1}{Titulo no definido}%
	\phantomsection%
	\needspace{3\baselineskip}%
	\section*{#1}%
	\ifthenelse{\equal{\anumsecaddtocounter}{true}}{\stepcounter{section}}{}%
	\changeheadertitle{#1}%
	\setcounter{subsection}{0}%
	\global\def\GLOBALsectionanumenabled {true}%
	\coreintializetitlenumbering%
}

% Insertar un título sin número sin cambiar el título del header
% 	#1	Título
\newcommand{\sectionanumheadless}[1]{%
	\emptyvarerr{\sectionanumnoheadless}{#1}{Titulo no definido}%
	\section*{#1}%
	\addcontentsline{toc}{section}{#1}%
	\ifthenelse{\equal{\anumsecaddtocounter}{true}}{\stepcounter{section}}{}%
	\setcounter{subsection}{0}%
	\global\def\GLOBALsectionanumenabled {true}%
	\coreintializetitlenumbering%
}

% Insertar un título sin número, sin indexar y sin cambiar el título del header
% 	#1	Título
\newcommand{\sectionanumnoiheadless}[1]{%
	\emptyvarerr{\sectionanumnoiheadless}{#1}{Titulo no definido}%
	\section*{#1}%
	\ifthenelse{\equal{\anumsecaddtocounter}{true}}{\stepcounter{section}}{}%
	\setcounter{subsection}{0}%
	\global\def\GLOBALsectionanumenabled {true}%
	\coreintializetitlenumbering%
}

% Insertar un subtítulo sin número
% 	#1	Subtítulo
\newcommand{\subsectionanum}[1]{%
	\emptyvarerr{\subsectionanum}{#1}{Subtitulo no definido}%
	\subsection*{#1}%
	\addcontentsline{toc}{subsection}{#1}
	\ifthenelse{\equal{\anumsecaddtocounter}{true}}{\stepcounter{subsection}}{}%
	\setcounter{subsubsection}{0}%
	\global\def\GLOBALsubsectionanumenabled {true}%
	\coreintializetitlenumbering%
}

% Insertar un subtítulo sin número y sin indexar
% 	#1	Subtítulo
\newcommand{\subsectionanumnoi}[1]{%
	\emptyvarerr{\subsectionanumnoi}{#1}{Subtitulo no definido}%
	\subsection*{#1}%
	\ifthenelse{\equal{\anumsecaddtocounter}{true}}{\stepcounter{subsection}}{}%
	\setcounter{subsubsection}{0}%
	\global\def\GLOBALsubsectionanumenabled {true}%
	\coreintializetitlenumbering%
}

% Insertar un sub-subtítulo sin número
% 	#1	Sub-subtítulo
\newcommand{\subsubsectionanum}[1]{%
	\emptyvarerr{\subsubsectionanum}{#1}{Sub-subtitulo no definido}%
	\subsubsection*{#1}%
	\addcontentsline{toc}{subsubsection}{#1}%
	\ifthenelse{\equal{\anumsecaddtocounter}{true}}{\stepcounter{subsubsection}}{}%
	\setcounter{subsubsubsection}{0}%
	\global\def\GLOBALsubsubsectionanumenabled {true}%
	\coreintializetitlenumbering%
}

% Insertar un sub-subtítulo sin número y sin indexar
% 	#1	Sub-subtítulo
\newcommand{\subsubsectionanumnoi}[1]{%
	\emptyvarerr{\subsubsectionanumnoi}{#1}{Sub-subtitulo no definido}%
	\subsubsection*{#1}%
	\ifthenelse{\equal{\anumsecaddtocounter}{true}}{\stepcounter{subsubsection}}{}%
	\setcounter{subsubsubsection}{0}%
	\global\def\GLOBALsubsubsectionanumenabled {true}%
	\coreintializetitlenumbering%
}

% Insertar un sub-sub-subtítulo sin número
% 	#1	Sub-sub-subtítulo
\newcommand{\subsubsubsectionanum}[1]{%
	\emptyvarerr{\subsubsubsectionanum}{#1}{Sub-sub-subtitulo no definido}%
	\subsubsubsection*{#1}%
	\addcontentsline{toc}{subsubsubsection}{#1}%
	\ifthenelse{\equal{\anumsecaddtocounter}{true}}{\stepcounter{subsubsubsection}}{}%
}

% Insertar un sub-sub-subtítulo sin número y sin indexar
% 	#1	Sub-sub-subtítulo
\newcommand{\subsubsubsectionanumnoi}[1]{%
	\emptyvarerr{\subsubsubsectionanumnoi}{#1}{Sub-sub-subtitulo no definido}%
	\subsubsection*{#1}%
	\ifthenelse{\equal{\anumsecaddtocounter}{true}}{\stepcounter{subsubsubsection}}{}%
}

% Cambia el título del encabezado (header)
%	#1	Título
\newcommand{\changeheadertitle}[1]{%
	\emptyvarerr{\changeheadertitle}{#1}{Titulo no definido}%
	\markboth{#1}{}%
}

% Elimina el título del encabezado (header)
\newcommand{\clearheadertitle}{%
	\markboth{}{}%
}

% Insertar un título en un índice, sin número de página
%	#1	Margen superior en pt. (opcional)
%	#2	Título
\newcommand{\insertindextitle}[2][]{%
	\emptyvarerr{\insertindextitle}{#2}{Titulo no definido}%
	\ifx\hfuzz#1\hfuzz%
		\addtocontents{toc}{\protect\addvspace{\indextitlemargin pt}}%
	\else%
		\addtocontents{toc}{\protect\addvspace{#1 pt}}%
	\fi%
	\addtocontents{toc}{\noindent\hyperref[swpn]{\textbf{#2}}}%
}

% Insertar un título en un índice, con número de página
%	#1	Margen superior en pt. (opcional)
%	#2	Título
\newcommand{\insertindextitlepage}[2][]{%
	\emptyvarerr{\insertindextitlepage}{#2}{Titulo no definido}%
	\ifx\hfuzz#1\hfuzz%
		\addtocontents{toc}{\protect\addvspace{\indextitlemargin pt}}%
	\else%
		\addtocontents{toc}{\protect\addvspace{#1 pt}}%
	\fi%
	\addcontentsline{toc}{section}{#2}%
}

% Crea una sección en el índice y en el header
%	#1	Margen superior en pt. (opcional)
%	#2	Título
\newcommand{\createhiddensection}[2][]{%
	\changeheadertitle{#2}%
	\insertindextitlepage[#1]{#2}%
}

% Crear un capítulo como una sección
%	#1	Título
\newcommand{\newchapter}[1]{%
	\emptyvarerr{\newchapter}{#1}{Titulo no definido}%
	\clearpage%
	\stepcounter{section}%
	\phantomsection%
	\needspace{3\baselineskip}%
	\vspace* {3cm}%
	\noindent {\huge{\textbf{\namechapter\ \thesection}}} \\%
	\vspace* {0.5cm} \\%
	\noindent {\Huge{\textbf{#1}}} \\%
	\vspace {0.5cm} \\%
	\addcontentsline{toc}{section}{\protect\numberline{\thesection}#1}%
	\markboth{#1}{}%
}

% Insertar párrafo
\newcommand{\newp}{%
	\hbadness=10000 \vspace{\baselinestretch\baselineskip} \par%
}

% Crea un salto de columna en el entorno multicol
\ifthenelse{\isundefined{\newcolumn}}{
	\newcommand{\newcolumn}{%
		\checkinsidemulticol\vfill\null\columnbreak%
	}
}{
	\renewcommand{\newcolumn}{%
		\checkinsidemulticol\vfill\null\columnbreak%
	}
}

% Salto de página en entorno multicol
\newcommand{\newpagemulticol}{%
	\newcolumn\newcolumn\clearpage%
}

% Redimensiona un ítem
% 	#1	Tamaño del nuevo objeto (En linewidth)
%	#2	Objeto a redimensionar
\newcommand{\itemresize}[2]{%
	\emptyvarerr{\itemresize}{#1}{Tamano del nuevo objeto no definido}%
	\emptyvarerr{\itemresize}{#2}{Objeto a redimensionar no definido}%
	\resizebox{#1\linewidth}{!}{#2}%
}

% Crea una página vacía sin header o footer
\newcommand{\insertemptypage}{%
	\clearpage%
	\thispagestyle{empty}%
	\null%
	\clearpage%
}

% Inserta una página vacía, aunque conserva header, footer y numeración
\newcommand{\insertblankpage}{%
	\clearpage%
	\null%
	\clearpage%
}

% Función personalizada \cleardoublepage
\def\corecleardoublepage {%
	\clearpage %
	\ifthenelse{\equal{\GLOBALtwoside}{true}}{%
		\ifodd\thepage %
		\else%
			\emptypagespredocformat%
		\fi%
	}{}%
}

% Ejecuta una función dependiendo si la página es par o impar
%	#1	Par
%	#2	Impar
\newcommand{\coretriggeronpage}[2]{%
	\ifthenelse{\isodd{\value{templatePageCounter}}}{%
		#2%
	}{%
		#1%
	}%
}

% Añade un archivo pdf con el header
%	#1	Parámetros (opcional)
%	#2	Nombre del archivo pdf
\newcommand{\includehfpdf}[2][]{%
	\includepdf[pagecommand={\pagestyle{fancy}},#1]{#2}%
}

% Añade un archivo pdf con el header
%	#1	Parámetros (opcional)
%	#2	Nombre del archivo pdf
\newcommand{\includefullhfpdf}[2][]{%
	\includepdf[pages=-,pagecommand={\pagestyle{fancy}},#1]{#2}%
}

% Inserta un texto entre comillas
%	#1 	Texto
\newcommand{\quotes}[1]{%
	\enquote*{#1}%
}

% Inserta un texto entre comillas y negrita
%	#1 	Texto
\newcommand{\quotesbf}[1]{%
	\quotes{\textbf{#1}}%
}

% Inserta un texto entre comillas e itálico
%	#1 	Texto
\newcommand{\quotesit}[1]{%
	\quotes{\textit{#1}}%
}

% Inserta un texto entre comillas y typewriter
%	#1 	Texto
\newcommand{\quotesttt}[1]{%
	\quotes{\texttt{#1}}%
}

% Inserta un texto entre comillas dobles
%	#1 	Texto
\newcommand{\doublequotes}[1]{%
	\enquote{#1}%
}

% Inserta una cita con texto elevado
%	#1	Cita
\newcommand{\scite}[1]{%
	\textsuperscript{\cite{#1}}%
}

% Fuerza la indentación
\newcommand{\forceindent}{%
	~ \\ %
	
	\vspace{-2\baselineskip}%
}

% Inserta un texto con el formato de enlace
% 	#1 	Enlace
\newcommand{\hreftext}[1]{%
	\ifthenelse{\equal{\fonturl}{same}}{%
		#1%
	}{%
	\ifthenelse{\equal{\fonturl}{tt}}{%
		\texttt{#1}%
	}{%
	\ifthenelse{\equal{\fonturl}{rm}}{%
		\textrm{#1}%
	}{%
	\ifthenelse{\equal{\fonturl}{sf}}{%
		\textsf{#1}%
	}{}}}}%
}

% Inserta un email con un link cliqueable
%	#1 	Dirección email
\newcommand{\insertemail}[1]{%
	\href{mailto:#1}{\hreftext{#1}}%
}

% Inserta un teléfono celular
%	#1	Teléfono celular
\newcommand{\insertphone}[1]{%
	\href{tel:#1}{\hreftext{#1}}%
}

% Reinicia el número de ecuaciones
\newcommand{\restartequation}{%
	\setcounter{equation}{0}%
}

% Desactiva el margen de las leyendas
\newcommand{\disablecaptionmargin}{%
	\setcaptionmargincm{0}%
}

% Reinicia el margen de las leyendas
\newcommand{\resetcaptionmargin}{%
	\setcaptionmargincm{\captionlrmargin}%
}

% Modifica el color de las tablas
%	#1	Posición inicial del inicio de colores
\newcommand{\settablerowcolors}[1]{%
	\emptyvarerr{\settablerowcolors}{#1}{Posicion de fila no definida}%
	\ifthenelse{\equal{\GLOBALtablerowcolorswitch}{false}}{ % Usa colores normales
		\ifthenelse{\equal{\tablerowfirstcolor}{none}}{%
			\ifthenelse{\equal{\tablerowsecondcolor}{none}}{%
				\rowcolors{#1}{}{}%
			}{%
				\rowcolors{#1}{\tablerowsecondcolor}{}%
			}%
		}{%
			\ifthenelse{\equal{\tablerowsecondcolor}{none}}{%
				\rowcolors{#1}{}{\tablerowfirstcolor}%
			}{%
				\rowcolors{#1}{\tablerowsecondcolor}{\tablerowfirstcolor}%
			}%
		}%
	}{ % Usa colores alternados
		\ifthenelse{\equal{\tablerowfirstcolor}{none}}{%
			\ifthenelse{\equal{\tablerowsecondcolor}{none}}{%
				\rowcolors{#1}{}{}%
			}{%
				\rowcolors{#1}{}{\tablerowsecondcolor}%
			}%
		}{%
			\ifthenelse{\equal{\tablerowsecondcolor}{none}}{%
				\rowcolors{#1}{\tablerowfirstcolor}{}%
			}{%
				\rowcolors{#1}{\tablerowfirstcolor}{\tablerowsecondcolor}%
			}%
		}%
	}%
	
	% Actualiza el índice previo
	\global\def\GLOBALtablerowcolorindex {#1}%
}

% Alterna los colores de las tablas a la última ejecución
\newcommand{\settablerowcolorslast}{
	\ifthenelse{\equal{\GLOBALtablerowcolorswitch}{false}}{ % Usa colores normales
		\ifthenelse{\equal{\tablerowfirstcolor}{none}}{%
			\ifthenelse{\equal{\tablerowsecondcolor}{none}}{%
				\rowcolors{\GLOBALtablerowcolorindex}{}{}%
			}{%
				\rowcolors{\GLOBALtablerowcolorindex}{\tablerowsecondcolor}{}%
			}%
		}{%
			\ifthenelse{\equal{\tablerowsecondcolor}{none}}{%
				\rowcolors{\GLOBALtablerowcolorindex}{}{\tablerowfirstcolor}%
			}{%
				\rowcolors{\GLOBALtablerowcolorindex}{\tablerowsecondcolor}{\tablerowfirstcolor}%
			}%
		}%
	}{ % Usa colores alternados
		\ifthenelse{\equal{\tablerowfirstcolor}{none}}{%
			\ifthenelse{\equal{\tablerowsecondcolor}{none}}{%
				\rowcolors{\GLOBALtablerowcolorindex}{}{}%
			}{%
				\rowcolors{\GLOBALtablerowcolorindex}{}{\tablerowsecondcolor}%
			}%
		}{%
			\ifthenelse{\equal{\tablerowsecondcolor}{none}}{%
				\rowcolors{\GLOBALtablerowcolorindex}{\tablerowfirstcolor}{}%
			}{%
				\rowcolors{\GLOBALtablerowcolorindex}{\tablerowfirstcolor}{\tablerowsecondcolor}%
			}%
		}%
	}%
}

% Activa el color de las filas de las tablas
%	#1	Posición inicial del inicio de colores
\newcommand{\enabletablerowcolor}[1][]{%
	\ifx\hfuzz#1\hfuzz%
		\settablerowcolors{2}%
	\else%
		\settablerowcolors{#1}%
	\fi%
}

% Desactiva el color de las filas de las tablas
\newcommand{\disabletablerowcolor}{\rowcolors{2}{}{}}

% Alterna los colores de las filas de las tablas
\newcommand{\switchtablerowcolors}{%
	\ifthenelse{\equal{\GLOBALtablerowcolorswitch}{false}}{%
		\global\def\GLOBALtablerowcolorswitch {true}%
	}{%
		\global\def\GLOBALtablerowcolorswitch {false}%
	}%
	\settablerowcolorslast%
}

% Actualiza el padding de las celdas de las tablas
%	#1	Padding horizontal (em)
%	#2	Padding vertical (em)
\newcommand{\settablecellpadding}[2]{%
	\emptyvarerr{\settablecellpadding}{#1}{Padding horizontal no definido}%
	\emptyvarerr{\settablecellpadding}{#2}{Padding vertical no definido}%
	\setlength{\tabcolsep}{#1 em} % Horizontal
	\def\arraystretch {#2} % Vertical
}

% Resetea el padding de las celdas de las tablas
\newcommand{\resettablecellpadding}{%
	\settablecellpadding{\tablepaddingh}{\tablepaddingv}%
}

% Cambia el tamaño de la página
%	#1	Orientacion de la página, puede ser 0 o 90. Por defecto es cero
%	#2	Ancho de la página (cm)
%	#3	Alto de la página (cm)
\newcommand{\changepagesize}[3][]{%
	% \emptyvarerr{\changepagesize}{#1}{Orientacion de la pagina}
	\emptyvarerr{\changepagesize}{#2}{Ancho de la pagina no definida}%
	\emptyvarerr{\changepagesize}{#3}{Altura de la pagina no definida}%
	\ifthenelse{\equal{\compilertype}{lualatex}}{%
		\throwwarning{Funcion no valida en compilador lualatex}%
	}{%
		\clearpage%
		\ifthenelse{\equal{#1}{}}{%
			\newgeometry{left=\pagemarginleft cm, top=\pagemargintop cm, right=\pagemarginright cm, bottom=\pagemarginbottom cm, paperwidth=#2 cm, paperheight=#3 cm}%
		}{%
		\ifthenelse{\equal{#1}{0}}{%
			\newgeometry{left=\pagemarginleft cm, top=\pagemargintop cm, right=\pagemarginright cm, bottom=\pagemarginbottom cm, paperwidth=#2 cm, paperheight=#3 cm}%
		}{%
		\ifthenelse{\equal{#1}{90}}{%
			\newgeometry{left=\pagemarginleft cm, top=\pagemargintop cm, right=\pagemarginright cm, bottom=\pagemarginbottom cm, paperwidth=#3 cm, paperheight=#2 cm}%
		}{%
			\throwbadconfig{Orientacion de pagina no valido}{\changepagesize}{0,90}}}%
		}%
	}%
}

% Ofrece diferentes formatos de pagina
% https://www.prepressure.com/library/paper-size
%	#1	Indica la rotación, puede ser 0 o 90
%	#2	Formato de la pagina
\newcommand{\changepagesizeformat}[2][]{%
	\emptyvarerr{\changepagesizeformat}{#2}{Formato de pagina no definido}%
	\ifthenelse{\equal{#2}{4A0}}{%
		\changepagesize[#1]{168.2}{237.8}%
	}{%
	\ifthenelse{\equal{#2}{2A0}}{%
		\changepagesize[#1]{118.9}{168.2}%
	}{%
	\ifthenelse{\equal{#2}{A0}}{%
		\changepagesize[#1]{84.1}{118.9}%
	}{%
	\ifthenelse{\equal{#2}{A1}}{%
		\changepagesize[#1]{59.4}{84.1}%
	}{%
	\ifthenelse{\equal{#2}{A2}}{%
		\changepagesize[#1]{42.0}{84.1}%
	}{%
	\ifthenelse{\equal{#2}{A3}}{%
		\changepagesize[#1]{29.7}{42.0}%
	}{%
	\ifthenelse{\equal{#2}{A4}}{%
		\changepagesize[#1]{21.0}{29.7}%
	}{%
	\ifthenelse{\equal{#2}{A5}}{%
		\changepagesize[#1]{14.8}{21.0}%
	}{%
	\ifthenelse{\equal{#2}{A6}}{%
		\changepagesize[#1]{10.5}{14.8}%
	}{%
	\ifthenelse{\equal{#2}{letter}}{%
		\changepagesize[#1]{21.59}{27.94}%
	}{%
	\ifthenelse{\equal{#2}{legal}}{%
		\changepagesize[#1]{21.59}{35.6}%
	}{%
	\ifthenelse{\equal{#2}{foolscap}}{%
		\changepagesize[#1]{20.3}{33.0}%
	}{%
	\ifthenelse{\equal{#2}{executive}}{%
		\changepagesize[#1]{18.41}{26.67}%
	}{%
	\ifthenelse{\equal{#2}{ledger}}{%
		\changepagesize[#1]{27.94}{43.18}%
	}{%
	\ifthenelse{\equal{#2}{tabloid}}{%
		\changepagesize[#1]{43.18}{27.94}%
	}{%
	\ifthenelse{\equal{#2}{ANSIC}}{%
		\changepagesize[#1]{55.9}{43.2}%
	}{%
	\ifthenelse{\equal{#2}{ANSID}}{%
		\changepagesize[#1]{86.4}{55.9}%
	}{%
	\ifthenelse{\equal{#2}{ANSIE}}{%
		\changepagesize[#1]{111.8}{86.4}%
	}{%
	\ifthenelse{\equal{#2}{B0}}{%
		\changepagesize[#1]{100}{141.4}%
	}{%
	\ifthenelse{\equal{#2}{B1}}{%
		\changepagesize[#1]{70.7}{100}%
	}{%
	\ifthenelse{\equal{#2}{B2}}{%
		\changepagesize[#1]{50}{70.7}%
	}{%
	\ifthenelse{\equal{#2}{B3}}{%
		\changepagesize[#1]{35.3}{50}%
	}{%
	\ifthenelse{\equal{#2}{B4}}{%
		\changepagesize[#1]{25}{35.3}%
	}{%
	\ifthenelse{\equal{#2}{B5}}{%
		\changepagesize[#1]{17.6}{25}%
	}{%
	\ifthenelse{\equal{#2}{B6}}{%
		\changepagesize[#1]{12.5}{17.6}%
	}{%
		\throwbadconfig{Estilo de pagina no valido}{\changepagesizeformat}{4A0,2A0,A0,A1,A2,A3,A4,A5,A6,letter,legal,foolscap,executive,ledger,tabloid,ANSIC,ANSID,ANSIE,B0,B1,B2,B3,B4,B5,B6}}}}}}}}}}}}}}}}}}}}}}}}}%
	}%
}

% Crea variables para guardar configuraciones de columnas
\global\def\GLOBALtwocolumnap {l}
\global\def\GLOBALtwocolumnav {t}
\global\def\GLOBALtwocolumnbp {l}
\global\def\GLOBALtwocolumnbv {t}
\global\def\GLOBALthreecolumnap {l}
\global\def\GLOBALthreecolumnav {t}
\global\def\GLOBALthreecolumnbp {l}
\global\def\GLOBALthreecolumnbv {t}
\global\def\GLOBALthreecolumncp {l}
\global\def\GLOBALthreecolumncv {t}

% Chequea posición columna
%	#1	Valor posición (c, t, b)
\newcommand{\corecheckcolumnvvalue}[1]{%
	\ifthenelse{\equal{#1}{c}}{}{%
	\ifthenelse{\equal{#1}{t}}{}{%
	\ifthenelse{\equal{#1}{b}}{}{%
		\errmessage{LaTeX Warning: Posicion vertical columna invalido, valores esperados: c,t,b}%
	}}}%
}

% Chequea alineación columna
%	#1	Valor alineación (c, l, r)
\newcommand{\corecheckcolumnpvalue}[1]{%
	\ifthenelse{\equal{#1}{c}}{}{%
	\ifthenelse{\equal{#1}{l}}{}{%
	\ifthenelse{\equal{#1}{r}}{}{%
		\errmessage{LaTeX Warning: Alineacion columna invalida, valores esperados: c,l,r}%
	}}}%
}

% Configura las columnas dobles
%	#1	Posición vertical columna izquierda (c, t, b)
%	#2	Alineación horizontal columna izquierda (c, l, r)
%	#3	Posición vertical columna derecha (c, t, b)
%	#4	Alineación horizontal columna derecha (c, l, r)
\newcommand{\createtwocolumncfg}[4]{%
	\corecheckcolumnvvalue{#1}%
	\corecheckcolumnpvalue{#2}%
	\corecheckcolumnvvalue{#3}%
	\corecheckcolumnpvalue{#4}%
	\global\def\GLOBALtwocolumnav {#1}%
	\global\def\GLOBALtwocolumnap {#2}%
	\global\def\GLOBALtwocolumnbv {#3}%
	\global\def\GLOBALtwocolumnbp {#4}%
}

% Restaura la configuración de dos columnas
\newcommand{\resettwocolumncfg}{%
	\createtwocolumncfg{t}{l}{t}{l}%
}

% Configura las columnas triples
%	#1	Posición vertical columna izquierda (c, t, b)
%	#2	Alineación horizontal columna izquierda (c, l, r)
%	#3	Posición vertical columna central (c, t, b)
%	#4	Alineación horizontal columna central (c, l, r)
%	#5	Posición vertical columna derecha (c, t, b)
%	#6	Alineación horizontal columna derecha (c, l, r)
\newcommand{\createthreecolumncfg}[6]{%
	\corecheckcolumnvvalue{#1}%
	\corecheckcolumnpvalue{#2}%
	\corecheckcolumnvvalue{#3}%
	\corecheckcolumnpvalue{#4}%
	\corecheckcolumnvvalue{#5}%
	\corecheckcolumnpvalue{#6}%
	\global\def\GLOBALthreecolumnav {#1}%
	\global\def\GLOBALthreecolumnap {#2}%
	\global\def\GLOBALthreecolumnbv {#3}%
	\global\def\GLOBALthreecolumnbp {#4}%
	\global\def\GLOBALthreecolumncv {#3}%
	\global\def\GLOBALthreecolumncp {#4}%
}

% Restaura la configuración de tres columnas
\newcommand{\resetthreecolumncfg}{%
	\createthreecolumncfg{t}{l}{t}{l}{t}{l}%
}

% Crea dos columnas con contenido
%	#1	Altura de las columnas (opcional)
%	#2 	Dimensión de la columna izquierda (En linewidth)
%	#3	Dimensión de la columna derecha (En linewidth)
%	#4	Distancia entre columnas (En cm)
%	#5 	Contenido de la columna izquierda
%	#6	Contenido de la columna derecha
\newcommand{\createtwocolumn}[6][]{%
	\setcaptionmargincm{0}%
	\begin{samepage}%
	\begin{flushleft}%
		\vspace{-0.5\baselineskip}%
		\begin{minipage}{1\linewidth}%
			\begin{minipage}[t][#1][\GLOBALtwocolumnav]{#2\linewidth}%
				\ifthenelse{\equal{\GLOBALtwocolumnap}{c}}{%
					\begin{center}#5\end{center}%
				}{%
				\ifthenelse{\equal{\GLOBALtwocolumnap}{l}}{%
					\begin{raggedright}#5\end{raggedright}%
				}{%
				\ifthenelse{\equal{\GLOBALtwocolumnap}{r}}{%
					\hfill\begin{raggedleft}#5\end{raggedleft}%
				}{%
					\errmessage{LaTeX Warning: Alineacion columna izquierda incorrecta, valores esperados: c,l,r}%
				}}}%
			\end{minipage}%
			\hspace{#4 cm}%
			\begin{minipage}[t][#1][\GLOBALtwocolumnbv]{#3\linewidth}%
				\ifthenelse{\equal{\GLOBALtwocolumnbp}{c}}{%
					\begin{center}#6\end{center}%
				}{%
				\ifthenelse{\equal{\GLOBALtwocolumnbp}{l}}{%
					\begin{raggedright}#6\end{raggedright}%
				}{%
				\ifthenelse{\equal{\GLOBALtwocolumnbp}{r}}{%
					\hfill\begin{raggedleft}#6\end{raggedleft}%
				}{%
					\errmessage{LaTeX Warning: Alineacion columna derecha incorrecta, valores esperados: c,l,r}%
				}}}%
			\end{minipage}
		\end{minipage}
	\end{flushleft}
	~ \vspace{-0.5\baselineskip}%
	\end{samepage}
	\setcaptionmargincm{\captionlrmargin}%
}

% Crea dos columnas idénticas
%	#1	Altura de las columnas (opcional)
%	#2 	Contenido de la columna izquierda
%	#3	Contenido de la columna derecha
\newcommand{\createhalfcolumn}[3][]{%
	\createtwocolumn[#1]{0.5}{0.5}{0}{#2}{#3}%
}

% Crea tres columnas con contenido
%	#1	Altura de las columnas (opcional)
%	#2 	Dimensión de la columna izquierda (En linewidth)
%	#3	Dimensión de la columna central (En linewidth)
%	#4	Dimensión de la columna derecha (En linewidth)
%	#5	Distancia entre columna 1-2 (En cm)
%	#6	Distancia entre columna 2-3 (En cm)
%	#7 	Contenido de la columna izquierda
%	#8	Contenido de la columna central
%	#9	Contenido de la columna derecha
\newcommand{\createthreecolumn}[9][]{%
	\setcaptionmargincm{0}%
	\begin{samepage}%
	\begin{flushleft}%
		\vspace{-0.5\baselineskip}%
		\begin{minipage}{1\linewidth}%
			\begin{minipage}[t][#1][\GLOBALthreecolumnav]{#2\linewidth}%
				\ifthenelse{\equal{\GLOBALthreecolumnap}{c}}{%
					\begin{center}#7\end{center}%
				}{%
				\ifthenelse{\equal{\GLOBALthreecolumnap}{l}}{%
					\begin{raggedright}#7\end{raggedright}%
				}{%
				\ifthenelse{\equal{\GLOBALthreecolumnap}{r}}{%
					\hfill\begin{raggedleft}#7\end{raggedleft}%
				}{%
					\errmessage{LaTeX Warning: Alineacion columna izquierda incorrecta, valores esperados: c,l,r}%
				}}}%
			\end{minipage}
			\hspace{#5 cm}%
			\begin{minipage}[t][#1][\GLOBALthreecolumnbv]{#3\linewidth}%
				\ifthenelse{\equal{\GLOBALthreecolumnbp}{c}}{%
					\begin{center}#8\end{center}%
				}{%
				\ifthenelse{\equal{\GLOBALthreecolumnbp}{l}}{%
					\begin{raggedright}#8\end{raggedright}%
				}{%
				\ifthenelse{\equal{\GLOBALthreecolumnbp}{r}}{%
					\hfill\begin{raggedleft}#8\end{raggedleft}%
				}{%
					\errmessage{LaTeX Warning: Alineacion columna central incorrecta, valores esperados: c,l,r}%
				}}}%
			\end{minipage}
			\hspace{#6 cm}%
			\begin{minipage}[t][#1][\GLOBALthreecolumncv]{#4\linewidth}%
				\ifthenelse{\equal{\GLOBALthreecolumncp}{c}}{%
					\begin{center}#9\end{center}%
				}{%
				\ifthenelse{\equal{\GLOBALthreecolumncp}{l}}{%
					\begin{raggedright}#9\end{raggedright}%
				}{%
				\ifthenelse{\equal{\GLOBALthreecolumncp}{r}}{%
					\hfill\begin{raggedleft}#9\end{raggedleft}%
				}{%
					\errmessage{LaTeX Warning: Alineacion columna derecha incorrecta, valores esperados: c,l,r}%
				}}}%
			\end{minipage}
		\end{minipage}
	\end{flushleft}
	~ \vspace{-0.5\baselineskip}%
	\end{samepage}
	\setcaptionmargincm{\captionlrmargin}%
}

% Crea tres columnas idénticas
%	#1 	Contenido de la columna izquierda
%	#2	Contenido de la columna central
%	#3	Contenido de la columna derecha
\newcommand{\createthirdcolumn}[3]{%
	\createthreecolumn{0.3333}{0.3333}{0.3333}{0}{0}{#1}{#2}{#3}%
}

% Crea una sección de referencias solo para bibtex
\newenvironment{references}{%
	\ifthenelse{\equal{\stylecitereferences}{bibtex}}{ % Verifica configuraciones
	}{%
		\throwerror{\references}{Solo se puede usar entorno references con estilo citas \noexpand\stylecitereferences=bibtex}%
	}%
	\phantomsection%
	\addcontentsline{toc}{chapter}{\namereferences}%
	\begin{thebibliography}{} % Inicia la bibliografía
		\ifthenelse{\equal{\bibtextextalign}{justify}}{ % Formato ajuste de línea
		}{%
		\ifthenelse{\equal{\bibtextextalign}{left}}{%
			\raggedright%
		}{%
		\ifthenelse{\equal{\bibtextextalign}{right}}{%
			\raggedleft%
		}{%
		\ifthenelse{\equal{\bibtextextalign}{center}}{%
			\centering%
		}{%
			\throwbadconfig{Ajuste de linea referencias bibtex desconocido}{\bibtextextalign}{justified,left,right,center}}}}%
		}%
	}%
	{%
	\end{thebibliography}
}

% Crea un entorno para definir el tamaño de bloque
%	#1	Tamaño de fuente en pt
\newenvironment{fontsizeblock}[1][\documentfontsize]{%
	\changefontsizes{#1 pt}%
}{%
	\changefontsizes{\documentfontsize pt}%
}

% Crea una sección de anexos
\newenvironment{appendixd}{%
	\appendix%
	\global\def\GLOBALenvappendix {true}%
	\global\def\GLOBALtitlerequirechapter {true}%
	\begingroup%
	\phantomsection%
	\changeheadertitle{\nameltappendixsection} % Cambia el nombre del header
	% Define formato números para appendix
	\global\def\GLOBALformatnumchapter {\formatnumapchapter}%
	\global\def\GLOBALformatnumsection {\formatnumapsection}%
	\global\def\GLOBALformatnumssection {\formatnumapssection}%
	\global\def\GLOBALformatnumsssection {\formatnumapsssection}%
	\global\def\GLOBALformatnumssssection {\formatnumapssssection}%
	% Define estado de numeración
	\global\def\GLOBALtitleinitchapter {false}%
	\global\def\GLOBALtitleinitsection {false}%
	\global\def\GLOBALtitleinitsubsection {false}%
	\global\def\GLOBALtitleinitsubsubsection {false}%
	\global\def\GLOBALtitleinitsubsubsubsection {false}%
	\bookmarksetup{%
		numbered={true},
		openlevel={\thetemplateBookmarksLevelPrev}
	}%
	\appendixtitleon%
	\appendixtitletocon%
	\bookmarksetupnext{level=part}%
	\begin{appendices} % Crea la sección
		\ifthenelse{\equal{\showappendixsecindex}{true}}{}{%
			\pdfbookmark{\nameappendixsection}{appendix} % Si false
		}%
		% \setcounter{secnumdepth}{4}
		% \setcounter{tocdepth}{4}
		\ifthenelse{\equal{\appendixindepobjnum}{true}}{%
			\counterwithin{equation}{chapter}
			\counterwithin{figure}{chapter}
			\counterwithin{lstlisting}{chapter}
			\counterwithin{table}{chapter}
		}{}%
	}{%
	\end{appendices}
	% Restablece formato de números
	\global\def\GLOBALformatnumchapter {\formatnumchapter}%
	\global\def\GLOBALformatnumsection {\formatnumsection}%
	\global\def\GLOBALformatnumssection {\formatnumssection}%
	\global\def\GLOBALformatnumsssection {\formatnumsssection}%
	\global\def\GLOBALformatnumssssection {\formatnumssssection}%
	% Reestablece estado de numeración
	\global\def\GLOBALtitleinitchapter {false}%
	\global\def\GLOBALtitleinitsection {false}%
	\global\def\GLOBALtitleinitsubsection {false}%
	\global\def\GLOBALtitleinitsubsubsection {false}%
	\global\def\GLOBALtitleinitsubsubsubsection {false}%
	\bookmarksetupnext{level={\thetemplateBookmarksLevelPrev}} % Restablece índice marcador
	\bookmarksetup{%
		numbered={\cfgpdfsecnumbookmarks},
		openlevel={\cfgbookmarksopenlevel}
	}%
	\endgroup%
	\global\def\GLOBALenvappendix {false}%
	\global\def\GLOBALtitlerequirechapter {true}%
}

% Entorno simple de apéndices
\newenvironment{appendixs}{%
	\appendix%
	\global\def\GLOBALenvappendix {true}%
	\global\def\GLOBALtitlerequirechapter {false}%
	\begingroup%
	\chapteranum{\nameappendixsection}%
	% Define etiqueta secciones
	\global\def\GLOBALtitlepresectionstr {\nameltappendixsection~}%
	\changeheadertitle{\nameltappendixsection} % Cambia el nombre del header
	% Define formato números para appendix
	\global\def\GLOBALformatnumchapter {\formatnumapchapter}%
	\global\def\GLOBALformatnumsection {\formatnumapchapter}%
	\global\def\GLOBALformatnumssection {\formatnumapsection}%
	\global\def\GLOBALformatnumsssection {\formatnumapssection}%
	\global\def\GLOBALformatnumssssection {\formatnumapsssection}%
	% Define estado de numeración
	\global\def\GLOBALtitleinitchapter {false}%
	\global\def\GLOBALtitleinitsection {false}%
	\global\def\GLOBALtitleinitsubsection {false}%
	\global\def\GLOBALtitleinitsubsubsection {false}%
	\global\def\GLOBALtitleinitsubsubsubsection {false}%
	% Otras configuraciones
	\disablechapter%
	\ifthenelse{\equal{\appendixindepobjnum}{true}}{%
		\counterwithin{equation}{section}
		\counterwithin{figure}{section}
		\counterwithin{lstlisting}{section}
		\counterwithin{table}{section}
	}{}%
	}{%
	% Restablece formato de números
	\global\def\GLOBALformatnumchapter {\formatnumchapter}%
	\global\def\GLOBALformatnumsection {\formatnumsection}%
	\global\def\GLOBALformatnumssection {\formatnumssection}%
	\global\def\GLOBALformatnumsssection {\formatnumsssection}%
	\global\def\GLOBALformatnumssssection {\formatnumssssection}%
	% Reestablece estado de numeración
	\global\def\GLOBALtitleinitchapter {false}%
	\global\def\GLOBALtitleinitsection {false}%
	\global\def\GLOBALtitleinitsubsection {false}%
	\global\def\GLOBALtitleinitsubsubsection {false}%
	\global\def\GLOBALtitleinitsubsubsubsection {false}%
	% Resetea etiqueta secciones
	\global\def\GLOBALtitlepresectionstr {}%
	\enablechapter%
	\endgroup%
	\global\def\GLOBALenvappendix {false}%
}

% Entorno capítulos apéndices con título
\newenvironment{appendixdtitle}[1][style1]{
	\chapter*{\nameappendixsection}%
	\let\clearpage\relax%
	\vspace{-1.75cm}%
	% Configura el tipo de capítulo
	\ifthenelse{\equal{#1}{style1}}{ % Default
	}{%
	\ifthenelse{\equal{#1}{style2}}{%
		\titleformat{\chapter}[hang]{\huge\bfseries}{\thechapter.\hspace{20pt}}{0pt}{\huge\bfseries}%
	}{%
	\ifthenelse{\equal{#1}{style3}}{%
		\titleformat{\chapter}[hang]{\huge\bfseries}{\nameltappendixsection\ \thechapter.\hspace{20pt}}{0pt}{\huge\bfseries}%
	}{%
	\ifthenelse{\equal{#1}{style4}}{%
		\titleformat{\chapter}[hang]{\LARGE\bfseries}{\nameltappendixsection\ \thechapter.\hspace{20pt}}{0pt}{\LARGE\bfseries}%
	}{%
		\throwerror{appendixdtitle}{Estilo capitulo apendice incorrecto. Estilos validos style1..style4}%
	}{}}}}%
	\begin{appendixd}%
		}{%
	\end{appendixd}%
}

% Inicia código fuente con parámetros
%	#1	Label (opcional)
%	#2	Estilo de código
%	#3	Parámetros
%	#4	Caption
\newcommand{\coreinitsourcecodep}[4]{%
	\emptyvarerr{\coreinitsourcecodep}{#2}{Estilo de codigo no definido}%
	\checkvalidsourcecodestyle{#2}%
	\ifthenelse{\equal{\showlinenumbers}{true}}{%
		\rightlinenumbers}{%
	}%
	\lstset{
		backgroundcolor=\color{\sourcecodebgcolor}
	}%
	\ifthenelse{\equal{\codecaptiontop}{true}}{%
		\ifx\hfuzz#4\hfuzz%
			\ifx\hfuzz#3\hfuzz%
				\lstset{
					escapeinside={(*@}{@*)},
					style=#2
				}%
			\else%
				\lstset{
					escapeinside={(*@}{@*)},
					style=#2,
					#3
				}%
			\fi%
		\else%
			\ifx\hfuzz#3\hfuzz%
				\lstset{
					caption={#4 #1},
					captionpos=t,
					escapeinside={(*@}{@*)},
					style=#2
				}%
			\else%
				\lstset{
					caption={#4 #1},
					captionpos=t,
					escapeinside={(*@}{@*)},
					style=#2,
					#3
				}%
			\fi%
		\fi%
	}{%
		\ifx\hfuzz#4\hfuzz%
			\ifx\hfuzz#3\hfuzz%
				\lstset{
					escapeinside={(*@}{@*)},
					style=#2
				}%
			\else%
				\lstset{
					escapeinside={(*@}{@*)},
					style=#2,
					#3
				}%
			\fi%
		\else%
			\ifx\hfuzz#3\hfuzz%
				\lstset{
					caption={#4 #1},
					captionpos=b,
					style=#2
				}%
			\else%
				\lstset{
					caption={#4 #1},
					captionpos=b,
					escapeinside={(*@}{@*)},
					style=#2,
					#3
				}%
			\fi%
		\fi%
	}%
}

% Inserta código fuente con parámetros
%	#1	Label (opcional)
%	#2	Estilo de código
%	#3	Parámetros
%	#4	Caption
\lstnewenvironment{sourcecodep}[4][]{%
	\coreinitsourcecodep{#1}{#2}{#3}{#4}%
}{%
	\ifthenelse{\equal{\showlinenumbers}{true}}{%
		\leftlinenumbers}{%
	}%
}

% Importa código fuente desde un archivo con parámetros
%	#1	Label (opcional)
%	#2	Estilo de código
%	#3	Parámetros
%	#4	Archivo de código fuente
%	#5	Caption
\newcommand{\importsourcecodep}[5][]{%
	\coreinitsourcecodep{#1}{#2}{#3}{#5}%
	\inputlisting{#4}%
	\ifthenelse{\equal{\showlinenumbers}{true}}{%
		\leftlinenumbers}{%
	}%
}

% Inicia código fuente sin parámetros
%	#1	Label (opcional)
%	#2	Estilo de código
%	#3	Caption
\newcommand{\coreinitsourcecode}[3]{%
	\emptyvarerr{\coreinitsourcecode}{#2}{Estilo de codigo no definido}%
	\checkvalidsourcecodestyle{#2}%
	\ifthenelse{\equal{\showlinenumbers}{true}}{%
		\rightlinenumbers}{%
	}%
	\lstset{
		backgroundcolor=\color{\sourcecodebgcolor}
	}%
	\ifthenelse{\equal{\codecaptiontop}{true}}{
		\ifx\hfuzz#3\hfuzz%
			\lstset{
				escapeinside={(*@}{@*)},
				style=#2
			}%
		\else%
			\lstset{
				escapeinside={(*@}{@*)},
				caption={#3 #1},
				captionpos=t,
				style=#2
			}%
		\fi%
	}{%
		\ifx\hfuzz#3\hfuzz%
			\lstset{
				escapeinside={(*@}{@*)},
				style=#2
			}%
		\else%
			\lstset{
				escapeinside={(*@}{@*)},
				caption={#3 #1},
				captionpos=b,
				style=#2
			}%
		\fi%
	}%
}

% Inserta código fuente sin parámetros
%	#1	Label (opcional)
%	#2	Estilo de código
%	#3	Caption
\lstnewenvironment{sourcecode}[3][]{%
	\coreinitsourcecode{#1}{#2}{#3}%
}{%
	\ifthenelse{\equal{\showlinenumbers}{true}}{%
		\leftlinenumbers}{%
	}%
}

% Importa código fuente desde un archivo sin parámetros
%	#1	Label (opcional)
%	#2	Estilo de código
%	#3	Archivo de código fuente
%	#4	Caption
\newcommand{\importsourcecode}[4][]{%
	\coreinitsourcecode{#1}{#2}{#4}%
	\lstinputlisting{#3}%
	\ifthenelse{\equal{\showlinenumbers}{true}}{%
		\leftlinenumbers}{%
	}%
}

% Itemize en negrita
%	#1	Parámetros opcionales
\newenvironment{itemizebf}[1][]{%
	\begin{itemize}[font=\bfseries,#1]%
	}{%
	\end{itemize}
}

% Enumerate en negrita
%	#1	Parámetros opcionales
\newenvironment{enumeratebf}[1][]{%
	\begin{enumerate}[font=\bfseries,#1]%
	}{%
	\end{enumerate}
}

% Crea una sección de resumen
%	#1	Tabla resumen
%	#2	Título de la tesis
%	#3	Título de la sección
%	#4	Etiqueta del marcador del pdf
\newenvironment{abstractenv}[4]{%
	\clearpage%
	\ifthenelse{\equal{\GLOBALtwoside}{true}}{%
		\coretriggeronpage{\emptypagespredocformat}{}%
	}{}%
	\emptyvarerr{\abstractenv}{#1}{Tabla resumen no definida}%
	\emptyvarerr{\abstractenv}{#2}{Titulo tesis no definido}%
	\emptyvarerr{\abstractenv}{#3}{Titulo seccion no definida}%
	\emptyvarerr{\abstractenv}{#4}{Etiqueta marcador del pdf}%
	% Añade a los marcadores
	\ifthenelse{\equal{\addabstracttobookmarks}{true}}{%
		\phantomsection%
		\pdfbookmark{#3}{#4}}{%
	}%
	% Inserta la tabla resumen
	\ifthenelse{\equal{#1}{}}{%
		\vspace*{0\baselineskip}%
	}{%
		\begin{flushright}%
			\small%
			#1%
		\end{flushright}%
		\vspace*{0.5\baselineskip}%
	}%
	% Título
	\begin{center}%
		\textcolor{\sectioncolor}{\MakeUppercase{\textbf{#2}}}%
	\end{center} \newp%
	\ifthenelse{\equal{#1}{}}{%
		\vspace{-0.5\baselineskip}%
	}{%
		\vspace{-\baselineskip}%
	}%
	}{%
	\vfill\null%
}

% Llama al entorno de resumen
\newenvironment{abstractd}{%
	\ifthenelse{\isundefined{\abstracttable}}{%
		\def\abstracttable {}}{%
	}%
	\begin{abstractenv}{\abstracttable}{\documenttitle}{\nameabstract}{abstractbookmark}%
	}{%
	\end{abstractenv}
}



% Crea una sección de dedicatoria
\newenvironment{dedicatory}{%
	\clearpage%
	\ifthenelse{\equal{\GLOBALtwoside}{true}}{%
		\coretriggeronpage{\emptypagespredocformat}{}%
	}{}%
	\null%
	\phantomsection%
	% \ifthenelse{\equal{\adddedictobookmarks}{true}}{
	% 	\pdfbookmark{\nameadedic}{contents}}{
	% }
	\vspace{\stretch{1}}%
	\begin{flushright}%
		\itshape}{%
	\end{flushright}%
	\vspace{\stretch{2}}%
	\null%
}

% Crea una sección de agradecimientos
\newenvironment{acknowledgments}{%
	\clearpage%
	\ifthenelse{\equal{\GLOBALtwoside}{true}}{%
		\coretriggeronpage{\emptypagespredocformat}{}%
	}{}%
	\section*{\nameagradec}%
	\ifthenelse{\equal{\addagradectobookmarks}{true}}{%
		\phantomsection%
		\pdfbookmark{\nameagradec}{acknowledgments}}{%
	}%
	\forceindent%
	}{%
}

% Crea una sección de imágenes múltiples
%	#1	Label (opcional)
%	#2	Caption
\newenvironment{images}[2][]{%
	% Modifica globales
	\def\envimageslabelvar {#1}%
	\def\envimagescaptioncf {false}%
	\def\envimagescaptionvar {#2}%
	\global\def\GLOBALenvimageadded {false}%
	\global\def\GLOBALenvimageinitialized {true}%
	% Configura caption y márgenes
	\corevspacevarcm{\marginimagetop}%
	\setcaptionmargincm{\captionmarginmultimg} % Eso es para los wrapfig
	% Inicia la figura
	\begin{samepage}%
	\begin{figure}[H] \centering%
		\ifthenelse{\equal{\GLOBALenvimagecf}{true}}{%
			\ContinuedFloat%
			\global\def\GLOBALenvimagecf {false}%
			\def\envimagescaptioncf {true}%
		}{}%
		\corevspacevarcm{\marginimagemulttop}%
		}{%
		\setcaptionmargincm{\captionlrmargin}%
		\ifthenelse{\equal{\envimagescaptionvar}{}}{%
			\corevspacevarcm{\captionlessmarginimage}%
		}{%
			\corevspacevarcm{\captionmarginimages}%
			\ifthenelse{\equal{\envimagescaptioncf}{true}}{%
				\caption[]{\envimagescaptionvar\envimageslabelvar}%
			}{%
				\caption{\envimagescaptionvar\envimageslabelvar}%
			}%
		}%
	\end{figure}%
	% Restablece caption y márgenes
	\setcaptionmargincm{\captionlrmargin}%
	\corevspacevarcm{\marginimagebottom}%
	\end{samepage}
	% Restablece globales
	\global\def\GLOBALenvimageinitialized {false}%
}

% Crea una sección de imágenes múltiples completa dentro de un multicol
%	#1	Label (opcional)
%	#2	Posición de la imagen, "bottom", "top"
%	#3	Caption
\newenvironment{imagesmc}[3][]{%
	% Modifica globales
	\def\envimageslabelvar {#1}%
	\def\envimagesmcpos {#2}%
	\def\envimagescaptioncf {false}%
	\def\envimagescaptionvar {#3}%
	\global\def\GLOBALenvimageadded {false}%
	\global\def\GLOBALenvimageinitialized {true}%
	\checkinsidemulticol%
	\checkoutsideappendix%
	% Configura caption y márgenes
	\setcaptionmargincm{\captionmarginmultimg} % Eso es para los wrapfig
	% Inicia la figura
	\ifthenelse{\equal{#2}{bottom}}{%
		\begin{figure*}[!b] \centering%
	}{%
	\ifthenelse{\equal{#2}{top}}{%
		\begin{figure*}[!t] \centering%
	}{%
		\errmessage{LaTeX Warning: Posicion de imagen invalida, valores esperados: bottom,top}
		\stop
	}}%
		\ifthenelse{\equal{\GLOBALenvimagecf}{true}}{%
			\ContinuedFloat%
			\global\def\GLOBALenvimagecf {false}%
			\def\envimagescaptioncf {true}%
		}{}%
		\corevspacevarcm{\marginimagemulttop}%
	}{%
		\setcaptionmargincm{\captionlrmargin}%
		\ifthenelse{\equal{\envimagescaptionvar}{}}{%
			\corevspacevarcm{\captionlessmarginimage}%
		}{%
			\corevspacevarcm{\captionmarginimagesmc}%
			\ifthenelse{\equal{\envimagescaptioncf}{true}}{%
				\caption[]{\envimagescaptionvar\envimageslabelvar}%
			}{%
				\caption{\envimagescaptionvar\envimageslabelvar}%
			}%
		}%
	\end{figure*}%
	% Restablece caption y márgenes
	\setcaptionmargincm{\captionlrmarginmc}%
	% Restablece globales
	\global\def\GLOBALenvimageinitialized {false}%
}

% -----------------------------------------------------------------------------
% IMPORTACIÓN DE ESTILOS
% -----------------------------------------------------------------------------
% Definición de colores
\colorlet{numb}{magenta!60!black}
\colorlet{punct}{red!60!black}
\definecolor{delim}{RGB}{20,105,176}
\definecolor{dkcyan}{RGB}{0,123,167}
\definecolor{dkgray}{RGB}{90,90,90}
\definecolor{dkgreen}{RGB}{0,150,0}
\definecolor{gray}{RGB}{127,127,127}
\definecolor{lbrown}{RGB}{255,252,249}
\definecolor{lgray}{RGB}{240,240,240}
\definecolor{mauve}{RGB}{150,0,210}
\definecolor{ocre}{RGB}{243,102,25}

% ABAP
\lstdefinestyle{abap}{
	language=ABAP
}

% Ada
\lstdefinestyle{ada}{
	language=[2005]Ada
}

% Assembler
\lstdefinelanguage[x64]{Assembler}[x86masm]{Assembler}{
	morekeywords={
		CDQE,CMPSQ,CMPXCHG16B,CQO,IRETQ,JRCXZ,LODSQ,MOVSXD,POPFQ,PUSHFQ,r8,r8b,r8d,r8w,r9,r9b,r9d,r9w,r10,r10b,r10d,r10w,r11,r11b,r11d,r11w,r12,r12b,r12d,r12w,r13,r13b,r13d,r13w,r14,r14b,r14d,r14w,r15,r15b,r15d,r15w,rax,rbp,rbx,rcx,rdi,RDTSCP,rdx,rsi,rsp,SCASQ,STOSQ,SWAPGS
	}
}
\lstdefinestyle{assemblerx64}{
	language=[x64]Assembler
}
\lstdefinestyle{assemblerx86}{
	language=[x86masm]Assembler
}

% Awk
\lstdefinestyle{awk}{
	language=[gnu]Awk
}

% Bash
\lstdefinestyle{bash}{
	language=bash,
	breakatwhitespace=false,
	morecomment=[l]{rem},
	morecomment=[s]{::}{::},
	morekeywords={
		call,cp,dig,gcc,git,grep,ls,mv,python,rm,sudo,vim
	},
	sensitive=false
}

% Basic
\lstdefinestyle{basic}{
	language=[Visual]Basic
}

% C
\lstdefinestyle{c}{
	language=C,
	breakatwhitespace=false,
	keepspaces=true
}

% Caml
\lstdefinestyle{caml}{
	language=[light]Caml
}

% CMake
\lstdefinestyle{cmake}{
	language=[gnu] make,
	keywordstyle=[2]\color{dkcyan},
	morekeywords=[1]{
		add_custom_command,add_custom_target,add_definitions,add_executable,add_library,add_subdirectory,cmake_minimum_required,cmake_policy,configure_file,cuda_add_library,cuda_include_directories,else,elseif,endforeach,endfunction,endif,endmacro,execute_process,file,find_library,find_package,find_path,find_program,foreach,function,get_directory_property,get_filename_component,get_filename_component,get_source_file_property,get_target_property,if,include,include_directories,install,link_directories,list,macro,mark_as_advanced,message,option,PKG_CHECK_MODULES,project,set,SET_CHECK_CXX_FLAGS,set_property,set_source_files_properties,set_target_properties,string,target_compile_options,target_include_directories,target_link_libraries,unset
	},
	morekeywords=[2]{
		AND,APPEND,APPLE,ARCHIVE,CACHE,CMAKE_CURRENT_LIST_DIR,CMAKE_CXX_STANDARD,CMAKE_MODULE_PATH,CMAKE_SYSTEM_NAME,COMMAND,COMMENT,COMPILE_DEFINITIONS,CONFIG,DEFINED,DEPENDS,DESTINATION,DIRECTORY,ENDIF,ENV,EQUAL,ERROR_QUIET,EXISTS,FATAL_ERROR,FILES,FILES_MATCHING,FIND,FIND,FIND_LIBRARY,FORCE,GLOB,GREATER,IF,INCLUDE_DIRECTORIES,IS_ABSOLUTE,LESS,LIBRARY,LINK_PRIVATE,LIST,MAIN_DEPENDENCY,MAKE_DIRECTORY,MARK_AS_ADVANCED,MATCHALL,MATCHES,NOT,OBJECT,OFF,ON,OPTIONAL,OR,OUTPUT,OUTPUT_STRIP_TRAILING_WHITESPACE,OUTPUT_VARIABLE,PARENT_SCOPE,PATTERN,PRE_BUILD_COMMAND,PRE_LINK,PRIVATE,PROJECT_NAME,PROPERTIES,PROPERTY,PUBLIC,REGEX,RELEASE,RENAME,REQUIRED,RUNTIME,SET,STATIC,STREQUAL,SYSTEM,TARGET,TARGETS,TOUPPER,UNIX,VERSION,VERSION_EQUAL,VERSION_LESS,WIN32,WORKING_DIRECTORY
	}
}

% Cobol
\lstdefinestyle{cobol}{
	language=Cobol
}

% C++
\lstdefinestyle{cpp}{
	language=C++,
	breakatwhitespace=false,
	morekeywords={NULL}
}

% C#
\lstdefinestyle{csharp}{
	language=csh,
	morecomment=[l]{//},
	morecomment=[s]{/*}{*/},
	morekeywords={
		abstract,as,base,bool,break,byte,case,catch,char,checked,class,const,continue,decimal,default,delegate,do,double,else,enum,event,explicit,extern,false,finally,fixed,float,for,foreach,goto,if,implicit,in,int,interface,internal,is,lock,long,namespace,new,null,object,operator,out,override,params,private,protected,public,readonly,ref,return,sbyte,sealed,short,sizeof,stackalloc,static,string,struct,switch,this,throw,true,try,typeof,uint,ulong,unchecked,unsafe,ushort,using,virtual,void,volatile,while
	}
}

% CSS
\lstdefinelanguage{CSS}{
	morecomment=[s]{/*}{*/},
	morekeywords={
		-moz-binding,-moz-border-bottom-colors,-moz-border-left-colors,-moz-border-radius,-moz-border-radius-bottomleft,-moz-border-radius-bottomright,-moz-border-radius-topleft,-moz-border-radius-topright,-moz-border-right-colors,-moz-border-top-colors,-moz-opacity,-moz-outline,-moz-outline-color,-moz-outline-style,-moz-outline-width,-moz-user-focus,-moz-user-input,-moz-user-modify,-moz-user-select,-replace,-set-link-source,-use-link-source,accelerator,azimuth,background,background-attachment,background-color,background-image,background-position,background-position-x,background-position-y,background-repeat,behavior,border,border-bottom,border-bottom-color,border-bottom-style,border-bottom-width,border-collapse,border-color,border-left,border-left-color,border-left-style,border-left-width,border-right,border-right-color,border-right-style,border-right-width,border-spacing,border-style,border-top,border-top-color,border-top-style,border-top-width,border-width,bottom,caption-side,clear,clip,color,content,counter-increment,counter-reset,cue,cue-after,cue-before,cursor,direction,display,elevation,empty-cells,filter,float,font,font-family,font-size,font-size-adjust,font-stretch,font-style,font-variant,font-weight,height,ime-mode,include-source,layer-background-color,layer-background-image,layout-flow,layout-grid,layout-grid-char,layout-grid-char-spacing,layout-grid-line,layout-grid-mode,layout-grid-type,left,letter-spacing,line-break,line-height,list-style,list-style-image,list-style-position,list-style-type,margin,margin-bottom,margin-left,margin-right,margin-top,marker-offset,marks,max-height,max-width,min-height,min-width,orphans,outline,outline-color,outline-style,outline-width,overflow,overflow-X,overflow-Y,padding,padding-bottom,padding-left,padding-right,padding-top,page,page-break-after,page-break-before,page-break-inside,pause,pause-after,pause-before,pitch,pitch-range,play-during,position,quotes,richness,right,ruby-align,ruby-overhang,ruby-position,scrollbar-3d-light-color,scrollbar-arrow-color,scrollbar-base-color,scrollbar-dark-shadow-color,scrollbar-face-color,scrollbar-highlight-color,scrollbar-shadow-color,scrollbar-track-color,size,speak,speak-header,speak-numeral,speak-punctuation,speech-rate,stress,table-layout,text-align,text-align-last,text-autospace,text-decoration,text-indent,text-justify,text-kashida-space,text-overflow,text-shadow,text-transform,text-underline-position,top,unicode-bidi,vertical-align,visibility,voice-family,volume,white-space,widows,width,word-break,word-spacing,word-wrap,writing-mode,z-index,zoom
	},
	morestring=[s]{:}{;},
	sensitive=true
}
\lstdefinestyle{css}{
	language=CSS,
	breakatwhitespace=true
}

% CSV
\lstdefinestyle{csv}{
	language={}
}

% CUDA
\lstdefinestyle{cuda}{
	language=C++,
	breakatwhitespace=false,
	emph={
		cudaFree,cudaMalloc,__device__,__global__,__host__,__shared__,__syncthreads
	},
	emphstyle=\color{dkcyan}\ttfamily,
	morecomment=[l][\color{magenta}]{\#},
	moredelim=[s][\ttfamily]{<<<}{>>>}
}

% Dart
\lstdefinestyle{dart}{
	language=Java,
	emph=[2]{
		findAllElements,findElements
	},
	morekeywords={
		*,get,library,List,num,set,String,var
	}
}

% Docker
\lstdefinelanguage{docker}{
	comment=[l]{\#},
	keywords={
		ADD,CMD,COPY,ENTRYPOINT,ENV,EXPOSE,FROM,LABEL,MAINTAINER,ONBUILD,RUN,STOPSIGNAL,USER,VOLUME,WORKDIR
	},
	morestring=[b]',
	morestring=[b]"
}
\lstdefinestyle{docker}{
	language=docker,
	breakatwhitespace=true
}

% Elisp
\lstdefinestyle{elisp}{
	language=elisp
}

% Elixir
\lstdefinestyle{elixir}{
	morekeywords={
		case,catch,def,do,else,false,use,alias,receive,timeout,defmacro,defp,for,if,import,defmodule,defprotocol,nil,defmacrop,defoverridable,defimpl,super,fn,raise,true,try,end,with,unless
	},
	otherkeywords={
		<-,->, |>, \%\{, \}, \{, \, (, )
	},
	morecomment=[l]{\#},
	morecomment=[n]{/*}{*/},
	morecomment=[s][\color{purple}]{:}{\ },
	morestring=[s][\color{mauve}]"",
	sensitive=true
}

% Erlang
\lstdefinestyle{erlang}{
	language=erlang
}

% Fortran-95
\lstdefinestyle{fortran}{
	language=[95]Fortran,
	breakatwhitespace=false
}

% F#
\lstdefinestyle{fsharp}{
	morecomment=[l][\color{dkgreen}]{///},
	morecomment=[l][\color{dkgreen}]{//},
	morecomment=[s][\color{dkgreen}]{{(*}{*)}},
	morestring=[b]",
	morekeywords={
		abstract,and,Application,Array,Async,async,begin,cloud,do,else,end,false,finally,for,fun,function,if,in,inherit,interface,let,List,match,member,module,mutable,namespace,new,of,open,rec,return,Seq,static,System,then,true,try,type,use,while,with,yield
	},
	otherkeywords={
		by,do!,from,let!,order,return!,select,use!,var,where,yield!
	},
	sensitive=true
}

% GLSL
\lstdefinelanguage{GLSL}{
	alsoletter={\#},
	morekeywords=[1]{
		attribute,bool,break,bvec2,bvec3,bvec4,case,centroid,const,continue,default,discard,do,else,false,flat,float,for,highp,if,in,inout,int,invariant,isampler1D,isampler1DArray,isampler2D,isampler2DArray,isampler2DMS,isampler2DMSArray,isampler2DRect,isampler3D,isamplerBuffer,isamplerCube,ivec2,ivec3,ivec4,layout,lowp,mat2,mat2x2,mat2x3,mat2x4,mat3,mat3x2,mat3x3,mat3x4,mat4,mat4x2,mat4x3,mat4x4,mediump,noperspective,out,precision,return,sampler1D,sampler1DArray,sampler1DArrayShadow,sampler1DShadow,sampler2D,sampler2DArray,sampler2DArrayShadow,sampler2DMS,sampler2DMSArray,sampler2DRect,sampler2DRectShadow,sampler2DShadow,sampler3D,samplerBuffer,samplerCube,samplerCubeShadow,smooth,struct,switch,true,uint,uniform,usampler1D,usampler1DArray,usampler2D,usampler2DArray,usampler2DMS,usampler2DMSArray,usampler2DRect,usampler3D,usamplerBuffer,usamplerCube,uvec2,uvec3,uvec4,varying,vec2,vec3,vec4,void,while
	},
	morekeywords=[2]{
		abs,acos,acosh,all,any,asin,asinh,atan,atan,atanh,ceil,clamp,cos,cosh,cross,degrees,determinant,dFdx,dFdy,distance,dot,EmitVertex,EndPrimitive,equal,exp,exp2,faceforward,floatBitsToInt,floatBitsToUint,floor,fract,fwidth,greaterThan,greaterThanEqual,intBitsToFloat,inverse,inversesqrt,isinf,isnan,length,lessThan,lessThanEqual,log,log2,matrixCompMult,max,min,mix,mod,modf,noise1,noise2,noise3,noise4,normalize,not,notEqual,outerProduct,pow,radians,reflect,refract,round,roundEven,shadow1D,shadow1DLod,shadow1DProj,shadow1DProjLod,shadow2D,shadow2DLod,shadow2DProj,shadow2DProjLod,sign,sin,sinh,smoothstep,sqrt,step,tan,tanh,texelFetch,texelFetchOffset,texture,texture1D,texture1DProj,texture1DProjLod,texture2D,texture2DLod,texture2DProj,texture2DProjLod,texture3D,texture3DLod,texture3DProj,texture3DProjLod,textureCube,textureCubeLod,textureGrad,textureGradOffset,textureLod,textureLodOffset,textureOffset,textureProj,textureProjGrad,textureProjGradOffset,textureProjLod,textureProjLodOffset,textureProjOffset,textureSize,transpose,trunc,uintBitsToFloat
	},
	morekeywords=[3]{
		\#version,core,gl_ClipDistance,gl_ClipDistance,gl_ClipVertex,gl_DepthRange,gl_FragColor,gl_FragCoord,gl_FragData,gl_FragDepth,gl_FrontFacing,gl_InstanceID,gl_Layer,gl_MaxClipDistances,gl_MaxCombinedTextureImageUnits,gl_MaxDrawBuffers,gl_MaxDrawBuffers,gl_MaxFragmentInputComponents,gl_MaxFragmentUniformComponents,gl_MaxGeometryInputComponents,gl_MaxGeometryOutputComponents,gl_MaxGeometryOutputVertices,gl_MaxGeometryOutputVertices,gl_MaxGeometryTextureImageUnits,gl_MaxGeometryTotalOutputComponents,gl_MaxGeometryUniformComponents,gl_MaxGeometryVaryingComponents,gl_MaxTextureImageUnits,gl_MaxVaryingComponents,gl_MaxVaryingFloats,gl_MaxVertexAttribs,gl_MaxVertexOutputComponents,gl_MaxVertexTextureImageUnits,gl_MaxVertexUniformComponents,gl_PerVertex,gl_PointCoord,gl_PointSize,gl_Position,gl_PrimitiveID,gl_VertexID
	},
	morecomment=[l]{//},
	morecomment=[s]{/*}{*/}
}
\lstdefinestyle{glsl}{
	language=GLSL,
	keywordstyle=[3]\color{dkcyan}\ttfamily,
	prebreak=\raisebox{0ex}[0ex][0ex]{\ensuremath{\hookleftarrow}},
	sensitive=true,
	upquote=true
}

% Gnuplot
\lstdefinestyle{gnuplot}{
	language=Gnuplot
}

% Go
\lstdefinestyle{go}{
	language=Go
}

% Haskell
\lstdefinestyle{haskell}{
	language=haskell,
	morecomment=[l]\%
}

% HTML5
\lstdefinelanguage{HTML5}{
	language=html,
	alsoletter={<>=-},
	morecomment=[s]{<!--}{-->},
	ndkeywords={
		% General
		=,
		% Atributos HTML
		accept-charset=,accept=,accesskey=,action=,align=,alt=,async=,autocomplete=,autofocus=,autoplay=,autosave=,bgcolor=,border=,buffered=,challenge=,charset=,checked=,cite=,class=,code=,codebase=,color=,cols=,colspan=,content=,contenteditable=,contextmenu=,controls=,coords=,data=,datetime=,default=,defer=,dir=,dirname=,disabled=,download=,draggable=,dropzone=,enctype=,for=,form=,formaction=,headers=,height=,hidden=,high=,href=,hreflang=,http-equiv=,icon=,id=,ismap=,itemprop=,keytype=,kind=,label=,lang=,language=,list=,loop=,low=,manifest=,max=,maxlength=,media=,method=,min=,multiple=,name=,novalidate=,open=,optimum=,pattern=,ping=,placeholder=,poster=,preload=,pubdate=,radiogroup=,readonly=,rel=,required=,reversed=,rows=,rowspan=,sandbox=,scope=,scoped=,seamless=,selected=,shape=,size=,sizes=,span=,spellcheck=,src=,srcdoc=,srclang=,start=,step=,style=,summary=,tabindex=,target=,title=,type=,usemap=,value=,width=,wrap=,
		% Propiedades CSS
		-moz-binding:,-moz-border-bottom-colors:,-moz-border-left-colors:,-moz-border-radius-bottomleft:,-moz-border-radius-bottomright:,-moz-border-radius-topleft:,-moz-border-radius-topright:,-moz-border-radius:,-moz-border-right-colors:,-moz-border-top-colors:,-moz-opacity:,-moz-outline-color:,-moz-outline-style:,-moz-outline-width:,-moz-outline:,-moz-transform:,-moz-user-focus:,-moz-user-input:,-moz-user-modify:,-moz-user-select:,-replace:,-set-link-source:,-use-link-source:,accelerator:,azimuth:,background-attachment:,background-color:,background-image:,background-position-x:,background-position-y:,background-position:,background-repeat:,background:,behavior:,border-bottom-color:,border-bottom-style:,border-bottom-width:,border-bottom:,border-collapse:,border-color:,border-left-color:,border-left-style:,border-left-width:,border-left:,border-right-color:,border-right-style:,border-right-width:,border-right:,border-spacing:,border-style:,border-top-color:,border-top-style:,border-top-width:,border-top:,border-width:,border:,bottom:,caption-side:,clear:,clip:,color:,content:,counter-increment:,counter-reset:,cue-after:,cue-before:,cue:,cursor:,direction:,display:,elevation:,empty-cells:,filter:,float:,font-family:,font-size-adjust:,font-size:,font-stretch:,font-style:,font-variant:,font-weight:,font:,height:,ime-mode:,include-source:,layer-background-color:,layer-background-image:,layout-flow:,layout-grid-char-spacing:,layout-grid-char:,layout-grid-line:,layout-grid-mode:,layout-grid-type:,layout-grid:,left:,letter-spacing:,line-break:,line-height:,list-style-image:,list-style-position:,list-style-type:,list-style:,margin-bottom:,margin-left:,margin-right:,margin-top:,margin:,marker-offset:,marks:,max-height:,max-width:,min-height:,min-width:,orphans:,outline-color:,outline-style:,outline-width:,outline:,overflow-X:,overflow-Y:,overflow:,padding-bottom:,padding-left:,padding-right:,padding-top:,padding:,page-break-after:,page-break-before:,page-break-inside:,page:,pause-after:,pause-before:,pause:,pitch-range:,pitch:,play-during:,position:,quotes:,richness:,right:,ruby-align:,ruby-overhang:,ruby-position:,scrollbar-3d-light-color:,scrollbar-arrow-color:,scrollbar-base-color:,scrollbar-dark-shadow-color:,scrollbar-face-color:,scrollbar-highlight-color:,scrollbar-shadow-color:,scrollbar-track-color:,size:,speak-header:,speak-numeral:,speak-punctuation:,speak:,speech-rate:,stress:,table-layout:,text-align-last:,text-align:,text-autospace:,text-decoration:,text-indent:,text-justify:,text-kashida-space:,text-overflow:,text-shadow:,text-transform:,text-underline-position:,top:,transform:,transition-duration:,transition-property:,transition-timing-function:,unicode-bidi:,vertical-align:,visibility:,voice-family:,volume:,white-space:,widows:,width:,word-break:,word-spacing:,word-wrap:,writing-mode:,z-index:,zoom:
	},
	otherkeywords={
		<,</,>,</a,<a,</a>,</abbr,<abbr,</abbr>,</address,<address,</address>,</area,<area,</area>,</area,<area,</area>,</article,<article,</article>,</aside,<aside,</aside>,</audio,<audio,</audio>,</audio,<audio,</audio>,</b,<b,</b>,</base,<base,</base>,</bdi,<bdi,</bdi>,</bdo,<bdo,</bdo>,</blockquote,<blockquote,</blockquote>,</body,<body,</body>,</br,<br,</br>,</button,<button,</button>,</canvas,<canvas,</canvas>,</caption,<caption,</caption>,</cite,<cite,</cite>,</code,<code,</code>,</col,<col,</col>,</colgroup,<colgroup,</colgroup>,</data,<data,</data>,</datalist,<datalist,</datalist>,</dd,<dd,</dd>,</del,<del,</del>,</details,<details,</details>,</dfn,<dfn,</dfn>,</div,<div,</div>,</dl,<dl,</dl>,</dt,<dt,</dt>,</em,<em,</em>,</embed,<embed,</embed>,</fieldset,<fieldset,</fieldset>,</figcaption,<figcaption,</figcaption>,</figure,<figure,</figure>,</footer,<footer,</footer>,</form,<form,</form>,</h1,<h1,</h1>,</h2,<h2,</h2>,</h3,<h3,</h3>,</h4,<h4,</h4>,</h5,<h5,</h5>,</h6,<h6,</h6>,</head,<head,</head>,</header,<header,</header>,</hr,<hr,</hr>,</html,<html,</html>,</i,<i,</i>,</iframe,<iframe,</iframe>,</img,<img,</img>,</input,<input,</input>,</ins,<ins,</ins>,</kbd,<kbd,</kbd>,</keygen,<keygen,</keygen>,</label,<label,</label>,</legend,<legend,</legend>,</li,<li,</li>,</link,<link,</link>,</main,<main,</main>,</map,<map,</map>,</mark,<mark,</mark>,</math,<math,</math>,</menu,<menu,</menu>,</menuitem,<menuitem,</menuitem>,</meta,<meta,</meta>,</meter,<meter,</meter>,</nav,<nav,</nav>,</noscript,<noscript,</noscript>,</object,<object,</object>,</ol,<ol,</ol>,</optgroup,<optgroup,</optgroup>,</option,<option,</option>,</output,<output,</output>,</p,<p,</p>,</param,<param,</param>,</pre,<pre,</pre>,</progress,<progress,</progress>,</q,<q,</q>,</rp,<rp,</rp>,</rt,<rt,</rt>,</ruby,<ruby,</ruby>,</s,<s,</s>,</samp,<samp,</samp>,</script,<script,</script>,</section,<section,</section>,</select,<select,</select>,</small,<small,</small>,</source,<source,</source>,</span,<span,</span>,</strong,<strong,</strong>,</style,<style,</style>,</summary,<summary,</summary>,</sup,<sup,</sup>,</svg,<svg,</svg>,</table,<table,</table>,</tbody,<tbody,</tbody>,</td,<td,</td>,</template,<template,</template>,</textarea,<textarea,</textarea>,</tfoot,<tfoot,</tfoot>,</th,<th,</th>,</thead,<thead,</thead>,</time,<time,</time>,</title,<title,</title>,</tr,<tr,</tr>,</track,<track,</track>,</u,<u,</u>,</ul,<ul,</ul>,</var,<var,</var>,</video,<video,</video>,</wbr,<wbr,</wbr>,/>,<!
	},
	sensitive=true,
	tag=[s]
}
\lstdefinestyle{html}{
	language=HTML5,
	alsodigit={.:;},
	alsolanguage=JavaScript,
	firstnumber=1,
	ndkeywordstyle=\color{dkgreen}\bfseries,
	numberfirstline=true
}

% INI, Archivos de configuraciones
\lstdefinestyle{ini}{
	language={},
	commentstyle=\color{gray}\ttfamily,
	keywordstyle={\color{black}\bfseries},
	morecomment=[l]{;},
	morecomment=[l]{\#},
	morecomment=[s][\color{dkgreen}\bfseries]{[}{]},
	morekeywords={},
	otherkeywords={=,:}
}

% Java
\lstdefinestyle{java}{
	language=Java,
	breakatwhitespace=true,
	keepspaces=true
}

% Javascript
\lstdefinelanguage{JavaScript}{
	comment=[l]{//},
	keepspaces=true,
	keywords={
		break,else,false,for,function,if,in,new,null,return,true,typeof,var,while
	},
	morecomment=[s]{/*}{*/},
	morestring=[b]',
	morestring=[b]",
	morestring=[b]`,
	ndkeywords={
		await,async,case,catch,class,const,default,do,enum,export,extends,finally,from,implements,import,instanceof,let,static,super,switch,then,this,throw,try
	},
	ndkeywordstyle=\color{blue}\bfseries,
	sensitive=false
}
\lstdefinestyle{javascript}{
	language=JavaScript
}

% JSON
\lstdefinestyle{json}{
	literate=*{0}{{{\color{numb}0}}}{1}{1}{{{\color{numb}1}}}{1}{2}
	{{{\color{numb}2}}}{1}{3}{{{\color{numb}3}}}{1}{4}{{{\color{numb}4}}}
	{1}{5}{{{\color{numb}5}}}{1}{6}{{{\color{numb}6}}}{1}{7}{{{\color{numb}7}}}
	{1}{8}{{{\color{numb}8}}}{1}{9}{{{\color{numb}9}}}{1}{:}
	{{{\color{punct}{:}}}}{1}{,}{{{\color{punct}{,}}}}{1}{\{}
	{{{\color{delim}{\{}}}}{1}{\}}{{{\color{delim}{\}}}}}
	{1}{[}{{{\color{delim}{[}}}}{1}{]}{{{\color{delim}{]}}}}{1},
	tabsize=2
}

% Julia
\lstdefinestyle{julia}{
	keywordsprefix=\@,
	morecomment=[l]{\#},
	morekeywords={
		abstract,Any,applicable,assert,baremodule,begin,bitstype,Bool,break,catch,ccall,Complex64,Complex128,const,continue,convert,dlopen,dlsym,do,edit,else,elseif,end,eps,error,exit,export,finalizer,Float32,Float64,for,function,global,hash,if,im,immutable,import,importall,in,Inf,Int,Int8,Int16,Int32,Int64,invoke,is,isa,isequal,let,load,local,macro,method_exists,module,Nan,new,None,Nothing,ntuple,pi,promote,promote_type,quote,realmax,realmin,return,sizeof,subtype,system,throw,try,tuple,type,typealias,typemax,typemin,typeof,uid,Uint,Uint8,Uint16,Uint32,Uint64,using,while,whos
	},
	morestring=[b]',
	morestring=[b]",
	sensitive=true
}

% Kotlin
\lstdefinestyle{kotlin}{
	comment=[l]{//},
	emph={delegate,filter,first,firstOrNull,forEach,lazy,map,mapNotNull,println,
		return@},
	emphstyle={\color{blue}},
	keywords={
		abstract,actual,as,as?,break,by,class,companion,continue,data,do,dynamic,else,enum,expect,false,final,for,fun,get,if,import,in,interface,internal,is,null,object,override,package,private,public,return,set,super,suspend,this,throw,true,try,typealias,val,var,vararg,when,where,while
	},
	morecomment=[s]{/*}{*/},
	morestring=[b]",
	morestring=[s]{"""*}{*"""},
	ndkeywords={
		@Deprecated,@JvmField,@JvmName,@JvmOverloads,@JvmStatic,@JvmSynthetic,Array,Byte,Double,Float,Int,Integer,Iterable,Long,Runnable,Short,String
	},
	ndkeywordstyle=\color{BurntOrange}\bfseries,
	sensitive=true
}

% LaTeX
\lstdefinestyle{latex}{
	language=TeX,
	morekeywords={
		aacos,aasin,aatan,acos,addimage,addimageanum,addimageboxed,align,asin,atan,begin,bibitem,bibliography,bigstrut,boldmath,bookmarksetup,boxed,cancelto,caption,changeheadertitle,checkmark,checkvardefined,cite,clearpage,dd,degree,eqref,equal,frac,fracnpartial,fullcite,hline,href,ifthenelse,imageshspace,imagesnewline,imagesvspace,includefullhfpdf,includehfpdf,insertalign,insertalignanum,insertaligncaptioned,insertaligncaptioned,insertaligncaptionedanum,insertaligned,insertalignedanum,insertalignedcaptioned,insertalignedcaptionedanum,insertemail,insertemptypage,inserteqimage,insertequation,insertequationanum,insertequationcaptioned,insertequationcaptionedanum,insertgather,insertgatheranum,insertgathercaptioned,insertgathercaptionedanum,insertgathered,insertgatheredanum,insertgatheredcaptioned,insertgatheredcaptionedanum,insertimage,insertimageleft,insertimageright,insertindextitle,insertindextitlepage,insertphone,isundefined,itemresize,label,LaTeX,lipsum,lpow,makeatletter,makeatother,newcommand,newcounter,newp,newpage,pow,quotes,ref,renewcommand,section,sectionanum,setcounter,setlength,shortcite,sourcecode,sourcecodep,subsection,subsectionanum,subsubsection,subsubsectionanum,subsubsubsection,subsubsubsection,subsubsubsectionanum,textbf,textit,textregistered,textsuperscript,texttt,throwbadconfig,unboldmath,url,xspace
	}
}

% Lisp
\lstdefinestyle{lisp}{
	language=Lisp,
	morekeywords={if}
}

% LLVM
\lstdefinestyle{llvm}{
	language=LLVM
}

% Lua
\lstdefinestyle{lua}{
	language={[5.3]Lua}
}

% Make
\lstdefinestyle{make}{
	language=[gnu] make
}

% Maple
\lstdefinelanguage{Maple}{
	morecomment=[l]\#,
	morekeywords={
		and,assuming,break,by,catch,description,do,done,elif,else,end,error,export,fi,finally,for,from,global,if,implies,in,intersect,local,minus,mod,module,next,not,o,option,options,or,proc,quit,read,restart,return,save,stop,subset,then,to,try,union,use,uses,with,while,xor
	},
	morestring=[b]",
	morestring=[d],
	sensitive=true
} 
\lstdefinestyle{maple}{
	language=Maple
}

% Mathematica
\lstdefinestyle{mathematica}{
	language=Mathematica
}

% Matlab
\lstdefinestyle{matlab}{
	language=Matlab,
	deletekeywords={fft},
	keepspaces=true,
	morecomment=[l]\%,
	morecomment=[n]{\%\{\^^M}{\%\}\^^M},
	morekeywords={
		addOptional,box,break,catch,cell,classdef,continue,deal,double,end,factorial,for,gradient,hessian,if,isa,ltitr,matlab2tikz,methods,minor,movegui,normcdf,normpdf,on,ones,parse,persistent,poissrnd,properties,repmat,solve,strcat,subs,syms,try,var,warning,xlim,ylim
	}
}

% Mercury
\lstdefinestyle{mercury}{
	language=Mercury
}

% Modula-2
\lstdefinestyle{modula2}{
	language=Modula-2
}

% Objective-C
\lstdefinestyle{objectivec}{
	language=[Objective]C,
	breakatwhitespace=false,
	keepspaces=true,
	moredirectives={
		import
	},
	morekeywords={
		@catch,@class,@dynamic,@encode,@end,@finally,@implementation,@interface,@package,@private,@property,@protected,@protocol,@public,@selector,@synchronized,@synthesize,@throw,@try,assign,BOOL,bycopy,byref,Class,copy,id,IMP,in,inout,Nil,nil,NO,nonatomic,oneway,out,readonly,readwrite,retain,SEL,self,super,YES,_cmd
	}
}

% Octave
\lstdefinestyle{octave}{
	language=Octave,
	keepspaces=true,
	morecomment=[l]\%,
	morecomment=[n]{\%\{\^^M}{\%\}\^^M}
}

% OpenCL
\lstdefinestyle{opencl}{
	language=C++,
	breakatwhitespace=false,
	emph={
		bool2,bool3,bool4,bool8,bool16,char2,char3,char4,char8,char16,complex,constant,event_t,float2,float3,float4,float8,float16,global,half2,half3,half4,half8,half16,image2d_t,image3d_t,imaginary,int2,int3,int4,int8,int16,kernel,local,long2,long3,long4,long8,long16,private,quad,quad2,quad3,quad4,quad8,quad16,sampler_t,short2,short3,short4,short8,short16,uchar2,uchar3,uchar4,uchar8,uchar16,uint2,uint3,uint4,uint8,uint16,ulong2,ulong3,ulong4,ulong8,ulong16,ushort2,ushort3,ushort4,ushort8,ushort16,__constant,__global,__kernel,__local,__private
	},
	emphstyle=\color{dkcyan}\ttfamily,
	morecomment=[l][\color{magenta}]{\#}
}

% OpenSees
\lstdefinestyle{opensees}{
	language=tcl,
	breakatwhitespace=false,
	emph=[1]{
		-accel,-beamUniform,-dir,-dof,-ele,-eleRange,-file,-height,-increment,-initial,-iNode,-integration,-iterate,-jNode,-kNode,-mass,-mat,-matConcrete,-matShear,-matSteel,-max,-maxDim,-maxEta,-maxIter,-min,-minEta,-ndf,-ndm,-node,-nodeRange,-numSublevels,-numSubSteps,-perpDirn,-region,-rho,-sections,-thick,-time,-tol,-type,-width
	},
	emphstyle=[1]\color{black}\bfseries\em,
	keepspaces=true,
	morecomment=[l]{\#},
	morekeywords={
		algorithm,analysis,analyze,constraints,deformation,disp,eleLoad,element,equalDOF,fix,fixX,fixY,fixZmodel,geomTransf,initialize,integrator,layer,loadConst,mass,model,node,numberer,patch,pattern,printA,PySimple1Gen,reaction,recorder,region,rigidDiaphragm,section,system,test,uniaxialMaterial,wipe,wipeAnalysis
	},
	ndkeywords={
		9_4_QuadUP,20_8_BrickUP,AC3D8,Aggregator,ArcLength,ASI3D8,AV3D4,AxialSp,AxialSpHD,BandGeneral,BARSLIP,BasicBuilder,bbarBrick,bbarBrickUP,bbarQuad,bbarQuadUP,BeamColumnJoint,BeamContact2D,BeamContact3D,BeamEndContact3D,BFGS,Bilin,BilinearOilDamper,Bond_SP01,BoucWen,Brick20N,brickUP,Broyden,BWBN,Cast,CatenaryCable,CentralDifference,CFSSSWP,CFSWSWP,Concrete01,Concrete01WithSITC,Concrete02,Concrete03,Concrete04,Concrete06,Concrete07,ConcreteCM,ConcreteD,ConfinedConcrete01,constraintsTypeGravity,Corotational,corotTruss,corotTrussSection,CoupledZeroLength,DeformedShape,dispBeamColumn,dispBeamColumnInt,DisplacementControl,Dodd_Restrepo,Drift,ECC01,Elastic,elasticBeamColumn,ElasticBilin,ElasticMultiLinear,ElasticPP,ElasticPPGap,ElasticTimoshenkoBeam,ElasticTubularJoint,elastomericBearingBoucWen,elastomericBearingPlasticity,ElastomericX,Element,EnergyIncr,enhancedQuad,ENT,Explicitdifference,Fatigue,flatSliderBearing,forceBeamColumn,forceBeamColumn,FourNodeTetrahedron,FPBearingPTV,FRPConfinedConcrete,GeneralizedAlpha,Hardening,HDR,HHT,HyperbolicGapMaterial,Hysteretic,ImpactMaterial,InitStrainMaterial,InitStressMaterial,Joint2D,KikuchiAikenHDR,KikuchiAikenLRB,KikuchiBearing,KrylovNewton,Lagrange,LeadRubberX,LimitState,Linear,LoadControl,LoadControl,MinMax,MinUnbalDispNorm,mkdir,ModElasticBeam2d,ModifiedNewton,ModIMKPeakOriented,ModIMKPinching,MultiLinear,multipleShearSpring,MVLEM,Newmark,Newton,NewtonLineSearch,Node,NodeNumbers,nonlinearBeamColumn,NormDispIncr,numberer,Parallel,PathIndependentMaterial,pattern,PDelta,Pinching4,PinchingLimitStateMaterial,Plain,PyLiq1,PySimple1,quad,quadr,quadUP,QzSimple1,RambergOsgoodSteel,rayleigh,RCM,rect,ReinforcingSteel,RJWatsonEqsBearing,SAWS,SecantNewton,SelfCentering,Series,SFI_MVLEM,ShallowFoundationGen,ShellDKGQ,ShellDKGT,ShellMITC4,ShellNL,ShellNLDKGQ,ShellNLDKGT,SimpleContact2D,SimpleContact3D,singleFPBearing,SparseGeneral,SSPbrick,SSPbrickUP,SSPquad,SSPquadUP,Static,stdBrick,Steel01,Steel01,Steel02,Steel4,SteelMPF,straight,SurfaceLoad,TFP,Transient,TRBDF2,tri31,TripleFrictionPendulum,truss,trussSection,twoNodeLink,TzLiq1,TzSimple1,UniformExcitation,ViewScale,Viscous,ViscousDamper,VS3D4,YamamotoBiaxialHDR,zeroLength,zeroLengthContact,zeroLengthContactNTS2D,zeroLengthImpact3D,zeroLengthImpact3D,zeroLengthInterface2D,zeroLengthND,zeroLengthSection
	},
	ndkeywordstyle=\color{dkcyan}\ttfamily
}

% Pascal
\lstdefinestyle{pascal}{
	language=Pascal,
	morecomment=[l]{//},
	sensitive=false
}

% Perl
\lstdefinestyle{perl}{
	language=Perl,
	alsoletter={\%},
	breakatwhitespace=false,
	keepspaces=true
}

% PHP
\lstdefinestyle{php}{
	language=php,
	emph=[1]{
		php
	},
	emph=[2]{
		if,and,or,else
	},
	emph=[3]{
		abstract,as,const,else,elseif,endfor,endforeach,endif,extends,final,for,foreach,global,if,implements,private,protected,public,static,var
	},
	emphstyle=[1]\color{black},
	emphstyle=[2]\color{blue},
	keywords={
		abstract,and,array,as,break,callable,case,catch,class,clone,const,continue,declare,default,die,do,echo,else,elseif,empty,enddeclare,endfor,endforeach,endif,endswitch,endwhile,eval,exit,extends,final,finally,for,foreach,function,global,goto,if,implements,include,include_once,instanceof,insteadof,interface,isset,list,namespace,new,or,print,private,protected,public,require,require_once,return,static,switch,throw,trait,try,unset,use,var,while,xor,yield,__halt_compiler
	},
	showlines=true,
	upquote=true
}

% Texto plano
\lstdefinestyle{plaintext}{
	language={},
	keepspaces=true,
	postbreak={},
	tabsize=4
}

% Postscript
\lstdefinestyle{postscript}{
	language=PostScript,
	keepspaces=true
}

% Powershell
% https://github.com/rmainer/latex-listings-powershell/blob/master/src/latex-listings-powershell.tex
\lstdefinestyle{powershell}{
	alsodigit={-},
	morecomment=[l]{\#},
	morecomment=[n]{<\#}{\#>},
	morekeywords={
		Add-Content,Add-PSSnapin,Clear-Content,Clear-History,Clear-Host,Clear-Item,Clear-ItemProperty,Clear-Variable,Compare-Object,Connect-PSSession,Convert-Path,ConvertFrom-String,Copy-Item,Copy-ItemProperty,Disable-PSBreakpoint,Disconnect-PSSession,Enable-PSBreakpoint,Enter-PSSession,Exit-PSSession,Export-Alias,Export-Csv,Export-PSSession,ForEach-Object,Format-Custom,Format-Hex,Format-List,Format-Table,Format-Wide,Get-Alias,Get-ChildItem,Get-Clipboard,Get-Command,Get-ComputerInfo,Get-Content,Get-History,Get-Item,Get-ItemProperty,Get-ItemPropertyValue,Get-Job,Get-Location,Get-Member,Get-Module,Get-Process,Get-PSBreakpoint,Get-PSCallStack,Get-PSDrive,Get-PSSession,Get-PSSnapin,Get-Service,Get-TimeZone,Get-Unique,Get-Variable,Get-WmiObject,Group-Object,help,Import-Alias,Import-Csv,Import-Module,Import-PSSession,Invoke-Command,Invoke-Expression,Invoke-History,Invoke-Item,Invoke-RestMethod,Invoke-WebRequest,Invoke-WmiMethod,Measure-Object,mkdir,Move-Item,Move-ItemProperty,New-Alias,New-Item,New-Module,New-PSDrive,New-PSSession,New-PSSessionConfigurationFile,New-Variable,Out-GridView,Out-Host,Out-Printer,Pop-Location,powershell_ise.exe,Push-Location,Receive-Job,Receive-PSSession,Remove-Item,Remove-ItemProperty,Remove-Job,Remove-Module,Remove-PSBreakpoint,Remove-PSDrive,Remove-PSSession,Remove-PSSnapin,Remove-Variable,Remove-WmiObject,Rename-Item,Rename-ItemProperty,Resolve-Path,Resume-Job,Select-Object,Select-String,Set-Alias,Set-Clipboard,Set-Content,Set-Item,Set-ItemProperty,Set-Location,Set-PSBreakpoint,Set-TimeZone,Set-Variable,Set-WmiInstance,Show-Command,Sort-Object,Start-Job,Start-Process,Start-Service,Start-Sleep,Stop-Job,Stop-Process,Stop-Service,Suspend-Job,Tee-Object,Trace-Command,Wait-Job,Where-Object,Write-Output
	},
	morekeywords={
		Do,Else,For,ForEach,Function,If,In,Until,While
	},
	morestring=[b]{"},
	morestring=[b]{'},
	morestring=[s]{@'}{'@},
	morestring=[s]{@"}{"@},
	sensitive=false
}

% Prolog
\lstdefinestyle{prolog}{
	language=Prolog
}

% Promela
\lstdefinestyle{promela}{
	language=Promela
}

% Pseudocódigo
\lstdefinelanguage{Pseudocode}{
	language={},
	breakatwhitespace=false,
	commentstyle=\color{gray}\upshape,
	keepspaces=true,
	keywords={
		and,be,begin,break,datatype,do,elif,else,end,for,foreach,fun,function,if,in,input,let,not,null,or,output,pop,procedure,push,repeat,return,swap,until,while,xor
	},
	keywordstyle=\color{black}\bfseries,
	mathescape=true,
	morecomment=[l]{//},
	morecomment=[l]{\#},
	morecomment=[s]{/*}{*/},
	morecomment=[s]{/**}{*/},
	sensitive=false,
	stringstyle=\color{dkgray}\bfseries\em
}
\lstdefinestyle{pseudocode}{
	language=Pseudocode,
	backgroundcolor=\color{white},
	frame=tb,
	numbers=none
}
\lstdefinestyle{pseudocodecolor}{
	language=Pseudocode
}

% Python
\lstdefinelanguage{pythonEXTENDED}{
	language=Python,
	breakatwhitespace=false,
	emph={
		AbstractSet,Any,AsyncContextManager,AsyncGenerator,AsyncIterable,AsyncIterator,Awaitable,AwaitableGenerator,BinaryIO,ByteString,Callable,Collection,Container,ContextManager,Coroutine,Dict,False,ForwardRef,Generator,GenericMeta,Hashable,IO,ItemsView,Iterable,Iterator,KeysView,List,Mapping,MappingView,Match,Meta,MutableMapping,MutableSequence,MutableSet,NamedTuple,None,Pattern,Reversible,Sequence,Sized,SupportInts,SupportsAbs,SupportsBytes,SupportsComplex,SupportsFloat,SupportsIndex,SupportsRound,TextIO,True,Tuple,TypeAlias,TYPE_CHECKING,Union,ValuesView,__add__,__and__,__eq__,__floordiv__,__ge__,__gt__,__init__,__le__,__lt__,__main__,__mod__,__mul__,__name__,__ne__,__or__,__pow__,__repr__,__str__,__sub__,__truediv__,__xor__
	},
	emphstyle=\color{dkcyan}\ttfamily,
	keepspaces=true,
	morecomment=[s][\color{BurntOrange}]{@}{\ },
	morekeywords={
		as,assert,close,listdir,self,sorted,split,strip,with
	}
}
\lstdefinestyle{python}{
	language=pythonEXTENDED
}

% Q#
\lstdefinestyle{qsharp}{
	mathescape=true,
	morecomment=[l]{//},
	morecomment=[l][\color{dkgreen}]{///},
	morekeywords={
		Adj,Adjoint,adjoint,and,apply,as,auto,BigInt,body,Bool,borrowing,Controlled,controlled,Ctl,distribute,Double,elif,else,fail,false,fixup,for,function,if,in,Int,intrinsic,invert,is,let,mutable,namespace,new,newtype,not,One,open,operation,or,Pauli,PauliI,PauliX,PauliY,PauliZ,Qubit,Range,repeat,Result,return,self,set,String,true,Unit,until,using,while,within,Zero
	},
	morekeywords=[2]{
		Assert,AssertProb,CCNOT,CNOT,Exp,ExpFrac,H,I,M,Measure,Message,R,R1,R1Frac,Random,Reset,ResetAll,RFrac,Rx,Ry,Rz,S,SWAP,T,X,Y,Z
	},
	sensitive=true
}

% R
\lstdefinestyle{r}{
	language=R,
	alsoletter={.<-},
	alsoother={._$},
	deletekeywords={
		df,data,frame,length,as,character
	},
	morecomment=[l]\#,
	morestring=[d]',
	morestring=[d]",
	otherkeywords={
		!,!=,~,$,*,\&,\%/\%,\%*\%,\%\%,<-,<<-,/
	}
}

% Racket
\lstdefinestyle{racket}{
	alsoletter={',`,-,/,>,<,\#,\%},
	morekeywords=[1]{
		define,define-macro,define-stream,define-syntax,lambda,stream-lambda
	},
	morekeywords=[2]{
		->,always_publish,and,\#',\#\%module-begin,\#lang,\#`,begin,begin-for-syntax,Boolean,call-with-current-continuation,call-with-input-file,call-with-output-file,callback,call/cc,case,cond,define-context,define-controller,define-struct/contract,define/contract,delay,do,else,environment,eval,fold,for,for-each,force,get,if,implement,in-range,Integer,label,let,let*,let*-values,let-syntax,let-values,letrec,letrec-syntax,map,maybe_publish,message-box,module,new,not,or,or/c,parent,provide,quasiquote,query,quote,rename-out,require,send,submod,syntax,syntax-case,syntax-rules,unquote,unquote-splicing,when,when-provided,when-required,with-syntax
	},
	morekeywords=[3]{
		export,import
	},
	morecomment=[l]{;},
	moredelim=**[is][\color{lgray}]{<<@<<}{>>@>>},
	moredelim=**[is][\itshape\color{mauve}]{<<;<<}{>>;>>},
	morecomment=[s]{\#|}{|\#},
	morestring=[s]{"}{"},
	sensitive=true
}

% Reil
\lstdefinestyle{reil}{
	comment=[l]{;},
	keywords=[1]{
		ADD,add,and,AND,BISZ,bisz,bsh,BSH,div,DIV,jcc,JCC,LDM,ldm,MOD,mod,mul,MUL,nop,NOP,or,OR,stm,STM,STR,str,sub,SUB,undef,UNDEF,unkn,UNKN,XOR,xor
	},
	keywords=[3]{
		ah,AH,al,AL,AX,ax,bh,BH,BL,bl,bp,BP,bpl,BPL,BX,bx,ch,CH,cl,CL,cx,CX,DH,dh,di,DI,dil,DIL,dl,DL,DX,dx,EAX,eax,EBP,ebp,ebx,EBX,ECX,ecx,EDI,edi,edx,EDX,esi,ESI,esp,ESP,r8,R8,r8b,R8B,r8d,R8D,r8w,R8W,r9,R9,R9B,r9b,R9D,r9d,r9w,R9W,r10,R10,R10B,r10b,R10D,r10d,r10w,R10W,r11,R11,r11b,R11B,r11d,R11D,R11W,r11w,R12,r12,R12B,r12b,r12d,R12D,r12w,R12W,R13,r13,r13b,R13B,R13D,r13d,R13W,r13w,r14,R14,R14B,r14b,r14d,R14D,R14W,r14w,r15,R15,r15b,R15B,r15d,R15D,R15W,r15w,RAX,rax,rbp,RBP,rbx,RBX,RCX,rcx,RDI,rdi,rdx,RDX,RSI,rsi,RSP,rsp,SI,si,SIL,sil,SP,sp,spl,SPL
	},
	sensitive=true
}

% Ruby
\lstdefinestyle{ruby}{
	language=Ruby,
	breakatwhitespace=true,
	morestring=[s][]{\#\{}{\}},
	morestring=*[d]{"},
	sensitive=true
}

% Rust
\lstdefinelanguage{Rust}{
	sensitive,
	alsodigit={},
	alsoletter={!},
	alsoother={},
	morecomment=[l]{//},
	morecomment=[s]{/*}{*/},
	moredelim=[s][{\itshape\color[rgb]{0,0,0.75}}]{\#[}{]},
	morekeywords=[2]{ % Traits
		Add,AddAssign,Any,AsciiExt,AsInner,AsInnerMut,AsMut,AsRawFd,AsRawHandle,AsRawSocket,AsRef,Binary,BitAnd,BitAndAssign,Bitor,BitOr,BitOrAssign,BitXor,BitXorAssign,Borrow,BorrowMut,Boxed,BoxPlace,BufRead,BuildHasher,CastInto,CharExt,Clone,CoerceUnsized,CommandExt,Copy,Debug,DecodableFloat,Default,Deref,DerefMut,DirBuilderExt,DirEntryExt,Display,Div,DivAssign,DoubleEndedIterator,DoubleEndedSearcher,Drop,EnvKey,Eq,Error,ExactSizeIterator,ExitStatusExt,Extend,FileExt,FileTypeExt,Float,Fn,FnBox,FnMut,FnOnce,Freeze,From,FromInner,FromIterator,FromRawFd,FromRawHandle,FromRawSocket,FromStr,FullOps,FusedIterator,Generator,Hash,Hasher,Index,IndexMut,InPlace,Int,Into,IntoCow,IntoInner,IntoIterator,IntoRawFd,IntoRawHandle,IntoRawSocket,IsMinusOne,IsZero,Iterator,JoinHandleExt,LargeInt,LowerExp,LowerHex,MetadataExt,Mul,MulAssign,Neg,Not,Octal,OpenOptionsExt,Ord,OsStrExt,OsStringExt,Packet,PartialEq,PartialOrd,Pattern,PermissionsExt,Place,Placer,Pointer,Product,Put,RangeArgument,RawFloat,Read,Rem,RemAssign,Seek,Shl,ShlAssign,Shr,ShrAssign,Sized,SliceConcatExt,SliceExt,SliceIndex,Stats,Step,StrExt,Sub,SubAssign,Sum,Sync,TDynBenchFn,Terminal,Termination,ToOwned,ToSocketAddrs,ToString,Try,TryFrom,TryInto,UnicodeStr,Unsize,UpperExp,UpperHex,WideInt,Write
	},
	morekeywords=[2]{
		Send
	},
	morekeywords=[3]{ % Primitivas
		bool,char,f32,f64,i8,i16,i32,i64,isize,str,u8,u16,u32,u64,unit,usize,i128,u128
	},
	morekeywords=[4]{ % Valor y tipo de constructores
		Err,false,None,Ok,Some,true
	},
	morekeywords=[5]{ % Identificadores
		assert!,assert_eq!,assert_ne!,cfg!,column!,compile_error!,concat!,concat_idents!,debug_assert!,debug_assert_eq!,debug_assert_ne!,env!,eprint!,eprintln!,file!,format!,format_args!,include!,include_bytes!,include_str!,line!,module_path!,option_env!,panic!,print!,println!,select!,stringify!,thread_local!,try!,unimplemented!,unreachable!,vec!,write!,writeln!
	},
	morekeywords={ % Palabras reservadas
		abstract,alignof,become,box,do,final,macro,offsetof,override,priv, proc,pure,sizeof,typeof,unsized,virtual,yield
	},
	morekeywords={
		as,const,let,move,mut,ref,static
	},
	morekeywords={
		break,continue,else,for,if,in,loop,match,return,while
	},
	morekeywords={
		crate,extern,mod,pub,super
	},
	morekeywords={
		dyn,enum,fn,impl,Self,self,struct,trait,type,union,use,where
	},
	morekeywords={
		unsafe
	},
	morestring=[b]{"}
}
\lstdefinestyle{rust}{
	language=Rust,
	keywordstyle=[2]\color[rgb]{0.75,0,0}, % Traits
	keywordstyle=[3]\color[rgb]{0,0.5,0}, % Primitivas
	keywordstyle=[4]\color[rgb]{0,0.5,0}, % Valor y tipo de constructores
	keywordstyle=[5]\color[rgb]{0,0,0.75} % Macros
}

% Scala
\lstdefinestyle{scala}{
	language=scala,
	breakatwhitespace=true,
	morecomment=[l]{//},
	morecomment=[n]{/*}{*/},
	morekeywords={
		abstract,case,catch,class,def,do,else,extends,false,final,finally,for,if,implicit,import,match,mixin,new,null,object,override,package,private,protected,requires,return,sealed,super,this,throw,trait,true,try,type,val,var,while,with,yield
	},
	morestring=[b]',
	morestring=[b]",
	morestring=[b]""",
	otherkeywords={
		=>,<-,<\%,<:,>:,\#,@
	}
}

% Scheme
\lstdefinestyle{scheme}{
	language=Lisp,
	morecomment=[l]{;},
	morekeywords={
		and,begin,case,case-lambda,cond,cond-expand,define,delay,delay-force,do,else,force,guard,if,lambda,let,let*,let*-values,let-syntax,let-values,letrec,letrec*,letrec-syntax,make-parameter,make-promise,map,or,parameterize,promise?,quasiquote,quote,set!,syntax-rules,unless,when
	},
	morestring=[b]"
}

% Scilab
\lstdefinestyle{scilab}{
	language=Scilab
}

% Simula
\lstdefinestyle{simula}{
	language=Simula
}

% SPARQL
\lstdefinestyle{sparql}{
	language=SPARQL
}

% SQL
\lstdefinestyle{sql}{
	language=SQL,
	breakatwhitespace=true
}

% Swift
\lstdefinestyle{swift}{
	language=Swift
}

% TCL
\lstdefinestyle{tcl}{
	language=tcl,
	breakatwhitespace=false,
	keepspaces=true,
	morecomment=[l]{\#}
}

% Visual Basic
\lstdefinestyle{vbscript}{
	language=[Visual]Basic,
	extendedchars=true
}

% Verilog
\lstdefinestyle{verilog}{
	language=Verilog
}

% VDHL
\lstdefinelanguage{VHDL}{
	morekeywords=[1]{
		ALL,all,and,architecture,begin,downto,end,entity,in,is,library,Not,of,or,out,port,use
	},
	morekeywords=[2]{
		IEEE,NUMERIC_STD,STD_LOGIC,std_logic,STD_LOGIC_1164,STD_LOGIC_ARITH,STD_LOGIC_UNSIGNED,STD_LOGIC_VECTOR,std_logic_vector
	},
	morecomment=[l]--
}
\lstdefinestyle{vhdl}{
	language=VHDL
}

% XML
\lstdefinelanguage{XML}{
	morecomment=[s]{<?}{?>},
	morekeywords={
		encoding,type,version,xmlns
	},
	morestring=[b]",
	morestring=[s]{>}{<}
}
\lstdefinestyle{xml}{
	language=XML,
	tabsize=2
}

% -----------------------------------------------------------------------------
% Configuración de códigos fuente
% -----------------------------------------------------------------------------
\lstset{
	aboveskip=\sourcecodeskipabove em,
	basicstyle={\sourcecodefonts\sourcecodefontf\color{\maintextcolor}},
	belowskip=\sourcecodeskipbelow em,
	breaklines=true,
	columns=fullflexible,
	commentstyle=\color{dkgreen}\upshape,
	extendedchars=true,
	fontadjust=true,
	frame=ltb,
	framerule=0pt,
	framexbottommargin=\sourcecodebgmarginbottom pt,
	framexleftmargin=\sourcecodebgmarginleft pt,
	framexrightmargin=\sourcecodebgmarginright pt,
	framextopmargin=\sourcecodebgmargintop pt,
	identifierstyle=\color{\maintextcolor},
	keepspaces=true,
	keywordstyle=\color{blue},
	literate={á}{{\'a}}1 {é}{{\'e}}1 {í}{{\'i}}1 {ó}{{\'o}}1 {ú}{{\'u}}1
		{Á}{{\'A}}1 {É}{{\'E}}1 {Í}{{\'I}}1 {Ó}{{\'O}}1 {Ú}{{\'U}}1 {à}{{\`a}}1
		{è}{{\`e}}1 {ì}{{\`i}}1 {ò}{{\`o}}1 {ù}{{\`u}}1 {À}{{\`A}}1 {È}{{\'E}}1
		{Ì}{{\`I}}1 {Ò}{{\`O}}1 {Ù}{{\`U}}1 {ä}{{\"a}}1 {ë}{{\"e}}1 {ï}{{\"i}}1
		{ö}{{\"o}}1 {ü}{{\"u}}1 {Ä}{{\"A}}1 {Ë}{{\"E}}1 {Ï}{{\"I}}1 {Ö}{{\"O}}1
		{Ü}{{\"U}}1 {â}{{\^a}}1 {ê}{{\^e}}1 {î}{{\^i}}1 {ô}{{\^o}}1 {û}{{\^u}}1
		{Â}{{\^A}}1 {Ê}{{\^E}}1 {Î}{{\^I}}1 {Ô}{{\^O}}1 {Û}{{\^U}}1 {œ}{{\oe}}1
		{Œ}{{\OE}}1 {æ}{{\ae}}1 {Æ}{{\AE}}1 {ß}{{\ss}}1 {ű}{{\H{u}}}1
		{Ű}{{\H{U}}}1 {ő}{{\H{o}}}1 {Ő}{{\H{O}}}1 {ç}{{\c c}}1 {Ç}{{\c C}}1
		{ø}{{\o}}1 {å}{{\r a}}1 {Å}{{\r A}}1 {€}{{\EUR}}1 {£}{{\pounds}}1
		{ñ}{{\~n}}1 {Ñ}{{\~N}}1 {¿}{{?``}}1 {¡}{{!``}}1 {«}{{\guillemotleft}}1
		{»}{{\guillemotright}}1 {°}{{\textdegree}}1 {∢}{{$\sphericalangle$}}1
		{¬}{{$\neg$}}1 {¨}{{\textasciidieresis}}1 {ã}{{\~a}}1 {Ã}{{\~a}}1
		{õ}{{\~o}}1 {Õ}{{\~O}}1 {Ð}{{\DJ}}1 {Ø}{{\O}}1 {Ý}{{\'Y}}1
		{¹}{{\textsuperscript{1}}}1 {²}{{\textsuperscript{2}}}1
		{³}{{\textsuperscript{3}}}1 {⁴}{{\textsuperscript{4}}}1
		{⁵}{{\textsuperscript{5}}}1 {⁶}{{\textsuperscript{6}}}1
		{⁷}{{\textsuperscript{7}}}1 {⁸}{{\textsuperscript{8}}}1
		{⁹}{{\textsuperscript{9}}}1 {⁰}{{\textsuperscript{0}}}1
		{ᵃ}{{\textsuperscript{a}}}1 {ᵇ}{{\textsuperscript{b}}}1
		{ᶜ}{{\textsuperscript{c}}}1 {ᵈ}{{\textsuperscript{d}}}1
		{ᵉ}{{\textsuperscript{e}}}1 {ᶠ}{{\textsuperscript{f}}}1
		{ᵍ}{{\textsuperscript{g}}}1 {ʰ}{{\textsuperscript{h}}}1
		{ᶦ}{{\textsuperscript{i}}}1 {ʲ}{{\textsuperscript{j}}}1
		{ᵏ}{{\textsuperscript{k}}}1 {ˡ}{{\textsuperscript{l}}}1
		{ᵐ}{{\textsuperscript{m}}}1 {ⁿ}{{\textsuperscript{n}}}1
		{ᵒ}{{\textsuperscript{o}}}1 {ᵖ}{{\textsuperscript{p}}}1
		{ᵠ}{{\textsuperscript{q}}}1 {ʳ}{{\textsuperscript{r}}}1
		{ˢ}{{\textsuperscript{s}}}1 {ᵗ}{{\textsuperscript{t}}}1
		{ᵘ}{{\textsuperscript{u}}}1 {ᵛ}{{\textsuperscript{v}}}1
		{ʷ}{{\textsuperscript{w}}}1 {ˣ}{{\textsuperscript{x}}}1
		{ʸ}{{\textsuperscript{y}}}1 {ᶻ}{{\textsuperscript{z}}}1
		{α}{{$\alpha$}}1 {ά}{{$\dot \alpha$}}1 {Γ}{{$\Gamma$}}1 {Þ}{{$\Thorn$}}1
		{γ}{{$\gamma$}}1 {Δ}{{$\Delta$}}1 {δ}{{$\delta$}}1 {þ}{{$\thorn$}}1
		{ε}{{$\epsilon$}}1 {έ}{{$\dot \epsilon$}}1 {ζ}{{$\zeta$}}1
		{η}{{$\eta$}}1 {ή}{{$\dot \eta$}}1 {Θ}{{$\Theta$}}1
		{θ}{{$\theta$}}1 {ι}{{$\iota$}}1 {ί}{{$\dot \iota$}}1
		{Ϊ}{{$\ddot I$}}1 {ϊ}{{$\ddot \iota$}}1 {ΐ}{{$\dddot \iota$}}1
		{κ}{{$\kappa$}}1 {Λ}{{$\Lambda$}}1 {λ}{{$\lambda$}}1
		{μ}{{$\mu$}}1 {ν}{{$\nu$}}1 {Ξ}{{$\Xi$}}1 {ξ}{{$\xi$}}1
		{ό}{{$\dot o$}}1 {Π}{{$\Pi$}}1 {π}{{$\pi$}}1 {ρ}{{$\rho$}}1
		{Σ}{{$\Sigma$}}1 {σ}{{$\sigma$}}1 {ς}{{$\varsigma$}}1 {τ}{{$\tau$}}1
		{υ}{{$\upsilon$}}1 {ύ}{{$\dot \upsilon$}}1 {Ϋ}{{$\ddot Y$}}1
		{ϋ}{{$\ddot \upsilon$}}1 {ΰ}{{$\dddot \upsilon$}}1
		{Φ}{{$\Phi$}}1 {φ}{{$\phi$}}1 {Ψ}{{$\Psi$}}1 {ψ}{{$\psi$}}1
		{Ω}{{$\Omega$}}1 {ω}{{$\omega$}}1 {ώ}{{$\dot \omega$}}1
		{“}{{``}}1 {”}{{''}}1 {…}{{$\ldots$}}1 {`}{\`}1 {–}{{--}}1 { }{{ }}1,
	numbers=left,
	numbersep={\sourcecodenumbersep pt},
	numberstyle=\sourcecodenumbersize\color{dkgray},
	postbreak=\mbox{$\hookrightarrow$\space},
	showspaces=false,
	showstringspaces=false,
	showtabs=false,
	stepnumber=1,
	stringstyle=\color{mauve},
	tabsize={\sourcecodetabsize}
}

% -----------------------------------------------------------------------------
% Chequeo de estilos, cualquier nuevo estilo añadirlo a esta lista
% -----------------------------------------------------------------------------
\newcommand{\checkvalidsourcecodestyle}[1]{%
	\ifthenelse{\equal{#1}{abap}}{}{%
	\ifthenelse{\equal{#1}{ada}}{}{%
	\ifthenelse{\equal{#1}{assemblerx64}}{}{%
	\ifthenelse{\equal{#1}{assemblerx86}}{}{%
	\ifthenelse{\equal{#1}{awk}}{}{%
	\ifthenelse{\equal{#1}{bash}}{}{%
	\ifthenelse{\equal{#1}{basic}}{}{%
	\ifthenelse{\equal{#1}{c}}{}{%
	\ifthenelse{\equal{#1}{caml}}{}{%
	\ifthenelse{\equal{#1}{cmake}}{}{%
	\ifthenelse{\equal{#1}{cobol}}{}{%
	\ifthenelse{\equal{#1}{cpp}}{}{%
	\ifthenelse{\equal{#1}{csharp}}{}{%
	\ifthenelse{\equal{#1}{css}}{}{%
	\ifthenelse{\equal{#1}{csv}}{}{%
	\ifthenelse{\equal{#1}{cuda}}{}{%
	\ifthenelse{\equal{#1}{dart}}{}{%
	\ifthenelse{\equal{#1}{docker}}{}{%
	\ifthenelse{\equal{#1}{elisp}}{}{%
	\ifthenelse{\equal{#1}{elixir}}{}{%
	\ifthenelse{\equal{#1}{erlang}}{}{%
	\ifthenelse{\equal{#1}{fortran}}{}{%
	\ifthenelse{\equal{#1}{fsharp}}{}{%
	\ifthenelse{\equal{#1}{glsl}}{}{%
	\ifthenelse{\equal{#1}{gnuplot}}{}{%
	\ifthenelse{\equal{#1}{go}}{}{%
	\ifthenelse{\equal{#1}{haskell}}{}{%
	\ifthenelse{\equal{#1}{html}}{}{%
	\ifthenelse{\equal{#1}{ini}}{}{%
	\ifthenelse{\equal{#1}{java}}{}{%
	\ifthenelse{\equal{#1}{javascript}}{}{%
	\ifthenelse{\equal{#1}{json}}{}{%
	\ifthenelse{\equal{#1}{julia}}{}{%
	\ifthenelse{\equal{#1}{kotlin}}{}{%
	\ifthenelse{\equal{#1}{latex}}{}{%
	\ifthenelse{\equal{#1}{lisp}}{}{%
	\ifthenelse{\equal{#1}{llvm}}{}{%
	\ifthenelse{\equal{#1}{lua}}{}{%
	\ifthenelse{\equal{#1}{make}}{}{%
	\ifthenelse{\equal{#1}{maple}}{}{%
	\ifthenelse{\equal{#1}{mathematica}}{}{%
	\ifthenelse{\equal{#1}{matlab}}{}{%
	\ifthenelse{\equal{#1}{mercury}}{}{%
	\ifthenelse{\equal{#1}{modula2}}{}{%
	\ifthenelse{\equal{#1}{objectivec}}{}{%
	\ifthenelse{\equal{#1}{octave}}{}{%
	\ifthenelse{\equal{#1}{opencl}}{}{%
	\ifthenelse{\equal{#1}{opensees}}{}{%
	\ifthenelse{\equal{#1}{pascal}}{}{%
	\ifthenelse{\equal{#1}{perl}}{}{%
	\ifthenelse{\equal{#1}{php}}{}{%
	\ifthenelse{\equal{#1}{plaintext}}{}{%
	\ifthenelse{\equal{#1}{postscript}}{}{%
	\ifthenelse{\equal{#1}{powershell}}{}{%
	\ifthenelse{\equal{#1}{prolog}}{}{%
	\ifthenelse{\equal{#1}{promela}}{}{%
	\ifthenelse{\equal{#1}{pseudocode}}{}{%
	\ifthenelse{\equal{#1}{pseudocodecolor}}{}{%
	\ifthenelse{\equal{#1}{python}}{}{%
	\ifthenelse{\equal{#1}{qsharp}}{}{%
	\ifthenelse{\equal{#1}{r}}{}{%
	\ifthenelse{\equal{#1}{racket}}{}{%
	\ifthenelse{\equal{#1}{reil}}{}{%
	\ifthenelse{\equal{#1}{ruby}}{}{%
	\ifthenelse{\equal{#1}{rust}}{}{%
	\ifthenelse{\equal{#1}{scala}}{}{%
	\ifthenelse{\equal{#1}{scheme}}{}{%
	\ifthenelse{\equal{#1}{scilab}}{}{%
	\ifthenelse{\equal{#1}{simula}}{}{%
	\ifthenelse{\equal{#1}{sparql}}{}{%
	\ifthenelse{\equal{#1}{sql}}{}{%
	\ifthenelse{\equal{#1}{swift}}{}{%
	\ifthenelse{\equal{#1}{tcl}}{}{%
	\ifthenelse{\equal{#1}{vbscript}}{}{%
	\ifthenelse{\equal{#1}{verilog}}{}{%
	\ifthenelse{\equal{#1}{vhdl}}{}{%
	\ifthenelse{\equal{#1}{xml}}{}{%
		\errmessage{LaTeX Warning: Estilo de codigo desconocido. Valores esperados: abap,ada,assemblerx64,assemblerx86,awk,bash,basic,c,caml,cmake,cobol,cpp,csharp,css,csv,cuda,dart,docker,elisp,elixir,erlang,fortran,fsharp,glsl,gnuplot,go,haskell,html,ini,java,javascript,json,julia,kotlin,latex,lisp,llvm,lua,make,maple,mathematica,matlab,mercury,modula2,objectivec,octave,opencl,opensees,pascal,perl,php,plaintext,postscript,powershell,prolog,promela,pseudocode,pseudocodecolor,python,qsharp,r,racket,reil,ruby,rust,scala,scheme,scilab,simula,sparql,sql,swift,tcl,vbscript,verilog,vhdl,xml}%
		\stop%
	}}}}}}}}}}}}}}}}}}}}}}}}}}}}}}}}}}}}}}}}}}}}}}}}}}}}}}}}}}}}}}}}}}}}}}}}}}}}}%
}

% Crea un entorno de código inline
%	#1	Estilo de código
%	#2	Código a insertar
\newcommand{\inlinesourcecode}[2]{%
	\inlinesourcecodeboxed[NOCOLOR]{#1}{#2}%
}

% Crea un entorno de código inline dentro de un recuadro de color
%	#1	Color del recuadro
%	#2	Estilo de código
%	#3	Código a insertar
\newcommand{\inlinesourcecodeboxed}[3][]{%
	\emptyvarerr{\inlinesourcecodeboxed}{#2}{Estilo de codigo no definido}%
	\emptyvarerr{\inlinesourcecodeboxed}{#3}{Codigo no definido}%
	\lstset{%
		basicstyle={\sourcecodeilfonts\sourcecodeilfontf\color{\maintextcolor}}%
	}%
	\checkvalidsourcecodestyle{#2}%
	\ifthenelse{\equal{#1}{}}{%
		\Colorbox{\sourcecodebgcolor}{\lstinline[style=#2]!#3!}%
	}{%
	\ifthenelse{\equal{#1}{NOCOLOR}}{%
		\lstinline[style=#2]!#3!%
	}{%
		\Colorbox{#1}{\lstinline[style=#2]!#3!}%
	}}%
	\lstset{%
		basicstyle={\sourcecodefonts\sourcecodefontf\color{\maintextcolor}}%
	}%
}

% Inserta una referencia en un código fuente
% 	#1	Referencia
\newcommand{\coderef}[1]{%
	\ensuremath{\text{\ref{#1}}}%
}

% Inserta una referencia en un código fuente
% 	#1	Referencia
\newcommand{\codeeqref}[1]{%
	\ensuremath{\text{(\ref{#1})}}%
}

% -----------------------------------------------------------------------------
% Estilo de enumeración en griego
% -----------------------------------------------------------------------------
\makeatletter
\def\greek#1{\expandafter\@greek\csname c@#1\endcsname}
\def\Greek#1{\expandafter\@Greek\csname c@#1\endcsname}
\def\@greek#1{%
	\ifcase#1%
		\or $\alpha$%
		\or $\beta$%
		\or $\gamma$%
		\or $\delta$%
		\or $\epsilon$%
		\or $\zeta$%
		\or $\eta$%
		\or $\theta$%
		\or $\iota$%
		\or $\kappa$%
		\or $\lambda$%
		\or $\mu$%
		\or $\nu$%
		\or $\xi$%
		\or $o$%
		\or $\pi$%
		\or $\rho$%
		\or $\sigma$%
		\or $\tau$%
		\or $\upsilon$%
		\or $\phi$%
		\or $\chi$%
		\or $\psi$%
		\or $\omega$%
	\fi%
}
\def\@Greek#1{%
	\ifcase#1%
		\or $\mathrm{A}$%
		\or $\mathrm{B}$%
		\or $\Gamma$%
		\or $\Delta$%
		\or $\mathrm{E}$%
		\or $\mathrm{Z}$%
		\or $\mathrm{H}$%
		\or $\Theta$%
		\or $\mathrm{I}$%
		\or $\mathrm{K}$%
		\or $\Lambda$%
		\or $\mathrm{M}$%
		\or $\mathrm{N}$%
		\or $\Xi$%
		\or $\mathrm{O}$%
		\or $\Pi$%
		\or $\mathrm{P}$%
		\or $\Sigma$%
		\or $\mathrm{T}$%
		\or $\mathrm{Y}$%
		\or $\Phi$%
		\or $\mathrm{X}$%
		\or $\Psi$%
		\or $\Omega$%
	\fi%
}
\makeatother
\AddEnumerateCounter{\greek}{\@greek}{24}
\AddEnumerateCounter{\Greek}{\@Greek}{12}

% -----------------------------------------------------------------------------
% CONFIGURACIÓN INICIAL DEL DOCUMENTO
% -----------------------------------------------------------------------------
% Se revisa si las variables no han sido borradas
\def\documentsubject {}
\def\predocpageromannumber {true}
\def\predocresetpagenumber {true}
\def\indexnewpagec {false}
\def\indexnewpagef {false}
\def\indexnewpaget {false}
\def\indexnewpagee {false}
\def\showindex {true}
\def\showindexofcontents {true}
\def\coursecode {}
\def\coursename {}
\def\indexsectionfontsize {\sectionfontsize}
\def\indexsectionstyle {\sectionfontstyle}

\checkvardefined{\coursecode}
\checkvardefined{\documentauthor}
\checkvardefined{\documentsubject}
\checkvardefined{\documenttitle}
\checkvardefined{\universitydepartment}
\checkvardefined{\universitydepartmentimagecfg}
\checkvardefined{\universityfaculty}
\checkvardefined{\universitylocation}
\checkvardefined{\universityname}

% -----------------------------------------------------------------------------
% Se añade \xspace a las variables
% -----------------------------------------------------------------------------
\makeatletter
	\g@addto@macro\coursecode\xspace
	\g@addto@macro\coursename\xspace
	\g@addto@macro\documentauthor\xspace
	\g@addto@macro\documentsubject\xspace
	\g@addto@macro\documenttitle\xspace
	\g@addto@macro\universitydepartment\xspace
	\g@addto@macro\universityfaculty\xspace
	\g@addto@macro\universitylocation\xspace
	\g@addto@macro\universityname\xspace
\makeatother

% -----------------------------------------------------------------------------
% Se crean variables si se borraron
% -----------------------------------------------------------------------------
\ifthenelse{\isundefined{\documentsubtitle}}{
	\errmessage{LaTeX Warning: Se borro la variable \noexpand\documentsubtitle, creando una vacia}
	\def\documentsubtitle {}}{
}

\ifthenelse{\equal{\documentsubtitle}{}}{
	\def\documenttitlehf {\documenttitle}
}{
	\def\documenttitlehf {\documentsubtitle}
}

% -----------------------------------------------------------------------------
% Se activan números en menú marcadores del pdf
% -----------------------------------------------------------------------------
\ifthenelse{\equal{\cfgpdfsecnumbookmarks}{true}}{
	\bookmarksetup{numbered}}{
}

% -----------------------------------------------------------------------------
% Se define metadata del pdf
% -----------------------------------------------------------------------------
\ifthenelse{\equal{\cfgshowbookmarkmenu}{true}}{
	\def\cfgpdfpagemode {UseOutlines}
	}{
	\def\cfgpdfpagemode {UseNone}
}
\ifthenelse{\equal{\usepdfmetadata}{true}}{
	\def\pdfmetainfoauthor {\documentauthor}
	\def\pdfmetainfocoursecode {\coursecode}
	\def\pdfmetainfocoursename {\coursename}
	\def\pdfmetainfosubject {\documentsubject}
	\def\pdfmetainfotitle {\documenttitle}
	\def\pdfmetainfouniversity {\universityname}
	\def\pdfmetainfouniversitydepartment {\universitydepartment}
	\def\pdfmetainfouniversityfaculty {\universityfaculty}
	\def\pdfmetainfouniversitylocation {\universitylocation}
	\author{\pdfmetainfoauthor}
	\title{\pdfmetainfotitle}
}{
	\def\pdfmetainfoauthor {}
	\def\pdfmetainfocoursecode {}
	\def\pdfmetainfocoursename {}
	\def\pdfmetainfosubject {}
	\def\pdfmetainfotitle {}
	\def\pdfmetainfouniversity {}
	\def\pdfmetainfouniversitydepartment {}
	\def\pdfmetainfouniversityfaculty {}
	\def\pdfmetainfouniversitylocation {}
}
\hypersetup{
	keeppdfinfo,
	bookmarksopen={\cfgpdfbookmarkopen},
	bookmarksopenlevel={\cfgbookmarksopenlevel},
	bookmarkstype={toc},
	pdfauthor={\pdfmetainfoauthor},
	pdfcenterwindow={\cfgpdfcenterwindow},
	pdfcopyright={\cfgpdfcopyright},
	pdfcreator={LaTeX},
	pdfdisplaydoctitle={\cfgpdfdisplaydoctitle},
	pdfencoding={unicode},
	pdffitwindow={\cfgpdffitwindow},
	pdfinfo={
		Course.Code={\pdfmetainfocoursecode},
		Course.Name={\pdfmetainfocoursename},
		Document.Author={\pdfmetainfoauthor},
		Document.Subject={\pdfmetainfosubject},
		Document.Title={\pdfmetainfotitle},
		Template.Author.Alias={ppizarror},
		Template.Author.Email={pablo@ppizarror.com},
		Template.Author.Web={https://ppizarror.com},
		Template.Author={Pablo Pizarro R.},
		Template.Date={29/04/2023},
		Template.Encoding={UTF-8},
		Template.Latex.Compiler={pdflatex},
		Template.License.Type={MIT},
		Template.License.Web={https://opensource.org/licenses/MIT},
		Template.Name={Template-Tesis},
		Template.Type={Normal},
		Template.Version.Dev={3.2.6-2-THS},
		Template.Version.Hash={21A68ACE232EB5A5944E5C715DAE3171},
		Template.Version.Release={3.2.6},
		Template.Web.Dev={https://github.com/Template-Latex/Template-Tesis},
		Template.Web.Manual={https://latex.ppizarror.com/tesis},
		University.Department={\pdfmetainfouniversitydepartment},
		University.Faculty={\pdfmetainfouniversityfaculty},
		University.Location={\pdfmetainfouniversitylocation},
		University.Name={\pdfmetainfouniversity}
	},
	pdfkeywords={\cfgpdfkeywords},
	pdfmenubar={\cfgpdfmenubar},
	pdfpagelayout={\cfgpdflayout},
	pdfpagemode={\cfgpdfpagemode},
	pdfproducer={Template-Tesis v3.2.6 | (Pablo Pizarro R.) ppizarror.com},
	pdfremotestartview={Fit},
	pdfstartpage={1},
	pdfstartview={\cfgpdfpageview},
	pdfsubject={\pdfmetainfosubject},
	pdftitle={\pdfmetainfotitle},
	pdftoolbar={\cfgpdftoolbar}
}

% -----------------------------------------------------------------------------
% Establece la carpeta de imágenes por defecto
% -----------------------------------------------------------------------------
\graphicspath{{./\defaultimagefolder}}

% -----------------------------------------------------------------------------
% Elimina el espacio vertical de los flotantes
% -----------------------------------------------------------------------------
\makeatletter
\ifthenelse{\equal{\fpremovetopbottomcenter}{true}}{
	\setlength{\@fptop}{0pt}
	\setlength{\@fpbot}{0pt}
}{}
\makeatother

% -----------------------------------------------------------------------------
% Definición de valores e dimensiones
% -----------------------------------------------------------------------------
\setstretch{\documentinterline} % Ajuste del entrelineado
\setlength{\headheight}{64 pt} % Tamaño de la cabecera sin fancyhdr
\setlength{\columnsep}{\columnsepwidth em} % Separación entre columnas
\ifthenelse{\equal{\showlinenumbers}{true}}{
	\setlength{\linenumbersep}{\marginlinenumbers pt}
	\renewcommand\linenumberfont{\normalfont\tiny\color{\linenumbercolor}}
	}{
}

% -----------------------------------------------------------------------------
% Posición inicial de los objetos
% -----------------------------------------------------------------------------
\floatplacement{figure}{\imagedefaultplacement}
\floatplacement{table}{\tabledefaultplacement}
\floatplacement{tikz}{\tikzdefaultplacement}

% -----------------------------------------------------------------------------
% Configuración de los colores
% -----------------------------------------------------------------------------
\color{\maintextcolor} % Color principal
\arrayrulecolor{\tablelinecolor} % Color de las líneas de las tablas
\sethlcolor{\highlightcolor} % Color del subrayado por defecto
\ifthenelse{\equal{\showborderonlinks}{true}}{
	% Color de links con borde
	\hypersetup{
		citebordercolor=\numcitecolor,
		linkbordercolor=\linkcolor,
		urlbordercolor=\urlcolor
	}
}{
	% Color de links sin borde
	\hypersetup{ % No reorganizar
		hidelinks,
		colorlinks=true,
		citecolor=\numcitecolor,
		filecolor=\urlcolor,
		linkcolor=\linkcolor,
		urlcolor=\urlcolor
	}
}
\ifthenelse{\equal{\pagescolor}{white}}{}{
	\pagecolor{\pagescolor}
}

% -----------------------------------------------------------------------------
% Configuración de las leyendas
% -----------------------------------------------------------------------------
% Márgenes de las leyendas por defecto
\setcaptionmargincm{\captionlrmargin}
\ifthenelse{\equal{\captiontextbold}{true}}{ % Texto en negrita en etiquetas
	\renewcommand{\captiontextbold}{bf}}{
	\renewcommand{\captiontextbold}{}
}
\ifthenelse{\equal{\captiontextsubnumbold}{true}}{ % Número en negritas
	\renewcommand{\captiontextsubnumbold}{bf}}{
	\renewcommand{\captiontextsubnumbold}{}
}

% Se configura el texto de los caption
\corecheckfontsize{\captionfontsize}
\captionsetup{
	font={\captionfontsize},
	labelfont={color=\captioncolor, \captiontextbold},
	labelformat={\captionlabelformat},
	labelsep={\captionlabelsep},
	textfont={color=\captiontextcolor},
	singlelinecheck=on
}

% Configura texto de los subcaption
\corecheckfontsize{\subcaptionfsize}
\captionsetup*[subfigure]{
	font={\subcaptionfsize},
	labelfont={color=\captioncolor, \captiontextsubnumbold},
	labelformat={\subcaptionlabelformat},
	labelsep={\subcaptionlabelsep},
	lofdepth=1,
	textfont={color=\captiontextcolor},
	singlelinecheck=on
}
\captionsetup*[subtable]{
	font={\subcaptionfsize},
	labelfont={color=\captioncolor, \captiontextsubnumbold},
	labelformat={\subcaptionlabelformat},
	labelsep={\subcaptionlabelsep},
	lofdepth=1,
	textfont={color=\captiontextcolor},
	singlelinecheck=on
}

\makeatletter
\renewcommand\p@subfigure{\thefigure\captionsubchar}
\renewcommand\p@subtable{\thetable\captionsubchar}
\makeatother

% Configuración de márgenes en las figuras
\floatsetup[figure]{
	captionskip=\captiontbmarginfigure pt
}

% Configuración de márgenes en las tablas
\floatsetup[table]{
	captionskip=\captiontbmargintable pt
}

% Caption superior en figuras
\ifthenelse{\equal{\figurecaptiontop}{true}}{
	\floatsetup[figure]{position=above}}{
}

% Caption superior en tablas
\ifthenelse{\equal{\tablecaptiontop}{true}}{
	\floatsetup[table]{position=top}
	}{
	\floatsetup[table]{position=bottom}
}

% Alineado de leyendas
\ifthenelse{\equal{\captionalignment}{justified}}{ % Leyenda justificada
	\captionsetup{
		format=plain,
		justification=justified
	}
}{
\ifthenelse{\equal{\captionalignment}{centered}}{ % Leyenda centrada
	\captionsetup{
		justification=centering
	}
}{
\ifthenelse{\equal{\captionalignment}{left}}{ % Leyenda alineada a la izquierda
	\captionsetup{
		justification=raggedright,
		singlelinecheck=false
	}
}{
\ifthenelse{\equal{\captionalignment}{right}}{ % Leyenda alineada a la derecha
	\captionsetup{
		justification=raggedleft,
		singlelinecheck=false
	}
}{
	\throwbadconfig{Posicion de leyendas desconocida}{\captionalignment}{justified,centered,left,right}}}}
}

% -----------------------------------------------------------------------------
% Configuración de referencias y citas
% -----------------------------------------------------------------------------
\ifthenelse{\equal{\stylecitereferences}{natbib}}{
	\def\twocolumnreferencesmargin{-0.35cm}
	\bibliographystyle{\natbibrefstyle}
	\setlength{\bibsep}{\natbibrefsep pt}
	\newcommand{\shortcite}[1]{\citep{#1}}
	\newcommand{\fullcite}[1]{\citet{#1}}
	% Caracteres citas
	\setcitestyle{open={\natbibrefcitecharopen},close={\natbibrefcitecharclose}}
	% Separador citas
	\ifthenelse{\equal{\natbibrefcitesepcomma}{true}}{
		\setcitestyle{comma}
	}{
		\setcitestyle{semicolon}
	}
	% Tipo citas
	\ifthenelse{\equal{\natbibrefcitetype}{numbers}}{
		\setcitestyle{numbers}
	}{
	\ifthenelse{\equal{\natbibrefcitetype}{authoryear}}{
		\setcitestyle{authoryear}
	}{
	\ifthenelse{\equal{\natbibrefcitetype}{super}}{
		\setcitestyle{super}
	}{
		\throwbadconfig{Tipo cita natbib desconocido}{\natbibrefcitetype}{numbers,authoryear,super}}}
	}
}{
\ifthenelse{\equal{\stylecitereferences}{apacite}}{
	\def\twocolumnreferencesmargin{-0.39cm}
	\bibliographystyle{\apacitestyle}
	\setlength{\bibitemsep}{\apaciterefsep pt}
	\newcommand{\citep}[1]{\fullcite{#1}}
	\newcommand{\citet}[1]{\shortcite{#1}}
}{
\ifthenelse{\equal{\stylecitereferences}{bibtex}}{
	\def\twocolumnreferencesmargin{-0.35cm}
	\bibliographystyle{\bibtexstyle}
	\newlength{\bibitemsep}
	\setlength{\bibitemsep}{.2\baselineskip plus .05\baselineskip minus .05\baselineskip}
	\newlength{\bibparskip}\setlength{\bibparskip}{0pt}
	\ifthenelse{\equal{\bibtexindexbibliography}{true}}{
		\let\oldbibliography\bibliography
		\renewcommand{\bibliography}[1]{
			\clearpage
			\phantomsection
			\addcontentsline{toc}{chapter}{\namereferences} % bibtex tesis en chapter
			\oldbibliography{#1}}}{
	}
	\let\oldthebibliography\thebibliography
	\renewcommand\thebibliography[1]{
		\oldthebibliography{#1}
		\setlength{\parskip}{\bibitemsep}
		\setlength{\itemsep}{\bibparskip}
	}
	\setlength{\bibitemsep}{\bibtexrefsep pt}
}{
\ifthenelse{\equal{\stylecitereferences}{custom}}{
	\coretemplatemessage{Usando estilo citas referencias custom, importar librerias y configuraciones posterior al llamado de template.tex en archivo principal}
}{
	\throwbadconfig{Estilo citas desconocido}{\stylecitereferences}{bibtex,apacite,natbib,custom}}}}
}

% Crea referencias enumeradas en apacite
\makeatletter
\ifthenelse{\equal{\stylecitereferences}{apacite}}{
	\ifthenelse{\equal{\apaciterefnumber}{true}}{
		\newcounter{apaciteNumberCounter}
		\renewcommand{\theapaciteNumberCounter}{ % Formato de número
			\apaciterefcitecharopen\arabic{apaciteNumberCounter}\apaciterefcitecharclose
		}
		\patchcmd{\@lbibitem}{\item[}{\item[\stepcounter{apaciteNumberCounter}{\hss\llap{\theapaciteNumberCounter}\quad}}{}{}
		\setlength{\bibleftmargin}{2.54em}
		\setlength{\bibindent}{-0.54em}
	}{}
}{}
\makeatother

% Desactiva la URL de apacite
\ifthenelse{\equal{\stylecitereferences}{apacite}}{
	\ifthenelse{\equal{\apaciteshowurl}{false}}{
		\renewenvironment{APACrefURL}[1][]{}{}
		\AtBeginEnvironment{APACrefURL}{\renewcommand{\url}[1]{}}
		\renewcommand{\doiprefix}{doi:~\kern-1pt}
	}{}
}{}

% Referencias en 2 columnas
\makeatletter
\ifthenelse{\equal{\twocolumnreferences}{true}}{
	\renewenvironment{thebibliography}[1]
	{\begin{multicols}{2}[\chapter*{\refname}]
		\@mkboth{\MakeUppercase\refname}{\MakeUppercase\refname}
		\list{\@biblabel{\@arabic\c@enumiv}}
		{\settowidth\labelwidth{\@biblabel{#1}}
			\leftmargin\labelwidth
			\advance\leftmargin\labelsep
			\@openbib@code
			\usecounter{enumiv}
			\let\p@enumiv\@empty
			\renewcommand\theenumiv{\@arabic\c@enumiv}}
		\sloppy
		\clubpenalty 4000
		\@clubpenalty \clubpenalty
		\widowpenalty 4000
		\sfcode`\.\@m}
		{\def\@noitemerr
		{\@latex@warning{Ambiente `thebibliography' no definido}}
		\endlist\end{multicols}}}{}
\makeatother

% -----------------------------------------------------------------------------
% Configuración anexo
% -----------------------------------------------------------------------------
\patchcmd{\appendices}{\quad}{\charappendixsection\spacingaftersection}{}{}

% -----------------------------------------------------------------------------
% Se añade listings (código fuente) a tocloft
% -----------------------------------------------------------------------------
\begingroup
	\makeatletter
	\let\newcounter\@gobble\let\setcounter\@gobbletwo
	\globaldefs\@ne\let\c@loldepth\@ne
	\newlistof{listings}{lol}{\lstlistlistingname}
	\newlistentry{lstlisting}{lol}{0}
	\makeatother
\endgroup

% -----------------------------------------------------------------------------
% Crea índice de ecuaciones
% -----------------------------------------------------------------------------
\newcommand{\listindexequationsname}{\namelteqn}
\newlistof{myindexequations}{equ}{\listindexequationsname}
\newcommand{\myindexequations}[1]{
	\addcontentsline{equ}{myindexequations}{\protect\numberline{\theequation}#1}
}
\setcounter{templateIndexEquations}{0}
\DeclareTotalCounter{templateIndexEquations}

% -----------------------------------------------------------------------------
% Reconfiguración de tamaño de páginas
% -----------------------------------------------------------------------------
\makeatletter
	\def\ifGm@preamble#1{\@firstofone}
	\appto\restoregeometry{
		\pdfpagewidth=\paperwidth
		\pdfpageheight=\paperheight}
	\apptocmd\newgeometry{
		\pdfpagewidth=\paperwidth
		\pdfpageheight=\paperheight}{}{}
\makeatother

% -----------------------------------------------------------------------------
% Configuración de hbox y vbox
% -----------------------------------------------------------------------------
\hfuzz=200pt
\vfuzz=200pt
\hbadness=\maxdimen
\vbadness=\maxdimen

% -----------------------------------------------------------------------------
% Configura las fuentes
% -----------------------------------------------------------------------------
\makeatletter
\def\Hv@scale {.95}
\makeatother

% -----------------------------------------------------------------------------
% Configuraciones de las tablas
% -----------------------------------------------------------------------------
\makeatletter % Reinicia el número de cada fila en todas las tablas
\preto\tabular{\global\rownum=\z@}
\preto\tabularx{\global\rownum=\z@}
\makeatother

% -----------------------------------------------------------------------------
% Se activa el word-wrap para textos con \texttt{}
% -----------------------------------------------------------------------------
\ttfamily \hyphenchar\the\font=`\-

% -----------------------------------------------------------------------------
% Se define el tipo de texto de los url
% -----------------------------------------------------------------------------
\urlstyle{\fonturl}

% -----------------------------------------------------------------------------
% Configuraciones del motor de compilación
% -----------------------------------------------------------------------------
\ifthenelse{\equal{\compilertype}{pdf2latex}}{
	% Nivel de compresión
	\pdfcompresslevel=\pdfcompilecompression
	
	% El óptimo es 2, según
	% https://texdoc.org/serve/pdftex-a.pdf/0 p.20
	\pdfdecimaldigits=2
	
	% Inclusión de PDF
	\pdfinclusionerrorlevel=0
	
	% Versión
	\pdfminorversion=\pdfcompileversion
	
	% Compresión de objetos
	\pdfobjcompresslevel=\pdfcompileobjcompression
}{
\ifthenelse{\equal{\compilertype}{xelatex}}{
}{
\ifthenelse{\equal{\compilertype}{lualatex}}{
}{
	\throwbadconfig{Compilador desconocido}{\compilertype}{pdf2latex,xelatex,lualatex}}}
}

% -----------------------------------------------------------------------------
% Crea las sub-sub-sub-secciones
% -----------------------------------------------------------------------------
\newcounter{subsubsubsection}[subsubsection]

% Límite máximo profundidad
\setcounter{secnumdepth}{4}

% Agrega compatibilidad de sub-sub-sub-secciones al TOC
\makeatletter
	\def\toclevel@subsubsubsection {4}
	\def\toclevel@paragraph {5}
	\def\toclevel@subparagraph {6}
	\ifthenelse{\equal{\charaftersectionnum}{}}{ % Sin caracter
		\def\l@subsubsubsection {\@dottedtocline{4}{6.97em}{4em}}
		\def\l@paragraph {\@dottedtocline{5}{10.97em}{5em}}
		\def\l@subparagraph {\@dottedtocline{6}{14em}{6em}}
	}{ % Posee caracter, Incremento 0.77+3.35 a 3.35
		\def\l@subsubsubsection {\@dottedtocline{4}{7.83em}{4.15em}}
		\def\l@paragraph {\@dottedtocline{5}{11.98em}{4.92em}}
		\def\l@subparagraph {\@dottedtocline{6}{14.65em}{5.69em}}
	}
\makeatother

% -----------------------------------------------------------------------------
% Configura el número de las secciones
% -----------------------------------------------------------------------------
% Funciones de bajo nivel
\makeatletter
\newcommand\sectionpunct[2]{%
	\expandafter\def\csname @seccntfmt@#1\endcsname##1{%
		\csname the##1\endcsname#2%
	}%
}
\def\@seccntformat#1{\@ifundefined{#1@cntformat}%
	{\csname the#1\endcsname} % Default
	{\csname #1@cntformat\endcsname} % Control individual
}
% Configura secciones
\newcommand\section@cntformat{\GLOBALtitlepresectionstr\thesection\charaftersectionnum\spacingaftersection}
\newcommand\subsection@cntformat{\GLOBALtitlepresubsectionstr\thesubsection\charaftersectionnum\spacingaftersection}
\newcommand\subsubsection@cntformat{\GLOBALtitlepresubsubsectionstr\thesubsubsection\charaftersectionnum\spacingaftersection}
\makeatother

% -----------------------------------------------------------------------------
% Actualización margen títulos
% -----------------------------------------------------------------------------
\titlespacing*{\section}{\sectionspacingleft pt}{\sectionspacingtop pt plus 0pt minus 4pt}{\sectionspacingbottom pt plus 0pt minus 2pt}
\titlespacing*{\subsection}{\ssectionspacingleft pt}{\ssectionspacingtop pt plus 0pt minus 2pt}{\ssectionspacingbottom pt plus 0pt minus 2pt}
\titlespacing*{\subsubsection}{\sssectionspacingleft pt}{\sssectionspacingtop pt plus 0pt minus 2pt}{\sssectionspacingbottom pt plus 0pt minus 2pt}
\titlespacing*{\subsubsubsection}{\ssssectionspacingleft pt}{\ssssectionspacingtop pt plus 0pt minus 2pt}{\ssssectionspacingbottom pt plus 0pt minus 2pt}
\chaptertitlefont{\color{\chaptercolor} \chapterfontsize \chapterfontstyle \selectfont}
\makeatletter
\renewcommand\paragraph{\@startsection{paragraph}{5}{\paragspacingleft pt}
	{\paragspacingtop pt \@plus 0pt \@minus 2pt}
	{\paragspacingbottom pt \@plus 0pt \@minus 2pt}
	{\color{\paragcolor}\normalfont\paragfontsize\paragfontstyle}}
\renewcommand\subparagraph{\@startsection{subparagraph}{6}{\paragsubspacingleft pt}
	{\paragsubspacingtop pt \@plus 0pt \@minus 2pt}
	{\paragsubspacingbottom pt \@plus 0pt \@minus 2pt}
	{\color{\paragsubcolor}\normalfont\paragsubfontsize\paragsubfontstyle}}
\makeatother

% -----------------------------------------------------------------------------
% Profundidad del índice y bookmarks pdf
% -----------------------------------------------------------------------------
\setcounter{tocdepth}{\indexdepth}

% -----------------------------------------------------------------------------
% Configuración footnotes
% -----------------------------------------------------------------------------
% Restaura número
\ifthenelse{\equal{\footnoterestart}{none}}{
	\counterwithout*{footnote}{chapter}
}{
\ifthenelse{\equal{\footnoterestart}{sec}}{
	\counterwithin*{footnote}{section}
}{
\ifthenelse{\equal{\footnoterestart}{ssec}}{
	\counterwithin*{footnote}{subsection}
}{
\ifthenelse{\equal{\footnoterestart}{sssec}}{
	\counterwithin*{footnote}{subsubsection}
}{
\ifthenelse{\equal{\footnoterestart}{ssssec}}{
	\counterwithin*{footnote}{subsubsubsection}
}{
\ifthenelse{\equal{\footnoterestart}{page}}{
	\counterwithin*{footnote}{page}
}{
\ifthenelse{\equal{\footnoterestart}{chap}}{
	\counterwithin*{footnote}{chapter}
}{
	\throwbadconfig{Formato reinicio numero footnote desconocido}{\footnoterestart}{none,chap,page,sec,ssec,sssec,ssssec}}}}}}}
}

% Define el tamaño del margen
\setlength{\footnotemargin}{\footnotelmargin pt}

% Previene footnote en otras páginas
\interfootnotelinepenalty=10000

% Configura tablas y figuras
\ifthenelse{\equal{\footnoterulefigure}{false}}{
	\floatsetup[figure]{footnoterule=none}}{
}
\ifthenelse{\equal{\footnoteruletable}{false}}{
	\floatsetup[table]{footnoterule=none}}{
}

% -----------------------------------------------------------------------------
% Restauración número ecuación, NOTA: NO hace nada, sólo se modifica en title.tex
% -----------------------------------------------------------------------------
\ifthenelse{\equal{\equationrestart}{none}}{
}{
\ifthenelse{\equal{\equationrestart}{chap}}{
}{
\ifthenelse{\equal{\equationrestart}{sec}}{
}{
\ifthenelse{\equal{\equationrestart}{ssec}}{
}{
\ifthenelse{\equal{\equationrestart}{sssec}}{
}{
\ifthenelse{\equal{\equationrestart}{ssssec}}{
}{
	\throwbadconfig{Formato reinicio numero ecuacion desconocido}{\equationrestart}{none,chap,sec,ssec,sssec,ssssec}}}}}}
}

% -----------------------------------------------------------------------------
% Configuración elementos matemáticos
% -----------------------------------------------------------------------------
\newtheoremstyle{templatetheorem}{\baselineskip}{3pt}{\itshape}{}{\bfseries}{}{.5em}{}
\newtheoremstyle{templateobs}{\baselineskip}{3pt}{}{}{\bfseries}{}{.5em}{}
\theoremstyle{templatetheorem}

% Configura números
\ifthenelse{\equal{\showsectioncaptionmat}{none}}{
	\newtheorem{defn}{\namemathdefn}
	\newtheorem{teo}{\namemaththeorem}
	\newtheorem{cor}{\namemathcol}
	\newtheorem{lema}{\namemathlem}
	\newtheorem{prop}{\namemathprp}
}{
\ifthenelse{\equal{\showsectioncaptionmat}{chap}}{
	\newtheorem{defn}{\namemathdefn}[chapter]
	\newtheorem{teo}{\namemaththeorem}[chapter]
	\newtheorem{cor}{\namemathcol}[chapter]
	\newtheorem{lema}{\namemathlem}[chapter]
	\newtheorem{prop}{\namemathprp}[chapter]
}{
\ifthenelse{\equal{\showsectioncaptionmat}{sec}}{
	\newtheorem{defn}{\namemathdefn}[section]
	\newtheorem{teo}{\namemaththeorem}[section]
	\newtheorem{cor}{\namemathcol}[section]
	\newtheorem{lema}{\namemathlem}[section]
	\newtheorem{prop}{\namemathprp}[section]
}{
\ifthenelse{\equal{\showsectioncaptionmat}{ssec}}{
	\newtheorem{defn}{\namemathdefn}[subsection]
	\newtheorem{teo}{\namemaththeorem}[subsection]
	\newtheorem{cor}{\namemathcol}[subsection]
	\newtheorem{lema}{\namemathlem}[subsection]
	\newtheorem{prop}{\namemathprp}[subsection]
}{
\ifthenelse{\equal{\showsectioncaptionmat}{sssec}}{
	\newtheorem{defn}{\namemathdefn}[subsubsection]
	\newtheorem{teo}{\namemaththeorem}[subsubsection]
	\newtheorem{cor}{\namemathcol}[subsubsection]
	\newtheorem{lema}{\namemathlem}[subsubsection]
	\newtheorem{prop}{\namemathprp}[subsubsection]
}{
\ifthenelse{\equal{\showsectioncaptionmat}{ssssec}}{
	\newtheorem{defn}{\namemathdefn}[subsubsubsection]
	\newtheorem{teo}{\namemaththeorem}[subsubsubsection]
	\newtheorem{cor}{\namemathcol}[subsubsubsection]
	\newtheorem{lema}{\namemathlem}[subsubsubsection]
	\newtheorem{prop}{\namemathprp}[subsubsubsection]
}{
	\throwbadconfig{Valor configuracion incorrecto}{\showsectioncaptionmat}{none,chap,sec,ssec,sssec,ssssec}}}}}}
}
\theoremstyle{templateobs}
\newtheorem*{ej}{\namemathej}
\newtheorem*{obs}{\namemathobs}

% -----------------------------------------------------------------------------
% Configura el formato oneside/twoside
% -----------------------------------------------------------------------------
% Normaliza el formato de páginas
\raggedbottom

% Desactiva \cleardoublepage hasta el inicio del documento
\let\oldcleardoublepage\cleardoublepage
\let\cleardoublepage\clearpage

% Modifica el formato de nuevas páginas predoc y \cleardoublepage 
\ifthenelse{\equal{\twopagesclearformat}{blank}}{
	\let\emptypagespredocformat\insertblankpage
}{
\ifthenelse{\equal{\twopagesclearformat}{empty}}{
	\let\emptypagespredocformat\insertemptypage
}{
	\throwbadconfig{Valor configuracion incorrecto}{\twopagesclearformat}{blank,empty}}
}

% -----------------------------------------------------------------------------
% Configuraciones del idioma
% -----------------------------------------------------------------------------
% Desactiva caracteres acentuados en operaciones matemáticas
\unaccentedoperators

% -----------------------------------------------------------------------------
% Configura número de objetos en el final del documento
% -----------------------------------------------------------------------------
\AtEndDocument{
	\addtocounter{equation}{\value{templateEquations}}
	\addtocounter{figure}{\value{templateFigures}}
	\addtocounter{lstlisting}{\value{templateListings}}
	\addtocounter{table}{\value{templateTables}}
}

% -----------------------------------------------------------------------------
% Formato de columnas
% -----------------------------------------------------------------------------
% Centrado
\newcolumntype{C}[1]{>{\centering\let\newline\\\arraybackslash\hspace{0pt}}m{#1}}
\newcolumntype{\CColor}[2]{>{\columncolor{#1}\centering\let\newline\\\arraybackslash\hspace{0pt}}m{#2}}

\newcolumntype{P}[1]{>{\centering\let\newline\\\arraybackslash\hspace{0pt}}p{#1}}
\newcolumntype{\PColor}[2]{>{\columncolor{#1}\centering\let\newline\\\arraybackslash\hspace{0pt}}p{#2}}

\newcolumntype{B}[1]{>{\centering\let\newline\\\arraybackslash\hspace{0pt}}b{#1}}
\newcolumntype{\BColor}[2]{>{\columncolor{#1}\centering\let\newline\\\arraybackslash\hspace{0pt}}b{#2}}

% Izquierda
\newcolumntype{L}[1]{>{\raggedright\let\newline\\\arraybackslash\hspace{0pt}}m{#1}}
\newcolumntype{\LColor}[2]{>{\columncolor{#1}\raggedright\let\newline\\\arraybackslash\hspace{0pt}}m{#2}}
\newcolumntype{T}[1]{>{\raggedright\let\newline\\\arraybackslash\hspace{0pt}}p{#1}}
\newcolumntype{\TColor}[2]{>{\columncolor{#1}\raggedright\let\newline\\\arraybackslash\hspace{0pt}}p{#2}}
\newcolumntype{F}[1]{>{\raggedright\let\newline\\\arraybackslash\hspace{0pt}}b{#1}}
\newcolumntype{\FColor}[2]{>{\columncolor{#1}\raggedright\let\newline\\\arraybackslash\hspace{0pt}}b{#2}}

% Derecha
\newcolumntype{R}[1]{>{\raggedleft\let\newline\\\arraybackslash\hspace{0pt}}m{#1}}
\newcolumntype{\RColor}[2]{>{\columncolor{#1}\raggedleft\let\newline\\\arraybackslash\hspace{0pt}}m{#2}}
\newcolumntype{H}[1]{>{\raggedleft\let\newline\\\arraybackslash\hspace{0pt}}p{#1}}
\newcolumntype{\HColor}[2]{>{\columncolor{#1}\raggedleft\let\newline\\\arraybackslash\hspace{0pt}}p{#2}}
\newcolumntype{G}[1]{>{\raggedleft\let\newline\\\arraybackslash\hspace{0pt}}b{#1}}
\newcolumntype{\GColor}[2]{>{\columncolor{#1}\raggedleft\let\newline\\\arraybackslash\hspace{0pt}}b{#2}}

% -----------------------------------------------------------------------------
% Parcha el entorno tablenotes
% -----------------------------------------------------------------------------
\BeforeBeginEnvironment{tablenotes}{%
	\tablenotesfontsize\selectfont%
}
\AfterEndEnvironment{tablenotes}{%
	\normalsize\selectfont%
}

% -----------------------------------------------------------------------------
% Parcha el entorno multicols
% -----------------------------------------------------------------------------
\let\SOURCEcaptionlrmargin\captionlrmargin
\newcounter{multicoldepth}
\setcounter{multicoldepth}{0}
\BeforeBeginEnvironment{multicols}{%
	\def\captionlrmargin {\captionlrmarginmc}%
	\global\def\GLOBALenvmulticol {true}%
	\setcaptionmargincm{\captionlrmargin}%
	\addtocounter{multicoldepth}{1}%
}
\AfterEndEnvironment{multicols}{%
	\def\captionlrmargin {\SOURCEcaptionlrmargin}%
	\setcaptionmargincm{\captionlrmargin}%
	\addtocounter{multicoldepth}{-1}
	\ifnumequal{\number\value{multicoldepth}}{0}{%
		\global\def\GLOBALenvmulticol {false}
	}{}
}

% -----------------------------------------------------------------------------
% Configura estilos de listas
% -----------------------------------------------------------------------------
% Enumerate
\def\labelenumi {\textcolor{\enumerateitemcolor}{\senumerti}}
\def\labelenumii {\textcolor{\enumerateitemcolor}{\senumertii}}
\def\labelenumiii {\textcolor{\enumerateitemcolor}{\senumertiii}}
\def\labelenumiv {\textcolor{\enumerateitemcolor}{\senumertiv}}

% Itemize
\def\labelitemi {\textcolor{\itemizeitemcolor}{\sitemizei}}
\def\labelitemii {\textcolor{\itemizeitemcolor}{\sitemizeii}}
\def\labelitemiii {\textcolor{\itemizeitemcolor}{\sitemizeiii}}
\def\labelitemiv {\textcolor{\itemizeitemcolor}{\sitemizeiv}}

% Márgenes
\setlength\leftmargini{\sitemsmargini pt}
\setlength\leftmarginii{\sitemsmarginii pt}
\setlength\leftmarginiii{\sitemsmarginiii pt}
\setlength\leftmarginiv{\sitemsmarginiv pt}

% -----------------------------------------------------------------------------
% Chequea que ciertos módulos no hayan sido cargados antes del inicio del documento
% -----------------------------------------------------------------------------
\checkmodulenotloaded{tcolorbox}

% -----------------------------------------------------------------------------
% Da soporte a \hl{} del paquete soul
% -----------------------------------------------------------------------------
\soulregister\cite7
\soulregister\eqref7
\soulregister\eqref7
\soulregister\footnote7
\soulregister\href7
\soulregister\pageref7
\soulregister\quotes7
\soulregister\ref7
\soulregister\scite7

% -----------------------------------------------------------------------------
% Configura métodos aplicados al iniciar el documento
% -----------------------------------------------------------------------------
\AtBeginDocument{%
	\normalfont%
	\setlength{\parindent}{\documentparindent pt}%
	\setlength{\parskip}{\documentparskip pt}%
}

% -----------------------------------------------------------------------------
% Estilos de capítulos
% -----------------------------------------------------------------------------
\ifthenelse{\equal{\chapterstyle}{style1}}{
	% Default
}{
\ifthenelse{\equal{\chapterstyle}{style2}}{
	\definecolor{gray75}{gray}{0.75}
	\newcommand{\hsp}{\hspace{20pt}}
	\titleformat{\chapter}[hang]{\Huge\bfseries}{\thechapter\hsp\textcolor{gray75}{|}\hsp}{0pt}{\Huge\bfseries}
}{
\ifthenelse{\equal{\chapterstyle}{style3}}{
	\usepackage[Sonny]{fncychap}
	\ChNameVar{\Large}
	\ChTitleVar{\Large}
}{
\ifthenelse{\equal{\chapterstyle}{style4}}{
	\usepackage[Lenny]{fncychap}
	\ChNameVar{\Large}
	\ChTitleVar{\Large}
}{
\ifthenelse{\equal{\chapterstyle}{style5}}{
	\usepackage[Glenn]{fncychap}
	\ChNameVar{\Large}
	\ChTitleVar{\Large}
}{
\ifthenelse{\equal{\chapterstyle}{style6}}{
	\usepackage[Conny]{fncychap}
}{
\ifthenelse{\equal{\chapterstyle}{style7}}{
	\usepackage[Rejne]{fncychap}
}{
\ifthenelse{\equal{\chapterstyle}{style8}}{
	\usepackage[Bjarne]{fncychap}
}{
\ifthenelse{\equal{\chapterstyle}{style9}}{
	\usepackage[Bjornstrup]{fncychap}
}{
\ifthenelse{\equal{\chapterstyle}{style10}}{
	\titleformat{\chapter}[hang]{\Huge\bfseries}{\thechapter.\hspace{20pt}}{0pt}{\Huge\bfseries}
}{
\ifthenelse{\equal{\chapterstyle}{style11}}{
	\titleformat{\chapter}[hang]{\Huge\bfseries}{\thechapter\hspace{20pt}}{0pt}{\Huge\bfseries}
}{
\ifthenelse{\equal{\chapterstyle}{style12}}{
	\titleformat{\chapter}[hang]{\Huge\bfseries}{}{0pt}{\Huge\bfseries}
}{
	\throwbadconfigondoc{Estilo de capitulo incorrecto}{\chapterstyle}{style1 .. style12}}}}}}}}}}}}
}

% -----------------------------------------------------------------------------
% DECLARACIÓN DE CARACTERES ADICIONALES UNICODE
% -----------------------------------------------------------------------------
% Definición letras griegas
\def\Alpha{A}
\def\Beta{B}
\def\Chi{X}
\def\Epsilon{E}
\def\Eta{H}
\def\Iota{I}
\def\Kappa{K}
\def\Mu{M}
\def\Nu{N}
\def\Omicron{O}
\def\omicron{o}
\def\Rho{P}
\def\Tau{T}
\def\Zeta{Z}

\def\LOCALunknownchar {\ensuremath{\mathrm{UNKNOWN\;CHAR}}}

% Definición de símbolos
\makeatletter
\newsavebox{\@brxanglelr}
\newcommand{\llangle}[1][]{\savebox{\@brxanglelr}{\(\m@th{#1\langle}\)}%
	\mathopen{\copy\@brxanglelr\kern-0.5\wd\@brxanglelr\usebox{\@brxanglelr}}}
\newcommand{\rrangle}[1][]{\savebox{\@brxanglelr}{\(\m@th{#1\rangle}\)}%
	\mathclose{\copy\@brxanglelr\kern-0.5\wd\@brxanglelr\usebox{\@brxanglelr}}}
\makeatother

% Creación de comandos si no existen
\ifx\DeclareUnicodeCharacter\undefined%
	\def\DeclareUnicodeCharacter#1#2{%
		\def\tmp{#2}\uccode`\~="#1 \catcode"#1 \active%
		\uppercase{\global\let~\tmp}%
		\uccode`\~=0%
	}
\fi%
\ifx\mapsfrom\undefined%
	\newcommand\mapsfrom{\mathrel{\reflectbox{\ensuremath{\mapsto}}}}%
\fi%

% Agrega caracteres unicode
\ifdefined\DeclareUnicodeCharacter
\DeclareUnicodeCharacter{000B}{~}
\DeclareUnicodeCharacter{00A0}{~}
\DeclareUnicodeCharacter{00A1}{\textexclamdown}
\DeclareUnicodeCharacter{00A2}{\textcent}
\DeclareUnicodeCharacter{00A3}{\pounds}
\DeclareUnicodeCharacter{00A4}{\textcurrency}
\DeclareUnicodeCharacter{00A5}{\textyen}
\DeclareUnicodeCharacter{00A6}{\textbrokenbar}
\DeclareUnicodeCharacter{00A7}{{\mathhexbox 278}}
\DeclareUnicodeCharacter{00A8}{\"{ }}
\DeclareUnicodeCharacter{00A9}{\copyright}
\DeclareUnicodeCharacter{00AA}{\textordfeminine}
\DeclareUnicodeCharacter{00AB}{\guillemotleft}
\DeclareUnicodeCharacter{00AC}{\ensuremath{\neg}}
\DeclareUnicodeCharacter{00AE}{\textregistered}
\DeclareUnicodeCharacter{00AF}{\textasciimacron}
\DeclareUnicodeCharacter{00B0}{\textsuperscript{o}}
\DeclareUnicodeCharacter{00B1}{\ensuremath{\pm}}
\DeclareUnicodeCharacter{00B2}{\textsuperscript{2}}
\DeclareUnicodeCharacter{00B3}{\textsuperscript{3}}
\DeclareUnicodeCharacter{00B5}{\textmu}
\DeclareUnicodeCharacter{00B6}{{\mathhexbox 27B}}
\DeclareUnicodeCharacter{00B7}{\ensuremath{\cdot}}
\DeclareUnicodeCharacter{00B9}{\textsuperscript{1}}
\DeclareUnicodeCharacter{00BA}{\textordmasculine}
\DeclareUnicodeCharacter{00BB}{\guillemotright}
\DeclareUnicodeCharacter{00BC}{\ensuremath{\frac{1}{4}}}
\DeclareUnicodeCharacter{00BD}{\ensuremath{\frac{1}{2}}}
\DeclareUnicodeCharacter{00BE}{\ensuremath{\frac{3}{4}}}
\DeclareUnicodeCharacter{00BF}{\textquestiondown}
\DeclareUnicodeCharacter{00D7}{\ensuremath{\times}}
\DeclareUnicodeCharacter{00F7}{\ensuremath{\div}}
\DeclareUnicodeCharacter{0131}{\ensuremath{\imath}}
\DeclareUnicodeCharacter{02102}{\ensuremath{\mathbb{C}}}
\DeclareUnicodeCharacter{0210D}{\ensuremath{\mathbb{H}}}
\DeclareUnicodeCharacter{02115}{\ensuremath{\mathbb{N}}}
\DeclareUnicodeCharacter{02119}{\ensuremath{\mathbb{P}}}
\DeclareUnicodeCharacter{0211A}{\ensuremath{\mathbb{Q}}}
\DeclareUnicodeCharacter{0211D}{\ensuremath{\mathbb{R}}}
\DeclareUnicodeCharacter{02124}{\ensuremath{\mathbb{Z}}}
\DeclareUnicodeCharacter{0237}{\ensuremath{\jmath}}
\DeclareUnicodeCharacter{02B0}{\ensuremath{^h}}
\DeclareUnicodeCharacter{02B2}{\ensuremath{^j}}
\DeclareUnicodeCharacter{02B3}{\ensuremath{^r}}
\DeclareUnicodeCharacter{02B7}{\ensuremath{^w}}
\DeclareUnicodeCharacter{02B8}{\ensuremath{^y}}
\DeclareUnicodeCharacter{02E1}{\ensuremath{^l}}
\DeclareUnicodeCharacter{02E2}{\ensuremath{^s}}
\DeclareUnicodeCharacter{02E3}{\ensuremath{^x}}
\DeclareUnicodeCharacter{0302}{\ensuremath{\hat{\phantom{x}}}}
\DeclareUnicodeCharacter{0308}{\ensuremath{\ddot{\phantom{x}}}}
\DeclareUnicodeCharacter{0332}{\ensuremath{\underline{\phantom{x}}}}
\DeclareUnicodeCharacter{0391}{\ensuremath{\Alpha}}
\DeclareUnicodeCharacter{0392}{\ensuremath{\Beta}}
\DeclareUnicodeCharacter{0393}{\ensuremath{\Gamma}}
\DeclareUnicodeCharacter{0394}{\ensuremath{\Delta}}
\DeclareUnicodeCharacter{0395}{\ensuremath{\Epsilon}}
\DeclareUnicodeCharacter{0396}{\ensuremath{\Zeta}}
\DeclareUnicodeCharacter{0397}{\ensuremath{\Eta}}
\DeclareUnicodeCharacter{0398}{\ensuremath{\Theta}}
\DeclareUnicodeCharacter{0399}{\Iota}
\DeclareUnicodeCharacter{039A}{\Kappa}
\DeclareUnicodeCharacter{039B}{\ensuremath{\Lambda}}
\DeclareUnicodeCharacter{039C}{\Mu}
\DeclareUnicodeCharacter{039D}{\Nu}
\DeclareUnicodeCharacter{039E}{\ensuremath{\Xi}}
\DeclareUnicodeCharacter{039F}{\Omicron}
\DeclareUnicodeCharacter{03A0}{\ensuremath{\Pi}}
\DeclareUnicodeCharacter{03A1}{\Rho}
\DeclareUnicodeCharacter{03A3}{\ensuremath{\Sigma}}
\DeclareUnicodeCharacter{03A4}{\Tau}
\DeclareUnicodeCharacter{03A5}{\ensuremath{\Upsilon}}
\DeclareUnicodeCharacter{03A6}{\ensuremath{\Phi}}
\DeclareUnicodeCharacter{03A7}{\Chi}
\DeclareUnicodeCharacter{03A8}{\ensuremath{\Psi}}
\DeclareUnicodeCharacter{03A9}{\ensuremath{\Omega}}
\DeclareUnicodeCharacter{03B1}{\ensuremath{\alpha}}
\DeclareUnicodeCharacter{03B2}{\ensuremath{\beta}}
\DeclareUnicodeCharacter{03B3}{\ensuremath{\gamma}}
\DeclareUnicodeCharacter{03B4}{\ensuremath{\delta}}
\DeclareUnicodeCharacter{03B5}{\ensuremath{\varepsilon}}
\DeclareUnicodeCharacter{03B6}{\ensuremath{\zeta}}
\DeclareUnicodeCharacter{03B7}{\ensuremath{\eta}}
\DeclareUnicodeCharacter{03B8}{\ensuremath{\theta}}
\DeclareUnicodeCharacter{03B9}{\ensuremath{\iota}}
\DeclareUnicodeCharacter{03BA}{\ensuremath{\kappa}}
\DeclareUnicodeCharacter{03BB}{\ensuremath{\lambda}}
\DeclareUnicodeCharacter{03BC}{\ensuremath{\mu}}
\DeclareUnicodeCharacter{03BD}{\ensuremath{\nu}}
\DeclareUnicodeCharacter{03BE}{\ensuremath{\xi}}
\DeclareUnicodeCharacter{03BF}{\omicron}
\DeclareUnicodeCharacter{03C0}{\ensuremath{\pi}}
\DeclareUnicodeCharacter{03C1}{\ensuremath{\rho}}
\DeclareUnicodeCharacter{03C2}{\ensuremath{\varsigma}}
\DeclareUnicodeCharacter{03C3}{\ensuremath{\sigma}}
\DeclareUnicodeCharacter{03C4}{\ensuremath{\tau}}
\DeclareUnicodeCharacter{03C5}{\ensuremath{\upsilon}}
\DeclareUnicodeCharacter{03C6}{\ensuremath{\phi}}
\DeclareUnicodeCharacter{03C7}{\ensuremath{\chi}}
\DeclareUnicodeCharacter{03C8}{\ensuremath{\psi}}
\DeclareUnicodeCharacter{03C9}{\ensuremath{\omega}}
\DeclareUnicodeCharacter{03D0}{\ensuremath{\beta}}
\DeclareUnicodeCharacter{03D1}{\ensuremath{\theta}}
\DeclareUnicodeCharacter{03D5}{\ensuremath{\phi}}
\DeclareUnicodeCharacter{03D6}{\ensuremath{\pi}}
\DeclareUnicodeCharacter{03D8}{\ensuremath{Q}}
\DeclareUnicodeCharacter{03D9}{\ensuremath{q}}
\DeclareUnicodeCharacter{03DA}{\ensuremath{S}}
\DeclareUnicodeCharacter{03DB}{\ensuremath{s}}
\DeclareUnicodeCharacter{03DC}{\ensuremath{D}}
\DeclareUnicodeCharacter{03DD}{\ensuremath{d}}
\DeclareUnicodeCharacter{03DE}{\ensuremath{K}}
\DeclareUnicodeCharacter{03DF}{\ensuremath{k}}
\DeclareUnicodeCharacter{03E0}{\ensuremath{S}}
\DeclareUnicodeCharacter{03E1}{\ensuremath{s}}
\DeclareUnicodeCharacter{03F0}{\ensuremath{\varkappa}}
\DeclareUnicodeCharacter{03F1}{\ensuremath{\rho}}
\DeclareUnicodeCharacter{03F5}{\ensuremath{\epsilon}}
\DeclareUnicodeCharacter{03F6}{\ensuremath{\backepsilon}}
\DeclareUnicodeCharacter{041F}{\LOCALunknownchar}
\DeclareUnicodeCharacter{0432}{\LOCALunknownchar}
\DeclareUnicodeCharacter{0435}{\LOCALunknownchar}
\DeclareUnicodeCharacter{0438}{\LOCALunknownchar}
\DeclareUnicodeCharacter{043C}{\LOCALunknownchar}
\DeclareUnicodeCharacter{0440}{\LOCALunknownchar}
\DeclareUnicodeCharacter{0442}{\LOCALunknownchar}
\DeclareUnicodeCharacter{0BA8}{\LOCALunknownchar}
\DeclareUnicodeCharacter{0BBF}{\LOCALunknownchar}
\DeclareUnicodeCharacter{1100}{\LOCALunknownchar}
\DeclareUnicodeCharacter{11F9}{\LOCALunknownchar}
\DeclareUnicodeCharacter{1D2C}{\ensuremath{^A}}
\DeclareUnicodeCharacter{1D2E}{\ensuremath{^B}}
\DeclareUnicodeCharacter{1D30}{\ensuremath{^D}}
\DeclareUnicodeCharacter{1D31}{\ensuremath{^E}}
\DeclareUnicodeCharacter{1D33}{\ensuremath{^G}}
\DeclareUnicodeCharacter{1D34}{\ensuremath{^H}}
\DeclareUnicodeCharacter{1D35}{\ensuremath{^I}}
\DeclareUnicodeCharacter{1D36}{\ensuremath{^J}}
\DeclareUnicodeCharacter{1D37}{\ensuremath{^K}}
\DeclareUnicodeCharacter{1D38}{\ensuremath{^L}}
\DeclareUnicodeCharacter{1D39}{\ensuremath{^M}}
\DeclareUnicodeCharacter{1D3A}{\ensuremath{^N}}
\DeclareUnicodeCharacter{1D3C}{\ensuremath{^O}}
\DeclareUnicodeCharacter{1D3E}{\ensuremath{^P}}
\DeclareUnicodeCharacter{1D3F}{\ensuremath{^R}}
\DeclareUnicodeCharacter{1D40}{\ensuremath{^T}}
\DeclareUnicodeCharacter{1D400}{\ensuremath{\mathbf{A}}}
\DeclareUnicodeCharacter{1D401}{\ensuremath{\mathbf{B}}}
\DeclareUnicodeCharacter{1D402}{\ensuremath{\mathbf{C}}}
\DeclareUnicodeCharacter{1D403}{\ensuremath{\mathbf{D}}}
\DeclareUnicodeCharacter{1D404}{\ensuremath{\mathbf{E}}}
\DeclareUnicodeCharacter{1D405}{\ensuremath{\mathbf{F}}}
\DeclareUnicodeCharacter{1D406}{\ensuremath{\mathbf{G}}}
\DeclareUnicodeCharacter{1D407}{\ensuremath{\mathbf{H}}}
\DeclareUnicodeCharacter{1D408}{\ensuremath{\mathbf{I}}}
\DeclareUnicodeCharacter{1D409}{\ensuremath{\mathbf{J}}}
\DeclareUnicodeCharacter{1D40A}{\ensuremath{\mathbf{K}}}
\DeclareUnicodeCharacter{1D40B}{\ensuremath{\mathbf{L}}}
\DeclareUnicodeCharacter{1D40C}{\ensuremath{\mathbf{M}}}
\DeclareUnicodeCharacter{1D40D}{\ensuremath{\mathbf{N}}}
\DeclareUnicodeCharacter{1D40E}{\ensuremath{\mathbf{O}}}
\DeclareUnicodeCharacter{1D40F}{\ensuremath{\mathbf{P}}}
\DeclareUnicodeCharacter{1D41}{\ensuremath{^U}}
\DeclareUnicodeCharacter{1D410}{\ensuremath{\mathbf{Q}}}
\DeclareUnicodeCharacter{1D411}{\ensuremath{\mathbf{R}}}
\DeclareUnicodeCharacter{1D412}{\ensuremath{\mathbf{S}}}
\DeclareUnicodeCharacter{1D413}{\ensuremath{\mathbf{T}}}
\DeclareUnicodeCharacter{1D414}{\ensuremath{\mathbf{U}}}
\DeclareUnicodeCharacter{1D415}{\ensuremath{\mathbf{V}}}
\DeclareUnicodeCharacter{1D416}{\ensuremath{\mathbf{W}}}
\DeclareUnicodeCharacter{1D417}{\ensuremath{\mathbf{X}}}
\DeclareUnicodeCharacter{1D418}{\ensuremath{\mathbf{Y}}}
\DeclareUnicodeCharacter{1D419}{\ensuremath{\mathbf{Z}}}
\DeclareUnicodeCharacter{1D41A}{\ensuremath{\mathbf{a}}}
\DeclareUnicodeCharacter{1D41B}{\ensuremath{\mathbf{b}}}
\DeclareUnicodeCharacter{1D41C}{\ensuremath{\mathbf{c}}}
\DeclareUnicodeCharacter{1D41D}{\ensuremath{\mathbf{d}}}
\DeclareUnicodeCharacter{1D41E}{\ensuremath{\mathbf{e}}}
\DeclareUnicodeCharacter{1D41F}{\ensuremath{\mathbf{f}}}
\DeclareUnicodeCharacter{1D42}{\ensuremath{^W}}
\DeclareUnicodeCharacter{1D420}{\ensuremath{\mathbf{g}}}
\DeclareUnicodeCharacter{1D421}{\ensuremath{\mathbf{h}}}
\DeclareUnicodeCharacter{1D422}{\ensuremath{\mathbf{i}}}
\DeclareUnicodeCharacter{1D423}{\ensuremath{\mathbf{j}}}
\DeclareUnicodeCharacter{1D424}{\ensuremath{\mathbf{k}}}
\DeclareUnicodeCharacter{1D425}{\ensuremath{\mathbf{l}}}
\DeclareUnicodeCharacter{1D426}{\ensuremath{\mathbf{m}}}
\DeclareUnicodeCharacter{1D427}{\ensuremath{\mathbf{n}}}
\DeclareUnicodeCharacter{1D428}{\ensuremath{\mathbf{o}}}
\DeclareUnicodeCharacter{1D429}{\ensuremath{\mathbf{p}}}
\DeclareUnicodeCharacter{1D42A}{\ensuremath{\mathbf{q}}}
\DeclareUnicodeCharacter{1D42B}{\ensuremath{\mathbf{r}}}
\DeclareUnicodeCharacter{1D42C}{\ensuremath{\mathbf{s}}}
\DeclareUnicodeCharacter{1D42D}{\ensuremath{\mathbf{t}}}
\DeclareUnicodeCharacter{1D42E}{\ensuremath{\mathbf{u}}}
\DeclareUnicodeCharacter{1D42F}{\ensuremath{\mathbf{v}}}
\DeclareUnicodeCharacter{1D43}{\ensuremath{^a}}
\DeclareUnicodeCharacter{1D430}{\ensuremath{\mathbf{w}}}
\DeclareUnicodeCharacter{1D431}{\ensuremath{\mathbf{x}}}
\DeclareUnicodeCharacter{1D432}{\ensuremath{\mathbf{y}}}
\DeclareUnicodeCharacter{1D433}{\ensuremath{\mathbf{z}}}
\DeclareUnicodeCharacter{1D434}{\ensuremath{\mathit{A}}}
\DeclareUnicodeCharacter{1D435}{\ensuremath{\mathit{B}}}
\DeclareUnicodeCharacter{1D436}{\ensuremath{\mathit{C}}}
\DeclareUnicodeCharacter{1D437}{\ensuremath{\mathit{D}}}
\DeclareUnicodeCharacter{1D438}{\ensuremath{\mathit{E}}}
\DeclareUnicodeCharacter{1D439}{\ensuremath{\mathit{F}}}
\DeclareUnicodeCharacter{1D43A}{\ensuremath{\mathit{G}}}
\DeclareUnicodeCharacter{1D43B}{\ensuremath{\mathit{H}}}
\DeclareUnicodeCharacter{1D43C}{\ensuremath{\mathit{I}}}
\DeclareUnicodeCharacter{1D43D}{\ensuremath{\mathit{J}}}
\DeclareUnicodeCharacter{1D43E}{\ensuremath{\mathit{K}}}
\DeclareUnicodeCharacter{1D43F}{\ensuremath{\mathit{L}}}
\DeclareUnicodeCharacter{1D440}{\ensuremath{\mathit{M}}}
\DeclareUnicodeCharacter{1D441}{\ensuremath{\mathit{N}}}
\DeclareUnicodeCharacter{1D442}{\ensuremath{\mathit{O}}}
\DeclareUnicodeCharacter{1D443}{\ensuremath{\mathit{P}}}
\DeclareUnicodeCharacter{1D444}{\ensuremath{\mathit{Q}}}
\DeclareUnicodeCharacter{1D445}{\ensuremath{\mathit{R}}}
\DeclareUnicodeCharacter{1D446}{\ensuremath{\mathit{S}}}
\DeclareUnicodeCharacter{1D447}{\ensuremath{\mathit{T}}}
\DeclareUnicodeCharacter{1D448}{\ensuremath{\mathit{U}}}
\DeclareUnicodeCharacter{1D449}{\ensuremath{\mathit{V}}}
\DeclareUnicodeCharacter{1D44A}{\ensuremath{\mathit{W}}}
\DeclareUnicodeCharacter{1D44B}{\ensuremath{\mathit{X}}}
\DeclareUnicodeCharacter{1D44C}{\ensuremath{\mathit{Y}}}
\DeclareUnicodeCharacter{1D44D}{\ensuremath{\mathit{Z}}}
\DeclareUnicodeCharacter{1D44E}{\ensuremath{\mathit{a}}}
\DeclareUnicodeCharacter{1D44F}{\ensuremath{\mathit{b}}}
\DeclareUnicodeCharacter{1D450}{\ensuremath{\mathit{c}}}
\DeclareUnicodeCharacter{1D451}{\ensuremath{\mathit{d}}}
\DeclareUnicodeCharacter{1D452}{\ensuremath{\mathit{e}}}
\DeclareUnicodeCharacter{1D453}{\ensuremath{\mathit{f}}}
\DeclareUnicodeCharacter{1D454}{\ensuremath{\mathit{g}}}
\DeclareUnicodeCharacter{1D456}{\ensuremath{\mathit{i}}}
\DeclareUnicodeCharacter{1D457}{\ensuremath{\mathit{j}}}
\DeclareUnicodeCharacter{1D458}{\ensuremath{\mathit{k}}}
\DeclareUnicodeCharacter{1D459}{\ensuremath{\mathit{l}}}
\DeclareUnicodeCharacter{1D45A}{\ensuremath{\mathit{m}}}
\DeclareUnicodeCharacter{1D45B}{\ensuremath{\mathit{n}}}
\DeclareUnicodeCharacter{1D45C}{\ensuremath{\mathit{o}}}
\DeclareUnicodeCharacter{1D45D}{\ensuremath{\mathit{p}}}
\DeclareUnicodeCharacter{1D45E}{\ensuremath{\mathit{q}}}
\DeclareUnicodeCharacter{1D45F}{\ensuremath{\mathit{r}}}
\DeclareUnicodeCharacter{1D460}{\ensuremath{\mathit{s}}}
\DeclareUnicodeCharacter{1D461}{\ensuremath{\mathit{t}}}
\DeclareUnicodeCharacter{1D462}{\ensuremath{\mathit{u}}}
\DeclareUnicodeCharacter{1D463}{\ensuremath{\mathit{v}}}
\DeclareUnicodeCharacter{1D464}{\ensuremath{\mathit{w}}}
\DeclareUnicodeCharacter{1D465}{\ensuremath{\mathit{x}}}
\DeclareUnicodeCharacter{1D466}{\ensuremath{\mathit{y}}}
\DeclareUnicodeCharacter{1D467}{\ensuremath{\mathit{z}}}
\DeclareUnicodeCharacter{1D47}{\ensuremath{^b}}
\DeclareUnicodeCharacter{1D48}{\ensuremath{^d}}
\DeclareUnicodeCharacter{1D49}{\ensuremath{^e}}
\DeclareUnicodeCharacter{1D49C}{\ensuremath{\mathscr{A}}}
\DeclareUnicodeCharacter{1D49E}{\ensuremath{\mathscr{C}}}
\DeclareUnicodeCharacter{1D49F}{\ensuremath{\mathscr{D}}}
\DeclareUnicodeCharacter{1D4A2}{\ensuremath{\mathscr{G}}}
\DeclareUnicodeCharacter{1D4A5}{\ensuremath{\mathscr{J}}}
\DeclareUnicodeCharacter{1D4A6}{\ensuremath{\mathscr{K}}}
\DeclareUnicodeCharacter{1D4A9}{\ensuremath{\mathscr{N}}}
\DeclareUnicodeCharacter{1D4AA}{\ensuremath{\mathscr{O}}}
\DeclareUnicodeCharacter{1D4AB}{\ensuremath{\mathscr{P}}}
\DeclareUnicodeCharacter{1D4AC}{\ensuremath{\mathscr{Q}}}
\DeclareUnicodeCharacter{1D4AE}{\ensuremath{\mathscr{S}}}
\DeclareUnicodeCharacter{1D4AF}{\ensuremath{\mathscr{T}}}
\DeclareUnicodeCharacter{1D4B0}{\ensuremath{\mathscr{U}}}
\DeclareUnicodeCharacter{1D4B1}{\ensuremath{\mathscr{V}}}
\DeclareUnicodeCharacter{1D4B2}{\ensuremath{\mathscr{W}}}
\DeclareUnicodeCharacter{1D4B3}{\ensuremath{\mathscr{X}}}
\DeclareUnicodeCharacter{1D4B4}{\ensuremath{\mathscr{Y}}}
\DeclareUnicodeCharacter{1D4B5}{\ensuremath{\mathscr{Z}}}
\DeclareUnicodeCharacter{1D4D}{\ensuremath{^g}}
\DeclareUnicodeCharacter{1D4D0}{\ensuremath{\mathcal{A}}}
\DeclareUnicodeCharacter{1D4D1}{\ensuremath{\mathcal{B}}}
\DeclareUnicodeCharacter{1D4D2}{\ensuremath{\mathcal{C}}}
\DeclareUnicodeCharacter{1D4D3}{\ensuremath{\mathcal{D}}}
\DeclareUnicodeCharacter{1D4D4}{\ensuremath{\mathcal{E}}}
\DeclareUnicodeCharacter{1D4D5}{\ensuremath{\mathcal{F}}}
\DeclareUnicodeCharacter{1D4D6}{\ensuremath{\mathcal{G}}}
\DeclareUnicodeCharacter{1D4D7}{\ensuremath{\mathcal{H}}}
\DeclareUnicodeCharacter{1D4D8}{\ensuremath{\mathcal{I}}}
\DeclareUnicodeCharacter{1D4D9}{\ensuremath{\mathcal{J}}}
\DeclareUnicodeCharacter{1D4DA}{\ensuremath{\mathcal{K}}}
\DeclareUnicodeCharacter{1D4DB}{\ensuremath{\mathcal{L}}}
\DeclareUnicodeCharacter{1D4DC}{\ensuremath{\mathcal{M}}}
\DeclareUnicodeCharacter{1D4DD}{\ensuremath{\mathcal{N}}}
\DeclareUnicodeCharacter{1D4DE}{\ensuremath{\mathcal{O}}}
\DeclareUnicodeCharacter{1D4DF}{\ensuremath{\mathcal{P}}}
\DeclareUnicodeCharacter{1D4E0}{\ensuremath{\mathcal{Q}}}
\DeclareUnicodeCharacter{1D4E1}{\ensuremath{\mathcal{R}}}
\DeclareUnicodeCharacter{1D4E2}{\ensuremath{\mathcal{S}}}
\DeclareUnicodeCharacter{1D4E3}{\ensuremath{\mathcal{T}}}
\DeclareUnicodeCharacter{1D4E4}{\ensuremath{\mathcal{U}}}
\DeclareUnicodeCharacter{1D4E5}{\ensuremath{\mathcal{V}}}
\DeclareUnicodeCharacter{1D4E6}{\ensuremath{\mathcal{W}}}
\DeclareUnicodeCharacter{1D4E7}{\ensuremath{\mathcal{X}}}
\DeclareUnicodeCharacter{1D4E8}{\ensuremath{\mathcal{Y}}}
\DeclareUnicodeCharacter{1D4E9}{\ensuremath{\mathcal{Z}}}
\DeclareUnicodeCharacter{1D4F}{\ensuremath{^k}}
\DeclareUnicodeCharacter{1D50}{\ensuremath{^m}}
\DeclareUnicodeCharacter{1D504}{\ensuremath{\mathfrak{A}}}
\DeclareUnicodeCharacter{1D505}{\ensuremath{\mathfrak{B}}}
\DeclareUnicodeCharacter{1D507}{\ensuremath{\mathfrak{D}}}
\DeclareUnicodeCharacter{1D508}{\ensuremath{\mathfrak{E}}}
\DeclareUnicodeCharacter{1D509}{\ensuremath{\mathfrak{F}}}
\DeclareUnicodeCharacter{1D50A}{\ensuremath{\mathfrak{G}}}
\DeclareUnicodeCharacter{1D50D}{\ensuremath{\mathfrak{J}}}
\DeclareUnicodeCharacter{1D50E}{\ensuremath{\mathfrak{K}}}
\DeclareUnicodeCharacter{1D50F}{\ensuremath{\mathfrak{L}}}
\DeclareUnicodeCharacter{1D510}{\ensuremath{\mathfrak{M}}}
\DeclareUnicodeCharacter{1D511}{\ensuremath{\mathfrak{N}}}
\DeclareUnicodeCharacter{1D512}{\ensuremath{\mathfrak{O}}}
\DeclareUnicodeCharacter{1D513}{\ensuremath{\mathfrak{P}}}
\DeclareUnicodeCharacter{1D514}{\ensuremath{\mathfrak{Q}}}
\DeclareUnicodeCharacter{1D516}{\ensuremath{\mathfrak{S}}}
\DeclareUnicodeCharacter{1D517}{\ensuremath{\mathfrak{T}}}
\DeclareUnicodeCharacter{1D518}{\ensuremath{\mathfrak{U}}}
\DeclareUnicodeCharacter{1D519}{\ensuremath{\mathfrak{V}}}
\DeclareUnicodeCharacter{1D51A}{\ensuremath{\mathfrak{W}}}
\DeclareUnicodeCharacter{1D51B}{\ensuremath{\mathfrak{X}}}
\DeclareUnicodeCharacter{1D51C}{\ensuremath{\mathfrak{Y}}}
\DeclareUnicodeCharacter{1D51E}{\ensuremath{\mathfrak{a}}}
\DeclareUnicodeCharacter{1D51F}{\ensuremath{\mathfrak{b}}}
\DeclareUnicodeCharacter{1D52}{\ensuremath{^o}}
\DeclareUnicodeCharacter{1D520}{\ensuremath{\mathfrak{c}}}
\DeclareUnicodeCharacter{1D521}{\ensuremath{\mathfrak{d}}}
\DeclareUnicodeCharacter{1D522}{\ensuremath{\mathfrak{e}}}
\DeclareUnicodeCharacter{1D523}{\ensuremath{\mathfrak{f}}}
\DeclareUnicodeCharacter{1D524}{\ensuremath{\mathfrak{g}}}
\DeclareUnicodeCharacter{1D525}{\ensuremath{\mathfrak{h}}}
\DeclareUnicodeCharacter{1D526}{\ensuremath{\mathfrak{i}}}
\DeclareUnicodeCharacter{1D527}{\ensuremath{\mathfrak{j}}}
\DeclareUnicodeCharacter{1D528}{\ensuremath{\mathfrak{k}}}
\DeclareUnicodeCharacter{1D529}{\ensuremath{\mathfrak{l}}}
\DeclareUnicodeCharacter{1D52A}{\ensuremath{\mathfrak{m}}}
\DeclareUnicodeCharacter{1D52B}{\ensuremath{\mathfrak{n}}}
\DeclareUnicodeCharacter{1D52C}{\ensuremath{\mathfrak{o}}}
\DeclareUnicodeCharacter{1D52D}{\ensuremath{\mathfrak{p}}}
\DeclareUnicodeCharacter{1D52E}{\ensuremath{\mathfrak{q}}}
\DeclareUnicodeCharacter{1D52F}{\ensuremath{\mathfrak{r}}}
\DeclareUnicodeCharacter{1D530}{\ensuremath{\mathfrak{s}}}
\DeclareUnicodeCharacter{1D531}{\ensuremath{\mathfrak{t}}}
\DeclareUnicodeCharacter{1D532}{\ensuremath{\mathfrak{u}}}
\DeclareUnicodeCharacter{1D533}{\ensuremath{\mathfrak{v}}}
\DeclareUnicodeCharacter{1D534}{\ensuremath{\mathfrak{w}}}
\DeclareUnicodeCharacter{1D535}{\ensuremath{\mathfrak{x}}}
\DeclareUnicodeCharacter{1D536}{\ensuremath{\mathfrak{y}}}
\DeclareUnicodeCharacter{1D537}{\ensuremath{\mathfrak{z}}}
\DeclareUnicodeCharacter{1D538}{\ensuremath{\mathbb{A}}}
\DeclareUnicodeCharacter{1D539}{\ensuremath{\mathbb{B}}}
\DeclareUnicodeCharacter{1D53B}{\ensuremath{\mathbb{D}}}
\DeclareUnicodeCharacter{1D53C}{\ensuremath{\mathbb{E}}}
\DeclareUnicodeCharacter{1D53D}{\ensuremath{\mathbb{F}}}
\DeclareUnicodeCharacter{1D53E}{\ensuremath{\mathbb{G}}}
\DeclareUnicodeCharacter{1D540}{\ensuremath{\mathbb{I}}}
\DeclareUnicodeCharacter{1D541}{\ensuremath{\mathbb{J}}}
\DeclareUnicodeCharacter{1D542}{\ensuremath{\mathbb{K}}}
\DeclareUnicodeCharacter{1D543}{\ensuremath{\mathbb{L}}}
\DeclareUnicodeCharacter{1D544}{\ensuremath{\mathbb{M}}}
\DeclareUnicodeCharacter{1D546}{\ensuremath{\mathbb{O}}}
\DeclareUnicodeCharacter{1D54A}{\ensuremath{\mathbb{S}}}
\DeclareUnicodeCharacter{1D54B}{\ensuremath{\mathbb{T}}}
\DeclareUnicodeCharacter{1D54C}{\ensuremath{\mathbb{U}}}
\DeclareUnicodeCharacter{1D54D}{\ensuremath{\mathbb{V}}}
\DeclareUnicodeCharacter{1D54E}{\ensuremath{\mathbb{W}}}
\DeclareUnicodeCharacter{1D54F}{\ensuremath{\mathbb{X}}}
\DeclareUnicodeCharacter{1D550}{\ensuremath{\mathbb{Y}}}
\DeclareUnicodeCharacter{1D552}{\ensuremath{\mathbb{a}}}
\DeclareUnicodeCharacter{1D553}{\ensuremath{\mathbb{b}}}
\DeclareUnicodeCharacter{1D554}{\ensuremath{\mathbb{c}}}
\DeclareUnicodeCharacter{1D555}{\ensuremath{\mathbb{d}}}
\DeclareUnicodeCharacter{1D556}{\ensuremath{\mathbb{e}}}
\DeclareUnicodeCharacter{1D557}{\ensuremath{\mathbb{f}}}
\DeclareUnicodeCharacter{1D558}{\ensuremath{\mathbb{g}}}
\DeclareUnicodeCharacter{1D559}{\ensuremath{\mathbb{h}}}
\DeclareUnicodeCharacter{1D55A}{\ensuremath{\mathbb{i}}}
\DeclareUnicodeCharacter{1D55B}{\ensuremath{\mathbb{j}}}
\DeclareUnicodeCharacter{1D55C}{\ensuremath{\mathbb{k}}}
\DeclareUnicodeCharacter{1D55D}{\ensuremath{\mathbb{l}}}
\DeclareUnicodeCharacter{1D55E}{\ensuremath{\mathbb{m}}}
\DeclareUnicodeCharacter{1D55F}{\ensuremath{\mathbb{n}}}
\DeclareUnicodeCharacter{1D56}{\ensuremath{^p}}
\DeclareUnicodeCharacter{1D560}{\ensuremath{\mathbb{o}}}
\DeclareUnicodeCharacter{1D561}{\ensuremath{\mathbb{p}}}
\DeclareUnicodeCharacter{1D562}{\ensuremath{\mathbb{q}}}
\DeclareUnicodeCharacter{1D563}{\ensuremath{\mathbb{r}}}
\DeclareUnicodeCharacter{1D564}{\ensuremath{\mathbb{s}}}
\DeclareUnicodeCharacter{1D565}{\ensuremath{\mathbb{t}}}
\DeclareUnicodeCharacter{1D566}{\ensuremath{\mathbb{u}}}
\DeclareUnicodeCharacter{1D567}{\ensuremath{\mathbb{v}}}
\DeclareUnicodeCharacter{1D568}{\ensuremath{\mathbb{w}}}
\DeclareUnicodeCharacter{1D569}{\ensuremath{\mathbb{x}}}
\DeclareUnicodeCharacter{1D56A}{\ensuremath{\mathbb{y}}}
\DeclareUnicodeCharacter{1D56B}{\ensuremath{\mathbb{z}}}
\DeclareUnicodeCharacter{1D57}{\ensuremath{^t}}
\DeclareUnicodeCharacter{1D58}{\ensuremath{^u}}
\DeclareUnicodeCharacter{1D5B}{\ensuremath{^v}}
\DeclareUnicodeCharacter{1D62}{\ensuremath{_i}}
\DeclareUnicodeCharacter{1D629}{\ensuremath{\mathit{h}}}
\DeclareUnicodeCharacter{1D63}{\ensuremath{_r}}
\DeclareUnicodeCharacter{1D64}{\ensuremath{_u}}
\DeclareUnicodeCharacter{1D65}{\ensuremath{_v}}
\DeclareUnicodeCharacter{1D6E4}{\ensuremath{\Gamma}}
\DeclareUnicodeCharacter{1D6E5}{\ensuremath{\Delta}}
\DeclareUnicodeCharacter{1D6F1}{\ensuremath{\Pi}}
\DeclareUnicodeCharacter{1D6F4}{\ensuremath{\Sigma}}
\DeclareUnicodeCharacter{1D6FA}{\ensuremath{\Omega}}
\DeclareUnicodeCharacter{1D6FC}{\ensuremath{\alpha}}
\DeclareUnicodeCharacter{1D6FD}{\ensuremath{\beta}}
\DeclareUnicodeCharacter{1D6FE}{\ensuremath{\gamma}}
\DeclareUnicodeCharacter{1D6FF}{\ensuremath{\delta}}
\DeclareUnicodeCharacter{1D700}{\ensuremath{\varepsilon}}
\DeclareUnicodeCharacter{1D701}{\ensuremath{\zeta}}
\DeclareUnicodeCharacter{1D702}{\ensuremath{\eta}}
\DeclareUnicodeCharacter{1D703}{\ensuremath{\theta}}
\DeclareUnicodeCharacter{1D704}{\ensuremath{\iota}}
\DeclareUnicodeCharacter{1D705}{\ensuremath{\kappa}}
\DeclareUnicodeCharacter{1D706}{\ensuremath{\lambda}}
\DeclareUnicodeCharacter{1D707}{\ensuremath{\mu}}
\DeclareUnicodeCharacter{1D708}{\ensuremath{\nu}}
\DeclareUnicodeCharacter{1D709}{\ensuremath{\xi}}
\DeclareUnicodeCharacter{1D70A}{\ensuremath{o}}
\DeclareUnicodeCharacter{1D70B}{\ensuremath{\pi}}
\DeclareUnicodeCharacter{1D70C}{\ensuremath{\rho}}
\DeclareUnicodeCharacter{1D70D}{\ensuremath{\varsigma}}
\DeclareUnicodeCharacter{1D70E}{\ensuremath{\sigma}}
\DeclareUnicodeCharacter{1D70F}{\ensuremath{\tau}}
\DeclareUnicodeCharacter{1D710}{\ensuremath{\upsilon}}
\DeclareUnicodeCharacter{1D711}{\ensuremath{\varphi}}
\DeclareUnicodeCharacter{1D712}{\ensuremath{\chi}}
\DeclareUnicodeCharacter{1D713}{\ensuremath{\psi}}
\DeclareUnicodeCharacter{1D714}{\ensuremath{\omega}}
\DeclareUnicodeCharacter{1D716}{\ensuremath{\epsilon}}
\DeclareUnicodeCharacter{1D717}{\ensuremath{\vartheta}}
\DeclareUnicodeCharacter{1D719}{\ensuremath{\phi}}
\DeclareUnicodeCharacter{1D71A}{\ensuremath{\varrho}}
\DeclareUnicodeCharacter{1D71B}{\ensuremath{\varpi}}
\DeclareUnicodeCharacter{1D7D8}{\ensuremath{\mathbb{0}}}
\DeclareUnicodeCharacter{1D7D9}{\ensuremath{\mathbb{1}}}
\DeclareUnicodeCharacter{1D7DA}{\ensuremath{\mathbb{2}}}
\DeclareUnicodeCharacter{1D7DB}{\ensuremath{\mathbb{3}}}
\DeclareUnicodeCharacter{1D7DC}{\ensuremath{\mathbb{4}}}
\DeclareUnicodeCharacter{1D7DD}{\ensuremath{\mathbb{5}}}
\DeclareUnicodeCharacter{1D7DE}{\ensuremath{\mathbb{6}}}
\DeclareUnicodeCharacter{1D7DF}{\ensuremath{\mathbb{7}}}
\DeclareUnicodeCharacter{1D7E0}{\ensuremath{\mathbb{8}}}
\DeclareUnicodeCharacter{1D7E1}{\ensuremath{\mathbb{9}}}
\DeclareUnicodeCharacter{1D9C}{\ensuremath{^c}}
\DeclareUnicodeCharacter{1DA0}{\ensuremath{^f}}
\DeclareUnicodeCharacter{1DBB}{\ensuremath{^z}}
\DeclareUnicodeCharacter{1F329}{\ensuremath{\lightning}}
\DeclareUnicodeCharacter{2013}{--}
\DeclareUnicodeCharacter{2014}{---}
\DeclareUnicodeCharacter{2016}{\textbardbl}
\DeclareUnicodeCharacter{2018}{\textquoteleft}
\DeclareUnicodeCharacter{2019}{\textquoteright}
\DeclareUnicodeCharacter{201A}{\quotesinglbase}
\DeclareUnicodeCharacter{201C}{\textquotedblleft}
\DeclareUnicodeCharacter{201D}{\textquotedblright}
\DeclareUnicodeCharacter{201E}{\quotedblbase}
\DeclareUnicodeCharacter{2020}{\ensuremath{\dagger}}
\DeclareUnicodeCharacter{2021}{\ddag}
\DeclareUnicodeCharacter{2022}{\ensuremath{\bullet}}
\DeclareUnicodeCharacter{2023}{\ensuremath{\RHD}}
\DeclareUnicodeCharacter{2026}{\ensuremath{\ldots}}
\DeclareUnicodeCharacter{202F}{\,}
\DeclareUnicodeCharacter{2030}{\textperthousand}
\DeclareUnicodeCharacter{2031}{\textpertenthousand}
\DeclareUnicodeCharacter{2032}{\ensuremath{\prime}}
\DeclareUnicodeCharacter{2033}{\ensuremath{''}}
\DeclareUnicodeCharacter{2034}{\ensuremath{'''}}
\DeclareUnicodeCharacter{2035}{\ensuremath{\backprime}}
\DeclareUnicodeCharacter{2038}{\ifmmode\widehat{}\else\textasciicircum\fi}
\DeclareUnicodeCharacter{2039}{\guilsinglleft}
\DeclareUnicodeCharacter{203A}{\guilsinglright}
\DeclareUnicodeCharacter{203B}{\textreferencemark}
\DeclareUnicodeCharacter{203C}{{!\kern -.5ex!}}
\DeclareUnicodeCharacter{203D}{\textinterrobang}
\DeclareUnicodeCharacter{203E}{\ensuremath{\overline{0}}}
\DeclareUnicodeCharacter{2042}{\LOCALunknownchar}
\DeclareUnicodeCharacter{2045}{\textlquill}
\DeclareUnicodeCharacter{2046}{\textrquill}
\DeclareUnicodeCharacter{2047}{{?\kern -.5ex?}}
\DeclareUnicodeCharacter{2048}{{?\kern -.5ex!}}
\DeclareUnicodeCharacter{2049}{{!\kern -.5ex?}}
\DeclareUnicodeCharacter{2052}{\textdiscount}
\DeclareUnicodeCharacter{2062}{{}}
\DeclareUnicodeCharacter{2070}{\ensuremath{^0}}
\DeclareUnicodeCharacter{2071}{\ensuremath{^i}}
\DeclareUnicodeCharacter{2074}{\ensuremath{^4}}
\DeclareUnicodeCharacter{2075}{\ensuremath{^5}}
\DeclareUnicodeCharacter{2076}{\ensuremath{^6}}
\DeclareUnicodeCharacter{2077}{\ensuremath{^7}}
\DeclareUnicodeCharacter{2078}{\ensuremath{^8}}
\DeclareUnicodeCharacter{2079}{\ensuremath{^9}}
\DeclareUnicodeCharacter{207A}{\ensuremath{^+}}
\DeclareUnicodeCharacter{207B}{\ensuremath{^-}}
\DeclareUnicodeCharacter{207C}{\ensuremath{^=}}
\DeclareUnicodeCharacter{207D}{\ensuremath{^(}}
\DeclareUnicodeCharacter{207E}{\ensuremath{^)}}
\DeclareUnicodeCharacter{207F}{\ensuremath{^n}}
\DeclareUnicodeCharacter{2080}{\ensuremath{_0}}
\DeclareUnicodeCharacter{2081}{\ensuremath{_1}}
\DeclareUnicodeCharacter{2082}{\ensuremath{_2}}
\DeclareUnicodeCharacter{2083}{\ensuremath{_3}}
\DeclareUnicodeCharacter{2084}{\ensuremath{_4}}
\DeclareUnicodeCharacter{2085}{\ensuremath{_5}}
\DeclareUnicodeCharacter{2086}{\ensuremath{_6}}
\DeclareUnicodeCharacter{2087}{\ensuremath{_7}}
\DeclareUnicodeCharacter{2088}{\ensuremath{_8}}
\DeclareUnicodeCharacter{2089}{\ensuremath{_9}}
\DeclareUnicodeCharacter{208A}{\ensuremath{_+}}
\DeclareUnicodeCharacter{208B}{\ensuremath{_-}}
\DeclareUnicodeCharacter{208C}{\ensuremath{_=}}
\DeclareUnicodeCharacter{208D}{\ensuremath{_(}}
\DeclareUnicodeCharacter{208E}{\ensuremath{_)}}
\DeclareUnicodeCharacter{2090}{\ensuremath{_a}}
\DeclareUnicodeCharacter{2091}{\ensuremath{_e}}
\DeclareUnicodeCharacter{2092}{\ensuremath{_o}}
\DeclareUnicodeCharacter{2093}{\ensuremath{_x}}
\DeclareUnicodeCharacter{2095}{\ensuremath{_h}}
\DeclareUnicodeCharacter{2096}{\ensuremath{_k}}
\DeclareUnicodeCharacter{2097}{\ensuremath{_l}}
\DeclareUnicodeCharacter{2098}{\ensuremath{_m}}
\DeclareUnicodeCharacter{2099}{\ensuremath{_n}}
\DeclareUnicodeCharacter{209A}{\ensuremath{_p}}
\DeclareUnicodeCharacter{209B}{\ensuremath{_s}}
\DeclareUnicodeCharacter{209C}{\ensuremath{_t}}
\DeclareUnicodeCharacter{20AC}{\euro}
\DeclareUnicodeCharacter{2102}{\ensuremath{\mathbb{C}}}
\DeclareUnicodeCharacter{2107}{\ensuremath{\mathbb{E}}}
\DeclareUnicodeCharacter{210A}{\ensuremath{\mathcal g}}
\DeclareUnicodeCharacter{210B}{\ensuremath{\mathcal H}}
\DeclareUnicodeCharacter{210C}{\ensuremath{\mathfrak H}}
\DeclareUnicodeCharacter{210D}{\ensuremath{\mathbb{H}}}
\DeclareUnicodeCharacter{210E}{\ensuremath{\mathit{h}}}
\DeclareUnicodeCharacter{210F}{\ensuremath{\hbar}}
\DeclareUnicodeCharacter{2110}{\ensuremath{\mathcal I}}
\DeclareUnicodeCharacter{2111}{\ensuremath{\IM}}
\DeclareUnicodeCharacter{2112}{\ensuremath{\mathcal L}}
\DeclareUnicodeCharacter{2113}{\ensuremath{\ell}}
\DeclareUnicodeCharacter{2115}{\ensuremath{\mathbb{N}}}
\DeclareUnicodeCharacter{2118}{\ensuremath{\wp}}
\DeclareUnicodeCharacter{2119}{\ensuremath{\mathbb{P}}}
\DeclareUnicodeCharacter{211A}{\ensuremath{\mathbb{Q}}}
\DeclareUnicodeCharacter{211B}{\ensuremath{\mathscr{R}}}
\DeclareUnicodeCharacter{211C}{\ensuremath{\Re}}
\DeclareUnicodeCharacter{211D}{\ensuremath{\mathbb{R}}}
\DeclareUnicodeCharacter{2122}{\texttrademark}
\DeclareUnicodeCharacter{2124}{\ensuremath{\mathbb{Z}}}
\DeclareUnicodeCharacter{2126}{\ensuremath{\Omega}}
\DeclareUnicodeCharacter{2127}{\ensuremath{\mho}}
\DeclareUnicodeCharacter{2128}{\ensuremath{\mathfrak Z}}
\DeclareUnicodeCharacter{212A}{\ensuremath{\mathrm K}}
\DeclareUnicodeCharacter{212B}{\ensuremath{\mathring{\mathrm A}}}
\DeclareUnicodeCharacter{212C}{\ensuremath{\mathcal B}}
\DeclareUnicodeCharacter{212D}{\ensuremath{\mathfrak C}}
\DeclareUnicodeCharacter{212E}{\textestimated}
\DeclareUnicodeCharacter{212F}{\ensuremath{\mathcal e}}
\DeclareUnicodeCharacter{2130}{\ensuremath{\mathcal E}}
\DeclareUnicodeCharacter{2131}{\ensuremath{\mathcal F}}
\DeclareUnicodeCharacter{2132}{\ensuremath{\Finv}}
\DeclareUnicodeCharacter{2133}{\ensuremath{\mathscr{M}}}
\DeclareUnicodeCharacter{2135}{\ensuremath{\aleph}}
\DeclareUnicodeCharacter{2136}{\ensuremath{\beth}}
\DeclareUnicodeCharacter{2137}{\ensuremath{\gimel}}
\DeclareUnicodeCharacter{2138}{\ensuremath{\daleth}}
\DeclareUnicodeCharacter{213C}{\ensuremath{\mathbb{\pi}}}
\DeclareUnicodeCharacter{213D}{\ensuremath{\mathbb{\gamma}}}
\DeclareUnicodeCharacter{213E}{\ensuremath{\mathbb{\Pi}}}
\DeclareUnicodeCharacter{213F}{\ensuremath{\mathbb{\Gamma}}}
\DeclareUnicodeCharacter{2140}{\ensuremath{\mathbb{\Sigma}}}
\DeclareUnicodeCharacter{2141}{\ensuremath{\Game}}
\DeclareUnicodeCharacter{2144}{\ensuremath{Y}}
\DeclareUnicodeCharacter{2146}{\ensuremath{\mathrm{d}}}
\DeclareUnicodeCharacter{2148}{\ensuremath{\imath}}
\DeclareUnicodeCharacter{2149}{\ensuremath{\jmath}}
\DeclareUnicodeCharacter{214B}{\LOCALunknownchar}
\DeclareUnicodeCharacter{2153}{\ensuremath{\frac{1}{3}}}
\DeclareUnicodeCharacter{2154}{\ensuremath{\frac{2}{3}}}
\DeclareUnicodeCharacter{2155}{\ensuremath{\frac{1}{5}}}
\DeclareUnicodeCharacter{2156}{\ensuremath{\frac{2}{5}}}
\DeclareUnicodeCharacter{2157}{\ensuremath{\frac{3}{5}}}
\DeclareUnicodeCharacter{2158}{\ensuremath{\frac{4}{5}}}
\DeclareUnicodeCharacter{2159}{\ensuremath{\frac{1}{6}}}
\DeclareUnicodeCharacter{215A}{\ensuremath{\frac{5}{6}}}
\DeclareUnicodeCharacter{215B}{\ensuremath{\frac{1}{8}}}
\DeclareUnicodeCharacter{215D}{\ensuremath{\frac{5}{8}}}
\DeclareUnicodeCharacter{215E}{\ensuremath{\frac{7}{8}}}
\DeclareUnicodeCharacter{2190}{\ensuremath{\leftarrow}}
\DeclareUnicodeCharacter{2191}{\ensuremath{\uparrow}}
\DeclareUnicodeCharacter{2192}{\ensuremath{\rightarrow}}
\DeclareUnicodeCharacter{2193}{\ensuremath{\downarrow}}
\DeclareUnicodeCharacter{2194}{\ensuremath{\leftrightarrow}}
\DeclareUnicodeCharacter{2195}{\ensuremath{\updownarrow}}
\DeclareUnicodeCharacter{2196}{\ensuremath{\nwarrow}}
\DeclareUnicodeCharacter{2197}{\ensuremath{\nearrow}}
\DeclareUnicodeCharacter{2198}{\ensuremath{\searrow}}
\DeclareUnicodeCharacter{2199}{\ensuremath{\swarrow}}
\DeclareUnicodeCharacter{219A}{\ensuremath{\nleftarrow}}
\DeclareUnicodeCharacter{219B}{\ensuremath{\nrightarrow}}
\DeclareUnicodeCharacter{219E}{\ensuremath{\twoheadleftarrow}}
\DeclareUnicodeCharacter{21A0}{\ensuremath{\twoheadrightarrow}}
\DeclareUnicodeCharacter{21A2}{\ensuremath{\leftarrowtail}}
\DeclareUnicodeCharacter{21A3}{\ensuremath{\rightarrowtail}}
\DeclareUnicodeCharacter{21A4}{\ensuremath{\mapsfrom}}
\DeclareUnicodeCharacter{21A6}{\ensuremath{\mapsto}}
\DeclareUnicodeCharacter{21A9}{\ensuremath{\hookleftarrow}}
\DeclareUnicodeCharacter{21AA}{\ensuremath{\hookrightarrow}}
\DeclareUnicodeCharacter{21AB}{\ensuremath{\looparrowleft}}
\DeclareUnicodeCharacter{21AC}{\ensuremath{\looparrowright}}
\DeclareUnicodeCharacter{21AD}{\ensuremath{\leftrightsquigarrow}}
\DeclareUnicodeCharacter{21AE}{\ensuremath{\nleftrightarrow}}
\DeclareUnicodeCharacter{21AF}{\ensuremath{\lightning}}
\DeclareUnicodeCharacter{21B0}{\ensuremath{\Lsh}}
\DeclareUnicodeCharacter{21B1}{\ensuremath{\Rsh}}
\DeclareUnicodeCharacter{21B6}{\ensuremath{\curvearrowleft}}
\DeclareUnicodeCharacter{21B7}{\ensuremath{\curvearrowright}}
\DeclareUnicodeCharacter{21BA}{\ensuremath{\circlearrowleft}}
\DeclareUnicodeCharacter{21BB}{\ensuremath{\circlearrowright}}
\DeclareUnicodeCharacter{21BC}{\ensuremath{\leftharpoonup}}
\DeclareUnicodeCharacter{21BD}{\ensuremath{\leftharpoondown}}
\DeclareUnicodeCharacter{21BE}{\ensuremath{\upharpoonright}}
\DeclareUnicodeCharacter{21BF}{\ensuremath{\upharpoonleft}}
\DeclareUnicodeCharacter{21C0}{\ensuremath{\rightharpoonup}}
\DeclareUnicodeCharacter{21C1}{\ensuremath{\rightharpoondown}}
\DeclareUnicodeCharacter{21C2}{\ensuremath{\downharpoonright}}
\DeclareUnicodeCharacter{21C3}{\ensuremath{\downharpoonleft}}
\DeclareUnicodeCharacter{21C4}{\ensuremath{\rightleftarrows}}
\DeclareUnicodeCharacter{21C5}{\LOCALunknownchar}
\DeclareUnicodeCharacter{21C6}{\ensuremath{\leftrightarrows}}
\DeclareUnicodeCharacter{21C7}{\ensuremath{\leftleftarrows}}
\DeclareUnicodeCharacter{21C8}{\ensuremath{\upuparrows}}
\DeclareUnicodeCharacter{21C9}{\ensuremath{\rightrightarrows}}
\DeclareUnicodeCharacter{21CA}{\ensuremath{\downdownarrows}}
\DeclareUnicodeCharacter{21CB}{\ensuremath{\leftrightharpoons}}
\DeclareUnicodeCharacter{21CC}{\ensuremath{\rightleftharpoons}}
\DeclareUnicodeCharacter{21CD}{\ensuremath{\nLeftarrow}}
\DeclareUnicodeCharacter{21CE}{\ensuremath{\nLeftrightarrow}}
\DeclareUnicodeCharacter{21CF}{\ensuremath{\nRightarrow}}
\DeclareUnicodeCharacter{21D0}{\ensuremath{\Leftarrow}}
\DeclareUnicodeCharacter{21D1}{\ensuremath{\Uparrow}}
\DeclareUnicodeCharacter{21D2}{\ensuremath{\Rightarrow}}
\DeclareUnicodeCharacter{21D3}{\ensuremath{\Downarrow}}
\DeclareUnicodeCharacter{21D4}{\ensuremath{\Leftrightarrow}}
\DeclareUnicodeCharacter{21D5}{\ensuremath{\Updownarrow}}
\DeclareUnicodeCharacter{21D6}{\ensuremath{\nwarrow}}
\DeclareUnicodeCharacter{21D7}{\ensuremath{\nearrow}}
\DeclareUnicodeCharacter{21D8}{\ensuremath{\searrow}}
\DeclareUnicodeCharacter{21D9}{\ensuremath{\swarrow}}
\DeclareUnicodeCharacter{21DA}{\ensuremath{\Lleftarrow}}
\DeclareUnicodeCharacter{21DB}{\ensuremath{\Rrightarrow}}
\DeclareUnicodeCharacter{21DC}{\LOCALunknownchar}
\DeclareUnicodeCharacter{21DD}{\ensuremath{\rightsquigarrow}}
\DeclareUnicodeCharacter{21E0}{\ensuremath{\dashleftarrow}}
\DeclareUnicodeCharacter{21E2}{\ensuremath{\dashrightarrow}}
\DeclareUnicodeCharacter{21E4}{\ensuremath{\Leftarrow}}
\DeclareUnicodeCharacter{21E5}{\ensuremath{\Rightarrow}}
\DeclareUnicodeCharacter{21F0}{\ensuremath{\mapsto}}
\DeclareUnicodeCharacter{21FD}{\ensuremath{\leftarrow}}
\DeclareUnicodeCharacter{21FE}{\ensuremath{\rightarrow}}
\DeclareUnicodeCharacter{21FF}{\ensuremath{\leftrightarrow}}
\DeclareUnicodeCharacter{2200}{\ensuremath{\forall}}
\DeclareUnicodeCharacter{2201}{\ensuremath{\complement}}
\DeclareUnicodeCharacter{2202}{\ensuremath{\partial}}
\DeclareUnicodeCharacter{2203}{\ensuremath{\exists}}
\DeclareUnicodeCharacter{2204}{\ensuremath{\not\exists}}
\DeclareUnicodeCharacter{2205}{\ensuremath{\varnothing}}
\DeclareUnicodeCharacter{2207}{\ensuremath{\nabla}}
\DeclareUnicodeCharacter{2208}{\ensuremath{\in}}
\DeclareUnicodeCharacter{2209}{\ensuremath{\notin}}
\DeclareUnicodeCharacter{220B}{\ensuremath{\ni}}
\DeclareUnicodeCharacter{220C}{\ensuremath{!\ni}}
\DeclareUnicodeCharacter{220D}{\ensuremath{\bullet}}
\DeclareUnicodeCharacter{220E}{{\tiny \ensuremath{\blacksquare}}}
\DeclareUnicodeCharacter{220F}{\ensuremath{\prod}}
\DeclareUnicodeCharacter{2210}{\ensuremath{\coprod}}
\DeclareUnicodeCharacter{2211}{\ensuremath{\sum}}
\DeclareUnicodeCharacter{2212}{-}
\DeclareUnicodeCharacter{2213}{\ensuremath{\mp}}
\DeclareUnicodeCharacter{2214}{\ensuremath{\dotplus}}
\DeclareUnicodeCharacter{2215}{\ensuremath{/}}
\DeclareUnicodeCharacter{2216}{\ensuremath{\smallsetminus}}
\DeclareUnicodeCharacter{2217}{\ensuremath{\star}}
\DeclareUnicodeCharacter{2218}{\ensuremath{\circ}}
\DeclareUnicodeCharacter{2219}{\ensuremath{\bullet}}
\DeclareUnicodeCharacter{221A}{\ensuremath{\sqrt{}}}
\DeclareUnicodeCharacter{221B}{\ensuremath{\sqrt[3]{}}}
\DeclareUnicodeCharacter{221C}{\ensuremath{\sqrt[4]{}}}
\DeclareUnicodeCharacter{221D}{\ensuremath{\propto}}
\DeclareUnicodeCharacter{221E}{\ensuremath{\infty}}
\DeclareUnicodeCharacter{2220}{\ensuremath{\angle}}
\DeclareUnicodeCharacter{2221}{\ensuremath{\measuredangle}}
\DeclareUnicodeCharacter{2222}{\ensuremath{\sphericalangle}}
\DeclareUnicodeCharacter{2223}{\ensuremath{\mid}}
\DeclareUnicodeCharacter{2224}{\ensuremath{\nmid}}
\DeclareUnicodeCharacter{2225}{\ensuremath{\parallel}}
\DeclareUnicodeCharacter{2226}{\ensuremath{\nparallel}}
\DeclareUnicodeCharacter{2227}{\ensuremath{\wedge}}
\DeclareUnicodeCharacter{2228}{\ensuremath{\vee}}
\DeclareUnicodeCharacter{2229}{\ensuremath{\cap}}
\DeclareUnicodeCharacter{222A}{\ensuremath{\cup}}
\DeclareUnicodeCharacter{222B}{\ensuremath{\int}}
\DeclareUnicodeCharacter{222C}{\ensuremath{\iint}}
\DeclareUnicodeCharacter{222D}{\ensuremath{\iiint}}
\DeclareUnicodeCharacter{222E}{\ensuremath{\oint}}
\DeclareUnicodeCharacter{222F}{\LOCALunknownchar}
\DeclareUnicodeCharacter{2230}{\LOCALunknownchar}
\DeclareUnicodeCharacter{2232}{\LOCALunknownchar}
\DeclareUnicodeCharacter{2233}{\LOCALunknownchar}
\DeclareUnicodeCharacter{2234}{\ensuremath{\therefore}}
\DeclareUnicodeCharacter{2235}{\ensuremath{\because}}
\DeclareUnicodeCharacter{2236}{:}
\DeclareUnicodeCharacter{2237}{\LOCALunknownchar}
\DeclareUnicodeCharacter{2238}{\LOCALunknownchar}
\DeclareUnicodeCharacter{2239}{\ensuremath{\eqcolon}}
\DeclareUnicodeCharacter{223C}{\ensuremath{\sim}}
\DeclareUnicodeCharacter{223D}{\ensuremath{\backsim}}
\DeclareUnicodeCharacter{223F}{\AC}
\DeclareUnicodeCharacter{2240}{\ensuremath{\wr}}
\DeclareUnicodeCharacter{2241}{\ensuremath{\nsim}}
\DeclareUnicodeCharacter{2243}{\ensuremath{\simeq}}
\DeclareUnicodeCharacter{2244}{\ensuremath{\not\simeq}}
\DeclareUnicodeCharacter{2245}{\ensuremath{\cong}}
\DeclareUnicodeCharacter{2247}{\ensuremath{\ncong}}
\DeclareUnicodeCharacter{2248}{\ensuremath{\approx}}
\DeclareUnicodeCharacter{2249}{\ensuremath{\not\approx}}
\DeclareUnicodeCharacter{224A}{\ensuremath{\approxeq}}
\DeclareUnicodeCharacter{224D}{\ensuremath{\asymp}}
\DeclareUnicodeCharacter{224E}{\ensuremath{\Bumpeq}}
\DeclareUnicodeCharacter{224F}{\ensuremath{\bumpeq}}
\DeclareUnicodeCharacter{2250}{\ensuremath{\doteq}}
\DeclareUnicodeCharacter{2251}{\ensuremath{\doteqdot}}
\DeclareUnicodeCharacter{2252}{\ensuremath{\fallingdotseq}}
\DeclareUnicodeCharacter{2253}{\ensuremath{\risingdotseq}}
\DeclareUnicodeCharacter{2254}{\ensuremath{\coloneqq}}
\DeclareUnicodeCharacter{2255}{\ensuremath{\eqqcolon}}
\DeclareUnicodeCharacter{2256}{\ensuremath{\eqcirc}}
\DeclareUnicodeCharacter{2257}{\ensuremath{\circeq}}
\DeclareUnicodeCharacter{2258}{\ensuremath{\stackrel{\frown}{=}}}
\DeclareUnicodeCharacter{2259}{\ensuremath{\stackrel{\wedge}{=}}}
\DeclareUnicodeCharacter{225A}{\ensuremath{\stackrel{\vee}{=}}}
\DeclareUnicodeCharacter{225B}{\ensuremath{\stackrel{\star}{=}}}
\DeclareUnicodeCharacter{225C}{\ensuremath{\triangleq}}
\DeclareUnicodeCharacter{225D}{\ensuremath{\stackrel{\text{\tiny def}}{=}}}
\DeclareUnicodeCharacter{225F}{\ensuremath{\stackrel{\text{\tiny ?}}{=}}}
\DeclareUnicodeCharacter{2260}{\ensuremath{\ne}}
\DeclareUnicodeCharacter{2261}{\ensuremath{\equiv}}
\DeclareUnicodeCharacter{2262}{\ensuremath{\not\equiv}}
\DeclareUnicodeCharacter{2263}{\ensuremath{\stackrel{=}{=}}}
\DeclareUnicodeCharacter{2264}{\ensuremath{\le}}
\DeclareUnicodeCharacter{2265}{\ensuremath{\ge}}
\DeclareUnicodeCharacter{2266}{\ensuremath{\leqq}}
\DeclareUnicodeCharacter{2267}{\ensuremath{\geqq}}
\DeclareUnicodeCharacter{2268}{\ensuremath{\lneqq}}
\DeclareUnicodeCharacter{2269}{\ensuremath{\gneqq}}
\DeclareUnicodeCharacter{226A}{\ensuremath{\ll}}
\DeclareUnicodeCharacter{226B}{\ensuremath{\gg}}
\DeclareUnicodeCharacter{226C}{\ensuremath{\between}}
\DeclareUnicodeCharacter{226D}{\ensuremath{\not\asymp}}
\DeclareUnicodeCharacter{226E}{\ensuremath{\nless}}
\DeclareUnicodeCharacter{226F}{\ensuremath{\ngtr}}
\DeclareUnicodeCharacter{2270}{\ensuremath{\nleq}}
\DeclareUnicodeCharacter{2271}{\ensuremath{\ngeq}}
\DeclareUnicodeCharacter{2272}{\ensuremath{\lesssim}}
\DeclareUnicodeCharacter{2273}{\ensuremath{\gtrsim}}
\DeclareUnicodeCharacter{2274}{\ensuremath{\not\lesssim}}
\DeclareUnicodeCharacter{2275}{\ensuremath{\not\gtrsim}}
\DeclareUnicodeCharacter{2276}{\ensuremath{\lessgtr}}
\DeclareUnicodeCharacter{2277}{\ensuremath{\gtrless}}
\DeclareUnicodeCharacter{2278}{\ensuremath{\not\lessgtr}}
\DeclareUnicodeCharacter{2279}{\ensuremath{\not\gtrless}}
\DeclareUnicodeCharacter{227A}{\ensuremath{\prec}}
\DeclareUnicodeCharacter{227B}{\ensuremath{\succ}}
\DeclareUnicodeCharacter{227C}{\ensuremath{\preccurlyeq}}
\DeclareUnicodeCharacter{227D}{\LOCALunknownchar}
\DeclareUnicodeCharacter{227E}{\ensuremath{\precsim}}
\DeclareUnicodeCharacter{227F}{\ensuremath{\succsim}}
\DeclareUnicodeCharacter{2280}{\ensuremath{\nprec}}
\DeclareUnicodeCharacter{2281}{\ensuremath{\nsucc}}
\DeclareUnicodeCharacter{2282}{\ensuremath{\subset}}
\DeclareUnicodeCharacter{2283}{\ensuremath{\supset}}
\DeclareUnicodeCharacter{2284}{\ensuremath{\not\subset}}
\DeclareUnicodeCharacter{2285}{\ensuremath{\not\supset}}
\DeclareUnicodeCharacter{2286}{\ensuremath{\subseteq}}
\DeclareUnicodeCharacter{2287}{\ensuremath{\supseteq}}
\DeclareUnicodeCharacter{2288}{\ensuremath{\nsubseteq}}
\DeclareUnicodeCharacter{2289}{\ensuremath{\nsupseteq}}
\DeclareUnicodeCharacter{228A}{\ensuremath{\subsetneq}}
\DeclareUnicodeCharacter{228B}{\ensuremath{\supsetneq}}
\DeclareUnicodeCharacter{228E}{\ensuremath{\uplus}}
\DeclareUnicodeCharacter{228F}{\ensuremath{\sqsubset}}
\DeclareUnicodeCharacter{2290}{\ensuremath{\sqsupset}}
\DeclareUnicodeCharacter{2291}{\ensuremath{\sqsubseteq}}
\DeclareUnicodeCharacter{2292}{\ensuremath{\sqsupseteq}}
\DeclareUnicodeCharacter{2293}{\ensuremath{\sqcap}}
\DeclareUnicodeCharacter{2294}{\ensuremath{\sqcup}}
\DeclareUnicodeCharacter{2295}{\ensuremath{\oplus}}
\DeclareUnicodeCharacter{2296}{\ensuremath{\ominus}}
\DeclareUnicodeCharacter{2297}{\ensuremath{\otimes}}
\DeclareUnicodeCharacter{2298}{\ensuremath{\oslash}}
\DeclareUnicodeCharacter{2299}{\ensuremath{\odot}}
\DeclareUnicodeCharacter{229A}{\ensuremath{\circledcirc}}
\DeclareUnicodeCharacter{229B}{\ensuremath{\circledast}}
\DeclareUnicodeCharacter{229D}{\ensuremath{\circleddash}}
\DeclareUnicodeCharacter{229E}{\ensuremath{\boxplus}}
\DeclareUnicodeCharacter{229F}{\ensuremath{\boxminus}}
\DeclareUnicodeCharacter{22A0}{\ensuremath{\boxtimes}}
\DeclareUnicodeCharacter{22A1}{\ensuremath{\boxdot}}
\DeclareUnicodeCharacter{22A2}{\ensuremath{\vdash}}
\DeclareUnicodeCharacter{22A3}{\ensuremath{\dashv}}
\DeclareUnicodeCharacter{22A4}{\ensuremath{\top}}
\DeclareUnicodeCharacter{22A5}{\ensuremath{\bot}}
\DeclareUnicodeCharacter{22A6}{\ensuremath{\vdash}}
\DeclareUnicodeCharacter{22A7}{\ensuremath{\models}}
\DeclareUnicodeCharacter{22A9}{\ensuremath{\Vdash}}
\DeclareUnicodeCharacter{22AA}{\ensuremath{\Vvdash}}
\DeclareUnicodeCharacter{22AB}{\LOCALunknownchar}
\DeclareUnicodeCharacter{22AC}{\ensuremath{\not\vdash}}
\DeclareUnicodeCharacter{22AD}{\ensuremath{\not\vDash}}
\DeclareUnicodeCharacter{22AE}{\ensuremath{\not\Vdash}}
\DeclareUnicodeCharacter{22AF}{\LOCALunknownchar}
\DeclareUnicodeCharacter{22B2}{\ensuremath{\triangleleft}}
\DeclareUnicodeCharacter{22B3}{\ensuremath{\triangleright}}
\DeclareUnicodeCharacter{22B4}{\ensuremath{\unlhd}}
\DeclareUnicodeCharacter{22B5}{\ensuremath{\unrhd}}
\DeclareUnicodeCharacter{22B8}{\ensuremath{\multimap}}
\DeclareUnicodeCharacter{22BA}{\ensuremath{\intercal}}
\DeclareUnicodeCharacter{22BB}{\ensuremath{\veebar}}
\DeclareUnicodeCharacter{22BC}{\ensuremath{\barwedge}}
\DeclareUnicodeCharacter{22C0}{\ensuremath{\bigwedge}}
\DeclareUnicodeCharacter{22C1}{\ensuremath{\bigvee}}
\DeclareUnicodeCharacter{22C2}{\ensuremath{\bigcap}}
\DeclareUnicodeCharacter{22C3}{\ensuremath{\bigcup}}
\DeclareUnicodeCharacter{22C4}{\ensuremath{\diamond}}
\DeclareUnicodeCharacter{22C5}{\ensuremath{\cdot}}
\DeclareUnicodeCharacter{22C6}{\ensuremath{\star}}
\DeclareUnicodeCharacter{22C7}{\ensuremath{\divideontimes}}
\DeclareUnicodeCharacter{22C8}{\ensuremath{\bowtie}}
\DeclareUnicodeCharacter{22C9}{\ensuremath{\ltimes}}
\DeclareUnicodeCharacter{22CA}{\ensuremath{\rtimes}}
\DeclareUnicodeCharacter{22CB}{\ensuremath{\leftthreetimes}}
\DeclareUnicodeCharacter{22CC}{\ensuremath{\rightthreetimes}}
\DeclareUnicodeCharacter{22CD}{\ensuremath{\backsimeq}}
\DeclareUnicodeCharacter{22CE}{\ensuremath{\curlyvee}}
\DeclareUnicodeCharacter{22CF}{\ensuremath{\curlywedge}}
\DeclareUnicodeCharacter{22D0}{\ensuremath{\Subset}}
\DeclareUnicodeCharacter{22D1}{\ensuremath{\Supset}}
\DeclareUnicodeCharacter{22D2}{\ensuremath{\Cap}}
\DeclareUnicodeCharacter{22D3}{\ensuremath{\Cup}}
\DeclareUnicodeCharacter{22D4}{\ensuremath{\pitchfork}}
\DeclareUnicodeCharacter{22D6}{\ensuremath{\lessdot}}
\DeclareUnicodeCharacter{22D7}{\ensuremath{\gtrdot}}
\DeclareUnicodeCharacter{22D8}{\ensuremath{\lll}}
\DeclareUnicodeCharacter{22D9}{\ensuremath{\ggg}}
\DeclareUnicodeCharacter{22DA}{\ensuremath{\lesseqgtr}}
\DeclareUnicodeCharacter{22DB}{\ensuremath{\gtreqless}}
\DeclareUnicodeCharacter{22DE}{\ensuremath{\curlyeqprec}}
\DeclareUnicodeCharacter{22DF}{\ensuremath{\curlyeqsucc}}
\DeclareUnicodeCharacter{22E0}{\ensuremath{\not\preceq}}
\DeclareUnicodeCharacter{22E1}{\ensuremath{\not\succeq}}
\DeclareUnicodeCharacter{22E2}{\ensuremath{\not\sqsubseteq}}
\DeclareUnicodeCharacter{22E3}{\ensuremath{\not\sqsupseteq}}
\DeclareUnicodeCharacter{22E4}{\LOCALunknownchar}
\DeclareUnicodeCharacter{22E5}{\LOCALunknownchar}
\DeclareUnicodeCharacter{22E6}{\ensuremath{\lnsim}}
\DeclareUnicodeCharacter{22E7}{\ensuremath{\gnsim}}
\DeclareUnicodeCharacter{22E8}{\ensuremath{\precnsim}}
\DeclareUnicodeCharacter{22E9}{\ensuremath{\succnsim}}
\DeclareUnicodeCharacter{22EA}{\ensuremath{\not\triangleleft}}
\DeclareUnicodeCharacter{22EB}{\ensuremath{\not\triangleright}}
\DeclareUnicodeCharacter{22EC}{\ensuremath{\not\trianglelefteq}}
\DeclareUnicodeCharacter{22ED}{\ensuremath{\not\trianglerighteq}}
\DeclareUnicodeCharacter{22EE}{\ensuremath{\vdots}}
\DeclareUnicodeCharacter{22EF}{\ensuremath{\cdots}}
\DeclareUnicodeCharacter{22F0}{\LOCALunknownchar}
\DeclareUnicodeCharacter{22F1}{\ensuremath{\ddots}}
\DeclareUnicodeCharacter{2300}{\ensuremath{\diameter}}
\DeclareUnicodeCharacter{2308}{\ensuremath{\lceil}}
\DeclareUnicodeCharacter{2309}{\ensuremath{\rceil}}
\DeclareUnicodeCharacter{230A}{\ensuremath{\lfloor}}
\DeclareUnicodeCharacter{230B}{\ensuremath{\rfloor}}
\DeclareUnicodeCharacter{2322}{\ensuremath{\frown}}
\DeclareUnicodeCharacter{2323}{\ensuremath{\smile}}
\DeclareUnicodeCharacter{2329}{\ensuremath{\langle}}
\DeclareUnicodeCharacter{232A}{\ensuremath{\rangle}}
\DeclareUnicodeCharacter{23CE}{\ensuremath{\hookleftarrow}}
\DeclareUnicodeCharacter{2460}{\ensuremath{\text{1}}}
\DeclareUnicodeCharacter{2461}{\ensuremath{\text{2}}}
\DeclareUnicodeCharacter{2462}{\ensuremath{\text{3}}}
\DeclareUnicodeCharacter{2463}{\ensuremath{\text{4}}}
\DeclareUnicodeCharacter{2464}{\ensuremath{\text{5}}}
\DeclareUnicodeCharacter{2465}{\ensuremath{\text{6}}}
\DeclareUnicodeCharacter{2466}{\ensuremath{\text{7}}}
\DeclareUnicodeCharacter{2467}{\ensuremath{\text{8}}}
\DeclareUnicodeCharacter{2468}{\ensuremath{\text{9}}}
\DeclareUnicodeCharacter{25A1}{\ensuremath{\square}}
\DeclareUnicodeCharacter{25B3}{\ensuremath{\triangle}}
\DeclareUnicodeCharacter{25C5}{\ensuremath{\triangleleft}}
\DeclareUnicodeCharacter{2610}{\fbox{\ensuremath{\phantom{{\checkmark}}}}}
\DeclareUnicodeCharacter{2611}{\fbox{\ensuremath{\checkmark}}}
\DeclareUnicodeCharacter{2615}{\LOCALunknownchar}
\DeclareUnicodeCharacter{2621}{\LOCALunknownchar}
\DeclareUnicodeCharacter{2627}{\LOCALunknownchar}
\DeclareUnicodeCharacter{2639}{\ensuremath{\frownie}}
\DeclareUnicodeCharacter{263A}{\ensuremath{\smiley}}
\DeclareUnicodeCharacter{2660}{\ensuremath{\spadesuit}}
\DeclareUnicodeCharacter{2661}{\ensuremath{\heartsuit}}
\DeclareUnicodeCharacter{2662}{\ensuremath{\diamondsuit}}
\DeclareUnicodeCharacter{2663}{\ensuremath{\clubsuit}}
\DeclareUnicodeCharacter{266D}{\ensuremath{\flat}}
\DeclareUnicodeCharacter{266E}{\ensuremath{\natural}}
\DeclareUnicodeCharacter{266F}{\ensuremath{\sharp}}
\DeclareUnicodeCharacter{26A0}{\ensuremath{\lower .25ex\hbox{\Large $\triangle$\hskip -1.25ex}!\;\,}}
\DeclareUnicodeCharacter{2713}{\ensuremath{\checkmark}}
\DeclareUnicodeCharacter{27C2}{\ensuremath{\perp}}
\DeclareUnicodeCharacter{27E6}{\ensuremath{[}}
\DeclareUnicodeCharacter{27E7}{\ensuremath{]}}
\DeclareUnicodeCharacter{27E8}{\ensuremath{\langle}}
\DeclareUnicodeCharacter{27E9}{\ensuremath{\rangle}}
\DeclareUnicodeCharacter{27EA}{\ensuremath{\llangle}}
\DeclareUnicodeCharacter{27EB}{\ensuremath{\rrangle}}
\DeclareUnicodeCharacter{27F5}{\ensuremath{\longleftarrow}}
\DeclareUnicodeCharacter{27F6}{\ensuremath{\longrightarrow}}
\DeclareUnicodeCharacter{2983}{\LOCALunknownchar}
\DeclareUnicodeCharacter{2984}{\LOCALunknownchar}
\DeclareUnicodeCharacter{2985}{\LOCALunknownchar}
\DeclareUnicodeCharacter{2986}{\LOCALunknownchar}
\DeclareUnicodeCharacter{2987}{\ensuremath{(}}
\DeclareUnicodeCharacter{2988}{\ensuremath{)}}
\DeclareUnicodeCharacter{29F5}{\ensuremath{\setminus}}
\DeclareUnicodeCharacter{2A00}{\ensuremath{\bigodot}}
\DeclareUnicodeCharacter{2A01}{\ensuremath{\bigoplus}}
\DeclareUnicodeCharacter{2A02}{\ensuremath{\bigotimes}}
\DeclareUnicodeCharacter{2A05}{\LOCALunknownchar}
\DeclareUnicodeCharacter{2A06}{\ensuremath{\bigsqcup}}
\DeclareUnicodeCharacter{2A0C}{\ensuremath{\iiiint}}
\DeclareUnicodeCharacter{2A1D}{\ensuremath{\Join}}
\DeclareUnicodeCharacter{2A3F}{\ensuremath{\amalg}}
\DeclareUnicodeCharacter{2A7D}{\ensuremath{\leqslant}}
\DeclareUnicodeCharacter{2A7E}{\ensuremath{\geqslant}}
\DeclareUnicodeCharacter{2AA8}{\LOCALunknownchar}
\DeclareUnicodeCharacter{2AA9}{\LOCALunknownchar}
\DeclareUnicodeCharacter{2AAF}{\ensuremath{\preceq}}
\DeclareUnicodeCharacter{2AB0}{\ensuremath{\succeq}}
\DeclareUnicodeCharacter{2C7C}{\ensuremath{_j}}
\DeclareUnicodeCharacter{2E18}{\textinterrobangdown}
\DeclareUnicodeCharacter{301A}{\ensuremath{[}}
\DeclareUnicodeCharacter{301B}{\ensuremath{]}}
\DeclareUnicodeCharacter{33D1}{\ensuremath{\ln}}
\DeclareUnicodeCharacter{33D2}{\ensuremath{\log}}
\DeclareUnicodeCharacter{D7B0}{\LOCALunknownchar}
\fi

% -----------------------------------------------------------------------------
% PORTADA
% -----------------------------------------------------------------------------
\newcommand{\templatePortrait}{
	
	% Configura la página
	\clearpage
	\def\arraystretch {\tablepaddingv} % Ajusta espaciamiento de las tablas
	\renewcommand{\thepage}{\nameportraitpage}
	\setpagemargincm{\pagemarginleftportrait}{\pagemargintop}{\pagemarginright}{\pagemarginbottom}

	\pagestyle{fancy}
	\fancyhf{}
	\renewcommand{\headrulewidth}{0pt}
	\renewcommand{\footrulewidth}{0pt}

	% Logo de la universidad
	\hspace*{0.5cm} % Necesario para centrar el logo con el título
	\coreinsertkeyimage{\universitydepartmentimagecfg}{\universitydepartmentimage}%
	\hspace*{0.1cm}%
	\begin{minipage}{0.8\linewidth}%
		\MakeUppercase \universityname  ~ \\%
		\MakeUppercase \universityfaculty ~ \\%
		\MakeUppercase \universitydepartment%
		\vspace*{2.8cm}\mbox{}%
	\end{minipage}

	% Importada la portada
	\portrait%
	
	% Ajusta la fuente
	\normalfont%
	
}

% -----------------------------------------------------------------------------
% CONFIGURACIÓN DE PÁGINA Y ENCABEZADOS
% -----------------------------------------------------------------------------
\newcommand{\templatePagecfg}{
	
	% -------------------------------------------------------------------------
	% Numeración de páginas
	% -------------------------------------------------------------------------
	\clearpage
	\ifthenelse{\equal{\predocpageromannumber}{true}}{ % Si se usan números romanos en el pre-documento
		\ifthenelse{\equal{\predocpageromanupper}{true}}{
			\pagenumbering{Roman}
		}{
			\pagenumbering{roman}
		}}{
		\pagenumbering{arabic}
	}
	\setcounter{page}{1}
	\setcounter{footnote}{0}
	
	% -------------------------------------------------------------------------
	% Márgenes de páginas y tablas
	% -------------------------------------------------------------------------
	\setpagemargincm{\pagemarginleft}{\pagemargintop}{\pagemarginright}{\pagemarginbottom}
	\resettablecellpadding
	
	% -------------------------------------------------------------------------
	% Se define el punto decimal
	% -------------------------------------------------------------------------
	\ifthenelse{\equal{\pointdecimal}{true}}{
		\decimalpoint}{
	}
	
	% -------------------------------------------------------------------------
	% Definición de nombres de objetos
	% -------------------------------------------------------------------------
	\renewcommand{\abstractname}{\nameabstract} % Nombre del abstract
	\renewcommand{\appendixname}{\nameltappendixsection} % Nombre del anexo (título)
	\renewcommand{\appendixpagename}{\nameappendixsection} % Nombre del anexo en índice
	\renewcommand{\appendixtocname}{\nameappendixsection} % Nombre del anexo en índice
	\renewcommand{\chaptername}{\namechapter}  % Nombre de los capítulos
	\renewcommand{\contentsname}{\nameltcont} % Nombre del índice
	\renewcommand{\figurename}{\nameltwfigure} % Nombre de la leyenda de las fig.
	\renewcommand{\listfigurename}{\nameltfigure} % Nombre del índice de figuras
	\renewcommand{\listtablename}{\namelttable} % Nombre del índice de tablas
	\renewcommand{\lstlistingname}{\nameltwsrc} % Nombre leyenda del código fuente
	\renewcommand{\lstlistlistingname}{\nameltsrc} % Nombre índice código fuente
	\renewcommand{\refname}{\namereferences} % Nombre de las referencias (bibtex)
	\renewcommand{\bibname}{\namereferences} % Nombre de las referencias (natbib)
	\renewcommand{\tablename}{\nameltwtable} % Nombre de la leyenda de tablas
	
	% -------------------------------------------------------------------------
	% Estilo de títulos
	% -------------------------------------------------------------------------
	\sectionfont{%
		\color{\sectioncolor} \sectionfontsize \sectionfontstyle \selectfont%
	}
	\subsectionfont{%
		\color{\ssectioncolor} \ssectionfontsize \ssectionfontstyle \selectfont%
	}
	\subsubsectionfont{%
		\color{\sssectioncolor} \sssectionfontsize \sssectionfontstyle \selectfont%
	}
	\titleformat{\subsubsubsection}{%
		\color{\ssssectioncolor} \ssssectionfontsz \ssssectionfontstyle%
	}{%
		\GLOBALtitlepresubsubsubsectionstr\thesubsubsubsection\charaftersectionnum\spacingaftersection%
		\corepatchaftersubsubsubsection%
	}{0em}{%
	}
	\def\bibfont {\fontsizerefbibl} % Tamaño de fuente de las referencias
	
	% -------------------------------------------------------------------------
	% Estilo citas
	% -------------------------------------------------------------------------
	\ifthenelse{\equal{\stylecitereferences}{apacite}}{%
		\renewcommand{\BOthers}[1]{\apacitebothers\hbox{}}%
	}{}
	
	% -------------------------------------------------------------------------
	% Se crean los header-footer
	% -------------------------------------------------------------------------
	\fancyheadoffset{0pt} % Desactiva el offset de los header-footer
	\def\hfheaderimageparamsA {height=\baselineskip} % Tamaño de las imágenes del encabezado estilo 3/13
	\ifthenelse{\equal{\hfstyle}{style1}}{
		\pagestyle{fancy}
		\newcommand{\COREstyledefinition}{
			\fancyhf{}
			\ifthenelse{\equal{\disablehfrightmark}{false}}{
				\fancyhead[L]{\nouppercase{\rightmark}}
			}{}
			\fancyhead[R]{\small \thepage}
			\ifthenelse{\equal{\hfwidthwrap}{true}}{
				\fancyfoot[L]{
					\begin{minipage}[t]{\hfwidthtitle\linewidth}
						\begin{flushleft}
							\small \textit{\documenttitlehf}
						\end{flushleft}
					\end{minipage}
				}
				\fancyfoot[R]{
					\begin{minipage}[t]{\hfwidthcourse\linewidth}
						\begin{flushright}
							\small \textit{\coursecode \coursename}
						\end{flushright}
					\end{minipage}
				}
			}{
				\fancyfoot[L]{\small \textit{\documenttitlehf}}
				\fancyfoot[R]{\small \textit{\coursecode \coursename}}
			}
			\renewcommand{\headrulewidth}{0.5pt}
			\renewcommand{\footrulewidth}{0.5pt}
		}
		\renewcommand{\sectionmark}[1]{\markboth{##1}{}}
		\COREstyledefinition
	}{
	\ifthenelse{\equal{\hfstyle}{style1-i}}{ % Impar izquierdo
		\pagestyle{fancy}
		\newcommand{\COREstyledefinition}{
			\fancyhf{}
			\ifthenelse{\equal{\disablehfrightmark}{false}}{
				\fancyhead[LE,RO]{\nouppercase{\rightmark}}
			}{}
			\fancyhead[RE,LO]{\small \thepage}
			\ifthenelse{\equal{\hfwidthwrap}{true}}{
				\fancyfoot[L]{
					\begin{minipage}[t]{\hfwidthtitle\linewidth}
						\begin{flushleft}
							\small \textit{\documenttitlehf}
						\end{flushleft}
					\end{minipage}
				}
				\fancyfoot[R]{
					\begin{minipage}[t]{\hfwidthcourse\linewidth}
						\begin{flushright}
							\small \textit{\coursecode \coursename}
						\end{flushright}
					\end{minipage}
				}
			}{
				\fancyfoot[L]{\small \textit{\documenttitlehf}}
				\fancyfoot[R]{\small \textit{\coursecode \coursename}}
			}
			\renewcommand{\headrulewidth}{0.5pt}
			\renewcommand{\footrulewidth}{0.5pt}
		}
		\renewcommand{\sectionmark}[1]{\markboth{##1}{}}
		\COREstyledefinition
	}{
	\ifthenelse{\equal{\hfstyle}{style1-d}}{ % Impar derecho
		\pagestyle{fancy}
		\newcommand{\COREstyledefinition}{
			\fancyhf{}
			\ifthenelse{\equal{\disablehfrightmark}{false}}{
				\fancyhead[LO,RE]{\nouppercase{\rightmark}}
			}{}
			\fancyhead[RO,LE]{\small \thepage}
			\ifthenelse{\equal{\hfwidthwrap}{true}}{
				\fancyfoot[L]{
					\begin{minipage}[t]{\hfwidthtitle\linewidth}
						\begin{flushleft}
							\small \textit{\documenttitlehf}
						\end{flushleft}
					\end{minipage}
				}
				\fancyfoot[R]{
					\begin{minipage}[t]{\hfwidthcourse\linewidth}
						\begin{flushright}
							\small \textit{\coursecode \coursename}
						\end{flushright}
					\end{minipage}
				}
			}{
				\fancyfoot[L]{\small \textit{\documenttitlehf}}
				\fancyfoot[R]{\small \textit{\coursecode \coursename}}
			}
			\renewcommand{\headrulewidth}{0.5pt}
			\renewcommand{\footrulewidth}{0.5pt}
		}
		\renewcommand{\sectionmark}[1]{\markboth{##1}{}}
		\COREstyledefinition
	}{
	\ifthenelse{\equal{\hfstyle}{style2}}{
		\pagestyle{fancy}
		\newcommand{\COREstyledefinition}{
			\fancyhf{}
			\ifthenelse{\equal{\disablehfrightmark}{false}}{
				\fancyhead[L]{\nouppercase{\rightmark}}
			}{}
			\fancyhead[R]{\small \thepage}
			\ifthenelse{\equal{\hfwidthwrap}{true}}{
				\fancyfoot[L]{
					\begin{minipage}[t]{\hfwidthtitle\linewidth}
						\begin{flushleft}
							\small \textit{\documenttitlehf}
						\end{flushleft}
					\end{minipage}
				}
				\fancyfoot[R]{
					\begin{minipage}[t]{\hfwidthcourse\linewidth}
						\begin{flushright}
							\small \textit{\coursecode \coursename}
						\end{flushright}
					\end{minipage}
				}
			}{
				\fancyfoot[L]{\small \textit{\documenttitlehf}}
				\fancyfoot[R]{\small \textit{\coursecode \coursename}}
			}
			\renewcommand{\headrulewidth}{0.5pt}
			\renewcommand{\footrulewidth}{0pt}
		}
		\renewcommand{\sectionmark}[1]{\markboth{##1}{}}
		\COREstyledefinition
	}{
	\ifthenelse{\equal{\hfstyle}{style2-i}}{ % Impar izquierdo
		\pagestyle{fancy}
		\newcommand{\COREstyledefinition}{
			\fancyhf{}
			\ifthenelse{\equal{\disablehfrightmark}{false}}{
				\fancyhead[LE,RO]{\nouppercase{\rightmark}}
			}{}
			\fancyhead[RE,LO]{\small \thepage}
			\ifthenelse{\equal{\hfwidthwrap}{true}}{
				\fancyfoot[L]{
					\begin{minipage}[t]{\hfwidthtitle\linewidth}
						\begin{flushleft}
							\small \textit{\documenttitlehf}
						\end{flushleft}
					\end{minipage}
				}
				\fancyfoot[R]{
					\begin{minipage}[t]{\hfwidthcourse\linewidth}
						\begin{flushright}
							\small \textit{\coursecode \coursename}
						\end{flushright}
					\end{minipage}
				}
			}{
				\fancyfoot[L]{\small \textit{\documenttitlehf}}
				\fancyfoot[R]{\small \textit{\coursecode \coursename}}
			}
			\renewcommand{\headrulewidth}{0.5pt}
			\renewcommand{\footrulewidth}{0pt}
		}
		\renewcommand{\sectionmark}[1]{\markboth{##1}{}}
		\COREstyledefinition
	}{
	\ifthenelse{\equal{\hfstyle}{style1-d}}{ % Impar derecho
		\pagestyle{fancy}
		\newcommand{\COREstyledefinition}{
			\fancyhf{}
			\ifthenelse{\equal{\disablehfrightmark}{false}}{
				\fancyhead[LO,RE]{\nouppercase{\rightmark}}
			}{}
			\fancyhead[RO,LE]{\small \thepage}
			\ifthenelse{\equal{\hfwidthwrap}{true}}{
				\fancyfoot[L]{
					\begin{minipage}[t]{\hfwidthtitle\linewidth}
						\begin{flushleft}
							\small \textit{\documenttitlehf}
						\end{flushleft}
					\end{minipage}
				}
				\fancyfoot[R]{
					\begin{minipage}[t]{\hfwidthcourse\linewidth}
						\begin{flushright}
							\small \textit{\coursecode \coursename}
						\end{flushright}
					\end{minipage}
				}
			}{
				\fancyfoot[L]{\small \textit{\documenttitlehf}}
				\fancyfoot[R]{\small \textit{\coursecode \coursename}}
			}
			\renewcommand{\headrulewidth}{0.5pt}
			\renewcommand{\footrulewidth}{0pt}
		}
		\renewcommand{\sectionmark}[1]{\markboth{##1}{}}
		\COREstyledefinition
	}{
	\ifthenelse{\equal{\hfstyle}{style3}}{
		\pagestyle{fancy}
		\newcommand{\COREstyledefinition}{
			\fancyhf{}
			\ifthenelse{\equal{\hfwidthwrap}{true}}{
				\fancyhead[L]{
					\begin{minipage}[t]{\hfwidthtitle\linewidth}
						\begin{flushleft}
							\small \textit{\coursecode \coursename}
						\end{flushleft}
					\end{minipage}
				}
			}{
				\fancyhead[L]{\small \textit{\coursecode \coursename}}
			}
			\fancyhead[R]{%
				\coreinsertkeyimage{\hfheaderimageparamsA}{\universitydepartmentimage}%
				\vspace{-0.15cm}%
			}
			\fancyfoot[C]{\thepage}
			\renewcommand{\headrulewidth}{0.5pt}
			\renewcommand{\footrulewidth}{0pt}
		}
		\COREstyledefinition
	}{
	\ifthenelse{\equal{\hfstyle}{style4}}{
		\pagestyle{fancy}
		\newcommand{\COREstyledefinition}{
			\fancyhf{}
			\ifthenelse{\equal{\disablehfrightmark}{false}}{
				\fancyhead[L]{\nouppercase{\rightmark}}
			}{}
			\fancyhead[R]{}
			\fancyfoot[C]{\small \thepage}
			\renewcommand{\headrulewidth}{0.5pt}
			\renewcommand{\footrulewidth}{0pt}
		}
		\renewcommand{\sectionmark}[1]{\markboth{##1}{}}
		\COREstyledefinition
	}{
	\ifthenelse{\equal{\hfstyle}{style5}}{
		\pagestyle{fancy}
		\newcommand{\COREstyledefinition}{
			\fancyhf{}
			\ifthenelse{\equal{\hfwidthwrap}{true}}{
				\fancyhead[L]{
					\begin{minipage}[t]{\hfwidthcourse\linewidth}
						\begin{flushleft}
							\coursecode \coursename
						\end{flushleft}
					\end{minipage}
				}
				\ifthenelse{\equal{\disablehfrightmark}{false}}{
					\fancyhead[R]{
						\begin{minipage}[t]{\hfwidthtitle\linewidth}
							\begin{flushright}
								\nouppercase{\rightmark}
							\end{flushright}
						\end{minipage}
					}
				}{}
			}{
				\fancyhead[L]{\coursecode \coursename}
				\ifthenelse{\equal{\disablehfrightmark}{false}}{
					\fancyhead[R]{\nouppercase{\rightmark}}
				}{}
			}
			\fancyfoot[L]{\universitydepartment, \universityname}
			\fancyfoot[R]{\small \thepage}
			\renewcommand{\headrulewidth}{0pt}
			\renewcommand{\footrulewidth}{0pt}
		}
		\renewcommand{\sectionmark}[1]{\markboth{##1}{}}
		\COREstyledefinition
	}{
	\ifthenelse{\equal{\hfstyle}{style5-d}}{ % Impar derecho
		\pagestyle{fancy}
		\newcommand{\COREstyledefinition}{
			\fancyhf{}
			\ifthenelse{\equal{\hfwidthwrap}{true}}{
				\fancyhead[L]{
					\begin{minipage}[t]{\hfwidthcourse\linewidth}
						\begin{flushleft}
							\coursecode \coursename
						\end{flushleft}
					\end{minipage}
				}
				\ifthenelse{\equal{\disablehfrightmark}{false}}{
					\fancyhead[R]{
						\begin{minipage}[t]{\hfwidthtitle\linewidth}
							\begin{flushright}
								\nouppercase{\rightmark}
							\end{flushright}
						\end{minipage}
					}
				}{}
			}{
				\fancyhead[L]{\coursecode \coursename}
				\ifthenelse{\equal{\disablehfrightmark}{false}}{
					\fancyhead[R]{\nouppercase{\rightmark}}
				}{}
			}
			\fancyfoot[LO,RE]{\universitydepartment, \universityname}
			\fancyfoot[RO,LE]{\small \thepage}
			\renewcommand{\headrulewidth}{0pt}
			\renewcommand{\footrulewidth}{0pt}
		}
		\renewcommand{\sectionmark}[1]{\markboth{##1}{}}
		\COREstyledefinition
	}{
	\ifthenelse{\equal{\hfstyle}{style5-i}}{ % Impar izquierdo
		\pagestyle{fancy}
		\newcommand{\COREstyledefinition}{
			\fancyhf{}
			\ifthenelse{\equal{\hfwidthwrap}{true}}{
				\fancyhead[L]{
					\begin{minipage}[t]{\hfwidthcourse\linewidth}
						\begin{flushleft}
							\coursecode \coursename
						\end{flushleft}
					\end{minipage}
				}
				\ifthenelse{\equal{\disablehfrightmark}{false}}{
					\fancyhead[R]{
						\begin{minipage}[t]{\hfwidthtitle\linewidth}
							\begin{flushright}
								\nouppercase{\rightmark}
							\end{flushright}
						\end{minipage}
					}
				}{}
			}{
				\fancyhead[L]{\coursecode \coursename}
				\ifthenelse{\equal{\disablehfrightmark}{false}}{
					\fancyhead[R]{\nouppercase{\rightmark}}
				}{}
			}
			\fancyfoot[LE,RO]{\universitydepartment, \universityname}
			\fancyfoot[RE,LO]{\small \thepage}
			\renewcommand{\headrulewidth}{0pt}
			\renewcommand{\footrulewidth}{0pt}
		}
		\renewcommand{\sectionmark}[1]{\markboth{##1}{}}
		\COREstyledefinition
	}{
	\ifthenelse{\equal{\hfstyle}{style6}}{
		\pagestyle{fancy}
		\newcommand{\COREstyledefinition}{
			\fancyhf{}
			\fancyfoot[L]{\universitydepartment}
			\fancyfoot[C]{\thepage}
			\fancyfoot[R]{\universityname}
			\renewcommand{\headrulewidth}{0pt}
			\renewcommand{\footrulewidth}{0pt}
		}
		\setlength{\headheight}{49pt}
		\COREstyledefinition
	}{
	\ifthenelse{\equal{\hfstyle}{style7}}{
		\pagestyle{fancy}
		\newcommand{\COREstyledefinition}{
			\fancyhf{}
			\fancyfoot[C]{\thepage}
			\renewcommand{\headrulewidth}{0pt}
			\renewcommand{\footrulewidth}{0pt}
		}
		\setlength{\headheight}{49pt}
		\COREstyledefinition
	}{
	\ifthenelse{\equal{\hfstyle}{style8}}{
		\pagestyle{fancy}
		\newcommand{\COREstyledefinition}{
			\fancyhf{}
			\fancyfoot[R]{\thepage}
			\renewcommand{\headrulewidth}{0pt}
			\renewcommand{\footrulewidth}{0pt}
		}
		\setlength{\headheight}{49pt}
		\COREstyledefinition
	}{
	\ifthenelse{\equal{\hfstyle}{style8-d}}{ % Impar derecho
		\pagestyle{fancy}
		\newcommand{\COREstyledefinition}{
			\fancyhf{}
			\fancyfoot[RO,LE]{\thepage}
			\renewcommand{\headrulewidth}{0pt}
			\renewcommand{\footrulewidth}{0pt}
		}
		\setlength{\headheight}{49pt}
		\COREstyledefinition
	}{
	\ifthenelse{\equal{\hfstyle}{style8-i}}{ % Impar izquierdo
		\pagestyle{fancy}
		\newcommand{\COREstyledefinition}{
			\fancyhf{}
			\fancyfoot[RE,LO]{\thepage}
			\renewcommand{\headrulewidth}{0pt}
			\renewcommand{\footrulewidth}{0pt}
		}
		\setlength{\headheight}{49pt}
		\COREstyledefinition
	}{
	\ifthenelse{\equal{\hfstyle}{style9}}{
		\pagestyle{fancy}
		\newcommand{\COREstyledefinition}{
			\fancyhf{}
			\ifthenelse{\equal{\disablehfrightmark}{false}}{
				\fancyhead[L]{\nouppercase{\rightmark}}
			}{}
			\fancyhead[R]{}
			\fancyfoot[L]{\small \textit{\documenttitlehf}}
			\fancyfoot[R]{\small \thepage}
			\renewcommand{\headrulewidth}{0.5pt}
			\renewcommand{\footrulewidth}{0.5pt}
		}
		\renewcommand{\sectionmark}[1]{\markboth{##1}{}}
		\COREstyledefinition
	}{
	\ifthenelse{\equal{\hfstyle}{style9-d}}{ % Impar derecho
		\pagestyle{fancy}
		\newcommand{\COREstyledefinition}{
			\fancyhf{}
			\ifthenelse{\equal{\disablehfrightmark}{false}}{
				\fancyhead[L]{\nouppercase{\rightmark}}
			}{}
			\fancyhead[R]{}
			\fancyfoot[RE,LO]{\small \textit{\documenttitlehf}}
			\fancyfoot[RO,LE]{\small \thepage}
			\renewcommand{\headrulewidth}{0.5pt}
			\renewcommand{\footrulewidth}{0.5pt}
		}
		\renewcommand{\sectionmark}[1]{\markboth{##1}{}}
		\COREstyledefinition
	}{
	\ifthenelse{\equal{\hfstyle}{style9-i}}{ % Impar izquierdo
		\pagestyle{fancy}
		\newcommand{\COREstyledefinition}{
			\fancyhf{}
			\ifthenelse{\equal{\disablehfrightmark}{false}}{
				\fancyhead[L]{\nouppercase{\rightmark}}
			}{}
			\fancyhead[R]{}
			\fancyfoot[RO,LE]{\small \textit{\documenttitlehf}}
			\fancyfoot[RE,LO]{\small \thepage}
			\renewcommand{\headrulewidth}{0.5pt}
			\renewcommand{\footrulewidth}{0.5pt}
		}
		\renewcommand{\sectionmark}[1]{\markboth{##1}{}}
		\COREstyledefinition
	}{
	\ifthenelse{\equal{\hfstyle}{style10}}{
		\pagestyle{fancy}
		\newcommand{\COREstyledefinition}{
			\fancyhf{}
			\ifthenelse{\equal{\hfwidthwrap}{true}}{
				\ifthenelse{\equal{\disablehfrightmark}{false}}{
					\fancyhead[L]{
						\begin{minipage}[t]{\hfwidthtitle\linewidth}
							\begin{flushleft}
								\nouppercase{\rightmark}
							\end{flushleft}
						\end{minipage}
					}
				}{}
				\fancyhead[R]{
					\begin{minipage}[t]{\hfwidthcourse\linewidth}
						\begin{flushright}
							\small \textit{\documenttitlehf}
						\end{flushright}
					\end{minipage}
				}
			}{
				\ifthenelse{\equal{\disablehfrightmark}{false}}{
					\fancyhead[L]{\nouppercase{\rightmark}}
				}{}
				\fancyhead[R]{\small \textit{\documenttitlehf}}
			}
			\fancyfoot[L]{}
			\fancyfoot[R]{\small \thepage}
			\renewcommand{\headrulewidth}{0.5pt}
			\renewcommand{\footrulewidth}{0.5pt}
		}
		\renewcommand{\sectionmark}[1]{\markboth{##1}{}}
		\COREstyledefinition
	}{
	\ifthenelse{\equal{\hfstyle}{style10-i}}{ % Impar izquierdo
		\pagestyle{fancy}
		\newcommand{\COREstyledefinition}{
			\fancyhf{}
			\ifthenelse{\equal{\hfwidthwrap}{true}}{
				\ifthenelse{\equal{\disablehfrightmark}{false}}{
					\fancyhead[L]{
						\begin{minipage}[t]{\hfwidthtitle\linewidth}
							\begin{flushleft}
								\nouppercase{\rightmark}
							\end{flushleft}
						\end{minipage}
					}
				}{}
				\fancyhead[R]{
					\begin{minipage}[t]{\hfwidthcourse\linewidth}
						\begin{flushright}
							\small \textit{\documenttitlehf}
						\end{flushright}
					\end{minipage}
				}
			}{
				\ifthenelse{\equal{\disablehfrightmark}{false}}{
					\fancyhead[L]{\nouppercase{\rightmark}}
				}{}
				\fancyhead[R]{\small \textit{\documenttitlehf}}
			}
			\fancyfoot[L]{}
			\fancyfoot[RE,LO]{\small \thepage}
			\renewcommand{\headrulewidth}{0.5pt}
			\renewcommand{\footrulewidth}{0.5pt}
		}
		\renewcommand{\sectionmark}[1]{\markboth{##1}{}}
		\COREstyledefinition
	}{
	\ifthenelse{\equal{\hfstyle}{style10-d}}{ % Impar derecho
		\pagestyle{fancy}
		\newcommand{\COREstyledefinition}{
			\fancyhf{}
			\ifthenelse{\equal{\hfwidthwrap}{true}}{
				\ifthenelse{\equal{\disablehfrightmark}{false}}{
					\fancyhead[L]{
						\begin{minipage}[t]{\hfwidthtitle\linewidth}
							\begin{flushleft}
								\nouppercase{\rightmark}
							\end{flushleft}
						\end{minipage}
					}
				}{}
				\fancyhead[R]{
					\begin{minipage}[t]{\hfwidthcourse\linewidth}
						\begin{flushright}
							\small \textit{\documenttitlehf}
						\end{flushright}
					\end{minipage}
				}
			}{
				\ifthenelse{\equal{\disablehfrightmark}{false}}{
					\fancyhead[L]{\nouppercase{\rightmark}}
				}{}
				\fancyhead[R]{\small \textit{\documenttitlehf}}
			}
			\fancyfoot[L]{}
			\fancyfoot[LE,RO]{\small \thepage}
			\renewcommand{\headrulewidth}{0.5pt}
			\renewcommand{\footrulewidth}{0.5pt}
		}
		\renewcommand{\sectionmark}[1]{\markboth{##1}{}}
		\COREstyledefinition
	}{
	\ifthenelse{\equal{\hfstyle}{style11}}{ % Similar a 1
		\pagestyle{fancy}
		\newcommand{\COREstyledefinition}{
			\fancyhf{}
			\ifthenelse{\equal{\disablehfrightmark}{false}}{
				\fancyhead[L]{\nouppercase{\rightmark}}
			}{}
			\fancyhead[R]{\small \thepage \namepageof \pageref{TotPages}}
			\ifthenelse{\equal{\hfwidthwrap}{true}}{
				\fancyfoot[L]{
					\begin{minipage}[t]{\hfwidthtitle\linewidth}
						\begin{flushleft}
							\small \textit{\documenttitlehf}
						\end{flushleft}
					\end{minipage}
				}
				\fancyfoot[R]{
					\begin{minipage}[t]{\hfwidthcourse\linewidth}
						\begin{flushright}
							\small \textit{\coursecode \coursename}
						\end{flushright}
					\end{minipage}
				}
			}{
				\fancyfoot[L]{\small \textit{\documenttitlehf}}
				\fancyfoot[R]{\small \textit{\coursecode \coursename}}
			}
			\renewcommand{\headrulewidth}{0.5pt}
			\renewcommand{\footrulewidth}{0.5pt}
		}
		\renewcommand{\sectionmark}[1]{\markboth{##1}{}}
		\COREstyledefinition
	}{
	\ifthenelse{\equal{\hfstyle}{style12}}{ % Similar a 6
		\pagestyle{fancy}
		\newcommand{\COREstyledefinition}{
			\fancyhf{}
			\fancyfoot[L]{\universitydepartment}
			\fancyfoot[C]{\thepage \namepageof \pageref{TotPages}}
			\fancyfoot[R]{\universityname}
			\renewcommand{\headrulewidth}{0pt}
			\renewcommand{\footrulewidth}{0pt}
		}
		\setlength{\headheight}{49pt}
		\COREstyledefinition
	}{
	\ifthenelse{\equal{\hfstyle}{style13}}{ % Similar a 3
		\pagestyle{fancy}
		\newcommand{\COREstyledefinition}{
			\fancyhf{}
			\ifthenelse{\equal{\hfwidthwrap}{true}}{
				\fancyhead[L]{
					\begin{minipage}[t]{\hfwidthtitle\linewidth}
						\begin{flushleft}
							\small \textit{\coursecode \coursename}
						\end{flushleft}
					\end{minipage}
				}
			}{
				\fancyhead[L]{\small \textit{\coursecode \coursename}}
			}
			\fancyhead[R]{%
				\coreinsertkeyimage{\hfheaderimageparamsA}{\universitydepartmentimage}%
				\vspace{-0.15cm}%
			}
			\fancyfoot[C]{\thepage \namepageof \pageref{TotPages}}
			\renewcommand{\headrulewidth}{0.5pt}
			\renewcommand{\footrulewidth}{0pt}
		}
		\COREstyledefinition
	}{
	\ifthenelse{\equal{\hfstyle}{style14}}{ % Similar a 4
		\pagestyle{fancy}
		\newcommand{\COREstyledefinition}{
			\fancyhf{}
			\ifthenelse{\equal{\disablehfrightmark}{false}}{
				\fancyhead[L]{\nouppercase{\rightmark}}
			}{}
			\fancyhead[R]{}
			\fancyfoot[C]{\small \thepage \namepageof \pageref{TotPages}}
			\renewcommand{\headrulewidth}{0.5pt}
			\renewcommand{\footrulewidth}{0pt}
		}
		\renewcommand{\sectionmark}[1]{\markboth{##1}{}}
		\COREstyledefinition
	}{
	\ifthenelse{\equal{\hfstyle}{style15}}{ % Similar a 1
		\pagestyle{fancy}
		\newcommand{\COREstyledefinition}{
			\fancyhf{}
			\ifthenelse{\equal{\disablehfrightmark}{false}}{
				\fancyhead[L]{\nouppercase{\rightmark}}
			}{}
			\fancyhead[R]{}
			\fancyfoot[L]{
				\small \coursecode \coursename
			}
			\fancyfoot[R]{
				\small \thepage
			}
			\renewcommand{\headrulewidth}{0.5pt}
			\renewcommand{\footrulewidth}{0.5pt}
		}
		\renewcommand{\sectionmark}[1]{\markboth{##1}{}}
		\COREstyledefinition
	}{
	\ifthenelse{\equal{\hfstyle}{style16}}{
		\pagestyle{fancy}
		\newcommand{\COREstyledefinition}{
			\fancyhf{}
			\renewcommand{\headrulewidth}{0pt}
			\renewcommand{\footrulewidth}{0pt}
		}
		\renewcommand{\sectionmark}[1]{\markboth{##1}{}}
		\COREstyledefinition
	}{
		\throwbadconfigondoc{Estilo de header-footer incorrecto}{\hfstyle}{style1 .. style16}}}}}}}}}}}}}}}}}}}}}}}}}}}}
	}
	% Aplica el estilo de página
	\fancypagestyle{plain}{
		\fancyheadoffset{0pt}
		\COREstyledefinition
	}
	% Define estilos por defecto en flotantes
	\floatpagestyle{plain}
	\rotfloatpagestyle{plain}
	
	% -------------------------------------------------------------------------
	% Muestra los números de línea
	% -------------------------------------------------------------------------
	\ifthenelse{\equal{\showlinenumbers}{true}}{
		\linenumbers}{
	}
	% Añade página en blanco
	\ifthenelse{\equal{\GLOBALtwoside}{true}}{%
		\insertemptypage%
		\addtocounter{page}{-1}}{
	}
}

% -----------------------------------------------------------------------------
% TABLA DE CONTENIDOS - ÍNDICE
% -----------------------------------------------------------------------------
\newcommand{\templateIndex}{

	% -------------------------------------------------------------------------
	% Inicio índice, desactiva espacio entre objetos
	% -------------------------------------------------------------------------
	\ifthenelse{\equal{\objectchaptermargin}{false}}{
		\let\origaddvspace\addvspace
		\renewcommand{\addvspace}[1]{}
	}{}
	
	% -------------------------------------------------------------------------
	% Crea nueva página y establece estilo de títulos
	% -------------------------------------------------------------------------
	\clearpage%
	\begingroup%
	\sectionfont{\color{\indextitlecolor} \indexsectionfontsize \indexsectionstyle \selectfont}
	
	% -------------------------------------------------------------------------
	% Salta de página si está imprimiendo por ambas caras
	% -------------------------------------------------------------------------
	\ifthenelse{\equal{\GLOBALtwoside}{true}}{%
		\coretriggeronpage{\emptypagespredocformat}{}%
	}{}
	
	% -------------------------------------------------------------------------
	% Añade la entrada del índice a los marcadores del pdf
	% -------------------------------------------------------------------------
	\ifthenelse{\equal{\addindextobookmarks}{true}}{
		\phantomsection
		\pdfbookmark{\nameltcont}{contents}}{
	}
	\tocloftpagestyle{fancy}
	
	% -------------------------------------------------------------------------
	% Configuración del punto en índice
	% -------------------------------------------------------------------------
	\def\cftchapaftersnum {\charaftersectionnum}
	\def\cftsecaftersnum {\charaftersectionnum}
	\def\cftsubsecaftersnum {\charaftersectionnum}
	\def\cftsubsubsecaftersnum {\charaftersectionnum}
	\def\cftsubsubsubsecaftersnum {\charaftersectionnum}
	
	% -------------------------------------------------------------------------
	% Configuración carácter número de página
	% -------------------------------------------------------------------------
	\renewcommand{\cftdot}{\charnumpageindex}
	
	% -------------------------------------------------------------------------
	% Configuración del punto en número de objetos
	% -------------------------------------------------------------------------
	\def\cftfigaftersnum {\charafterobjectindex\enspace} % Figuras
	\def\cftsubfigaftersnum {\charafterobjectindex\enspace} % Subfiguras
	\def\cfttabaftersnum {\charafterobjectindex\enspace} % Tablas
	\def\cftlstlistingaftersnum {\charafterobjectindex\enspace} % Códigos fuente
	\def\cftmyindexequationsaftersnum {\charafterobjectindex\enspace} % Ecuaciones
	
	% -------------------------------------------------------------------------
	% Desactiva los números de línea
	% -------------------------------------------------------------------------
	\ifthenelse{\equal{\showlinenumbers}{true}}{
		\nolinenumbers}{
	}
	
	% -------------------------------------------------------------------------
	% Cambia tabulación índice de objetos para alinear con contenidos
	% -------------------------------------------------------------------------
	\ifthenelse{\equal{\objectindexindent}{true}}{
		\setlength{\cfttabindent}{1.5em}
		\setlength{\cftfigindent}{1.5em}
		\def\cftlstlistingindent {1.5em}
	}{
		\setlength{\cfttabindent}{0em}
		\setlength{\cftfigindent}{0em}
		\def\cftlstlistingindent {0em}
	}
	
	% -------------------------------------------------------------------------
	% Calcula tamaño del margen de los números en objetos del índice
	% -------------------------------------------------------------------------
	% Código fuente
	\ifthenelse{\equal{\showsectioncaptioncode}{none}}{
		\def\cftdefautnumwidthcode {3.0em} % Añade +0.7em
		\def\cftdefaultnumwidthromancode {5.25em} % Añade +0.5em para no overflow
	}{
	\ifthenelse{\equal{\showsectioncaptioncode}{sec}}{
		\def\cftdefautnumwidthcode {3.7em}
		\def\cftdefaultnumwidthromancode {5.75em}
	}{
	\ifthenelse{\equal{\showsectioncaptioncode}{ssec}}{
		\def\cftdefautnumwidthcode {4.4em}
		\def\cftdefaultnumwidthromancode {6.25em}
	}{
	\ifthenelse{\equal{\showsectioncaptioncode}{sssec}}{
		\def\cftdefautnumwidthcode {5.1em}
		\def\cftdefaultnumwidthromancode {6.75em}
	}{
	\ifthenelse{\equal{\showsectioncaptioncode}{ssssec}}{
		\def\cftdefautnumwidthcode {5.8em}
		\def\cftdefaultnumwidthromancode {7.25em}
	}{
	\ifthenelse{\equal{\showsectioncaptioncode}{chap}}{
		\def\cftdefautnumwidthcode {3.0em}
		\def\cftdefaultnumwidthromancode {5.25em}
	}{
		\throwbadconfig{Valor configuracion incorrecto}{\showsectioncaptioncode}{none,chap,sec,ssec,sssec,ssssec}}}}}}
	}
	
	% Código fuente
	\ifthenelse{\equal{\showsectioncaptioneqn}{none}}{
		\def\cftdefautnumwidtheqn {3.0em} % Añade +0.7em
		\def\cftdefaultnumwidthromaneqn {5.25em} % Añade +0.5em para no overflow
	}{
	\ifthenelse{\equal{\showsectioncaptioneqn}{sec}}{
		\def\cftdefautnumwidtheqn {3.7em}
		\def\cftdefaultnumwidthromaneqn {5.75em}
	}{
	\ifthenelse{\equal{\showsectioncaptioneqn}{ssec}}{
		\def\cftdefautnumwidtheqn {4.4em}
		\def\cftdefaultnumwidthromaneqn {6.25em}
	}{
	\ifthenelse{\equal{\showsectioncaptioneqn}{sssec}}{
		\def\cftdefautnumwidtheqn {5.1em}
		\def\cftdefaultnumwidthromaneqn {6.75em}
	}{
	\ifthenelse{\equal{\showsectioncaptioneqn}{ssssec}}{
		\def\cftdefautnumwidtheqn {5.8em}
		\def\cftdefaultnumwidthromaneqn {7.25em}
	}{
	\ifthenelse{\equal{\showsectioncaptioneqn}{chap}}{
		\def\cftdefautnumwidtheqn {3.0em}
		\def\cftdefaultnumwidthromaneqn {5.25em}
	}{
		\throwbadconfig{Valor configuracion incorrecto}{\showsectioncaptioneqn}{none,chap,sec,ssec,sssec,ssssec}}}}}}
	}
	
	% Figuras
	\ifthenelse{\equal{\showsectioncaptionfig}{none}}{
		\def\cftdefautnumwidthfig {3.0em} % Añade +0.7em
		\def\cftdefaultnumwidthromanfig {5.25em} % Añade +0.5em
	}{
	\ifthenelse{\equal{\showsectioncaptionfig}{sec}}{
		\def\cftdefautnumwidthfig {3.7em}
		\def\cftdefaultnumwidthromanfig {5.75em}
	}{
	\ifthenelse{\equal{\showsectioncaptionfig}{ssec}}{
		\def\cftdefautnumwidthfig {4.4em}
		\def\cftdefaultnumwidthromanfig {6.25em}
	}{
	\ifthenelse{\equal{\showsectioncaptionfig}{sssec}}{
		\def\cftdefautnumwidthfig {5.1em}
		\def\cftdefaultnumwidthromanfig {6.75em}
	}{
	\ifthenelse{\equal{\showsectioncaptionfig}{ssssec}}{
		\def\cftdefautnumwidthfig {5.8em}
		\def\cftdefaultnumwidthromanfig {7.25em}
	}{
	\ifthenelse{\equal{\showsectioncaptionfig}{chap}}{
		\def\cftdefautnumwidthfig {3.0em}
		\def\cftdefaultnumwidthromanfig {5.25em}
	}{
		\throwbadconfig{Valor configuracion incorrecto}{\showsectioncaptionfig}{none,chap,sec,ssec,sssec,ssssec}}}}}}
	}
	
	% Tablas
	\ifthenelse{\equal{\showsectioncaptiontab}{none}}{
		\def\cftdefautnumwidthtab {3.0em} % Añade +0.7em
		\def\cftdefaultnumwidthromantab {5.25em} % Añade +0.5em
	}{
	\ifthenelse{\equal{\showsectioncaptiontab}{sec}}{
		\def\cftdefautnumwidthtab {3.7em}
		\def\cftdefaultnumwidthromantab {5.75em}
	}{
	\ifthenelse{\equal{\showsectioncaptiontab}{ssec}}{
		\def\cftdefautnumwidthtab {4.4em}
		\def\cftdefaultnumwidthromantab {6.25em}
	}{
	\ifthenelse{\equal{\showsectioncaptiontab}{sssec}}{
		\def\cftdefautnumwidthtab {5.1em}
		\def\cftdefaultnumwidthromantab {6.75em}
	}{
	\ifthenelse{\equal{\showsectioncaptiontab}{ssssec}}{
		\def\cftdefautnumwidthtab {5.8em}
		\def\cftdefaultnumwidthromantab {7.25em}
	}{
	\ifthenelse{\equal{\showsectioncaptiontab}{chap}}{
		\def\cftdefautnumwidthtab {3.0em}
		\def\cftdefaultnumwidthromantab {5.25em}
	}{
		\throwbadconfig{Valor configuracion incorrecto}{\showsectioncaptiontab}{none,chap,sec,ssec,sssec,ssssec}}}}}}
	}
	
	% Configuración identado de títulos de objetos después del número
	\def\cftfignumwidth {\cftdefautnumwidth}
	% \def\cftsubfignumwidth {\cftdefautnumwidth}
	\def\cfttabnumwidth {\cftdefautnumwidth}
	\def\cftlstlistingnumwidth {\cftdefautnumwidth}
	
	% Código fuente
	\ifthenelse{\equal{\captionnumcode}{arabic}}{ % No hace nada (default)
		\def\cftlstlistingnumwidth {\cftdefautnumwidthcode}
	}{
		\ifthenelse{\equal{\captionnumcode}{roman}}{
			\def\cftlstlistingnumwidth {\cftdefaultnumwidthromancode}
		}{
		\ifthenelse{\equal{\captionnumcode}{Roman}}{
			\def\cftlstlistingnumwidth {\cftdefaultnumwidthromancode}
		}{
			\def\cftlstlistingnumwidth {\cftdefautnumwidthcode}
		}}
	}
	
	% Ecuaciones
	\ifthenelse{\equal{\captionnumequation}{arabic}}{ % No hace nada (default)
		\def\cftmyindexequationsnumwidth {\cftdefautnumwidtheqn}
	}{
		\ifthenelse{\equal{\captionnumequation}{roman}}{
			\def\cftmyindexequationsnumwidth {\cftdefaultnumwidthromaneqn}
		}{
		\ifthenelse{\equal{\captionnumequation}{Roman}}{
			\def\cftmyindexequationsnumwidth {\cftdefaultnumwidthromaneqn}
		}{
			\def\cftmyindexequationsnumwidth {\cftdefautnumwidtheqn}
		}}
	}
	
	% Figuras
	\ifthenelse{\equal{\captionnumfigure}{arabic}}{ % No hace nada (default)
		\def\cftfignumwidth {\cftdefautnumwidthfig}
	}{
		\ifthenelse{\equal{\captionnumfigure}{roman}}{
			\def\cftfignumwidth {\cftdefaultnumwidthromanfig}
		}{
			\ifthenelse{\equal{\captionnumfigure}{Roman}}{
				\def\cftfignumwidth {\cftdefaultnumwidthromanfig}
			}{
				\def\cftfignumwidth {\cftdefautnumwidthfig}
			}}
	}
	
	% Tablas
	\ifthenelse{\equal{\captionnumtable}{arabic}}{ % No hace nada (default)
		\def\cfttabnumwidth {\cftdefautnumwidthtab}
	}{
		\ifthenelse{\equal{\captionnumtable}{roman}}{
			\def\cfttabnumwidth {\cftdefaultnumwidthromantab}
		}{
			\ifthenelse{\equal{\captionnumtable}{Roman}}{
				\def\cfttabnumwidth {\cftdefaultnumwidthromantab}
			}{
				\def\cfttabnumwidth {\cftdefautnumwidthtab}
			}}
	}
	
	% -------------------------------------------------------------------------
	% Genera las funciones para los índices
	% -------------------------------------------------------------------------
	\newcommand{\LoIf}{ % Lista de figuras
		\iftotalfigures
			\ifthenelse{\equal{\indexnewpagef}{true}}{\clearpage}{}
			\ifthenelse{\equal{\addindextobookmarks}{true}}{
				\ifthenelse{\equal{\addindexsubtobookmarks}{true}}{
					\phantomsection
					\belowpdfbookmark{\nameltfigure}{clof}}{}}{
			}
			\listoffigures
		\fi
	}
	\newcommand{\LoIt}{ % Tablas
		\iftotaltables
			\ifthenelse{\equal{\indexnewpaget}{true}}{\clearpage}{}
			\ifthenelse{\equal{\addindextobookmarks}{true}}{
				\ifthenelse{\equal{\addindexsubtobookmarks}{true}}{
					\phantomsection
					\belowpdfbookmark{\namelttable}{clot}}{}}{
			}
			\listoftables
		\fi
	}
	\newcommand{\LoIc}{ % Códigos fuente (listings)
		\iftotallstlistings
			\ifthenelse{\equal{\indexnewpagec}{true}}{\clearpage}{}
			\ifthenelse{\equal{\addindextobookmarks}{true}}{
				\ifthenelse{\equal{\addindexsubtobookmarks}{true}}{
					\phantomsection
					\belowpdfbookmark{\nameltsrc}{clsrc}}{}}{
			}
			\lstlistoflistings
		\fi
	}
	\newcommand{\LoIe}{ % Ecuaciones
		\iftotaltemplateIndexEquationss
			\ifthenelse{\equal{\indexnewpagee}{true}}{\clearpage}{}
			\ifthenelse{\equal{\addindextobookmarks}{true}}{
				\ifthenelse{\equal{\addindexsubtobookmarks}{true}}{
					\phantomsection
					\belowpdfbookmark{\namelteqn}{cleqn}}{}}{
			}
			\listofmyindexequations
		\fi
	}
	
	% -------------------------------------------------------------------------
	% Índice de contenidos
	% -------------------------------------------------------------------------
	\ifthenelse{\equal{\showindexofcontents}{true}}{
		\tableofcontents
	}{}
	
	% -------------------------------------------------------------------------
	% Índice de objetos
	% -------------------------------------------------------------------------
	\ifthenelse{\equal{\indexstyle}{ftc}}{%
		\LoIf\LoIt\LoIc
	}{
	\ifthenelse{\equal{\indexstyle}{}}{%
	}{
	\ifthenelse{\equal{\indexstyle}{e}}{%
		\LoIe
	}{
	\ifthenelse{\equal{\indexstyle}{c}}{%
		\LoIc
	}{
	\ifthenelse{\equal{\indexstyle}{f}}{%
		\LoIf
	}{
	\ifthenelse{\equal{\indexstyle}{t}}{%
		\LoIt
	}{
	\ifthenelse{\equal{\indexstyle}{ec}}{%
		\LoIe\LoIc
	}{
	\ifthenelse{\equal{\indexstyle}{ce}}{%
		\LoIc\LoIe
	}{
	\ifthenelse{\equal{\indexstyle}{ef}}{%
		\LoIe\LoIf
	}{
	\ifthenelse{\equal{\indexstyle}{fe}}{%
		\LoIf\LoIe
	}{
	\ifthenelse{\equal{\indexstyle}{et}}{%
		\LoIe\LoIt
	}{
	\ifthenelse{\equal{\indexstyle}{te}}{%
		\LoIt\LoIe
	}{
	\ifthenelse{\equal{\indexstyle}{cf}}{%
		\LoIc\LoIf
	}{
	\ifthenelse{\equal{\indexstyle}{fc}}{%
		\LoIf\LoIc
	}{
	\ifthenelse{\equal{\indexstyle}{ct}}{%
		\LoIc\LoIt
	}{
	\ifthenelse{\equal{\indexstyle}{tc}}{%
		\LoIt\LoIc
	}{
	\ifthenelse{\equal{\indexstyle}{ft}}{%
		\LoIf\LoIt
	}{
	\ifthenelse{\equal{\indexstyle}{tf}}{%
		\LoIt\LoIf
	}{
	\ifthenelse{\equal{\indexstyle}{ecf}}{%
		\LoIe\LoIc\LoIf
	}{
	\ifthenelse{\equal{\indexstyle}{efc}}{%
		\LoIe\LoIf\LoIc
	}{
	\ifthenelse{\equal{\indexstyle}{cef}}{%
		\LoIc\LoIe\LoIf
	}{
	\ifthenelse{\equal{\indexstyle}{cfe}}{%
		\LoIc\LoIf\LoIe
	}{
	\ifthenelse{\equal{\indexstyle}{fec}}{%
		\LoIf\LoIe\LoIc
	}{
	\ifthenelse{\equal{\indexstyle}{fce}}{%
		\LoIf\LoIc\LoIe
	}{
	\ifthenelse{\equal{\indexstyle}{ect}}{%
		\LoIe\LoIc\LoIt
	}{
	\ifthenelse{\equal{\indexstyle}{etc}}{%
		\LoIe\LoIt\LoIc
	}{
	\ifthenelse{\equal{\indexstyle}{cet}}{%
		\LoIc\LoIe\LoIt
	}{
	\ifthenelse{\equal{\indexstyle}{cte}}{%
		\LoIc\LoIt\LoIe
	}{
	\ifthenelse{\equal{\indexstyle}{tec}}{%
		\LoIt\LoIe\LoIc
	}{
	\ifthenelse{\equal{\indexstyle}{tce}}{%
		\LoIt\LoIc\LoIe
	}{
	\ifthenelse{\equal{\indexstyle}{eft}}{%
		\LoIe\LoIf\LoIt
	}{
	\ifthenelse{\equal{\indexstyle}{etf}}{%
		\LoIe\LoIt\LoIf
	}{
	\ifthenelse{\equal{\indexstyle}{fet}}{%
		\LoIf\LoIe\LoIt
	}{
	\ifthenelse{\equal{\indexstyle}{fte}}{%
		\LoIf\LoIt\LoIe
	}{
	\ifthenelse{\equal{\indexstyle}{tef}}{%
		\LoIt\LoIe\LoIf
	}{
	\ifthenelse{\equal{\indexstyle}{tfe}}{%
		\LoIt\LoIf\LoIe
	}{
	\ifthenelse{\equal{\indexstyle}{cft}}{%
		\LoIc\LoIf\LoIt
	}{
	\ifthenelse{\equal{\indexstyle}{ctf}}{%
		\LoIc\LoIt\LoIf
	}{
	\ifthenelse{\equal{\indexstyle}{fct}}{%
		\LoIf\LoIc\LoIt
	}{
	\ifthenelse{\equal{\indexstyle}{tcf}}{%
		\LoIt\LoIc\LoIf
	}{
	\ifthenelse{\equal{\indexstyle}{tfc}}{%
		\LoIt\LoIf\LoIc
	}{
	\ifthenelse{\equal{\indexstyle}{ecft}}{%
		\LoIe\LoIc\LoIf\LoIt
	}{
	\ifthenelse{\equal{\indexstyle}{ectf}}{%
		\LoIe\LoIc\LoIt\LoIf
	}{
	\ifthenelse{\equal{\indexstyle}{efct}}{%
		\LoIe\LoIf\LoIc\LoIt
	}{
	\ifthenelse{\equal{\indexstyle}{eftc}}{%
		\LoIe\LoIf\LoIt\LoIc
	}{
	\ifthenelse{\equal{\indexstyle}{etcf}}{%
		\LoIe\LoIt\LoIc\LoIf
	}{
	\ifthenelse{\equal{\indexstyle}{etfc}}{%
		\LoIe\LoIt\LoIf\LoIc
	}{
	\ifthenelse{\equal{\indexstyle}{ceft}}{%
		\LoIc\LoIe\LoIf\LoIt
	}{
	\ifthenelse{\equal{\indexstyle}{cetf}}{%
		\LoIc\LoIe\LoIt\LoIf
	}{
	\ifthenelse{\equal{\indexstyle}{cfet}}{%
		\LoIc\LoIf\LoIe\LoIt
	}{
	\ifthenelse{\equal{\indexstyle}{cfte}}{%
		\LoIc\LoIf\LoIt\LoIe
	}{
	\ifthenelse{\equal{\indexstyle}{ctef}}{%
		\LoIc\LoIt\LoIe\LoIf
	}{
	\ifthenelse{\equal{\indexstyle}{ctfe}}{%
		\LoIc\LoIt\LoIf\LoIe
	}{
	\ifthenelse{\equal{\indexstyle}{fect}}{%
		\LoIf\LoIe\LoIc\LoIt
	}{
	\ifthenelse{\equal{\indexstyle}{fetc}}{%
		\LoIf\LoIe\LoIt\LoIc
	}{
	\ifthenelse{\equal{\indexstyle}{fcet}}{%
		\LoIf\LoIc\LoIe\LoIt
	}{
	\ifthenelse{\equal{\indexstyle}{fcte}}{%
		\LoIf\LoIc\LoIt\LoIe
	}{
	\ifthenelse{\equal{\indexstyle}{ftec}}{%
		\LoIf\LoIt\LoIe\LoIc
	}{
	\ifthenelse{\equal{\indexstyle}{ftce}}{%
		\LoIf\LoIt\LoIc\LoIe
	}{
	\ifthenelse{\equal{\indexstyle}{tecf}}{%
		\LoIt\LoIe\LoIc\LoIf
	}{
	\ifthenelse{\equal{\indexstyle}{tefc}}{%
		\LoIt\LoIe\LoIf\LoIc
	}{
	\ifthenelse{\equal{\indexstyle}{tcef}}{%
		\LoIt\LoIc\LoIe\LoIf
	}{
	\ifthenelse{\equal{\indexstyle}{tcfe}}{%
		\LoIt\LoIc\LoIf\LoIe
	}{
	\ifthenelse{\equal{\indexstyle}{tfec}}{%
		\LoIt\LoIf\LoIe\LoIc
	}{
	\ifthenelse{\equal{\indexstyle}{tfce}}{%
		\LoIt\LoIf\LoIc\LoIe
	}{
		\throwbadconfig{Estilo desconocido del indice}{\indexstyle}{ftc,,e,c,f,t,ec,ce,ef,fe,et,te,cf,fc,ct,tc,ft,tf,ecf,efc,cef,cfe,fec,fce,ect,etc,cet,cte,tec,tce,eft,etf,fet,fte,tef,tfe,cft,ctf,fct,tcf,tfc,ecft,ectf,efct,eftc,etcf,etfc,ceft,cetf,cfet,cfte,ctef,ctfe,fect,fetc,fcet,fcte,ftec,ftce,tecf,tefc,tcef,tcfe,tfec,tfce}}}}}}}}}}}}}}}}}}}}}}}}}}}}}}}}}}}}}}}}}}}}}}}}}}}}}}}}}}}}}}}}}
	}
	
	% -------------------------------------------------------------------------
	% Termina el bloque de índice
	% -------------------------------------------------------------------------
	\sectionfont{\color{\sectioncolor} \sectionfontsize \sectionfontstyle \selectfont}
	\endgroup
	
	% -------------------------------------------------------------------------
	% Final del índice, restablece el espacio
	% -------------------------------------------------------------------------
	\ifthenelse{\equal{\objectchaptermargin}{false}}{
		\renewcommand{\addvspace}[1]{\origaddvspace{##1}}
	}{}
	
}

% -----------------------------------------------------------------------------
% CONFIGURACIONES FINALES
% -----------------------------------------------------------------------------
\newcommand{\templateFinalcfg}{
	
	% -------------------------------------------------------------------------
	% Se restablecen headers y footers
	% -------------------------------------------------------------------------
	\markboth{}{}
	\clearpage
	% Actualiza headers
	\ifthenelse{\equal{\disablehfrightmark}{false}}{
		% Define funciones generales
		\def\COREhfstyledefA { % 1, 2, 4, 9, 11, 14, 15
			\fancypagestyle{plain}{\fancyhead[L]{\nouppercase{\leftmark}}}
			\fancyhead[L]{\nouppercase{\leftmark}}
		}
		\def\COREhfstyledefB { % 5
			\fancypagestyle{plain}{
				\ifthenelse{\equal{\hfwidthwrap}{true}}{
					\fancyhead[R]{
						\begin{minipage}[t]{\hfwidthtitle\linewidth}
							\begin{flushright}
								\nouppercase{\leftmark}
							\end{flushright}
						\end{minipage}
					}
				}{
					\fancyhead[R]{\nouppercase{\leftmark}}
				}
			}
			\ifthenelse{\equal{\hfwidthwrap}{true}}{
				\fancyhead[R]{
					\begin{minipage}[t]{\hfwidthtitle\linewidth}
						\begin{flushright}
							\nouppercase{\leftmark}
						\end{flushright}
					\end{minipage}
				}
			}{
				\fancyhead[R]{\nouppercase{\leftmark}}
			}
		}
		\def\COREhfstyledefC { % 10
			\fancypagestyle{plain}{
				\ifthenelse{\equal{\hfwidthwrap}{true}}{
					\fancyhead[L]{
						\begin{minipage}[t]{\hfwidthtitle\linewidth}
							\begin{flushleft}
								\nouppercase{\leftmark}
							\end{flushleft}
						\end{minipage}
					}
				}{
					\fancyhead[L]{\nouppercase{\leftmark}}
				}
			}
			\ifthenelse{\equal{\hfwidthwrap}{true}}{
				\fancyhead[L]{
					\begin{minipage}[t]{\hfwidthtitle\linewidth}
						\begin{flushleft}
							\nouppercase{\leftmark}
						\end{flushleft}
					\end{minipage}
				}
			}{
				\fancyhead[L]{\nouppercase{\leftmark}}
			}
		}
		% Actualiza los header-footer
		\ifthenelse{\equal{\hfstyle}{style1}}{
			\COREhfstyledefA
		}{
		\ifthenelse{\equal{\hfstyle}{style1-i}}{ % Impar izquierdo
			\fancypagestyle{plain}{\fancyhead[LE,RO]{\nouppercase{\leftmark}}}
			\fancyhead[LE,RO]{\nouppercase{\leftmark}}
		}{
		\ifthenelse{\equal{\hfstyle}{style1-d}}{ % Impar derecho
			\fancypagestyle{plain}{\fancyhead[LO,RE]{\nouppercase{\leftmark}}}
			\fancyhead[LO,RE]{\nouppercase{\leftmark}}
		}{
		\ifthenelse{\equal{\hfstyle}{style2}}{
			\COREhfstyledefA
		}{
		\ifthenelse{\equal{\hfstyle}{style2-i}}{ % Impar izquierdo
			\fancypagestyle{plain}{\fancyhead[LE,RO]{\nouppercase{\leftmark}}}
			\fancyhead[LE,RO]{\nouppercase{\leftmark}}
		}{
		\ifthenelse{\equal{\hfstyle}{style2-d}}{ % Impar derecho
			\fancypagestyle{plain}{\fancyhead[LO,RE]{\nouppercase{\leftmark}}}
			\fancyhead[LO,RE]{\nouppercase{\leftmark}}
		}{
		\ifthenelse{\equal{\hfstyle}{style4}}{
			\COREhfstyledefA
		}{
		\ifthenelse{\equal{\hfstyle}{style5}}{
			\COREhfstyledefB
		}{
		\ifthenelse{\equal{\hfstyle}{style5-d}}{ % Impar derecho
			\COREhfstyledefB
		}{
		\ifthenelse{\equal{\hfstyle}{style5-i}}{ % Impar izquierdo
			\COREhfstyledefB
		}{
		\ifthenelse{\equal{\hfstyle}{style9}}{
			\COREhfstyledefA
		}{
		\ifthenelse{\equal{\hfstyle}{style9-d}}{ % Impar derecho
			\COREhfstyledefA
		}{
		\ifthenelse{\equal{\hfstyle}{style9-i}}{ % Impar izquierdo
			\COREhfstyledefA
		}{
		\ifthenelse{\equal{\hfstyle}{style10}}{
			\COREhfstyledefC
		}{
		\ifthenelse{\equal{\hfstyle}{style10-d}}{ % Impar derecho
			\COREhfstyledefC
		}{
		\ifthenelse{\equal{\hfstyle}{style10-i}}{ % Impar izquierdo
			\COREhfstyledefC
		}{
		\ifthenelse{\equal{\hfstyle}{style11}}{ % Similar a 1
			\COREhfstyledefA
		}{
		\ifthenelse{\equal{\hfstyle}{style14}}{ % Similar a 4
			\COREhfstyledefA
		}{
		\ifthenelse{\equal{\hfstyle}{style15}}{ % Similar a 1
			\COREhfstyledefA
		}{
			% No se encontró el header-footer, no hace nada
		}}}}}}}}}}}}}}}}}}}
	}{
	}
	
	% -------------------------------------------------------------------------
	% Crea funciones para numerar objetos
	% -------------------------------------------------------------------------
	% Numeración de la sección en los objetos código fuente
	\ifthenelse{\equal{\showsectioncaptioncode}{none}}{
		\def\sectionobjectnumcode {}
	}{
	\ifthenelse{\equal{\showsectioncaptioncode}{sec}}{
		\def\sectionobjectnumcode {\thesection\sectioncaptiondelimiter}
	}{
	\ifthenelse{\equal{\showsectioncaptioncode}{ssec}}{
		\def\sectionobjectnumcode {\thesubsection\sectioncaptiondelimiter}
	}{
	\ifthenelse{\equal{\showsectioncaptioncode}{sssec}}{
		\def\sectionobjectnumcode {\thesubsubsection\sectioncaptiondelimiter}
	}{
	\ifthenelse{\equal{\showsectioncaptioncode}{ssssec}}{
		\def\sectionobjectnumcode {\thesubsubsubsection\sectioncaptiondelimiter}
	}{
	\ifthenelse{\equal{\showsectioncaptioncode}{chap}}{
		\def\sectionobjectnumcode {\thechapter\sectioncaptiondelimiter}
	}{
		\throwbadconfig{Valor configuracion incorrecto}{\showsectioncaptioncode}{none,chap,sec,ssec,sssec,ssssec}}}}}}
	}
	
	% Numeración de la sección en los objetos ecuaciones
	\ifthenelse{\equal{\showsectioncaptioneqn}{none}}{
		\def\sectionobjectnumeqn {}
	}{
	\ifthenelse{\equal{\showsectioncaptioneqn}{sec}}{
		\def\sectionobjectnumeqn {\thesection\sectioncaptiondelimiter}
	}{
	\ifthenelse{\equal{\showsectioncaptioneqn}{ssec}}{
		\def\sectionobjectnumeqn {\thesubsection\sectioncaptiondelimiter}
	}{
	\ifthenelse{\equal{\showsectioncaptioneqn}{sssec}}{
		\def\sectionobjectnumeqn {\thesubsubsection\sectioncaptiondelimiter}
	}{
	\ifthenelse{\equal{\showsectioncaptioneqn}{ssssec}}{
		\def\sectionobjectnumeqn {\thesubsubsubsection\sectioncaptiondelimiter}
	}{
	\ifthenelse{\equal{\showsectioncaptioneqn}{chap}}{
		\def\sectionobjectnumeqn {\thechapter\sectioncaptiondelimiter}
	}{
		\throwbadconfig{Valor configuracion incorrecto}{\showsectioncaptioneqn}{none,chap,sec,ssec,sssec,ssssec}}}}}}
	}
	
	% Numeración de la sección en los objetos figuras
	\ifthenelse{\equal{\showsectioncaptionfig}{none}}{
		\def\sectionobjectnumfig {}
	}{
	\ifthenelse{\equal{\showsectioncaptionfig}{sec}}{
		\def\sectionobjectnumfig {\thesection\sectioncaptiondelimiter}
	}{
	\ifthenelse{\equal{\showsectioncaptionfig}{ssec}}{
		\def\sectionobjectnumfig {\thesubsection\sectioncaptiondelimiter}
	}{
	\ifthenelse{\equal{\showsectioncaptionfig}{sssec}}{
		\def\sectionobjectnumfig {\thesubsubsection\sectioncaptiondelimiter}
	}{
	\ifthenelse{\equal{\showsectioncaptionfig}{ssssec}}{
		\def\sectionobjectnumfig {\thesubsubsubsection\sectioncaptiondelimiter}
	}{
	\ifthenelse{\equal{\showsectioncaptionfig}{chap}}{
		\def\sectionobjectnumfig {\thechapter\sectioncaptiondelimiter}
	}{
		\throwbadconfig{Valor configuracion incorrecto}{\showsectioncaptionfig}{none,chap,sec,ssec,sssec,ssssec}}}}}}
	}
	
	% Numeración de la sección en los objetos tablas
	\ifthenelse{\equal{\showsectioncaptiontab}{none}}{
		\def\sectionobjectnumtab {}
	}{
	\ifthenelse{\equal{\showsectioncaptiontab}{sec}}{
		\def\sectionobjectnumtab {\thesection\sectioncaptiondelimiter}
	}{
	\ifthenelse{\equal{\showsectioncaptiontab}{ssec}}{
		\def\sectionobjectnumtab {\thesubsection\sectioncaptiondelimiter}
	}{
	\ifthenelse{\equal{\showsectioncaptiontab}{sssec}}{
		\def\sectionobjectnumtab {\thesubsubsection\sectioncaptiondelimiter}
	}{
	\ifthenelse{\equal{\showsectioncaptiontab}{ssssec}}{
		\def\sectionobjectnumtab {\thesubsubsubsection\sectioncaptiondelimiter}
	}{
	\ifthenelse{\equal{\showsectioncaptiontab}{chap}}{
		\def\sectionobjectnumtab {\thechapter\sectioncaptiondelimiter}
	}{
		\throwbadconfig{Valor configuracion incorrecto}{\showsectioncaptiontab}{none,chap,sec,ssec,sssec,ssssec}}}}}}
	}
	
	% -------------------------------------------------------------------------
	% Modifica numeración de objetos
	% -------------------------------------------------------------------------
	% Código fuente, incluir sección
	\ifthenelse{\equal{\captionnumcode}{arabic}}{
		\renewcommand{\thelstlisting}{\sectionobjectnumcode\arabic{lstlisting}}
	}{
	\ifthenelse{\equal{\captionnumcode}{alph}}{
		\renewcommand{\thelstlisting}{\sectionobjectnumcode\alph{lstlisting}}
	}{
	\ifthenelse{\equal{\captionnumcode}{Alph}}{
		\renewcommand{\thelstlisting}{\sectionobjectnumcode\Alph{lstlisting}}
	}{
	\ifthenelse{\equal{\captionnumcode}{roman}}{
		\renewcommand{\thelstlisting}{\sectionobjectnumcode\roman{lstlisting}}
	}{
	\ifthenelse{\equal{\captionnumcode}{Roman}}{
		\renewcommand{\thelstlisting}{\sectionobjectnumcode\Roman{lstlisting}}
	}{
		\throwbadconfig{Tipo numero codigo fuente desconocido}{\captionnumcode}{arabic,alph,Alph,roman,Roman}}}}}
	}
	
	% Ecuaciones, incluir sección
	\ifthenelse{\equal{\captionnumequation}{arabic}}{
		\renewcommand{\theequation}{\sectionobjectnumeqn\arabic{equation}}
	}{
	\ifthenelse{\equal{\captionnumequation}{alph}}{
		\renewcommand{\theequation}{\sectionobjectnumeqn\alph{equation}}
	}{
	\ifthenelse{\equal{\captionnumequation}{Alph}}{
		\renewcommand{\theequation}{\sectionobjectnumeqn\Alph{equation}}
	}{
	\ifthenelse{\equal{\captionnumequation}{roman}}{
		\renewcommand{\theequation}{\sectionobjectnumeqn\roman{equation}}
	}{
	\ifthenelse{\equal{\captionnumequation}{Roman}}{
		\renewcommand{\theequation}{\sectionobjectnumeqn\Roman{equation}}
	}{
		\throwbadconfig{Tipo numero ecuacion desconocido}{\captionnumequation}{arabic,alph,Alph,roman,Roman}}}}}
	}
	
	% Figuras, incluir sección
	\ifthenelse{\equal{\captionnumfigure}{arabic}}{
		\renewcommand{\thefigure}{\sectionobjectnumfig\arabic{figure}}
	}{
	\ifthenelse{\equal{\captionnumfigure}{alph}}{
		\renewcommand{\thefigure}{\sectionobjectnumfig\alph{figure}}
	}{
	\ifthenelse{\equal{\captionnumfigure}{Alph}}{
		\renewcommand{\thefigure}{\sectionobjectnumfig\Alph{figure}}
	}{
	\ifthenelse{\equal{\captionnumfigure}{roman}}{
		\renewcommand{\thefigure}{\sectionobjectnumfig\roman{figure}}
	}{
	\ifthenelse{\equal{\captionnumfigure}{Roman}}{
		\renewcommand{\thefigure}{\sectionobjectnumfig\Roman{figure}}
	}{
		\throwbadconfig{Tipo numero figura desconocido}{\captionnumfigure}{arabic,alph,Alph,roman,Roman}}}}}
	}
	
	% Subfiguras, no usar secciones ya que son hijas de figura
	\ifthenelse{\equal{\captionnumsubfigure}{arabic}}{
		\renewcommand{\thesubfigure}{\arabic{subfigure}}
	}{
	\ifthenelse{\equal{\captionnumsubfigure}{alph}}{
		\renewcommand{\thesubfigure}{\alph{subfigure}}
	}{
	\ifthenelse{\equal{\captionnumsubfigure}{Alph}}{
		\renewcommand{\thesubfigure}{\Alph{subfigure}}
	}{
	\ifthenelse{\equal{\captionnumsubfigure}{roman}}{
		\renewcommand{\thesubfigure}{\roman{subfigure}}
	}{
	\ifthenelse{\equal{\captionnumsubfigure}{Roman}}{
		\renewcommand{\thesubfigure}{\Roman{subfigure}}
	}{
		\throwbadconfig{Tipo numero subfigura desconocido}{\captionnumsubfigure}{arabic,alph,Alph,roman,Roman}}}}}
	}
	
	% Tablas, incluir sección
	\ifthenelse{\equal{\captionnumtable}{arabic}}{
		\renewcommand{\thetable}{\sectionobjectnumtab\arabic{table}}
	}{
	\ifthenelse{\equal{\captionnumtable}{alph}}{
		\renewcommand{\thetable}{\sectionobjectnumtab\alph{table}}
	}{
	\ifthenelse{\equal{\captionnumtable}{Alph}}{
		\renewcommand{\thetable}{\sectionobjectnumtab\Alph{table}}
	}{
	\ifthenelse{\equal{\captionnumtable}{roman}}{
		\renewcommand{\thetable}{\sectionobjectnumtab\roman{table}}
	}{
	\ifthenelse{\equal{\captionnumtable}{Roman}}{
		\renewcommand{\thetable}{\sectionobjectnumtab\Roman{table}}
	}{
		\throwbadconfig{Tipo numero tabla desconocido}{\captionnumtable}{arabic,alph,Alph,roman,Roman}}}}}
	}
	
	% Subtablas, no incluir sección ya que son hijas de las tablas
	\ifthenelse{\equal{\captionnumsubtable}{arabic}}{
		\renewcommand{\thesubtable}{\arabic{subtable}}
	}{
	\ifthenelse{\equal{\captionnumsubtable}{alph}}{
		\renewcommand{\thesubtable}{\alph{subtable}}
	}{
	\ifthenelse{\equal{\captionnumsubtable}{Alph}}{
		\renewcommand{\thesubtable}{\Alph{subtable}}
	}{
	\ifthenelse{\equal{\captionnumsubtable}{roman}}{
		\renewcommand{\thesubtable}{\roman{subtable}}
	}{
	\ifthenelse{\equal{\captionnumsubtable}{Roman}}{
		\renewcommand{\thesubtable}{\Roman{subtable}}
	}{
		\throwbadconfig{Tipo numero subtabla desconocido}{\captionnumsubtable}{arabic,alph,Alph,roman,Roman}}}}}
	}
	
	% -------------------------------------------------------------------------
	% Agrega páginas dependiendo del formato
	% -------------------------------------------------------------------------
	\ifthenelse{\equal{\GLOBALtwoside}{true}}{%
		\coretriggeronpage{\emptypagespredocformat}{}}{
	}
	
	% -------------------------------------------------------------------------
	% Reestablece \cleardoublepage
	% -------------------------------------------------------------------------
	% \let\cleardoublepage\oldcleardoublepage
	\let\cleardoublepage\corecleardoublepage
	
	% -------------------------------------------------------------------------
	% Se restablecen números de página y secciones
	% -------------------------------------------------------------------------
	% Se usa número de páginas en arábigo si es que se tenía activado los números romanos
	\ifthenelse{\equal{\predocpageromannumber}{true}}{
		\renewcommand{\thepage}{\arabic{page}}}{
	}
	
	% Reinicia número de página
	\ifthenelse{\equal{\predocresetpagenumber}{true}}{
		\setcounter{page}{1}}{
	}
	
	\setcounter{section}{0}
	\setcounter{footnote}{0}
	
	% -------------------------------------------------------------------------
	% Muestra los números de línea
	% -------------------------------------------------------------------------
	\ifthenelse{\equal{\showlinenumbers}{true}}{
		\linenumbers}{
	}
	
	% -------------------------------------------------------------------------
	% Establece el estilo de las sub-sub-sub-secciones
	% -------------------------------------------------------------------------
	\titleclass{\subsubsubsection}{straight}[\subsection]
	
	% -------------------------------------------------------------------------
	% Reestablece los valores del estado de los títulos
	% -------------------------------------------------------------------------
	\global\def\GLOBALtitlerequirechapter {true}
	\global\def\GLOBALtitleinitchapter {false}
	\global\def\GLOBALtitleinitsection {false}
	\global\def\GLOBALtitleinitsubsection {false}
	\global\def\GLOBALtitleinitsubsubsection {false}
	\global\def\GLOBALtitleinitsubsubsubsection {false}
	
}


\DeclareCaptionType{mytableWrapper}[Tabla][Índice de tablas]


% Nuevo entorno de tabla personalizado
\newenvironment{mytable}[3]{% Aceptar dos argumentos: caption y label
\setstretch{1} 
	% \label{#3}\\% Usar el argumento #3 como label
    \renewcommand{\arraystretch}{2} % Ajustar padding en eje y
    \setlength{\tabcolsep}{0.45cm} % Ajustar padding en eje x
	\rowcolors{1}{}{gray!10} % Colores intercalados, ajusta "gray!10" según tu preferencia de color
    \begin{longtable} {|>{\raggedright\arraybackslash}m{#1} |>{\raggedright\arraybackslash}m{#1}|}
    \caption{#2}\label{#3} \\% Usar el argumento #2 como caption
}{% 

\end{longtable}
\setstretch{\documentinterline} 

}

% INICIO DE LAS PÁGINAS
\begin{document}

% PORTADA
\templatePortrait

% CONFIGURACIÓN DE PÁGINA Y ENCABEZADOS
\templatePagecfg

\newcommand{\keywords}[1]{\par\noindent #1}
\newcommand{\abstracttext}[1]{\par #1}

% % Declaración de autoría
% \newpage
% \section*{Declaración de autoría}
% Por este medio doy a conocer que soy el único autor de este trabajo y
% autorizo a la Facultad de Ingeniería Automática y Biomédica, a la Universidad
% Tecnológica de La Habana (CUJAE) y a los Laboratorios Farmacéuticos AICA a
% que hagan uso del mismo para futuras inversiones en nuestro país.\\
% Como constancia firmo la presente a los 9 días del mes de junio del año
% 2023.\\

% \vspace{2cm}
% \begin{center}
% 	\rule{6cm}{0.4pt}\\
% 	\vspace{0.5cm}
% 	Armando Cesar Martin Calderón \\
% 	\vspace*{3cm}
% 	\rule{6cm}{0.4pt}
% 	\hspace*{2cm}
% 	\rule{6cm}{0.4pt} \\
% 	\vspace{0.5cm}
% 	Tutora:	Ing. Amanda Martí Coll
% 	\hspace*{2cm}
% 	Tutora: Ing. Rosaine Ayala Gispert
% \end{center}


% % Resumen en español
% \newpage
% \section*{Resumen}
% \abstracttext{
% 	Este trabajo de diploma se centra en la optimización del sistema de tratamiento de agua de la planta de bulbos en Laboratorios AICA+, mediante la introducción de la tecnología de electrodesionización (EDI). Actualmente, el tratamiento de agua es crucial en la industria farmacéutica, sin embargo, los métodos convencionales presentan desafíos en cuanto a la eficiencia y la calidad del agua. La implementación de la EDI promete superar estos desafíos proporcionando agua de alta pureza de manera constante.

% 	El objetivo principal de este estudio es proponer una solución de optimización del sistema existente a través de la implementación de EDI, la cual implica un análisis detallado de la instrumentación necesaria, desde los sensores hasta los sistemas de control y supervisión de datos (SCADA). Además, se propone un esquema general de configuración de EDI.

% 	Los resultados obtenidos sugieren que la implementación de la EDI no solo mejorará la eficiencia del proceso de tratamiento de agua, sino que también reducirá los costos operativos y de mantenimiento, mientras cumple con los requisitos y regulaciones estrictas aplicables al agua en la industria farmacéutica. En conclusión, este estudio sienta las bases para la implementación de la EDI en la planta de bulbos, promoviendo una mejora significativa en el tratamiento de agua en la industria farmacéutica.

% }

% \section*{Palabras claves}
% \keywords{Electrodesionización (EDI), Planta de tratamiento de agua, Industria farmacéutica, AICA, Agua purificada (PW), Agua para inyección (WFI), Conductividad del agua, Ósmosis inversa  }

% % Resumen en ingles
% \newpage
% \section*{Abstract}
% \abstracttext{
% 	Esta tesis se centra en optimizar el sistema de tratamiento de agua de la planta de bulbos en AICA+ Laboratories, mediante la introducción de la tecnología de electrodeionización (EDI). Actualmente, el tratamiento de agua es crucial en la industria farmacéutica; sin embargo, los métodos convencionales presentan desafíos en términos de eficiencia y calidad del agua. La implementación de EDI promete superar estos desafíos al proporcionar agua de alta pureza de manera consistente.

% 	El objetivo principal de este estudio es proponer una solución de optimización para el sistema existente mediante la implementación de EDI, lo cual implica un análisis detallado de la instrumentación necesaria, desde sensores hasta sistemas de control y adquisición de datos (SCADA). Además, se propone un esquema general para la configuración de EDI.

% 	Los resultados obtenidos sugieren que la implementación de EDI no solo mejorará la eficiencia del proceso de tratamiento de agua, sino que también reducirá los costos operativos y de mantenimiento, al tiempo que cumplirá con los estrictos requisitos y regulaciones aplicables al agua en la industria farmacéutica. En conclusión, este estudio sienta las bases para la implementación de EDI en la planta de bulbos, promoviendo una mejora significativa en el tratamiento de agua en la industria farmacéutica.
% }

% \section*{keywords}
% \keywords{
% 	Electrodesionization (EDI), Water treatment plant,
% 	Pharmaceutical industry, AICA, Purified water (PW),
% 	Water for injection (WFI), Water conductivity, Reverse osmosis
% }




% % DEDICATORIA
% \begin{dedicatory}
% 	A mi querido despertador, por ser el compañero de batalla en las madrugadas de estudio. Tus estridentes alarmas y tus intentos incansables de sacarme de la cama han sido fundamentales para que aproveche al máximo cada minuto y adelante en mi tesis.

% 	A mi fiel cafetera, por ser la fuente inagotable de energía en mis largas noches de investigación. Tus deliciosas dosis de cafeína han sido el combustible que me ha mantenido despierto y concentrado, incluso cuando la conexión a internet era un obstáculo.

% 	A mi lista de reproducción "Modo Nocturno", por llenar mis horas de estudio con melodías motivadoras y canciones pegajosas. Tú has sido mi fiel acompañante, amenizando el ambiente y dándome ese impulso extra para seguir adelante.

% 	A la luz tenue de mi lámpara, por ser mi aliada en las horas nocturnas de lectura y escritura. Tu suave brillo ha creado un ambiente acogedor y tranquilo, permitiéndome sumergirme en el mundo de la investigación y la escritura.

% 	Y a la noche misma, por brindarme la tranquilidad y la calma necesarias para concentrarme en mi tesis. Aprovechando la conexión más estable en esas horas, pude avanzar significativamente en mi trabajo y superar los desafíos que la conexión diurna presentaba.

% 	Esta dedicatoria es un homenaje a esos elementos que me han acompañado en las noches de estudio y han sido fundamentales para avanzar en mi tesis. Sin ustedes, mi experiencia de investigación y escritura no habría sido tan memorable ni efectiva.
% \end{dedicatory}

% % AGRADECIMIENTOS
% \begin{acknowledgments}
% 	Primero y ante todo, quiero expresar mi más profundo agradecimiento a mi familia, quienes siempre han sido mi faro en la vida. A mis padres, por su incondicional amor, apoyo y enseñanzas, que me han guiado hasta este punto en mi vida. A mis hermanas en especial a mi hermana mayor, que ha sido un pilar de apoyo, sabiduría y amor incondicional. Su presencia ha sido esencial en mi camino y me ha inspirado a ser una mejor persona cada día.

% 	A mis amigos, que se convirtieron en hermanos, gracias por compartir conmigo momentos de risas y lágrimas, por estar a mi lado en los momentos de tensión y alivio, y por ser mi red de apoyo durante este arduo camino. No tengo palabras para expresar cuánto valoro cada uno de ustedes. Mención especial para los tanques de ``Cuestionarios los viernes`` y a los de siempre, a mis hermanos de la Lenin.

% 	Quiero expresar mi más sincero agradecimiento a mis dos tutoras, Ing. Amanda Martí Coll e Ing. Rosaine Ayala Gispert, quienes han sido mis mentores y guías en este viaje académico. La dedicación y apoyo de la Ing. Rosaine Ayala Gispert durante el proceso en el centro de trabajo han sido invaluables, y la ayuda de la Ing. Amanda Martí Coll en la metodología ha sido crucial para el desarrollo y conclusión de esta investigación. Les estaré eternamente agradecido por su apoyo y confianza en mis habilidades.

% 	Por último, pero no menos importante, deseo agradecer a todas las personas e instituciones que de alguna manera contribuyeron a la realización de esta investigación, aportando recursos, conocimientos o simplemente un espacio donde reflexionar y crecer.

% 	Este logro no es solo mío, sino de todos los que me han acompañado en este viaje. Con profundo amor y gratitud, dedico esta tesis a cada uno de ustedes.
% \end{acknowledgments}

% % TABLA DE CONTENIDOS - ÍNDICE
% \templateIndex

% % CONFIGURACIONES FINALES
% \templateFinalcfg




% % ======================= INICIO DEL DOCUMENTO =======================
% % INTRODUCCION
% \chapter*{Introducción}
\addcontentsline{toc}{chapter}{Introducción}
La calidad del agua en la industria farmacéutica es de suma importancia,
ya que influye directamente en la calidad y seguridad de los productos
farmacéuticos, como los inyectables. La presente tesis se enfoca en la
implementación de un Electrodesionizador (EDI) en una planta
de tratamiento de agua de la industria farmacéutica, con el objetivo de
mejorar la calidad del agua purificada (PW) y el agua para inyección.
A continuación, se presenta el contexto y la justificación de este
proyecto, así como el problema a resolver, la hipótesis, el objeto
de estudio, el campo de acción, los objetivos generales y específicos,
y la estructura por capítulos.\\

% secciones
\section*{Contexto y justificación}
La industria farmacéutica desempeña un papel fundamental en la promoción y protección de la salud pública, ya que proporciona medicamentos y productos farmacéuticos que salvan vidas y mejoran la calidad de vida de millones de personas en todo el mundo. La producción de estos productos requiere la utilización de agua de alta calidad, especialmente en la fabricación de soluciones inyectables y otros medicamentos críticos. La calidad del agua utilizada en los procesos de fabricación de medicamentos es un factor esencial para garantizar la seguridad, eficacia y estabilidad de los productos finales.\\

La planta de tratamiento de agua de para bulbos de la empresa Laboratorios AICA, dedicada a la industria farmacéutica, actualmente utiliza un sistema de ósmosis inversa (OI) de doble etapa para la producción de agua purificada (PW). Sin embargo, la planta enfrenta desafíos en la estabilización de los parámetros de calidad del agua, lo que puede afectar negativamente la producción y la calidad de los medicamentos. Este problema se debe, en parte, a la inestabilidad de la calidad del agua potable proveniente del acueducto y otros factores externos.\\

La implementación de un equipo de Electrodesionización (EDI) como etapa posterior al proceso de OI de doble etapa tiene el potencial de mejorar significativamente la calidad del agua purificada y el agua para inyección, al estabilizar los parámetros de calidad y reducir la conductividad. El EDI es una tecnología de purificación de agua que combina procesos de intercambio iónico y electrodiálisis, eliminando efectivamente las partículas inorgánicas disueltas y reduciendo la concentración de iones en el agua.\\

La justificación para esta investigación radica en la importancia de garantizar la calidad del agua en la industria farmacéutica y la necesidad de encontrar soluciones efectivas y sostenibles para mejorar y estabilizar la calidad del agua en el proceso de producción. La implementación exitosa del EDI en la planta de tratamiento de agua de AICA podría resultar en una producción más eficiente y segura de medicamentos, reduciendo el riesgo de contaminación y garantizando el cumplimiento de los estándares regulatorios y de calidad. Además, la experiencia y el conocimiento adquiridos en este proyecto podrían ser aplicables a otras plantas de tratamiento de agua y procesos industriales, contribuyendo al avance del campo de la ingeniería automática y la optimización de procesos en la industria farmacéutica.\\

\section{Situación problemática}
La planta de AICA enfrenta inestabilidad en los parámetros de calidad del agua purificada y el agua para inyección debido a la variabilidad en la calidad del agua potable y otros factores. Esta situación afecta la producción y calidad de los productos farmacéuticos.\\
\textbf{Problema a resolver:}\\
El problema a resolver es cómo mejorar y estabilizar la calidad del agua purificada y el agua para inyección en la planta de AICA mediante la incorporación de un equipo de Electrodesionización (EDI) y posibles modificaciones en el sistema de control e instrumentación.


\textbf{ Hipótesis:}\\
La implementación del EDI como etapa posterior al proceso de OI de doble etapa mejorará significativamente la calidad y estabilidad del agua purificada y el agua para inyección en la planta de AICA.


\section*{Objeto de estudio}
El objeto de estudio es el proceso de tratamiento de agua en la planta de AICA y la implementación del EDI como una solución para mejorar y estabilizar la calidad del agua.
\section*{Campo de acción}
El campo de acción se centra en la evaluación y propuesta de 
modificación del sistema de tratamiento de agua en la planta de 
AICA, incluyendo la implementación del EDI y propuestas 
de el sistema de control e instrumentación.\\
\section{Objetivo general}
El objetivo general es mejorar y estabilizar la calidad del agua purificada y el agua para inyección en la planta de AICA mediante la implementación del EDI y ajustes en el sistema de control e instrumentación.
\section{Objetivos específicos}
\begin{enumerate}
    \item Evaluar la situación actual del proceso de tratamiento de agua en la planta de AICA.
    \item Investigar y proponer la implementación del EDI como etapa posterior al proceso de OI de doble etapa.
    \item Analizar los requisitos técnicos, económicos y regulatorios para la implementación del EDI en la planta.
    \item Proponer modificaciones en el sistema de control e instrumentación existente para la integración del EDI.
\end{enumerate}
\textbf{Alcance y limitaciones:}\\
El alcance de esta tesis incluye la evaluación del proceso de tratamiento de agua en la planta de AICA, la propuesta de implementación del EDI y posibles ajustes en el sistema de control e instrumentación existente. Las limitaciones pueden incluir la disponibilidad de información técnica, económica y regulatoria específica, así como restricciones en el acceso a la planta y los equipos involucrados en el proceso.


\section*{Metodología}
Para abordar el problema planteado en esta tesis, se seguirá una metodología estructurada en diversas etapas, que permitirá una aproximación sistemática al objetivo general. Las etapas de la metodología propuesta son las siguientes:
\begin{itemize}
    \item Diagnóstico del proceso actual: En esta etapa se analizará el proceso de tratamiento de agua en la planta de AICA, identificando las variables críticas, inestabilidades y limitaciones en la calidad del agua purificada y el agua para inyección. Se recopilarán y analizarán datos de producción, calidad del agua y rendimiento de los equipos involucrados en el proceso.
    \item Revisión bibliográfica y análisis del estado del arte: Se llevará a cabo una revisión exhaustiva de la literatura científica y técnica relacionada con el tratamiento de agua en la industria farmacéutica, el proceso de OI de doble etapa y la tecnología de EDI. Se buscarán estudios de caso, investigaciones y experiencias previas en la implementación de EDI en plantas similares para identificar las mejores prácticas y lecciones aprendidas.
    \item Propuesta de implementación del EDI: Basándose en el diagnóstico del proceso actual y el análisis del estado del arte, se propondrá la implementación del EDI como etapa posterior al proceso de OI de doble etapa en la planta de AICA. Se definirán los requisitos técnicos, de instrumentación y de control para la integración del EDI en el proceso existente.
    \item Análisis de costos y beneficios: Se llevará a cabo un análisis económico para estimar los costos asociados con la implementación del EDI y las posibles modificaciones en el sistema de control e instrumentación. Además, se evaluarán los beneficios esperados en términos de mejora en la calidad y estabilidad del agua, así como posibles ahorros en el consumo de energía y recursos.
    \item Evaluación de requisitos regulatorios y de cumplimiento: Se investigarán los requisitos legales y regulatorios aplicables a la implementación del EDI en la planta de AICA, así como las normas y estándares de la industria farmacéutica relacionados con el tratamiento de agua y la calidad del agua purificada y el agua para inyección.
    \item Desarrollo de modificaciones en el sistema de control e instrumentación: Basándose en la propuesta de implementación del EDI y los requisitos identificados, se desarrollarán las modificaciones necesarias en el sistema de control e instrumentación existente, incluyendo la actualización del HMI y la programación del PLC.
\end{itemize}







% \section{Estructura por capítulos}
La estructura de la tesis se presenta a continuación:\\
\textbf{Capítulo 1:}  ``Introducción``\\
En este capítulo se presenta el contexto y justificación, la situación problemática, el problema a resolver, la hipótesis, el objeto de estudio, el campo de acción, el objetivo general, los objetivos específicos, el alcance y las limitaciones, y la metodología de la investigación.



Capítulo 2: Estado del arte y descripción del proceso
Este capítulo aborda la revisión de la literatura sobre sistemas de tratamiento de agua en la industria farmacéutica, las tecnologías de purificación de agua (ósmosis inversa, EDI, etc.), la descripción del proceso actual en la planta de AICA, y la instrumentación y control en sistemas de tratamiento de agua.

Capítulo 3: Análisis de la instrumentación actual
En este capítulo se realiza la identificación y descripción de los elementos de instrumentación en el sistema de ósmosis inversa, el análisis de las señales y parámetros de control, y la evaluación del rendimiento y limitaciones de la instrumentación actual.

Capítulo 4: Propuesta de integración del EDI y nueva instrumentación
Este capítulo presenta la selección y justificación del EDI, las modificaciones necesarias en la instrumentación y control, el diseño del sistema de control e integración con el PLC existente, y el estudio de casos similares y lecciones aprendidas.

Capítulo 5: Análisis de costos y beneficios
En este capítulo se lleva a cabo la estimación de costos de adquisición e instalación del EDI e instrumentación adicional, la estimación de costos operativos y de mantenimiento, la evaluación de los beneficios en términos de mejora en la calidad del agua, eficiencia y confiabilidad del proceso, y el análisis de retorno de inversión y viabilidad económica.

Capítulo 6: Cumplimiento de normativas y regulaciones
Este capítulo aborda los requisitos regulatorios aplicables a la industria farmacéutica y sistemas de tratamiento de agua, así como la evaluación de la conformidad del sistema propuesto con las regulaciones y estándares relevantes.

Capítulo 7: Conclusiones y recomendaciones
En este último capítulo, se presentan las conclusiones generales y específicas derivadas de los resultados obtenidos en la investigación, así como las recomendaciones para la implementación del EDI en la planta de AICA y futuras investigaciones relacionadas con el tratamiento de agua en la industria farmacéutica.
 % Ejemplo, se puede borrar

% % ESTADO DEL ARTE
% % Capitulo 1
% \chapter{Estado del arte y descripción del proceso}
La purificación del agua es un aspecto crítico en la industria farmacéutica, ya que el agua es un componente
fundamental en la producción de medicamentos y otros productos sanitarios. La calidad del agua utilizada
en estos procesos puede afectar significativamente la eficacia y seguridad de los productos finales.
Por lo tanto, es esencial contar con sistemas de tratamiento de agua que sean confiables, eficientes y cumplan con los estándares regulatorios establecidos.

En este capítulo, se revisará el estado del arte en lo que respecta a los sistemas de tratamiento de
agua en la industria farmacéutica, con énfasis en las tecnologías de purificación más utilizadas, como la ósmosis inversa y la electrodesionización (EDI). Además,
se describirá el sistema de tratamiento de agua de la planta de bulbos en Laboratorios AICA+ y se analizarán los aspectos relacionados con el control en estos sistemas.

\section{Sistemas de tratamiento de agua en la industria farmacéutica}
Los sistemas de tratamiento de agua desempeñan un papel crucial en la industria farmacéutica al 
garantizar la calidad y pureza del agua utilizada en los procesos de fabricación. 
Estos sistemas están diseñados para eliminar impurezas, microorganismos y productos 
químicos indeseables, asegurando que el agua cumpla con los estándares regulatorios
 y las especificaciones de la industria. La implementación adecuada de estos sistemas asegura 
 que el agua utilizada 
en la producción farmacéutica sea segura, confiable y cumpla con los
 requisitos de calidad necesarios para la fabricación de medicamentos.

\subsection{ Importancia del tratamiento de agua}
El agua es un recurso indispensable en la industria farmacéutica debido a su amplia utilización en múltiples procesos, tales como la producción de medicamentos, la limpieza de equipos, la fabricación de soluciones y reactivos, y la generación de vapor, entre otros. Dada su relevancia, el tratamiento de agua en este sector es de suma importancia para garantizar la calidad, seguridad y eficacia de los productos farmacéuticos. A continuación, se detallan varias razones que explican la importancia del tratamiento de agua en la industria farmacéutica.\\

\textbf{Calidad del producto:} El agua utilizada en la producción de medicamentos debe cumplir con estándares estrictos de calidad y pureza, ya que su presencia en la composición de los productos puede afectar significativamente su estabilidad, potencia y seguridad. Por ejemplo, la presencia de impurezas en el agua, como iones metálicos, microorganismos o productos químicos, puede reaccionar con los ingredientes activos y excipientes de los medicamentos, alterando sus propiedades y generando efectos adversos en los pacientes.\\

\textbf{Regulaciones y normativas:} Las agencias reguladoras de todo el mundo, como la FDA (Administración de Alimentos y Medicamentos de EE. UU.) y la EMA (Agencia Europea de Medicamentos), establecen requisitos rigurosos y específicos en cuanto a la calidad del agua empleada en la producción farmacéutica. Estas regulaciones tienen como objetivo garantizar que el agua utilizada cumpla con ciertos niveles de pureza y seguridad, y que los sistemas de tratamiento de agua sean adecuados y efectivos para garantizar la calidad del producto final.\\

\textbf{Control de contaminación y biofilm:} La proliferación de microorganismos y la formación de biofilm en los sistemas de tratamiento de agua pueden tener consecuencias negativas para la calidad de los productos farmacéuticos. Un tratamiento de agua eficiente debe eliminar o reducir al mínimo la presencia de microorganismos y prevenir la formación de biofilm en las superficies de los equipos y tuberías. De esta manera, se asegura un ambiente adecuado para la producción de medicamentos y se evita la contaminación cruzada.\\

\textbf{Eficiencia en los procesos:} Un sistema de tratamiento de agua eficiente y bien diseñado puede optimizar los procesos de producción y reducir los costos operativos. El uso de tecnologías avanzadas, como la ósmosis inversa y la electrodesionización (EDI), permite obtener agua de alta calidad y pureza, lo que a su vez disminuye la necesidad de tratamientos adicionales y reduce el consumo de reactivos y energía.\\

\textbf{Responsabilidad medioambiental:} La industria farmacéutica tiene una responsabilidad ética y legal de minimizar su impacto ambiental. El tratamiento adecuado del agua permite reducir la cantidad de contaminantes y sustancias químicas liberadas al medio ambiente y optimizar el uso de los recursos hídricos. Además, las tecnologías de tratamiento de agua más avanzadas pueden contribuir a la reducción del consumo energético y la generación de residuos.\\

En resumen, el tratamiento de agua en la industria farmacéutica es fundamental para garantizar la calidad, seguridad y eficacia de los productos, cumplir con las regulaciones y normativas vigentes, controlar la contaminación y la formación de biofilm, optimizar la eficiencia en los procesos y reducir el impacto medioambiental.\\

El tratamiento adecuado del agua en la industria farmacéutica no sólo garantiza que se cumplan los requisitos de calidad y pureza del agua, sino que también contribuye a la prevención de problemas asociados con la presencia de impurezas y contaminantes. Por lo tanto, es fundamental que las empresas farmacéuticas inviertan en tecnologías de tratamiento de agua apropiadas y en la implementación de sistemas de control y monitoreo efectivos.\\


\subsection{Tipos y clasificaciones del agua}

El agua es un componente fundamental en la industria farmacéutica, y su calidad y
pureza son aspectos críticos para garantizar la seguridad y eficacia de los productos.
Dependiendo de su uso y aplicación, existen diferentes tipos y clasificaciones de agua en la industria farmacéutica.
A continuación, se presentan las categorías más comunes \cite{setaphtTratamientosAguaPara}:

\textbf{Agua purificada (PW):} Es el tipo básico de agua utilizada en la industria farmacéutica y se obtiene a través de procesos como ósmosis inversa, destilación, intercambio iónico o filtración. La calidad del agua purificada es menor que la del agua para inyección (WFI), pero es adecuada para la fabricación de productos no parenterales y para su uso en procesos de limpieza.

\textbf{Agua para inyección (WFI):} Es un tipo de agua de alta pureza que se utiliza en la fabricación de productos parenterales, es decir, aquellos que se administran por vías como intravenosa, intramuscular o subcutánea. La calidad del WFI es superior a la del agua purificada, y se obtiene mediante procesos de destilación, ósmosis inversa o por una combinación de ambos métodos.

\textbf{Agua altamente purificada (HPW):} Este tipo de agua tiene una calidad intermedia entre el agua purificada y el WFI. Se utiliza en ciertas aplicaciones farmacéuticas donde se requiere un nivel de pureza más elevado que el del agua purificada, pero no se necesita llegar al grado de pureza del WFI.

\textbf{Agua estéril:} Es agua que ha sido sometida a un proceso de esterilización, como la filtración estéril o la autoclave, para eliminar cualquier microorganismo presente. El agua estéril se utiliza en aplicaciones específicas, como en la fabricación de productos estériles o en procesos de limpieza y desinfección que requieren la eliminación de microorganismos.

Cabe destacar que las regulaciones y normativas, como las establecidas por la Farmacopea de Estados Unidos (USP), la Farmacopea Europea (EP) y la Organización Mundial de la Salud (OMS), definen los requisitos de calidad y las especificaciones para cada tipo de agua en la industria farmacéutica. Estas especificaciones incluyen parámetros como la conductividad, el pH, la presencia de sustancias orgánicas, inorgánicas y microbiológicas, entre otros.

\subsection{ Requisitos y regulaciones aplicables al agua}

La calidad del agua utilizada en la industria farmacéutica está sujeta a una serie de requisitos y regulaciones establecidos por diversas entidades y organismos a nivel nacional e internacional. Estas regulaciones aseguran que el agua cumpla con los estándares de calidad necesarios para garantizar la seguridad y eficacia de los productos farmacéuticos. Algunas de las principales regulaciones y requisitos aplicables al agua en la industria farmacéutica incluyen:\\ 

\textbf{Farmacopeas:} Las farmacopeas son documentos oficiales que contienen las especificaciones técnicas y requisitos de calidad para sustancias y productos farmacéuticos, incluidos los diferentes tipos de agua. Entre las farmacopeas más reconocidas a nivel mundial se encuentran la Farmacopea de Estados Unidos (USP), la Farmacopea Europea (EP) y la Farmacopea de Japón (JP). Cada farmacopea establece parámetros específicos de calidad, como la conductividad, el pH, la presencia de sustancias orgánicas, inorgánicas y microbiológicas, entre otros. \\ 

\textbf{ Buenas Prácticas de Fabricación (GMP):} Las GMP son normas que establecen los requisitos mínimos que deben cumplir los procesos de fabricación, control de calidad y distribución de productos farmacéuticos, incluida la gestión del agua. Estas normas son aplicables a nivel mundial y son emitidas por organismos como la Food and Drug Administration (FDA) en Estados Unidos, la European Medicines Agency (EMA) en Europa y la Organización Mundial de la Salud (OMS).\\ 

\textbf{ Directrices y guías técnicas:} Además de las farmacopeas y las GMP, existen directrices y guías técnicas emitidas por organismos internacionales y nacionales que abordan aspectos específicos relacionados con el agua en la industria farmacéutica. Estas directrices pueden incluir recomendaciones sobre el diseño y validación de sistemas de tratamiento de agua, el monitoreo de la calidad del agua y la prevención de la contaminación.\\ 

\textbf{ Normativas nacionales y locales:} Cada país puede tener sus propias normativas y requisitos legales aplicables al agua en la industria farmacéutica. Estas normativas pueden estar en línea con las farmacopeas y las GMP, pero también pueden incluir requisitos adicionales específicos para cada país o región.\\\\ 


El cumplimiento de estas regulaciones y requisitos garantiza la calidad y seguridad del agua utilizada en la fabricación de productos farmacéuticos y, en última instancia, protege la salud de los pacientes.

\subsection{Impurezas presentes en el agua y su impacto en los productos farmacéuticos}

El agua utilizada en la industria farmacéutica puede contener diversas impurezas, las cuales pueden afectar la calidad, seguridad y eficacia de los productos finales. Estas impurezas pueden clasificarse en tres categorías principales: impurezas inorgánicas, impurezas orgánicas y contaminantes microbiológicos.\\

\textbf{Impurezas inorgánicas: }Incluyen iones metálicos y no metálicos, como calcio, magnesio, sodio, cloruros, sulfatos y silicatos. Estas impurezas pueden afectar la calidad de los productos farmacéuticos al causar cambios en la solubilidad, la estabilidad y la eficacia de los ingredientes activos, así como en la formación de precipitados y la corrosión de equipos y recipientes. Además, algunos iones metálicos, como el hierro, el cobre y el cromo, pueden ser tóxicos y afectar la seguridad de los productos.\\
\textbf{Impurezas orgánicas: }Son compuestos de origen natural o sintético, como ácidos húmicos y fúlvicos, pesticidas, disolventes y productos químicos de desinfección. Las impurezas orgánicas pueden reaccionar con los ingredientes activos y otros excipientes, lo que puede alterar la estabilidad, la eficacia y la liberación de los fármacos. Además, algunos compuestos orgánicos pueden ser tóxicos y afectar la seguridad de los productos farmacéuticos.\\
\textbf{Contaminantes microbiológicos:} Incluyen bacterias, hongos, levaduras, virus y protozoos. La presencia de microorganismos en el agua puede causar la contaminación de los productos farmacéuticos, lo que puede llevar a infecciones y reacciones adversas en los pacientes. Además, algunos microorganismos pueden producir sustancias tóxicas, como endotoxinas y micotoxinas, que pueden afectar la seguridad y eficacia de los productos.\\

El tratamiento adecuado del agua es esencial para eliminar o reducir estas impurezas a niveles aceptables, de acuerdo con las regulaciones y requisitos aplicables en la industria farmacéutica. Un control riguroso de la calidad del agua, así como el uso de tecnologías de purificación adecuadas, como la ósmosis inversa, la desionización y la electrodesionización (EDI), son fundamentales para garantizar la calidad y seguridad de los productos farmacéuticos.\\

\subsection{Variables críticas en la calidad del agua}

El tratamiento y monitoreo de la calidad del agua en la industria farmacéutica requieren un enfoque riguroso y sistemático para garantizar la eliminación efectiva de impurezas y el cumplimiento de los requisitos regulatorios. A continuación, se presentan algunas de las variables críticas que deben considerarse durante el tratamiento y monitoreo del agua:\\

\textbf{Conductividad eléctrica:} La conductividad eléctrica es una medida de la capacidad del agua para conducir la corriente eléctrica, y está directamente relacionada con la concentración de iones disueltos en el agua. Un mayor valor de conductividad indica una mayor concentración de impurezas inorgánicas. El monitoreo de la conductividad es fundamental para evaluar la efectividad de los procesos de purificación y para asegurar el cumplimiento de los límites establecidos por las regulaciones aplicables.

\textbf{Contenido de carbono orgánico total (TOC ):} El TOC es una medida del contenido de carbono en compuestos
orgánicos disueltos en el agua. Un alto nivel de TOC indica una mayor concentración de impurezas orgánicas.
El monitoreo regular del TOC es esencial para garantizar que el agua cumpla con los requisitos de calidad y para evaluar la eficacia de los procesos de purificación en la eliminación de compuestos orgánicos.

\textbf{Conteo microbiano y endotoxinas:} El monitoreo del recuento microbiano y las endotoxinas es fundamental
para controlar la calidad microbiológica del agua y garantizar la seguridad de los productos farmacéuticos.
Los métodos de análisis microbiológico incluyen el recuento en placa, el método de filtración por membrana
y las técnicas de bioluminiscencia. Las endotoxinas, sustancias tóxicas liberadas por bacterias Gram-negativas,
se miden mediante el ensayo de lisado de amebocitos de Limulus (LAL).

\textbf{pH:} El pH es una medida de la acidez o alcalinidad del agua y puede afectar la solubilidad,
la estabilidad y la reactividad de los ingredientes activos y excipientes en los productos farmacéuticos.
El control del pH es esencial para mantener un ambiente adecuado en los sistemas de tratamiento de agua y
garantizar la calidad del agua producida.

\textbf{Turbidez:} La turbidez es una medida de la cantidad de partículas en suspensión en el agua, incluidas partículas inorgánicas, orgánicas y microbiológicas. Un nivel elevado de turbidez puede afectar la efectividad de los procesos de purificación y el rendimiento de los equipos. La turbidez se mide utilizando un turbidímetro y se expresa en unidades de turbidez nefelométrica (NTU).

El monitoreo y control de estas variables críticas durante el tratamiento y purificación del agua son fundamentales para garantizar la calidad, seguridad y eficacia de los productos farmacéuticos y cumplir con los requisitos regulatorios aplicables.

\subsection{Evolución histórica de las tecnologías de tratamiento de agua}

La historia del tratamiento de agua en la industria farmacéutica ha experimentado una evolución considerable a lo largo del tiempo. A medida que la industria ha crecido y los requisitos regulatorios han aumentado en complejidad, las tecnologías de tratamiento de agua han seguido mejorando para garantizar la calidad y la seguridad de los productos farmacéuticos.

\textbf{Pre-Siglo XX}: Antes del siglo XX, los métodos de purificación de agua eran bastante rudimentarios, 
enfocándose principalmente en la eliminación de sólidos y materia orgánica a través de procesos físicos como 
la sedimentación y la filtración a través de medios porosos como la arena. Estos procesos, aunque rudimentarios, 
establecieron la base para las técnicas modernas de tratamiento de agua \cite{higieneambientalHistoriaTratamientoAgua2018}.

\textbf{Principios del Siglo XX}: Con la introducción del uso del cloro como agente desinfectante en 1908 en Jersey City, 
Estados Unidos, las industrias empezaron a utilizar este método para garantizar la seguridad microbiológica de su agua. 
Por otro lado, la destilación, un proceso que se basa en la evaporación y condensación del agua para separarla de sus
 impurezas, también se empleaba aunque era energéticamente costoso \cite{higieneambientalHistoriaTratamientoAgua2018}.

\textbf{Mediados del Siglo XX}: A mediados del siglo XX, comenzó a ser común el uso de la filtración por membrana, 
específicamente la ósmosis inversa (RO), para la eliminación de partículas y solutos disueltos. Este proceso utiliza 
una membrana semipermeable para eliminar iones, moléculas y partículas más grandes del agua potable Además, 
la radiación ultravioleta (UV) empezó a ser utilizada como un método eficaz de esterilización del agua, 
matando o inactivando microorganismos al destruir su material genético \cite{higieneambientalHistoriaTratamientoAgua2018}.

\textbf{Finales del Siglo XX y principios del Siglo XXI}: Las técnicas de purificación de agua se volvieron más avanzadas y selectivas hacia 
finales del siglo XX y principios del XXI. La ósmosis inversa, la desionización y la electrodesionización (EDI) 
se volvieron estándares en la industria farmacéutica. La electrodesionización, en particular, es una tecnología 
que combina la desionización electroquímica y la desionización de lecho mixto para producir agua de alta pureza de 
manera eficiente y sin el uso de productos químicos peligrosos.

La evolución de las tecnologías de tratamiento de agua en la industria farmacéutica ha sido impulsada por la creciente
demanda de productos de alta calidad y la necesidad de cumplir con requisitos regulatorios cada vez más rigurosos. A
medida que la industria farmacéutica continúa avanzando, es probable que surjan nuevas tecnologías y enfoques para el
tratamiento y monitoreo del agua en el futuro. Algunas áreas de investigación y desarrollo incluyen:\\

\textbf{Nanotecnología:} La aplicación de nanomateriales y nanopartículas en el tratamiento de agua ofrece oportunidades para
mejorar la eficiencia de los procesos existentes y desarrollar nuevos enfoques para la eliminación de impurezas. Por ejemplo,
las membranas nanocompuestas y las nanopartículas funcionales pueden mejorar la selectividad y la eficiencia de las membranas de ósmosis inversa y EDI \cite{quxApplicationsNanotechnologyWater2013}.

\textbf{Tratamiento biológico:} Los enfoques biológicos, como la utilización de microorganismos para la degradación de contaminantes
orgánicos, pueden proporcionar alternativas sostenibles y de bajo costo a las tecnologías convencionales de tratamiento de agua \cite{siziriciyildizWaterWastewaterTreatment2012}.

\textbf{Sistemas avanzados de monitoreo y control:} Los avances en sensores, analítica en línea y tecnologías de control permiten
una mejor comprensión y control del proceso de tratamiento de agua en tiempo real. Esto puede llevar a una mayor eficiencia y
garantizar una calidad de agua más consistente \cite{kayayAdvancesRealtimeMonitoring2020}.



% \subsection{Tecnologías actuales y enfoques de investigación en sistemas de tratamiento de agua para la industria farmacéutica}

\input{tesis/estado_del_arte/revision_de_literatura/tecnologias_actuales/etapas.tex}
\input{tesis/estado_del_arte/revision_de_literatura/tecnologias_actuales/variantes.tex}
\input{tesis/estado_del_arte/revision_de_literatura/tecnologias_actuales/innovaciones.tex}

\section{Descripción del proceso actual}
La planta de AICA cuenta con un proceso integral de tratamiento y purificación de agua para abastecer
a sus instalaciones con agua de alta calidad y pureza. Este proceso es esencial para garantizar
el cumplimiento de las normativas y estándares aplicables en la industria farmacéutica y biotecnológica.
A continuación, se proporcionará una descripción detallada de las distintas etapas y componentes del
proceso actual en la planta de AICA, desde la captación del agua hasta su punto antes de la distribución y uso en las distintas áreas de producción.

Cabe destacar que en el Anexo \ref{sec:anexoA} se encuentra un diagrama
 P\&ID del sistema de ósmosis inversa en la planta de bulbos en cuestión, el cual constituye un 
 elemento clave dentro de este proceso de tratamiento y purificación de agua. 
 El P\&ID brinda una representación gráfica detallada de las distintas etapas 
 y componentes involucrados en el sistema implementado en 
 la planta de AICA.

% Sistema no tecnológico
\subsection*{Sistema Tecnológico y sus plantas de tratamiento}

El Sistema Tecnológico es el área de interés para esta investigación y se compone de dos plantas de tratamiento de agua.
La primera planta se dedica a la producción de ampolletas, mientras que la segunda planta se encarga de la producción de bulbos, esta última es en la que centra el estudio.



% Almacenamiento y bombeo del agua potable
\subsection*{Almacenamiento y bombeo del agua potable}

El Sistema de Tratamiento de Agua de Bulbos en Laboratorios AICA$^+$ se encarga de garantizar la eficiencia y calidad de los
diferentes tipos de aguas farmacéuticas, como el agua purificada y destilada, que se utilizan en la planta de producción de inyectables.
El proceso comienza con el almacenamiento del agua potable procedente del acueducto en dos cisternas con capacidades de 900 y 700 m$^3$ .
Posteriormente, el agua cruda es bombeada a través de las bombas de la estación de hidroneumáticos hacia las líneas de Servicios Generales y
al Sistema de Tratamiento de Agua, que se divide en dos partes: el Sistema No Tecnológico y el Sistema Tecnológico.


\subsection*{Dosificación de hipoclorito de sodio y filtración}

El agua proveniente de la cisterna llega al sistema de pretratamiento de aguas de Bulbo a una presión entre 4 - 5 bar. En la línea de entrada, se
dosifica hipoclorito de sodio al 3\% para desinfectar el agua y reducir la concentración de bacterias y microorganismos. El sistema de dosificación consta de un
tanque de solución de 50 L y una bomba con capacidad de 1.58 l/h, permitiendo una concentración de cloro residual cercana al 1\%.
Un contador de impulsos acoplado a la línea gobierna esta dosificación, enviando una señal a la bomba cada 100 L de agua, equivalente a 1
impulso. Posteriormente, el agua pasa por un filtro CF-60 de 50 micras, fabricado de acero inoxidable AISI 304, que cumple la función de
filtración y actúa como elemento mezclador después de la dosificación de cloro.

\subsection*{Almacenamiento y monitoreo de parámetros del agua}

Una vez filtrada, el agua sale del filtro CF-60 con un flujo que oscila entre 7-8 m$^3$/h y se almacena en el tanque de almacenamiento de agua potable,
TK-60, con capacidad de 3,000 L. Este tanque sirve como depósito de alimentación para los suavizadores. Se han instalado tomas de muestra antes y
después del filtro para monitorear el pH y el cloro residual del agua. Este monitoreo permite verificar la calidad del agua en esta etapa del proceso y
asegurar que los parámetros se encuentren dentro de los límites aceptables antes de continuar con el proceso de purificación.

% Suavizadores

\input{tesis/estado_del_arte/descripcion_proceso/suavizacion/1suavizacion.tex}
\input{tesis/estado_del_arte/descripcion_proceso/purificacion/1purificacion.tex}


% % Capitulo 2
% \chapter{Introducción y fundamentos de la Electrodeionización (EDI)}\label{cap:fundamentosEDI}
La electrodeionización (EDI) es una tecnología que combina la electroquímica y la resina de intercambio iónico para producir agua ultrapura, que es esencial en una variedad de aplicaciones industriales y de laboratorio. Desde su invención en la década de 1950, la EDI ha evolucionado para convertirse en una opción preferida para la purificación de agua, especialmente en industrias que requieren altos estándares de pureza, como la farmacéutica, la de semiconductores, la de energía y la de alimentos y bebidas. La EDI es especialmente útil en aplicaciones donde se necesita una desionización continua y sin químicos, y donde la conservación de agua es crucial. \\

El principio de la EDI se basa en la utilización de corriente eléctrica y resinas de intercambio iónico para eliminar iones y partículas disueltas en el agua. A diferencia de los métodos tradicionales de desionización, como la desionización por intercambio iónico, la EDI no requiere el uso de químicos para regenerar las resinas de intercambio iónico. En su lugar, utiliza corriente eléctrica para regenerar las resinas, lo que permite un proceso de desionización continua. Esta característica no solo elimina la necesidad de manipular y disponer de productos químicos dañinos, sino que también mejora la eficiencia del proceso de desionización y reduce el consumo de agua. \\

Este capítulo proporciona una introducción y una descripción detallada de los fundamentos de la EDI. Incluye una discusión sobre los principios básicos de la EDI, los componentes y el diseño de un sistema de EDI, así como los beneficios y desafíos asociados con la implementación de la tecnología EDI. El capítulo concluye con una discusión sobre las aplicaciones de la EDI en la industria farmacéutica, resaltando la importancia de la EDI en la producción de agua ultrapura para aplicaciones farmacéuticas. \\

\section{Principios de la EDI}
La Electrodeionización (EDI) es una tecnología que combina métodos físicos y químicos para eliminar iones disueltos del agua. En el corazón de este proceso se encuentra la resina de intercambio iónico, que actúa como un medio para la extracción de iones y partículas disueltas en el agua. \\

El principio de funcionamiento de la EDI se basa en dos procesos fundamentales: el intercambio iónico y la electrólisis. En el intercambio iónico, los iones en el agua son atraídos y retenidos por la resina de intercambio iónico, que es esencialmente una red de polímeros cargados. Los iones negativos (aniones) son atraídos por los sitios cargados positivamente en la resina, mientras que los iones positivos (cationes) son atraídos por los sitios cargados negativamente. Este proceso de intercambio de iones efectivamente atrapa y retiene los iones disueltos en el agua. \\

Por otro lado, la electrólisis implica el uso de una corriente eléctrica para estimular la migración de iones. En un sistema de EDI, la corriente eléctrica es aplicada a través de una serie de electrodos, que están situados en los extremos de la celda de EDI. Bajo la influencia de este campo eléctrico, los iones en el agua son instigados a moverse hacia los electrodos correspondientes - los cationes hacia el electrodo negativo y los aniones hacia el electrodo positivo. \\

Combinando estos dos procesos, la EDI logra desionizar el agua de manera efectiva. La resina de intercambio iónico atrae y retiene los iones disueltos, mientras que la corriente eléctrica impulsa a estos iones a través de la celda de EDI y hacia los electrodos. Durante este proceso, el agua es efectivamente despojada de sus impurezas iónicas. \\

La EDI también se caracteriza por un proceso de regeneración continua de las resinas de intercambio iónico. Tradicionalmente, las resinas de intercambio iónico deben ser regeneradas con frecuencia mediante el uso de productos químicos. Sin embargo, en la EDI, la corriente eléctrica proporciona la energía necesaria para la regeneración. Esto significa que la resina de intercambio iónico nunca se agota realmente, lo que permite un proceso de desionización continua y sin interrupciones. \\

Además, la EDI se beneficia de la separación de las corrientes de agua desionizada y concentrada. Esto se logra mediante el uso de membranas semipermeables, que están dispuestas en la celda de tal manera que separan la celda en compartimentos de concentración y dilución. Los iones disueltos, que son atraídos por los electrodos, son llevados a través de las membranas hacia el compartimento de concentración, mientras que el agua desionizada se recoge en el compartimento de dilución. \\

\section{Componentes y diseño de la EDI}
La efectividad y eficiencia de un sistema de EDI dependen en gran medida de su diseño y de los componentes utilizados. En este sentido, existen varios componentes clave en un sistema de EDI que son fundamentales para su operación. \\

\subsection{Cámara de dilución y concentración}
Las cámaras de dilución y concentración son un componente crítico en el diseño de la EDI. Estas cámaras permiten la separación física del agua desionizada del agua concentrada con iones. La cámara de dilución es donde el agua purificada se recoge después de que los iones disueltos son extraídos, mientras que la cámara de concentración es donde se recogen los iones extraídos. Esta separación es crucial para mantener la pureza del agua desionizada y para garantizar que los iones disueltos no se reintroduzcan en el agua. Las cámaras de dilución y concentración están separadas por membranas semipermeables, que permiten el paso de iones pero restringen el flujo de agua. \\

\subsection{Resina de intercambio iónico}
La resina de intercambio iónico es otro componente crítico de un sistema de EDI. La resina actúa como un medio para la atracción y retención de iones disueltos en el agua. La resina es esencialmente una red de polímeros cargados, con sitios de intercambio iónico que atraen iones de carga opuesta. Las resinas de intercambio iónico vienen en dos tipos principales: cationes y aniones, que atraen iones negativos y positivos respectivamente. En un sistema de EDI, se utiliza una mezcla de resinas de intercambio de cationes y aniones para asegurar la extracción de todos los tipos de iones disueltos. \\

\subsection{Electrodo y membrana}
Los electrodos y las membranas son componentes esenciales en la operación de un sistema de EDI. Los electrodos, situados en los extremos de la celda de EDI, proporcionan el campo eléctrico que impulsa la migración de iones a través de la celda. Dependiendo de la carga del ión, los iones disueltos son atraídos hacia el electrodo positivo o negativo. Las membranas, por otro lado, están diseñadas para permitir el paso de iones, pero no de agua. De esta manera, las membranas facilitan el movimiento de los iones disueltos hacia la cámara de concentración, mientras que el agua purificada se recoge en la cámara de dilución. \\

\subsection{Fuente de alimentación}
La fuente de alimentación es otro componente crucial de un sistema de EDI. Proporciona la corriente eléctrica necesaria para el proceso de electrólisis, que impulsa la migración de iones a través de la celda de EDI. La fuente de alimentación debe ser capaz de suministrar una corriente eléctrica constante y estable para garantizar una operación eficiente del sistema. \\


\section{Beneficios y desafíos de la EDI}
La tecnología de la EDI tiene numerosos beneficios, pero también presenta ciertos desafíos que deben ser reconocidos y superados para su efectiva implementación y operación. \\

\subsection{Beneficios de la EDI}
La Electrodeionización (EDI) ofrece una variedad de ventajas significativas en comparación con otras tecnologías de purificación de agua. En este apartado, se detallarán estos beneficios de manera exhaustiva, desde la simplicidad operativa hasta la eficacia en la eliminación de partículas inorgánicas.\\

Empezando con su operación, la EDI destaca por su simplicidad y continuidad. Dado que esta tecnología combina la electrodiálisis y el intercambio iónico, permite una producción ininterrumpida de agua de alta pureza. Esto significa que no es necesario interrumpir el proceso para la regeneración de las resinas, como ocurre con otros métodos de desionización. Esta característica contribuye a mejorar la eficiencia de los procesos productivos y a minimizar el tiempo de inactividad.\\

Un factor crítico que diferencia a la EDI de otros métodos de purificación es la eliminación casi total del uso de productos químicos en el proceso de regeneración. A diferencia de los sistemas tradicionales de intercambio iónico, la EDI utiliza una corriente eléctrica para regenerar las resinas de intercambio iónico. Este enfoque no solo elimina la necesidad de manejar y almacenar productos químicos peligrosos, sino que también reduce los costos operativos y el impacto ambiental asociado con la eliminación de productos químicos residuales.\\

Desde la perspectiva operativa y de mantenimiento, la EDI también tiene ventajas económicas significativas. Gracias a su diseño compacto y a la ausencia de partes móviles, el mantenimiento de los sistemas de EDI es relativamente simple y los riesgos de averías son bajos. Esta característica se traduce en ahorros en los costos de mantenimiento y en la reducción del tiempo de inactividad del sistema. Adicionalmente, la EDI se caracteriza por su eficiencia energética, lo cual se refleja en un menor consumo de energía en comparación con otros métodos de purificación de agua, como la destilación.\\

En lo que respecta al impacto ambiental, la EDI se considera una tecnología ecológica. Al no producir efluentes peligrosos y al eliminar la necesidad de manejo de productos químicos, se reduce significativamente el riesgo de contaminación ambiental. Asimismo, no requiere la disposición de resinas de intercambio iónico agotadas, lo que minimiza aún más su huella ambiental.\\

Finalmente, la EDI es altamente efectiva en la eliminación de partículas inorgánicas disueltas en el agua. Con su capacidad para eliminar hasta el 99,9\% de los iones presentes en el agua, incluyendo cationes y aniones, ofrece un grado de purificación que supera a la mayoría de los otros métodos disponibles. Esta efectividad la convierte en una solución de purificación de agua altamente atractiva para una amplia gama de aplicaciones.\\

\subsection{Desafíos de la EDI}
A pesar de sus numerosos beneficios, la implementación y operación de la EDI también presentan desafíos. \\

Uno de los principales desafíos es la necesidad de una pretratamiento del agua de alimentación. La EDI requiere agua de alimentación de baja conductividad, por lo general proporcionada por la ósmosis inversa (RO). Además, el agua de alimentación debe estar libre de cloro y otras sustancias oxidantes que pueden dañar las resinas de intercambio iónico y las membranas de la EDI. Por lo tanto, el diseño del pretratamiento del agua es crucial para el rendimiento de la EDI. \\

Otro desafío es el mantenimiento de los sistemas de EDI. Aunque la EDI reduce la necesidad de químicos regenerantes, todavía requiere limpieza periódica y reemplazo de componentes para mantener su rendimiento. La membrana de EDI y las resinas de intercambio iónico pueden necesitar ser reemplazadas después de un cierto período de tiempo, dependiendo de la calidad del agua de alimentación y de las condiciones operativas. \\

Finalmente, la EDI requiere un suministro de energía eléctrica constante para su operación. Cualquier fluctuación en el suministro de energía puede afectar el rendimiento de la EDI y resultar en una calidad de agua inconsistente. Por lo tanto, un suministro de energía confiable es esencial para la operación de la EDI. \\

\section{Aplicaciones de la EDI en la industria farmacéutica}
En la industria farmacéutica, la pureza y consistencia del agua utilizada en los procesos de producción son de suma importancia. Cualquier contaminante, ya sea orgánico, inorgánico o microbiológico, puede afectar la calidad del producto final y comprometer la seguridad del paciente. En este contexto, la EDI ha encontrado un lugar destacado debido a su capacidad para producir agua de alta pureza de manera confiable y continua.\\

La EDI es comúnmente utilizada en la producción de agua purificada (PW) y agua para inyección (WFI). El agua purificada es utilizada en una amplia gama de aplicaciones en la industria farmacéutica, como la preparación de soluciones para la producción de productos farmacéuticos y la limpieza de equipos y envases. El agua para inyección, que requiere un nivel aún mayor de pureza, es utilizada en la producción de productos parenterales, como soluciones para inyección y productos liofilizados.\\

La EDI se utiliza a menudo en combinación con otros procesos de purificación de agua, como la ósmosis inversa (RO) y la destilación. En un sistema típico, la RO se utiliza primero para reducir la concentración de sales y otros contaminantes en el agua. Luego, la EDI se utiliza para eliminar los iones restantes y lograr el nivel de pureza deseado. Finalmente, si se requiere agua para inyección, el agua producida por la EDI puede ser sometida a destilación para eliminar cualquier contaminante restante y garantizar la esterilidad.\\



% Capitulo 3
% \chapter{Análisis de la instrumentación}

La instrumentación, como pieza fundamental de cualquier proceso de ingeniería, es una serie de 
elementos que proporcionan el control y la supervisión necesarios para garantizar la eficiencia y la 
seguridad de un sistema. El propósito de este capítulo es analizar la instrumentación actualmente 
implementada en nuestro sistema de ósmosis inversa, examinando tanto los componentes físicos 
como la red y los protocolos de comunicación que permiten su funcionamiento integrado.\\

Iniciaremos con una mirada detallada a la instrumentación existente, explorando su funcionalidad, 
la interrelación entre los componentes y la forma en que cada pieza contribuye al objetivo global del 
sistema de tratamiento de agua. Al entender completamente la configuración actual, podremos identificar 
las áreas de mejora y explorar las oportunidades para la optimización y el crecimiento.\\

En la siguiente parte del análisis, abordaremos los detalles de la red y los protocolos de 
comunicación. Al igual que el sistema circulatorio en un organismo, la red y los protocolos de 
comunicación son los que mantienen viva la instrumentación, permitiendo la comunicación y la colaboración 
eficaces entre los diferentes elementos. Profundizaremos en cómo funcionan, cómo están configurados y 
cómo impactan en la eficiencia general del sistema.\\

Finalmente, este capítulo culminará con una propuesta de instrumentación actualizada. Con la 
implementación de un electrodeionizador (EDI), buscamos mejorar aún más los parámetros de purificación del agua, 
llevando nuestro sistema de tratamiento de agua a nuevas alturas de rendimiento y eficacia. Exploraremos qué 
elementos adicionales son necesarios para esta actualización, cómo se integrarán en el sistema existente y cómo
mejorarán el proceso de purificación de agua.\\

Es importante destacar que aunque hay numerosos instrumentos y equipos en la planta de tratamiento de agua en general, 
el foco de este capítulo es proporcionar una descripción y análisis detallado  para el sistema 
de ósmosis inversa y que serán esenciales para entender e implementar las mejoras propuestas.\\


\section{Instrumentación de la planta de ósmosis inversa}


La instrumentación en un sistema de ósmosis inversa es crucial para su funcionamiento eficiente y seguro.
Esta sección se centrará en los distintos dispositivos de control que permiten la operación continua y
segura de la planta de ósmosis inversa. Hablaremos de las válvulas, que regulan el flujo de agua y sustancias
químicas en el sistema, las bombas que impulsan la circulación y los sensores que monitorizan
las condiciones y parámetros clave, como la conductividad, la temperatura y el pH.\\

Además, discutiremos el controlador lógico programable (PLC), que es el cerebro de la operación. El PLC es responsable
de la gestión y el control de todas las señales de entrada y salida de la planta, lo que implica tomar decisiones basadas
en las lecturas de los sensores y ajustar los actuadores, como las bombas y las válvulas, para mantener el sistema
funcionando de manera óptima.\\

Por último, exploraremos los diferentes módulos que el PLC necesita para interactuar con los demás componentes del sistema.
Estos módulos son esenciales para la comunicación y el control eficaz del proceso de ósmosis inversa.\\


\subsection{Sensor de conductividad} \label{sec:sesor_conductividad}

En el proceso de ósmosis inversa, la conductividad es una variable esencial a ser controlada. Los sensores de
conductividad son, por tanto, componentes críticos en la planta, proporcionando datos en tiempo real que informan
sobre la eficiencia del proceso. Específicamente, son capaces de detectar cambios en la concentración de iones
en el agua, lo que puede ser indicativo de un problema con las membranas de ósmosis inversa.

En nuestro sistema el sensor de conductividad usado como el de la figura \ref{fig:sensorCLS16}, pertenece al fabricante Endress+Hauser tiene un principio de funcionamiento que radica en la medición de la
conductividad eléctrica del agua,
que refleja su capacidad para transmitir corriente eléctrica. Esta propiedad está directamente relacionada
con la concentración de iones disueltos en el agua. El sensor aplica un voltaje a dos electrodos situados a una
distancia fija y mide la corriente resultante. Como la conductividad depende del contenido de sales en el agua,
un aumento de la conductividad sugiere una mayor concentración de iones, indicando una posible eficacia reducida
de las membranas de ósmosis inversa \cite{endress+hauserAnalogConductivitySensor}.

\insertimageboxed[\label{fig:sensorCLS16}]{instrumentacion/sensorCLS16}{scale=0.7}{0}{Sensor de conductividad CLS16-3D1A1P}

Este tipo de sensores se pueden encontrar en lugares como en el punto de salida de la primera
etapa del sistema de ósmosis inversa (en el flujo de permeado).

\begin{mytable}{6cm}{Datos técnicos del sensor de conductividad CLS16-3D1A1P.}{table:sensorCLS16}

        \hline
        \textbf{Modelo}                        & CLS16-3D1A1P                        \\
        \hline
        \textbf{Material}                      & Acero inoxidable 316 L (DIN 1.4435) \\
        \hline
        \textbf{Acabado}                       & Electropulido Ra < 0,8µm            \\
        \hline
        \textbf{Constante}                     & 0,1 (0,04 / 500 µS/cm)              \\
        \hline
        \textbf{Rango}                         & 0 a 20 µS/cm                        \\
        \hline
        \textbf{Conexiones}                    & 1"½ (38,10 mm) Tri-Clamp            \\
        \hline
        \textbf{Temperatura máxima del fluido} & 120 °C                              \\
        \hline
        \textbf{Fabricante}                    & Endress+Hauser                      \\
        \hline

\end{mytable}




% ------------Sensores de pH-----------
\subsection{Sensor de pH} \label{sec:sensor_ph}

La medición y control del pH en el agua tratada es crucial en una planta de ósmosis inversa. Los sensores de pH
desempeñan un papel vital en este aspecto, permitiendo la monitorización constante del pH del agua y facilitando
el control de la dosificación de hidróxido de sodio (NaOH).

Los sensores de pH como el usado de modelo CPS 11D-7AA2G, operan basándose en el principio de medición del potencial electroquímico a través de una
celda compuesta por un electrodo de referencia y un electrodo de medición. La diferencia de potencial entre
estos electrodos está relacionada con el pH del medio acuoso. El electrodo de medición, fabricado generalmente
de vidrio, tiene una propiedad particular de presentar una diferencia de potencial con el agua que se encuentra en contacto, la cual es dependiente del pH \cite{endress+hauserTechnicalInformationOrbisint}.

Un sensor de pH como el CPS 11D-7AA2G de Endress+Hauser mostrado en la figura \ref{fig:sensorCPS} se utiliza
en la planta para controlar el pH después del filtro de 5 micras. La información de este sensor es utilizada
para el control de la dosificación de NaOH, ayudando a mantener el pH dentro de los rangos deseados, lo cual
es esencial para la eficiencia del proceso de ósmosis inversa.

\insertimageboxed[\label{fig:sensorCPS}]{instrumentacion/sensorCPS}{scale=0.8}{0}{Sensor de pH CPS 11D-7AA2G}

A continuación, se presenta la tabla \ref{table:sensorCPS} con las características técnicas principales del sensor de pH CPS 11D-7AA2G:\\

\begin{mytable}{6cm}{Características del sensor de pH CPS 11D-7AA2G.}{table:sensorCPS}
        \hline
        \textbf{Modelo}               & CPS 11D-7AA2G Memosens \\
        \hline
        \textbf{Material}             & Vidrio                 \\
        \hline
        \textbf{Rango pH}             & 0-12                   \\
        \hline
        \textbf{Rango de temperatura} & -5 a 80°C              \\
        \hline
        \textbf{Longitud de la sonda} & 120 x 12 mm            \\
        \hline
        \textbf{Conector}             & tipo N con PG13,5      \\
        \hline
        \textbf{Fabricante}           & Endress+Hauser         \\
        \hline

\end{mytable}

% ------------Sensores de Temperatura-----------
\subsection{Sensor de temperatura} \label{sec:sensor_temp}

Las sondas de temperatura son un componente crítico en cualquier proceso industrial que requiera control preciso
de la temperatura. Son dispositivos que detectan cambios en las condiciones físicas y convierten los datos en
señales eléctricas que pueden ser leídas y monitorizadas. En el contexto de los sistemas de ósmosis inversa,
estas sondas son esenciales para monitorear y mantener las condiciones óptimas de temperatura que permiten la
eficacia del proceso, un ejemplo de ello representado en la figura \ref{fig:sensor_temperatura} es de modelo SPT-6702UAC y fabricante Endress+Hauser .

El modelo es un ejemplo de una sonda de temperatura de alta calidad. Funciona bajo
la clase A, que se refiere a su alta precisión y consistencia en la medición de la temperatura.
Este tipo de sondas son generalmente más precisas y estables que las sondas de clase B, lo que
las hace ideales para aplicaciones industriales que requieren mediciones precisas y repetibles. El principio de funcionamiento del
sensor se basa en la variación de la resistencia eléctrica del platino con la temperatura.

Estas sondas están ubicadas en puntos clave del proceso de ósmosis inversa, como en la tubería antes de la
entrada de cada etapa de la ósmosis. Aquí, las sondas pueden monitorear continuamente la temperatura del agua,
proporcionando datos vitales que pueden ayudar a prevenir problemas y garantizar que el sistema funcione de manera óptima.

\insertimageboxed[\label{fig:sensor_temperatura}]{instrumentacion/sensor_temperatura}{scale=0.8}{0}{Sensor de temperatura TSPT-6702UAC}

A continuación, se proporcionan algunas de las características específicas de este sensor en la tabla \ref{table:sensor_temperatura}.

\begin{mytable}{6cm}{Características del sensor de temperatura TSPT-6702UAC .}{table:sensor_temperatura}
        \hline
        \textbf{Modelo}                 & TSPT-6702UAC         \\
        \hline
        \textbf{Tipo}                   & Clase A              \\
        \hline
        \textbf{Precisión típica}       & +/- 0.15°C a 0°C     \\
        \hline
        \textbf{Valor de Alfa}          & 0.00385 °C$^{-1}$    \\
        \hline
        \textbf{Valor de resistencia}   & 100 ohm al 0°C       \\
        \hline
        \textbf{Rango de medición}      & 0°C a 200°C          \\
        \hline
        \textbf{Longitud de los cables} & 102 mm               \\
        \hline
        \textbf{Conexiones}             & ø 6 mm               \\
        \hline
        \textbf{Fabricante}             & Endress+Hauser       \\
        \hline
\end{mytable}

% ------------Sensores de Redox-----------
\subsection{Sensor de Redox} \label{sec:sensor_redox}

Los electrodos de Redox son dispositivos que se utilizan para medir el potencial de óxido-reducción (Redox) en una
solución. Su principal utilidad en los procesos industriales, incluido el tratamiento de agua por ósmosis inversa,
es proporcionar información en tiempo real sobre el estado de la solución, lo que permite ajustar los parámetros
del proceso en consecuencia.

El modelo CPS12D-7PA21 MEMOSENS mostrado en la figura \ref{fig:sensor_redox} de Endress+Hauser es un electrodo de Redox del tipo Orbisint. Estos electrodos
funcionan generando una diferencia de potencial eléctrico entre el electrodo y la solución a medida que se establece
un equilibrio electroquímico. Esta diferencia de potencial es proporcional al potencial Redox de la solución y puede
ser interpretada por un transmisor de Redox.

En el sistema de ósmosis inversa en estudio, este electrodo
se sitúa justo después del filtro de 10 micras. Desde aquí, puede enviar sus mediciones a un transmisor de
Redox para su interpretación y uso en el control del proceso.

\insertimageboxed[\label{fig:sensor_redox}]{instrumentacion/sensor_redox}{scale=0.8}{0}{Sensor de Redox CPS12D-7PA21}


Aquí se detallan las características específicas del electrodo de Redox CPS12D-7PA21 MEMOSENS:\\

\begin{mytable}{6cm}{Características del sensor de Redox CPS12D-7PA21.}{table:sensor_redox}

        \hline
        \textbf{Modelo}                        & CPS12D-7PA21 MEMOSENS             \\
        \hline
        \textbf{Tipo}                          & Orbisint                          \\
        \hline
        \textbf{Material}                      & Vidrio inastillable               \\
        \hline
        \textbf{Rango}                         & +/- 1500 mV                       \\
        \hline
        \textbf{Rango de temperatura}          & -15 a 80°C                        \\
        \hline
        \textbf{Longitud del electrodo}        & 120 mm                            \\
        \hline
        \textbf{Conector}                      & Standard con acoplamiento coaxial \\
        \hline
        \textbf{Temperatura máxima del fluido} & 20°C                              \\
        \hline
        \textbf{Fabricante}                    & Endress+Hauser                    \\
        \hline

\end{mytable}


% ------------Sensores de flujo-----------
\subsection{Sensor-Transmisor de flujo} \label{sec:sensor_flujo}

En cualquier proceso industrial, la medición precisa y la transmisión de los datos de flujo son esenciales para
garantizar la eficiencia y el correcto funcionamiento del sistema. En particular, los instrumentos que combinan
ambas funciones, conocidos como medidores de flujo y transmisores, son especialmente valiosos en la industria de
tratamiento de agua, como en los sistemas de ósmosis inversa. Proporcionan mediciones exactas de la tasa de flujo de
líquidos en distintos puntos del proceso y transmiten estos datos en tiempo real para su monitorización y control.

El modelo RAMC05-S4-SS-64S2- E90424*P6/Z de YOKOGAWA de la figura \ref{fig:sensor_transmisor_flujo} es un ejemplo perfecto de un medidor de flujo y transmisor en uno.
Este dispositivo funciona como un rotámetro, y su diseño permite no solo medir el flujo de líquidos sino también transmitir
estos datos para su monitorización remota o automatizada. Su ubicación en la tubería de permeado en la segunda etapa de la
ósmosis es estratégica, ya que permite un control constante y preciso del flujo de permeado en este punto crucial del proceso.

Por otro lado, el modelo DS20 07 YJ de MADDALENA de la figura \ref{fig:sensor_transmisor_flujo2} es otro medidor de flujo y transmisor efectivo,
es un medidor de flujo de dial húmedo de chorro múltiple.
Este medidor se encuentra después del
filtro de 10 micras, proporcionando mediciones de flujo esenciales después de esta etapa de filtración.



Estos son los detalles específicos de ambos medidores de flujo y transmisores:\\

\insertimageboxed[\label{fig:sensor_transmisor_flujo}]{instrumentacion/sensor_transmisor_flujo}{scale=0.4}{0}{Sensor-Transmisor de flujo RAMC05-S4-SS-64S2- E90424}


\begin{mytable}{6cm}{Características del rotámetro RAMC05-S4-SS-64S2- E90424.}{table:sensor_transmisor_flujo}
        \hline
        \textbf{Modelo}                 & RAMC05-S4-SS-64S2- E90424*P6/Z \\
        \hline
        \textbf{Tipo}                   & Rotámetro                      \\
        \hline
        \textbf{Material}               & 316 L                          \\
        \hline
        \textbf{Conexiones}             & 2" Triclamp                    \\
        \hline
        \textbf{Rango}                  & 400 a 4000 l/h                 \\
        \hline
        \textbf{Material de la carcasa} & Acero inoxidable               \\
        \hline
        \textbf{Opción}                 & 4-20 mA - 24Vdc                \\
        \hline
        \textbf{Fabricante}             & YOKOGAWA                       \\
        \hline

\end{mytable}

\insertimageboxed[\label{fig:sensor_transmisor_flujo2}]{instrumentacion/sensor_transmisor_flujo2}{scale=0.8}{0}{Sensor-Transmisor de flujo DS20 07 YJ}


\begin{mytable}{6cm}{Características del medidor de flujo DS20 07 YJ.}{table:sensor_transmisor_flujo2}

        \hline
        \textbf{Modelo}         & DS20 07 YJ                                     \\
        \hline
        \textbf{Tipo}           & Multi-jet wet dial (Multi-jet con dial húmedo) \\
        \hline
        \textbf{Material}       & Latón recubierto de epoxi                      \\
        \hline
        \textbf{Conexiones}     & Roscado ø 1"½ gas                              \\
        \hline
        \textbf{Rango}          & 10 m³/h (caudal nominal)                       \\
        \hline
        \textbf{Fabricante}     & MADDALENA                                      \\
        \hline
\end{mytable}



\subsection{Sensor de Nivel}

El control del nivel de agua en el proceso de ósmosis inversa es una variable clave, específicamente en el tanque TK50 donde se almacena el agua pretratada.
Para esta tarea esencial, se utiliza el sensor de nivel Liquicap FMI51, un dispositivo de medición de nivel por capacitancia desarrollado
por Endress+Hauser, tal como se muestra en la Figura \ref{fig:sensor_nivel}. Este instrumento asegura que el proceso de ósmosis inversa se
inicie solo cuando el nivel de agua en el tanque alcanza un punto establecido, lo que contribuye a optimizar la eficiencia del proceso.

El Liquicap FMI51 opera bajo el principio de la capacitancia. En el interior de su varilla sensora, este instrumento cuenta con dos
electrodos que generan un campo eléctrico. Cuando el nivel del agua en el tanque varía, las propiedades dieléctricas del espacio entre
los electrodos cambian, lo cual se traduce en una variación de la capacitancia. Esta variación es interpretada por el sensor y convertida
en una señal de nivel que es utilizada para controlar el proceso.

\insertimageboxed[\label{fig:sensor_nivel}]{instrumentacion/sensor_nivel}{scale=0.4}{0}{Sensor de nivel Liquicap FMI51}

En la tabla \ref{table:sensor_nivel} se muestran los datos técnicos de este sensor.
\begin{mytable}{6cm}{Características del Sensor de nivel Liquicap FMI51}{table:sensor_nivel}
        \hline
        \textbf{Fabricante}                       & Endress+Hauser                                                                                      \\
        \hline
        \textbf{Principio de medición}            & Capacitivo                                                                                          \\
        \hline
        \textbf{Rango de temperatura del proceso} & -80°C a +200°C                                                                                      \\
        \hline
        \textbf{Presión del proceso}              & Vacío a 100 bar                                                                                     \\
        \hline
        \textbf{Precisión}                        & Repetibilidad: 0,1\%                        \\
        \hline
        \textbf{Longitud total del sensor}        & 6m                                                                                                  \\
        \hline
        \textbf{Distancia máxima de medición}     & 0.1 a 4.0 m                                                                                         \\
        \hline
        \textbf{Comunicación}                     & 4...20mA HART, PFM                                                                                  \\
        \hline
        \textbf{Certificaciones / Aprobaciones}   & ATEX, FM, CSA, IEC Ex, TIIS, INMETRO, NEPSI, EAC, SIL                                               \\
        \hline
        \textbf{Limitaciones de aplicación}       & Espacio insuficiente hacia el techo, medios cambiantes no conductivos con conductividad < 100 μS/cm \\
        \hline
\end{mytable}



% ------------Sensores de Presión-----------
\subsection{Sensor-Transmisor de Presión} \label{sec:sensor_presion}

Los sensores de presión desempeñan un papel esencial en numerosos procesos industriales,
incluyendo la ósmosis inversa. Estos instrumentos son responsables de medir la presión en diferentes puntos
del sistema y transmitir esa información a un sistema de control para su seguimiento y análisis. El empleando
en nuestro sistema es el PTP31-A1C13S1AF1A fabricado por Endress+Hauser, reflejado en al fig \ref{fig:sensor_transmisor_presion}.

El principio de funcionamiento de estos dispositivos se basa en la aplicación de presión a un diafragma de
metal sensible, que causa su deformación. Esta deformación es detectada por un sensor, que la convierte en
una señal eléctrica. En el caso de los transmisores de presión, esta señal se transmite luego a un sistema de control en forma
de una señal estandarizada (generalmente 4-20 mA), lo que permite un fácil seguimiento y control de la presión en el proceso.

En el sistema de ósmosis inversa estudiado, estos sensores se encuentran
ubicados en la tubería de concentrado en cada etapa de la ósmosis, así como a la entrada de cada etapa
de la ósmosis. Esta disposición permite el monitoreo constante y preciso de la presión, lo que es
vital para la operación eficiente y segura del sistema.

\insertimageboxed[\label{fig:sensor_transmisor_presion}]{instrumentacion/sensor_transmisor_presion}{scale=0.6}{0}{Sensor-Transmisor de presión PTP31-A1C13S1AF1A}


\begin{mytable}{6cm}{Características del sensor de presión PTP31-A1C13S1AF1A. }{table:sensor_transmisor_presion}
        \hline
        \textbf{Modelo}                 & PTP31-A1C13S1AF1A                  \\
        \hline
        \textbf{Rango}                  & 0 a 40 bar (calibración 0-20 bar)  \\
        \hline
        \textbf{Pantalla}               & LCD                                \\
        \hline
        \textbf{Alimentación eléctrica} & 12 a 30 Vdc                        \\
        \hline
        \textbf{Salida}                 & Interruptor PNP, 3 hilos + 4-20 mA \\
        \hline
        \textbf{Conexión eléctrica}     & Conector M12 x 1.5                 \\
        \hline
        \textbf{Protección IP}          & IP 65                              \\
        \hline
        \textbf{Diafragma}              & AISI 316 L                         \\
        \hline
        \textbf{Fluido de llenado}      & Aceite de grado alimenticio        \\
        \hline
        \textbf{Conexión del proceso}   & Roscado G½'' ISO228 macho           \\
        \hline
        \textbf{Fabricante}             & Endress+Hauser                     \\
        \hline
\end{mytable}


\subsection{Manómetro} \label{sec:indicador_manometro}

Los manómetros son instrumentos de medición de presión esenciales en cualquier proceso industrial, incluyendo el tratamiento de agua por ósmosis inversa. Proporcionan una medida de la presión existente en un punto específico del proceso, permitiendo ajustar y controlar parámetros críticos para garantizar la eficacia del sistema.

Los manómetros de tipo seco, como el modelo P600 de ITEC de la figura \ref{fig:manometro}, funcionan basándose en la flexión de un tubo Bourdon (un tubo delgado y hueco que se curva en forma de C) por la presión del fluido. Al aumentar la presión, el tubo se endereza y este movimiento se traduce a una aguja en la esfera del manómetro para proporcionar una lectura de presión. Su diseño resistente y su facilidad de lectura los hacen idóneos para una amplia gama de aplicaciones industriales.

En el sistema de ósmosis inversa en estudio, los manómetros de tipo P600 se sitúan en puntos estratégicos: en cada filtro (de 10 micras y de 5 micras) y antes de la bomba que impulsa el agua a la segunda etapa de la ósmosis. La correcta monitorización de la presión en estas ubicaciones es vital para garantizar el adecuado funcionamiento del sistema y prevenir posibles problemas, como la sobrepresión que podría dañar las membranas de ósmosis.


\insertimageboxed[\label{fig:manometro}]{instrumentacion/manometro}{scale=0.5}{0}{Manómetro P600}


\begin{mytable}{6cm}{Características del manómetro P600.}{table:manometro}
        \hline
        \textbf{Modelo}                  & P600                        \\
        \hline
        \textbf{Tipo}                    & Ejecución seca              \\
        \hline
        \textbf{Material}                & Acero inoxidable            \\
        \hline
        \textbf{Rango de presión}        & 0 a 10 bar                  \\
        \hline
        \textbf{Diámetro de la carcasa}  & 63 mm o 200 mm              \\
        \hline
        \textbf{Temperatura del proceso} & 20°C                        \\
        \hline
        \textbf{Conexión del proceso}    & Roscado ¼" gas radial o ½'' \\
        \hline
        \textbf{Fabricante}              & ITEC                        \\
        \hline
\end{mytable}


\subsection{Indicador de Flujo} \label{sec:indicador_flujo}

Los indicadores de flujo son instrumentos indispensables en cualquier proceso industrial,
incluyendo el tratamiento de agua por ósmosis inversa. Estos dispositivos permiten medir
y controlar la cantidad de líquido que fluye por una tubería, proporcionando datos cruciales
para el funcionamiento correcto y eficiente del sistema.

Los indicadores de flujo de tipo rotámetro y de área variable son particularmente comunes
en la industria. Los rotámetros, como el modelo RAMC02-S4-SS-61S1-T90NNNZ de Yokogawa de la figura \ref{fig:indicador_flujo},
funcionan basándose en la elevación de un flotador en un tubo cónico debido al flujo del
fluido.

En el sistema de ósmosis inversa en estudio, estos indicadores de flujo se encuentran en
ubicaciones clave: como por ejemplo en la tubería de concentrado
de la segunda etapa de la ósmosis,así como en la tubería de permeado de la primera etapa de la ósmosis. Monitorear
el flujo en estas ubicaciones es esencial para garantizar la eficiencia y seguridad del
proceso.


\insertimageboxed[\label{fig:indicador_flujo}]{instrumentacion/indicador_flujo}{scale=1.1}{0}{Indicador de flujo RAMC05-S4-SS-64V2-T90}

A continuación, se presentan una la tabla \ref{table:indicador_flujo} con las características específicas de este indicador:\\

\begin{mytable}{6cm}{Características del dispositivo RAMC02-S4-SS-61S1-T90NNN*Z.}{table:indicador_flujo}
        \hline
        \textbf{Modelo}                 & RAMC02-S4-SS-61S1-T90NNN*Z \\
        \hline
        \textbf{Tipo}                   & Rotámetro                  \\
        \hline
        \textbf{Material}               & 316 L                      \\
        \hline
        \textbf{Acabado}                & Decapado y pasivado        \\
        \hline
        \textbf{Conexiones}             & 1" clamp                   \\
        \hline
        \textbf{Rango}                  & 100 a 1000 lt/h            \\
        \hline
        \textbf{Material de la carcasa} & Acero inoxidable           \\
        \hline
        \textbf{Fabricante}             & Yokogawa                   \\
        \hline
\end{mytable}


% 
\subsection{Transmisores}

En la arquitectura de este sistema de ósmosis inversa, los transmisores son piezas esenciales que funcionan como vínculos de
comunicación entre los sensores o analizadores y el PLC (Controlador Lógico Programable). 
Su función principal es transformar las señales eléctricas recibidas de los sensores en una forma que el 
PLC pueda interpretar y utilizar para el control y monitorización del proceso. \\

Más allá de esta función de conversión de señales, los transmisores también cuentan con sistemas de alarmas e indicadores 
integrados. Estos sistemas permiten detectar y alertar sobre cualquier desviación o
anomalía en los parámetros medidos, permitiendo una respuesta rápida para mantener la eficiencia y seguridad del proceso. \\

\subsubsection{Transmisores de Conductividad } \label{sec:transmisor_conductividad}

Los transmisores de conductividad son elementos vitales en diversos procesos industriales, incluyendo la ósmosis inversa. Aunque estos dispositivos no miden directamente la conductividad, desempeñan un papel esencial en la interpretación y transmisión de las mediciones de conductividad realizadas por un sensor.\\

El principio de funcionamiento de estos dispositivos es bastante sencillo pero fundamental. Un sensor de conductividad mide la capacidad de un medio (en este caso, el agua) para conducir la corriente eléctrica. Esta señal eléctrica es luego enviada al transmisor, que la convierte en una señal normalizada, típicamente de 4-20 mA, que puede ser fácilmente interpretada por otros dispositivos o sistemas de control.\\

El modelo CLM223-CD8110 de Endress+Hauser, es un equipo que, además de convertir y transmitir la señal de conductividad, también presenta una funcionalidad de alarma para valores altos de conductividad o errores del sistema. Además, este transmisor puede proporcionar una señal de temperatura, lo que aumenta su utilidad en el control de procesos.\\

En el sistema de ósmosis inversa en estudio, el transmisor de conductividad CLM223-CD8110 se sitúa a continuación de los sensores de conductividad, que se encuentran en el permeado a la salida de cada etapa de la ósmosis. De este modo, el transmisor juega un papel esencial en la monitorización y control de la pureza del agua.\\

Aquí se detallan las características específicas del transmisor de conductividad CLM223-CD8110:\\

\insertimageboxed[\label{fig:transmisor_conductividad}]{instrumentacion/transmisor_conductividad}{scale=1.1}{0}{Características del transmisor CLM223-CD8110}


\begin{table}[H]
    \centering
    \caption{Características del transmisor CLM223-CD8110.}
    \label{table:transmisor_conductividad}
    \begin{tabular}{| L{5cm} | L{5cm} |}
        
        \hline
        \textbf{Modelo} & CLM223-CD8110  \\
        \hline
        \textbf{Rango} & 0 a 20µS/cm  \\
        \hline
        \textbf{Salida} & 2 x 4 a 20 mA (conductividad y temperatura)  \\
        \hline
        \textbf{Alarmas} & 2 relés x alta conductividad + error del sistema  \\
        \hline
        \textbf{Voltaje} & 24 V ac/dc  \\
        \hline
        \textbf{Opciones} & transmisor de temperatura  \\
        \hline
        \textbf{Fabricante} & Endress+Hauser  \\
        \hline
    \end{tabular}
\end{table}



\subsubsection{Transmisores de pH y REDOX } \label{sec:transmisor_ph_redox}

Los transmisores de pH y REDOX son instrumentos clave en el análisis y control de procesos químicos e industriales, especialmente en sistemas como el tratamiento de agua por ósmosis inversa. Proporcionan mediciones precisas y fiables de dos parámetros críticos: el pH, que es una medida de la acidez o alcalinidad de una solución, y el potencial de reducción-oxidación (REDOX), que indica la capacidad de una solución para ganar o perder electrones.\\

En el sistema de ósmosis inversa en estudio, el modelo CPM 223-MR8010 Memosens de Endress+Hauser se utiliza para transmitir datos de pH y REDOX desde las ubicaciones de medición hasta los sistemas de control o monitorización. Este dispositivo se encuentra después del analizador de Redox, que está ubicado después del filtro de 10 micras, y después de los sensores de pH, que están ubicados luego del filtro de 5 micras. Esta ubicación es esencial para controlar y ajustar el proceso de ósmosis inversa en función de las condiciones del agua.\\

Las características específicas del modelo CPM 223-MR8010 Memosens de Endress+Hauser son las siguientes:\\

\insertimageboxed[\label{fig:transmisor_pH}]{instrumentacion/transmisor_pH}{scale=1.1}{0}{Características del transmisor de pHy Redox CPM 223-MR8010}


\begin{table}[H]
    \centering
    \caption{Características del Transmisor de pH y REDOX.}
    \label{table:transmisor_pH}
    \begin{tabular}{| L{5cm} | L{5cm} |}
        
        \hline
        \textbf{Modelo} & CPM 223-MR8010 Memosens  \\
        \hline
      
        \textbf{Diseño} & Transmisor de pH/ORP en caja de panel (96x96 mm)  \\
        \hline
        \textbf{Alarmas} & 2 relay + error de sistema  \\
        \hline
        \textbf{Dimensiones} & 96 mm x 96 mm x 146 mm (profundidad de montaje)  \\
        \hline
        \textbf{Protección de ingreso} & IP65  \\
        \hline
        \textbf{Entrada} & Transmisor de un solo canal  \\
        \hline
        \textbf{Salida / comunicación} & 0/4-20 mA, Hart, Profibus  \\
        \hline
       
        \textbf{Fabricante} & Endress+Hauser  \\
        \hline
    \end{tabular}
\end{table}

\subsection{Equipos de control}

En el vasto y complejo universo de la ingeniería de procesos, los equipos de 
control son los actores silenciosos que juegan un papel crucial en el funcionamiento eficiente y 
efectivo de cualquier sistema de tratamiento. Desde mantener condiciones óptimas hasta permitir ajustes precisos y 
oportunos, estos equipos son la columna vertebral de cualquier proceso industrial, incluyendo el tratamiento de agua 
mediante ósmosis inversa.\\

En esta sección, centraremos nuestro análisis en los distintos equipos de control presentes en nuestro subsistema.
 Examinaremos detenidamente equipos como las bombas y válvulas que conforman 
la instrumentación de este sistema, estudiando su funcionamiento, características y 
ubicación en el proceso. Al hacerlo, esperamos proporcionar una visión clara y completa de la instrumentación 
actual del sistema y destacar su importancia en el mantenimiento de un proceso de ósmosis inversa seguro y eficaz.\\

\subsubsection{Bombas de Alta Presión}

Las bombas de alta presión son elementos fundamentales en el sistema de ósmosis inversa. Son responsables de aplicar la presión necesaria para que se produzca la ósmosis, un aspecto crucial para el adecuado funcionamiento del sistema.\\

En el proceso que estamos analizando, se utilizan bombas centrífugas verticales de múltiples etapas, específicamente del modelo CRN10-7 de la marca GRUNDFOS. Estas bombas son conocidas por su eficiencia y durabilidad, lo que las hace ideales para aplicaciones de alta presión como la ósmosis inversa.\\

El funcionamiento de las bombas centrífugas se basa en la conversión de la energía cinética en energía de presión. El agua entra en la bomba y es impulsada por un impulsor que gira a alta velocidad. Cuando el agua sale del impulsor, su energía cinética se transforma en energía de presión a medida que su velocidad disminuye en la voluta o carcasa de la bomba.\\

Estas bombas están ubicadas antes de cada etapa de la ósmosis, donde su tarea es generar la presión necesaria para forzar el paso del agua a través de la membrana semi-permeable del sistema de ósmosis inversa.\\

A continuación, se presenta una tabla con las características técnicas más relevantes de las bombas de alta presión CRN10-7:\\

\insertimageboxed[\label{fig:bomba_centrifuga}]{instrumentacion/bomba_centrifuga}{scale=1.1}{0}{Bombas centrífuga CRN10-7}


\begin{table}[H]
    \centering
    \caption{Características de la bomba centrífuga vertical multietapa CRN10-7.}
    \label{table:bomba_centrifuga}
    \begin{tabular}{| L{6cm} | L{6cm} |}
        \hline
        \textbf{Modelo} & CRN10-7  \\
        \hline
        \textbf{Tipo} & Centrífuga vertical multietapa  \\
        \hline
        \textbf{Material} & AISI 316  \\
        \hline
        \textbf{Sello} & HUUE (Carburo de Tungsteno / EPDM)  \\
        \hline
        \textbf{Medio} & Agua ablandada  \\
        \hline
        \textbf{Temperatura de trabajo} & 20°C  \\
        \hline
        \textbf{Caudal} & 8000 lt/h  \\
        \hline
        \textbf{Presión de descarga} & 10 bar  \\
        \hline
        \textbf{Diámetro del impulsor} & n.a  \\
        \hline
        \textbf{Puerto de entrada} & 2" Tri-Clamp  \\
        \hline
        \textbf{Puerto de salida} & 2" Tri-Clamp  \\
        \hline
        \textbf{Suministro eléctrico} & 3 x 380V 60 Hz  \\
        \hline
        \textbf{Potencia} & 5,5 kW  \\
        \hline
        \textbf{Amperios} & 10,8  \\
        \hline
        \textbf{RPM} & 3600  \\
        \hline
        \textbf{Opciones} & Base de acero inoxidable  \\
        \hline
        \textbf{Fabricante} & GRUNDFOS  \\
        \hline
    \end{tabular}
\end{table}



\subsubsection{Bombas Dosificadoras}

Las bombas dosificadoras son las encargadas de administrar con precisión pequeñas 
 cantidades de químicos para alterar las características del agua. Estos químicos incluyen 
 agentes como el NaOH y Na2S2O5, que respectivamente alteran el pH y reducen el oxígeno disuelto en el agua.\\

En nuestro sistema, se utilizan dos bombas dosificadoras específicas de la marca PROMINENT: 
los modelos GALA G/L G1005 NPB 200UA 103000 figura \ref{fig:bomba_dosificadora} y G/L 1601 NPB 220UA 103 000 figura \ref{fig:bomba_dosificadora2} . Ambos modelos 
son reconocidos por su precisión y fiabilidad, y utilizan la tecnología de diafragma 
solenoide para garantizar una dosificación exacta.\\

El principio de funcionamiento de estas bombas se basa en la acción de un solenoide que atrae y repele un diafragma, creando un movimiento oscilante. Este movimiento provoca la succión del medio (el químico a dosificar) durante la fase de retracción del diafragma y su posterior expulsión durante la fase de compresión.\\

La bomba GALA G/L G1005 NPB 200UA 103000 se encuentra en el sistema de dosificación bomba-tanque de NaOH, mientras que la bomba G/L 1601 NPB 220UA 103 000 se utiliza en el sistema de dosificación bomba-tanque de Na2S2O5.\\

A continuación, se presentan las características técnicas de cada una de estas bombas dosificadoras:\\

\insertimageboxed[\label{fig:bomba_dosificadora}]{instrumentacion/bomba_dosificadora}{scale=0.8}{0}{Bomba dosificadora G1005}


\begin{table}[H]
    \centering
    \caption{Características de la bomba dosificadora G1005.}
    \label{table:bomba_dosificadora}
    \begin{tabular}{| L{6cm} | L{6cm} |}
        \hline
        \textbf{Modelo} & GALA G/L G1005 NPB 200UA 103000  \\
        \hline
        \textbf{Material} & Plexiglás \\
        \hline
        \textbf{Caudal} & 4,4 lt @ 10 bar \\
        \hline
        \textbf{Voltaje} & 100-230 V / 50-60 Hz \\
        \hline
        \textbf{Protección IP} & 65 \\
        \hline
        \textbf{Potencia} & 12W \\
        \hline
        \textbf{Fabricante} & PROMINENT \\
        \hline
    \end{tabular}
\end{table}

\insertimageboxed[\label{fig:bomba_dosificadora2}]{instrumentacion/bomba_dosificadora}{scale=0.8}{0}{Bomba dosificadora G/L 1601}


\begin{table}[H]
    \centering
    \caption{Características de la bomba dosificadora G/L 1601.}
    \label{table:bomba_dosificadora2}
    \begin{tabular}{| L{6cm} | L{6cm} |}
        \hline
        \textbf{Modelo} & G/L 1601 NPB 220UA  \\
        \hline
        \textbf{Tipo} & Diafragma de solenoide \\
        \hline
        \textbf{Medio} & Solución acuosa de Na2S2O5 \\
        \hline
        \textbf{Materiales} & Cabeza de dosificación: Acrílico, elemento de succión / presión: PVC, sellos: FPM-B, bolas: cerámica \\
        \hline
        \textbf{Caudal} & 1,1 lt/h \\
        \hline
        \textbf{Presión de descarga} & 16 bar \\
        \hline
        \textbf{Suministro eléctrico} & 100-230 V / 50-60 Hz \\
        \hline
        \textbf{Potencia} & 12 W \\
        \hline
        \textbf{Fabricante} & PROMINENT \\
        \hline
    \end{tabular}
\end{table}


\subsubsection{Válvulas de Retención} \label{sec:valvula_retencion}

Las válvulas de retención o check valves son elementos clave en cualquier sistema de tratamiento de agua o proceso industrial, ya que garantizan la unidireccionalidad del flujo en las tuberías. Su papel es esencial para mantener la seguridad y la eficiencia del sistema, ya que evitan el flujo inverso que podría causar daños en los equipos o interrumpir el proceso.\\

El papel de las válvulas de retención en nuestro sistema de ósmosis inversa es multifacético. Están ubicadas en varios puntos estratégicos a lo largo del proceso, incluyendo, pero no limitándose a, justo después de las bombas de alta presión, donde evitan que el fluido regrese a la bomba en caso de una parada o apagado. También se utilizan en la línea de dosificación de químicos, para asegurar un suministro constante y seguro de los reactivos necesarios para el proceso. Sin embargo, es importante destacar que pueden encontrarse en otros puntos del sistema donde sea necesario evitar el retroceso del flujo.\\

El principio de funcionamiento de las válvulas de retención es relativamente sencillo. Contienen un componente que se mueve libremente y permite el flujo en una dirección, pero bloquea el flujo si intenta moverse en la dirección contraria.\\

Para nuestro sistema, empleamos el modelo de válvula de retención Art. 048 VRTCV2 de RATTI. Este modelo está construido con un cuerpo de acero inoxidable AISI 316L, lo que garantiza su resistencia a la corrosión, y tiene una junta de PTFE.\\

Válvula de Retención 048 VRTCV2\\

\begin{table}[H]
    \centering
    \caption{Características del cuerpo.}
    \label{table:cuerpo}
    \begin{tabular}{| L{6cm} | L{6cm} |}
        \hline
        \textbf{Material del cuerpo} & AISI 316L \\
        \hline
        \textbf{Junta} & PTFE \\
        \hline
        \textbf{Diámetro} & 1½" \\
        \hline
        \textbf{Conexiones} & Abrazadera (clamp) \\
        \hline
        \textbf{Resorte} & Estándar \\
        \hline
        \textbf{Fabricante} & RATTI \\
        \hline
    \end{tabular}
\end{table}


\subsubsection{Válvulas Multiusos} \label{sec:valvula_multi}

Las válvulas multiusos son componentes esenciales en cualquier sistema de tratamiento de agua. Actúan como puntos de control, permitiendo o impidiendo el paso de fluidos a través de las tuberías. La capacidad de controlar el flujo de agua y otros líquidos es crucial para el funcionamiento seguro y eficiente de todo el sistema. Son llamadas "multiusos" porque se utilizan en una variedad de aplicaciones dentro del sistema, dependiendo de las necesidades del proceso en particular.\\

En nuestro sistema de ósmosis inversa, las válvulas multiusos se encuentran en varios puntos críticos. Una ubicación importante es en las tuberías de concentrado de la ósmosis, en la línea que va al drenaje o que retorna al tanque de almacenamiento de agua de pretratamiento. Además, se utilizan en la línea de bypass que se encuentra después de la bomba de lavado químico. Estas ubicaciones no son exhaustivas, y es posible encontrar estas válvulas en otros puntos del sistema donde se requiera controlar el flujo de fluido.\\

El principio de funcionamiento de las válvulas multiusos es simple pero efectivo. Cuando la válvula está abierta, permite el flujo de fluido; cuando está cerrada, detiene el flujo.\\

Utilizamos el modelo J4M1G00 de RATTI para nuestras válvulas multiusos. Esta válvula está fabricada con acero inoxidable AISI 316L para una resistencia óptima a la corrosión y durabilidad a largo plazo. Tiene un diámetro de 1" - ¾" y se conecta mediante una conexión Tri-Clamp.\\

\begin{table}[H]
    \centering
    \caption{Características del cuerpo.}
    \label{table:valvula_multiusos}
    \begin{tabular}{| L{6cm} | L{6cm} |}
        \hline
        \textbf{Material del cuerpo} & AISI 316 L \\
        \hline
        \textbf{Diámetro} & 1" - ¾" \\
        \hline
        \textbf{Conexiones} & Tri-Clamp \\
        \hline
        \textbf{Fabricante} & OMAL \\
        \hline
    \end{tabular}
\end{table}

\subsubsection{Válvulas de Control ON/OFF} \label{sec:valvula_OnOff}

En cualquier proceso industrial, las válvulas de control ON/OFF son elementos críticos para la gestión y regulación del flujo de fluidos. Su importancia radica en su habilidad para controlar de manera precisa y eficiente el flujo a través de las tuberías, permitiendo un total paso del fluido o su completa interrupción, según las demandas del sistema.\\

Un ejemplo específico de este tipo de válvulas es el modelo S386FPLY004Y05 de la reconocida empresa OMAL. Esta válvula cuenta con un cuerpo de hierro fundido revestido con EPOXY y EPDM, lo que la hace resistente y duradera. La característica más notable de esta válvula es su actuador neumático de retorno por resorte que, junto con su conjunto de accesorios, garantiza un funcionamiento fiable y una integración eficiente con el sistema de control del proceso.\\

La planta de ósmosis inversa que analizamos está equipada con numerosas válvulas de control ON/OFF, distribuidas estratégicamente en diferentes puntos del sistema. Son componentes indispensables que aseguran la correcta operación del proceso, y debido a su importancia, están presentes en gran cantidad en todas las áreas de la planta. Algunos lugares estratégicos donde podemos encontrar este tipo de válvulas puede ser la línea que precede a la bomba de alta presión y en la línea de concentrado de la primera etapa de ósmosis que retorna al tanque de pretratamiento.\\

Válvula de Control ON/OFF S386FPLY004Y05\\

\begin{table}[H]
    \centering
    \caption{Características del cuerpo.}
    \label{table:valvula_on_off}
    \begin{tabular}{| L{6cm} | L{6cm} |}
        \hline
        \textbf{Material del cuerpo} & Hierro fundido con revestimiento de EPOXY \\
        \hline
        \textbf{Revestimiento} & EPDM \\
        \hline
        \textbf{Vástago y disco} & Acero inoxidable AISI 316 \\
        \hline
        \textbf{Estilo del cuerpo} & "LUG" roscado para brida EN1092-1 \\
        \hline
        \textbf{Tamaño} & DN40 \\
        \hline
        \textbf{Actuador} & Neumático de retorno por resorte N.O. tipo SR30 y tornillos de regulación \\
        \hline
        \textbf{Accesorios} & Indicador de posición del eje KI02PP10, regulador de flujo de aire comprimido KAPR00101, filtro de aire de bronce 9490S001 \\
        \hline
        \textbf{Fabricante} & OMAL \\
        \hline
    \end{tabular}
\end{table}

\subsubsection{Válvulas de Retención de Presión de Inyección}

Las válvulas de retención de presión de inyección juegan un papel importante en el sistema de dosificación, especialmente en procesos industriales que requieren una precisión en la dosificación de ciertos productos químicos. Estas válvulas mantienen una presión constante de salida, evitando fluctuaciones y garantizando una dosificación precisa y estable.\\

En nuestra planta de ósmosis inversa, estas válvulas son vitales en la dosificación precisa de sustancias químicas como NaOH y Na2S2O5. Son componentes esenciales para asegurar la eficacia de las operaciones de dosificación y se encuentran estratégicamente ubicadas en las líneas de dosificación correspondientes. Sin embargo, su presencia no se limita a estas áreas, y se podrían encontrar en otras partes del sistema donde se necesite una dosificación precisa.\\

El modelo MFV-DK de PROMINENT, una empresa reconocida por la fabricación de componentes de alta calidad, es una de las válvulas utilizadas en nuestro sistema. Con un cuerpo de PVDF y un diafragma de PTFE, esta válvula es capaz de mantener una presión de alivio de hasta 16 bar, lo que asegura su capacidad para trabajar bajo condiciones exigentes.\\

Válvula de Retención de Presión de Inyección MFV-DK\\

\begin{table}[H]
    \centering
    \caption{Características del tipo MFV-DK, PVDF.}
    \label{table:valvula_retencion}
    \begin{tabular}{| L{6cm} | L{6cm} |}
        \hline
        \textbf{Tipo} & MFV-DK, PVDF \\
        \hline
        \textbf{Tamaño} & 1 \\
        \hline
        \textbf{Presión de alivio} & 16 bar \\
        \hline
        \textbf{Conector} & 6-12 mm \\
        \hline
        \textbf{Conector de bypass} & 6/4 mm \\
        \hline
        \textbf{Materiales} & Cuerpo de PVDF, diafragma de PTFE, sello de FPM \\
        \hline
        \textbf{Fabricante} & PROMINENT \\
        \hline
    \end{tabular}
\end{table}



\section{Cominicación de la planta}

\section{Propuesta de instrumentación}

La ingeniería de procesos, y especialmente el tratamiento de agua mediante ósmosis inversa, requiere una cuidadosa selección de equipos y dispositivos de control, también conocidos como instrumentación. En esta sección, daremos un paso hacia adelante desde el análisis de la instrumentación actual, para abordar nuestra propuesta de mejoramiento: la implementación de un electrodesionizador (EDI) y la instrumentación requerida para su correcta operación.

La instrumentación adecuada es crucial para el buen funcionamiento de cualquier proceso industrial, ya que nos permite monitorizar y controlar de forma precisa las variables críticas de operación. En el caso de la ósmosis inversa y, más concretamente, del EDI, esta importancia se acentúa, dado que el rendimiento y la eficiencia del sistema dependen en gran medida de la capacidad de regular las condiciones de trabajo.

En las siguientes subsecciones, primero justificaremos por qué el EDI ha sido seleccionado como la mejor opción para mejorar el proceso de tratamiento de agua existente. Posteriormente, describiremos la instrumentación necesaria para implementar y operar de manera efectiva este dispositivo, siempre desde una perspectiva de control.


\subsection{Contexto de la Planta Actual}

La planta de tratamiento de agua existente, en funcionamiento durante años,
ha demostrado ser un recurso crucial en el suministro de agua pura (PW).
Sin embargo, recientemente, se han observado ciertas inestabilidades en el
proceso de tratamiento, lo que ha dificultado el logro constante de los parámetros
de calidad requeridos durante la producción de agua.

Es importante mencionar que estas inestabilidades no desacreditan mucho la eficacia
del sistema existente. Sin embargo, podrían ser indicativos de la necesidad
de mejoras o adaptaciones para hacer frente a los cambios en las condiciones
del agua de entrada o a los requisitos de calidad cada vez más exigentes.

Además, uno de los desafíos que ha surgido es la capacidad de la
planta para producir agua de la calidad necesaria para cumplir con la demanda.
La planta tiene la capacidad de producir una cantidad considerable de agua,
sin embargo, una porción de esta producción no alcanza los parámetros de
calidad requeridos. Esta agua de menor calidad debe ser desechada o
recirculada para un nuevo tratamiento, lo que resulta en un suministro
efectivo de agua de calidad inferior a la demanda.

\subsection{Tecnologías Alternativas y sus Limitaciones}

En la búsqueda de la mejora continua y optimización de la planta de tratamiento de agua,
es importante considerar las diversas tecnologías alternativas disponibles.
Sin embargo, cada tecnología tiene sus propias limitaciones, algunas de las
cuales pueden no adaptarse a las necesidades y condiciones específicas de nuestra planta.
Las siguientes son algunas de las tecnologías que se han evaluado:

\begin{itemize}
    \item \textbf{Reforzamiento de la Ósmosis Inversa (RO):}  Nuestra planta ya cuenta con
          un sistema de ósmosis inversa de dos etapas que cumple con las necesidades
          básicas de la planta. Sin embargo, incluso con un sistema RO de dos etapas,
          todavía existen limitaciones, especialmente en términos de la eliminación de
          ciertos iones y pequeñas moléculas. Los sistemas RO también son susceptibles a
          la acumulación de sarro y biofilm, lo que puede afectar el rendimiento y
          la vida útil de la membrana.

    \item \textbf{Destilación:}  Aunque la destilación puede ofrecer altos niveles de
          purificación, la energía requerida para este proceso es considerable,
          lo que resulta en costos operativos más altos. Además, la destilación
          no elimina eficientemente algunos contaminantes volátiles que pueden ser arrastrados con el vapor.

    \item \textbf{Desionización (DI): } Los sistemas de DI pueden ser eficientes para
          la eliminación de iones de agua, pero su capacidad para eliminar
          partículas no iónicas, gases disueltos y microorganismos es limitada.
          Además, los cartuchos de DI requieren un reemplazo frecuente, lo que
          implica costos adicionales de operación y mantenimiento.

    \item \textbf{Filtración de Carbón Activado:}  Esta tecnología es efectiva para la
          eliminación de cloro y ciertos otros contaminantes, pero su eficacia es
          limitada cuando se trata de la eliminación de sales disueltas y
          algunos contaminantes orgánicos.

\end{itemize}

Teniendo en cuenta las limitaciones y desafíos presentes en estas tecnologías alternativas, y
dadas las necesidades específicas de nuestra planta de tratamiento de agua, es evidente que se
necesita una solución más eficaz y sostenible. En este contexto, la Electrodesionización (EDI)
emerge como una solución potencialmente superior, debido a su capacidad para superar muchas de las
limitaciones de las tecnologías mencionadas anteriormente.

\subsection{La Electrodesionización}

La Electrodesionización (EDI) es una tecnología innovadora que combina dos procesos
fundamentales de purificación de agua: la desionización mediante intercambio iónico y
la electrodiálisis. La sinergia de estos métodos resulta en un sistema eficiente y
sustentable capaz de producir agua ultrapura de manera continua y sin la necesidad
de regenerar químicos.

En términos generales, la EDI utiliza una corriente eléctrica para mover los iones a
través de membranas semipermeables, eliminándolos del agua. Este proceso se realiza
en un entorno controlado en el que los iones son capturados por resinas de intercambio
iónico, siendo luego extraídos mediante la corriente eléctrica.

La elección de la EDI como una adición al sistema actual de Ósmosis Inversa de Doble
Etapa se justifica en gran medida por su capacidad para superar las limitaciones de
otras tecnologías y proporcionar un rendimiento superior en términos de calidad del
agua, eficiencia y sostenibilidad.

Cabe mencionar que este resumen brinda una visión general y concisa del funcionamiento y
las ventajas de la EDI. Sin embargo, para un entendimiento más profundo y detallado del
principio de funcionamiento de la EDI, sus componentes, así como los beneficios y
desafíos asociados, se remite al lector al Capítulo \ref{cap:fundamentosEDI} donde se proporciona un
análisis exhaustivo de esta tecnología.

\subsection{El Electrodesionizador}

El electrodesionizador seleccionado para la implementación en la planta de tratamiento de agua es el modelo LMX30-X-3 fabricado por Ionpure. Este equipo desempeña un papel crucial en el proceso de purificación del agua, ya que permite la eliminación de iones y moléculas no deseadas a través de un proceso de electrodesionización. A continuación se muestra la figura del Electrodesionizador seleccionado (Figura \ref{fig:edi_model}). Las especificaciones técnicas de este modelo se presentan en la Tabla \ref{table:edi_specs}.

\insertimageboxed[\label{fig:edi_model}]{instrumentacion/edi}{scale=0.8}{0}{Modelo LMX30-X-3 de Ionpure.}


La fuente de alimentación del Electrodesionizador, vital para su funcionamiento correcto, es el modelo PTM06 de STIL MAS. Esta fuente de alimentación proporciona la energía eléctrica necesaria para el funcionamiento del Electrodesionizador, permitiendo la ionización de las moléculas y facilitando su eliminación. La figura de la Fuente de alimentación seleccionada se muestra a continuación (Figura \ref{fig:edi_power}). Sus especificaciones se muestran en la Tabla \ref{table:power_supply_specs}.


\insertimageboxed[\label{fig:edi_power}]{instrumentacion/edi_power}{scale=0.8}{0}{Modelo PTM06 de STIL MAS.}


\begin{table}[H]
    \centering
    \caption{Especificaciones técnicas del Electrodesionizador LMX30-X-3 de Ionpure.}
    \label{table:edi_specs}
    \begin{tabular}{| L{6cm} | L{6cm} |}
        \hline
        \textbf{Fabricante}                                        & IONPURE                            \\
        \hline
        \textbf{Modelo}                                            & LMX30-X-3                          \\
        \hline
        \textbf{Tensión nominal}                                   & 0-600V DC                          \\
        \hline
        \textbf{Corriente nominal}                                 & 0-6 A                              \\
        \hline
        \textbf{Fuente de agua de alimentación}                    & Agua pretratada en ósmosis inversa \\
        \hline
        \textbf{Flujo de producto}                                 & 3300 l/h                           \\
        \hline
        \textbf{Flujo de concentrado}                              & 180 l/h                            \\
        \hline
        \textbf{Conexión de los flujos de alimentación y producto} & 1”                                 \\
        \hline
        \textbf{Conexión de los flujos rechazo y concentrado}      & ½”                                 \\
        \hline
        \textbf{Temperatura ambiente de operación}                 & ≤ 45°C                             \\
        \hline
    \end{tabular}
\end{table}

\begin{table}[H]
    \centering
    \caption{Especificaciones técnicas de la fuente de alimentación PTM06 de STIL MAS.}
    \label{table:power_supply_specs}
    \begin{tabular}{| L{6cm} | L{6cm} |}
        \hline
        \textbf{Fabricante}           & STIL MAS                                                     \\
        \hline
        \textbf{Modelo}               & PTM06                                                        \\
        \hline
        \textbf{Voltaje de entrada}   & 200-480 VAC (±5\%) - 50/60Hz                                 \\
        \hline
        \textbf{Corriente de entrada} & 1-20 A                                                       \\
        \hline
        \textbf{Voltaje de salida}    & 30-400 VDC                                                   \\
        \hline
        \textbf{Entradas de control}  & 2 x 4-20 mA + contactos de inicio/parada                     \\
        \hline
        \textbf{Salidas de control}   & 2 x 4-20 mA + contacto para establecer condiciones iniciales \\
        \hline
        \textbf{Potencia}             & 6KVA                                                         \\
        \hline
    \end{tabular}
\end{table}



\subsection{Válvulas y Sensores para el EDI}

La implementación del electrodesionizador (EDI) en la planta farmacéutica de AICA requiere una serie de válvulas y sensores para garantizar un control riguroso del proceso. Tras un detallado análisis de la instrumentación existente en la planta de tratamiento de agua, se decidió que los sensores y válvulas actuales cumplen a cabalidad con los requerimientos del sistema de EDI. Estos dispositivos han demostrado su eficacia en las operaciones de la planta y el personal tiene experiencia en su uso y mantenimiento. Por ello, no se consideró necesario incorporar nuevos modelos de sensores o válvulas en la implementación del EDI.

A continuación, se resumen los principales elementos de instrumentación que serán utilizados en el sistema de EDI, la explicación detallada de cada uno de estos se encuentra en capítulos anteriores:
\begin{itemize}
    \item \textbf{Sensores de Conductividad:} Como el sensor de conductividad presentado en la sección \ref{sec:sesor_conductividad}, estos dispositivos permiten monitorizar la calidad del agua de salida del EDI en tiempo real.

    \item \textbf{Sensores de Temperatura:} Los sensores de temperatura son necesarios para asegurar que el proceso se lleva a cabo en las condiciones de temperatura óptimas, ver sección \ref{sec:sensor_temp}.

    \item \textbf{Transmisores de Flujo y Presión:} Los transmisores de flujo y presión, como se describen en las secciones \ref{sec:sensor_flujo} y \ref{sec:sensor_presion}, permiten monitorizar y controlar el flujo de agua y las condiciones de presión dentro del sistema de EDI.

    \item \textbf{Indicadores de Flujo y Manómetros:} Los indicadores de flujo y los manómetros proporcionan una visualización inmediata de las condiciones del sistema, lo que facilita su operación y mantenimiento, ver secciones \ref{sec:indicador_flujo} y \ref{sec:indicador_manometro}.

    \item \textbf{Válvulas de Retención y Válvulas de operación:} Estas válvulas, referenciadas en las secciones \ref{sec:valvula_OnOff} , \ref{sec:valvula_retencion} y \ref{sec:valvula_multi}, son fundamentales para controlar el flujo de agua dentro del sistema de EDI.
\end{itemize}

En la siguiente sección, se presentará un esquema general de la configuración del EDI, en el que se identificarán las ubicaciones de las válvulas y sensores en el sistema.



\subsection{Esquema General de la Configuración del EDI}

El sistema de Electrodesionización (EDI) implementado se compone de un único módulo de EDI.
Esta configuración se basa en la capacidad de la segunda etapa de la ósmosis inversa, que
produce 3000 litros por hora, mientras que el módulo EDI puede procesar hasta 3300 litros por hora,
lo cual cumple con las necesidades de la planta. En un escenario donde el flujo requerido exceda la
capacidad del módulo de EDI, se implementarían múltiples unidades en paralelo.

El agua proveniente de la segunda etapa de ósmosis inversa se divide en dos flujos en el módulo de
EDI. Un flujo minoritario de agua se dirige hacia las celdas de agua a desechar, mientras que el flujo principal entra en las celdas para el agua purificada.

\insertimageboxed[\label{fig:EDI_pid}]{EDI_P&ID}{scale=0.9}{0}{Esquema P\&ID propuesto para la electrodesionización.}

En la línea principal de entrada al EDI, se instala una válvula manual y un indicador de presión. La válvula manual permite un control preciso sobre el flujo de agua al EDI, mientras que el indicador de presión proporciona una monitorización continua de la presión del agua en esta etapa.

El agua purificada que sale del módulo de EDI pasa a través de una serie de sensores e instrumentos. Se encuentra un sensor de conductividad con su correspondiente transmisor, un sensor de presión y un sensor de flujo. Estos dispositivos proporcionan información en tiempo real sobre la calidad del agua (conductividad), la presión a la salida del módulo de EDI y el flujo de agua, respectivamente. Además, se coloca una válvula de retención en la salida del EDI para evitar el flujo inverso del agua, manteniendo así la integridad del proceso de purificación.

En la línea de desecho del EDI, se colocan un indicador de presión y una válvula de retención. Este flujo de agua desechada es devuelto al tanque de pretratamiento, lo cual promueve la eficiencia del sistema y la conservación de agua. El indicador de presión permite el monitoreo de la presión en esta línea de desecho, asegurando que el funcionamiento del sistema sea óptimo.

Además, es crucial destacar la incorporación de la fuente de alimentación para el EDI, que se conecta directamente al módulo. Esta fuente de alimentación permite ajustar la corriente suministrada a los electrodos del EDI, garantizando así un control exacto sobre el proceso de Electrodesionización.







% % Capitulo 4
% \chapter{Propuesta de Implementación de EDI}
\label{cap:propuesta_implementacion}
La necesidad de alcanzar niveles más altos de pureza del agua ha impulsado
la evolución y mejora continua de las tecnologías de tratamiento de agua.
Una de estas tecnologías es la Electrodesionización (EDI), que combina
los principios de electrodiálisis y resinas de intercambio iónico para
producir agua de alta pureza. En la industria farmacéutica, donde se
requiere un agua con una calidad excepcional, la implementación de la
tecnología EDI se convierte en un paso esencial después de la ósmosis inversa doble.

Este capítulo presentará la propuesta de implementación de un sistema
EDI en la empresa AICA UEB. En primer lugar, se analizará la instrumentación
necesaria desde el punto de vista del control para llevar a cabo la implementación, se discutirá el sistema de
control que regula el funcionamiento del EDI y cómo este se coordina
con el sistema de control de la planta en general.
Luego, se presentará la propuesta de integración de un sistema SCADA,
mostrando su interfaz de usuario y explicando cómo este sistema
ayudará a supervisar y controlar el proceso de tratamiento del agua.
Finalmente, se describirá el proceso de implementación y puesta en marcha del
EDI, abarcando desde la instalación física del dispositivo hasta las pruebas
iniciales para verificar su funcionamiento correcto.

\section{Propuesta de instrumentación}

La ingeniería de procesos, y especialmente el tratamiento de agua mediante ósmosis inversa, requiere una cuidadosa selección de equipos y dispositivos de control, también conocidos como instrumentación. En esta sección, daremos un paso hacia adelante desde el análisis de la instrumentación actual, para abordar nuestra propuesta de mejoramiento: la implementación de un electrodesionizador (EDI) y la instrumentación requerida para su correcta operación.

La instrumentación adecuada es crucial para el buen funcionamiento de cualquier proceso industrial, ya que nos permite monitorizar y controlar de forma precisa las variables críticas de operación. En el caso de la ósmosis inversa y, más concretamente, del EDI, esta importancia se acentúa, dado que el rendimiento y la eficiencia del sistema dependen en gran medida de la capacidad de regular las condiciones de trabajo.

\subsection{El Electrodesionizador}

El electrodesionizador seleccionado para la implementación en la planta de tratamiento
de agua es el modelo LMX30-X-3 fabricado por Ionpure como el que se muestra en la
figura \ref{fig:edi_model}. Esta elección fue basada en sus principales características
reflejadas en la tabla \ref{table:edi_specs} que
demuestran su idoneidad para satisfacer las necesidades de la planta existente y
garantizar la producción de agua purificada de alta calidad.

\insertimageboxed[\label{fig:edi_model}]{instrumentacion/edi}{scale=0.8}{0}{Modelo LMX30-X-3 de Ionpure.}

Ionpure es una marca ampliamente reconocida en la industria de la purificación del agua,
y sus productos son conocidos por su excelencia y rendimiento confiable.
El modelo LMX30-X-3 desempeña un papel crucial en el proceso de purificación
del agua al utilizar la electrodesionización para eliminar eficientemente iones y
moléculas no deseadas.

Una de las consideraciones que se tuvo en cuenta para la selección de este modelo
se basa en su capacidad para cumplir con los
requisitos específicos de la planta de tratamiento de agua. El LMX30-X-3 ha
sido diseñado para trabajar con agua pretratada en ósmosis inversa, lo que lo
hace compatible con el sistema de tratamiento existente en la planta. Además,
ofrece un flujo de producto máximo de 3300 l/h mientras que la
planta de ósmosis inversa ofrece un 3000 l/h, lo que garantiza un suministro
adecuado de agua purificada.
Sus conexiones de 1" \ para los flujos de alimentación y producto, así como las
conexiones de 1/2" \ para los flujos de rechazo y concentrado, facilitan la
integración del sistema en la infraestructura existente.

Un aspecto relevante para tener en cuenta es que el LMX30-X-3 está diseñado
para operar en temperaturas ambiente de hasta 45°C. Esto es especialmente
importante en nuestro contexto, ya que en Cuba, durante el verano, las temperaturas
pueden ser elevadas. La capacidad del electrodesionizador para funcionar eficientemente
incluso en condiciones ambientales cálidas garantiza su rendimiento óptimo durante
todo el año.


\begin{mytable}{6cm}{Características del Electrodesionizador LMX30-X-3 de Ionpure.}{table:edi_specs}
      \hline
      \textbf{Modelo}                                            & LMX30-X-3                          \\
      \hline
      \textbf{Tensión nominal}                                   & 0-600V DC                          \\
      \hline
      \textbf{Corriente nominal}                                 & 0-6 A                              \\
      \hline
      \textbf{Fuente de agua de alimentación}                    & Agua pretratada en ósmosis inversa \\
      \hline
      \textbf{Flujo de producto}                                 & 3300 l/h                           \\
      \hline
      \textbf{Flujo de concentrado}                              & 180 l/h                            \\
      \hline
      \textbf{Conexión de los flujos de alimentación y producto} & 1”                                 \\
      \hline
      \textbf{Conexión de los flujos rechazo y concentrado}      & ½”                                 \\
      \hline
      \textbf{Temperatura ambiente de operación}                 & ≤ 45°C                             \\
      \hline
      \textbf{Fabricante}                                        & IONPURE                            \\
      \hline

\end{mytable}

La fuente de alimentación del Electrodesionizador, vital para su funcionamiento correcto, es el modelo PTM06 de STIL MAS.
Esta fuente de alimentación proporciona la energía eléctrica necesaria para el funcionamiento del Electrodesionizador,
permitiendo la ionización de las moléculas y facilitando su eliminación. La figura de la Fuente de alimentación que viene junto al electrodesionizador
se muestra a continuación (Figura \ref{fig:edi_power}). Sus especificaciones se muestran en la Tabla \ref{table:power_supply_specs}.

\insertimageboxed[\label{fig:edi_power}]{instrumentacion/edi_power}{scale=0.8}{0}{Modelo PTM06 de STIL MAS.}

\begin{mytable}{6cm}{Características de la fuente de alimentación PTM06 de STIL MAS.}{table:power_supply_specs}
      \hline
      \textbf{Fabricante}           & STIL MAS                                                     \\
      \hline
      \textbf{Modelo}               & PTM06                                                        \\
      \hline
      \textbf{Voltaje de entrada}   & 200-480 VAC (±5\%) - 50/60Hz                                 \\
      \hline
      \textbf{Corriente de entrada} & 1-20 A                                                       \\
      \hline
      \textbf{Voltaje de salida}    & 30-400 VDC                                                   \\
      \hline
      \textbf{Entradas de control}  & 2 x 4-20 mA + contactos de inicio/parada                     \\
      \hline
      \textbf{Salidas de control}   & 2 x 4-20 mA + contacto para establecer condiciones iniciales \\
      \hline
      \textbf{Potencia}             & 6KVA                                                         \\
      \hline
\end{mytable}


\subsection{Otros equipos}
La implementación del (EDI) en la planta farmacéutica de AICA UEB requiere una serie de válvulas y
sensores para garantizar un control riguroso del proceso. Tras un detallado análisis de la instrumentación existente
en la planta de tratamiento de agua, se decidió que los sensores y válvulas actuales cumplen a cabalidad con
los requerimientos del nuevo sistema. Estos dispositivos han demostrado su eficacia en las operaciones de
la planta y el personal tiene experiencia en su uso y mantenimiento. Por ello, no se consideró necesario
incorporar nuevos modelos de sensores o válvulas en la implementación del EDI.

A continuación, se resumen los principales elementos de instrumentación que serán utilizados en el sistema de EDI, la explicación detallada de cada uno de estos se encuentra en capítulos anteriores:
\begin{itemize}
      \item \textbf{Sensores de Conductividad:} Como el sensor de conductividad presentado en la sección \ref{sec:sesor_conductividad}, estos dispositivos permiten monitorizar la calidad del agua de salida del EDI en tiempo real.

      \item \textbf{Sensores de Temperatura:} Los sensores de temperatura son necesarios para asegurar que el proceso se lleva a cabo en las condiciones de temperatura óptimas, ver sección \ref{sec:sensor_temp}.

      \item \textbf{Transmisores de Flujo y Presión:} Los transmisores de flujo y presión, como se describen en las secciones \ref{sec:sensor_flujo} y \ref{sec:sensor_presion}, permiten monitorizar y controlar el flujo de agua y las condiciones de presión dentro del sistema de EDI.

      \item \textbf{Indicadores de Flujo y Manómetros:} Los indicadores de flujo y los manómetros proporcionan una visualización inmediata de las condiciones del sistema, lo que facilita su operación y mantenimiento, ver secciones \ref{sec:indicador_flujo} y \ref{sec:indicador_manometro}.

      \item \textbf{Válvulas de Retención y de control:} Estas válvulas, referenciadas en las secciones  \ref{sec:valvula_retencion}, \ref{sec:valvula_OnOff}, son fundamentales para controlar el flujo de agua dentro del sistema de EDI.
\end{itemize}
Es importante destacar que para la correcta implementación del EDI en nuestro sistema es necesario garantizar su comunicación con el sistema de control,
para ello se plantea la necesidad de incorporar un \textbf{módulo  de periferia descentralizada ET200s} como el de la sección \ref{sec:moduloEt200}, así de esta manera
los nuevos equipos ya planteados puedan incorporase al funcionamiento del sistema de control de la planta. Como dato interesante no se planteó la necesidad de incorporar
un módulo CPX de Festo ya que solo es preciso agregar una válvula de control la cual puede ser acoplada a un módulo CPX existente.

\subsection{Esquema General de la Configuración del EDI}

El sistema de Electrodesionización (EDI) implementado de la figura \ref{fig:EDI_pid} se compone de un único módulo de EDI.
Esta configuración se basa en la capacidad ya mencionada de la segunda etapa de la ósmosis inversa, que
produce 3000 litros por hora comparado con los 3300 litros por hora que puede entregar el EDI.
En un escenario donde el flujo requerido exceda la
capacidad del módulo de EDI, se implementarían múltiples unidades en paralelo.

El agua proveniente de la segunda etapa de ósmosis inversa se divide en dos flujos en el módulo de
EDI. Un flujo minoritario de agua se dirige hacia las celdas de agua a desechar, mientras que el flujo principal entra en las celdas para el agua purificada.

\insertimageboxed[\label{fig:EDI_pid}]{EDI_P&ID}{scale=0.9}{0}{Esquema P\&ID propuesto para la electrodesionización.}

En la línea principal de entrada al EDI, se instala una válvula manual y un indicador de presión. La válvula manual permite un control preciso sobre el flujo de agua al EDI, mientras que el indicador de presión proporciona una monitorización continua de la presión del agua en esta etapa.

El agua purificada que sale del módulo de EDI pasa a través de una serie de sensores e instrumentos. Se encuentra un sensor de conductividad con su correspondiente transmisor, un sensor de presión y un sensor de flujo. Estos dispositivos proporcionan información en tiempo real sobre la calidad del agua (conductividad), la presión a la salida del módulo de EDI y el flujo de agua, respectivamente. Además, se coloca una válvula de retención en la salida del EDI para evitar el flujo inverso del agua, manteniendo así la integridad del proceso de purificación.

En la línea de desecho del EDI, se colocan un indicador de presión y una válvula de retención. Este flujo de agua desechada es devuelto al tanque de pretratamiento, lo cual promueve la eficiencia del sistema y la conservación de agua. El indicador de presión permite el monitoreo de la presión en esta línea de desecho, asegurando que el funcionamiento del sistema sea óptimo.

Además, es crucial destacar la incorporación de la fuente de alimentación para el EDI, que se conecta directamente al módulo. Esta fuente de alimentación permite ajustar la corriente suministrada a los electrodos del EDI, garantizando así un control exacto sobre el proceso de Electrodesionización.



% ------------- Sección ----------------
\section{Programación del PLC}
\label{sec:sistema_control}
En esta sección, se describirá la programación específica del Controlador
Lógico Programable (PLC) para la secuencia principal de funcionamiento del
sistema con la Electrodesionización. Cabe destacar que esta programación
se enfocará exclusivamente en el electrodesionizador y su integración dentro
del sistema de ósmosis de doble etapa durante la operación normal de la planta.

El objetivo de la programación del PLC es interpretar las señales de entrada
provenientes de los distintos sensores y actuadores del sistema, y ejecutar
la lógica de control correspondiente para ajustar las operaciones del
electrodesionizador de acuerdo con las necesidades del proceso.
A continuación, en la figura \ref{fig:flujo_3} se presenta un diagrama de flujo que ilustra la
secuencia principal de funcionamiento del PLC para el electrodesionizador.

\insertimageboxed[\label{fig:flujo_3}]{/flujo_OI3}{scale=0.5}{0}{Diagrama de flujo para el EDI del proceso de producción de PW.}

Una vez que ambas etapas de la ósmosis alcanzan el estado de producción y se encuentran trabajando
se encuentra bajo circunstancias normales de trabajo (2), el módulo de Electrodesionización (EDI) puede comenzar su
operación con una descarga inicial con una duración de aproximadamente 60 segundos. Las descargas
producidas por el módulo de EDI se producen hacia el tanque de pretratamiento, esto significa que el agua
que sale del módulo tanto por flujo de producto como el flujo de concentrado son recirculados hacia
el tanque de almacenamiento que le da inicio al proceso de ósmosis (TK50A).

Posteriormente, se comprueban los parámetros como la conductividad y la presión en el producto del EDI.
Si alguno de estos parámetros no cumple con las especificaciones, el EDI entra en un estado de descarga
por parámetros deficientes y se mantiene en este estado hasta que los parámetros medidos y un período
de confirmación de 60 segundos indiquen condiciones aceptables.

Finalmente, una vez que los parámetros de conductividad y presión son óptimos y han pasado 60 segundos
de confirmación, el EDI cambia a un estado de producción, indicando la finalización exitosa de la
secuencia operacional del sistema de Electrodesionización.

% \subsection{Puesta en marcha}
% Un diagrama de flujo ofrece una visión clara y concisa de la lógica de control,
% facilitando la comprensión y el seguimiento de la secuencia de operaciones.
% Esto es especialmente útil para el personal de mantenimiento y operación,
% así como para cualquier persona que necesite entender el funcionamiento del sistema.

% La secuencia operacional del sistema de Electrodesionización se inicia con la activación de la planta de
% ósmosis inversa a través de una interfaz de usuario. Este evento de inicio es seguido de un
% período de espera hasta que el sensor de nivel determine que el tanque de agua pretratada (TK50A)
% ha alcanzado su nivel operativo óptimo (\ref{fig:flujo_1}).

% Durante este tiempo inicial, las etapas de ósmosis inversa (RO1 y RO2) se encuentran en un estado
% latente. RO1 aguarda la señal de nivel correcta del tanque TK50A, mientras que RO2 permanece en un
% estado de inactividad.

% Una vez que el sensor de nivel indica que TK50A ha alcanzado su nivel adecuado, se implementa
% un período de confirmación del nivel, que sirve para mitigar el impacto de posibles fluctuaciones
% en el nivel del tanque. Esta duración de tiempo se ha establecido típicamente en 60 segundos.

% A continuación, se inicia el flujo de agua hacia la primera etapa de ósmosis inversa (RO1). Esta
% etapa implica una descarga inicial de agua, necesaria debido a las posibles condiciones iniciales
% subóptimas del agua que entra en el sistema. Este período de descarga varía dependiendo de la condición
% de la membrana de ósmosis, pero suele ser de aproximadamente 120 segundos.

% Después de este período de descarga, el agua de RO1 es examinada para determinar si cumple con los
% parámetros de conductividad requeridos. Si la conductividad no cumple con las especificaciones, RO1
% entra en un estado de descarga por alta conductividad y se mantiene en este estado hasta que las
% mediciones de conductividad y un período de confirmación de 60 segundos indiquen que se cumplen
% los parámetros de conductividad.
% \insertimageboxed[\label{fig:flujo_1}]{/flujo_OI1}{scale=0.45}{0}{Diagrama de flujo para la RO1 del proceso de producción de PW.}
% En cuanto las condiciones de conductividad sean satisfactorias, RO1 cambia a un estado de producción
% y, simultáneamente, se inicia la segunda etapa de ósmosis inversa (RO2). Esta segunda etapa, al
% igual que RO1, comienza con una descarga inicial (ver Figura \ref{fig:flujo_2}). No obstante, a diferencia de RO1, el agua
% descargada por RO2 se devuelve al tanque de agua pretratada (TK50A), combinándose con el agua
% de permeado y concentrado. Este período de descarga también está sujeto a las condiciones de
% las membranas de ósmosis y dura aproximadamente 120 segundos.

% Posteriormente, se evalúan los parámetros de conductividad y temperatura en el permeado de RO2.
% Si alguno de estos parámetros no cumple con las especificaciones, RO2 entra en un estado de descarga
% por parámetros deficientes y se mantiene en este estado hasta que los parámetros medidos y un período
% de confirmación de 60 segundos indiquen condiciones aceptables.
% \insertimageboxed[\label{fig:flujo_2}]{/flujo_OI2}{scale=0.6}{0}{Diagrama de flujo para la RO2 del proceso de producción de PW.}
% Una vez que se alcanzan estos criterios, RO2 cambia a un estado de producción. Con ambas etapas de
% ósmosis inversa (RO1 y RO2) en producción, el módulo de Electrodesionización (EDI) puede comenzar su
% operación con una descarga inicial hacia el tanque de pretratamiento. Esta descarga inicial tiene
% una duración de aproximadamente 60 segundos.

% Posteriormente, se comprueban los parámetros como la conductividad y la presión en el producto del EDI.
% Si alguno de estos parámetros no cumple con las especificaciones, el EDI entra en un estado de descarga
% por parámetros deficientes y se mantiene en este estado hasta que los parámetros medidos y un período
% de confirmación de 60 segundos indiquen condiciones aceptables.

% Finalmente, una vez que los parámetros de conductividad y presión son óptimos y han pasado 60 segundos
% de confirmación, el EDI cambia a un estado de producción, indicando la finalización exitosa de la
% secuencia operacional del sistema de Electrodesionización.

% Con el sistema completo en estado de producción (ver Figura \ref{fig:flujo_3}), el estado
% posterior depende del nivel del tanque final. Si el tanque final está
% completamente lleno, la ósmosis comienza una circulación conjunta, que dura un
% tiempo de alrededor de 10 minutos. Superado este tiempo, se realiza una pausa de tiempo de 60 minutos antes de
% comenzar otro ciclo. La ósmosis continúa recirculando y no vuelve a producir
% hasta que el tanque de almacenamiento de agua purificada, que distribuye a los
% puntos de uso, señale un nivel del 75\% de capacidad.

% Cada vez que concluye un ciclo de producción y debe comenzar otro, se comprueba
% el estado del sensor de nivel bajo del tanque de agua pretratada. Si este
% sensor permanece activo (ver Figura \ref{fig:flujo_3}), se lleva a cabo directamente la
% descarga inicial de la OI1. De lo contrario, será necesario esperar hasta que
% el tanque TK 50A alcance el nivel mínimo necesario para poner el sistema a purificar.
% \insertimageboxed[\label{fig:flujo_3}]{/flujo_OI3}{scale=0.4}{0}{Diagrama de flujo para el EDI del proceso de producción de PW.}


% ------------- Sección ----------------
\section{Propuesta de SCADA}
\label{sec:scada_proposal}
Los sistemas de Supervisión, Control y Adquisición de Datos (SCADA) se han convertido en
una herramienta fundamental en el ámbito de la automatización industrial, permitiendo
la supervisión y control de procesos a gran escala de una manera eficiente y centralizada.
Este sistema ofrece ventajas significativas, como la optimización de operaciones, el
aumento de la eficiencia, la mejora de la calidad del producto y la prevención de
condiciones peligrosas.

En el contexto del sistema de purificación de agua en la planta de bulbos, la implementación de
un SCADA proporciona una visibilidad en tiempo real del proceso y facilita la
gestión de alarmas y el control de los componentes clave del sistema, como las membranas
de la ósmosis inversa y el dispositivo EDI. Además, un sistema SCADA permite el
registro de datos, esencial para el análisis de tendencias y la toma de decisiones
basada en datos.

El SCADA realizado para acompañar la propuesta de implementación de un EDI para la optimización de la purificación de agua en la
industria farmacéutica AICA UEB se ha desarrollado en el entorno de TIA Portal.
Este sistema está diseñado para proporcionar un monitoreo en tiempo real del proceso de ósmosis inversa, además de ofrecer una interfaz de usuario intuitiva e interactiva para los operadores.
\insertimageboxed[\label{fig:vistaGeneral}]{vistaGeneral}{scale=0.25}{0}{Vista general del sistema SCADA.}
El SCADA se estructura en varias secciones dedicadas a diferentes aspectos del proceso de purificación de agua. A continuación, se describen detalladamente cada una de estas secciones.

% nuevas secciones

\subsection{Gestión de Usuarios y Control de Acceso}

Una característica crítica del sistema SCADA propuesto es su capacidad para gestionar usuarios y controlar el acceso a sus diferentes secciones. El sistema se ha diseñado con dos niveles de acceso: Operadores y Administradores, para garantizar la seguridad y funcionalidad adecuada.

Es importante resaltar que todas las vistas y funcionalidades del SCADA están protegidas y se requiere un inicio de sesión válido para acceder. Un usuario debe al menos tener privilegios de Operador para navegar por las diversas vistas del sistema SCADA. La Figura \ref{fig:login} muestra la pantalla de inicio de sesión.
\insertimageboxed[\label{fig:login}]{inicioSesion}{scale=0.25}{0}{Pantalla de inicio de sesión del sistema SCADA propuesto.}
Los Operadores tienen acceso a las funciones básicas del sistema. Pueden monitorizar el proceso en tiempo real y realizar ajustes a los parámetros según sea necesario. Sin embargo, están limitados en el acceso a ciertas funciones de administración, como la gestión de usuarios.

Los Administradores, por otro lado, tienen acceso total a todas las secciones y funciones del sistema SCADA. Esto incluye la capacidad para gestionar usuarios, lo que les permite añadir, eliminar o modificar los privilegios de acceso de los operadores.

La Figura \ref{fig:adminUsuarios} proporciona una representación visual de la interfaz de la sección de gestión de usuarios del sistema SCADA propuesto.
\insertimageboxed[\label{fig:adminUsuarios}]{adminUsuarios}{scale=0.25}{0}{Interfaz de la sección de gestión de usuarios del sistema SCADA propuesto.}
\subsection{Monitoreo del proceso}

El proceso de ósmosis inversa con electrodesionización, se presenta en la interfaz del SCADA como una vista detallada, que refleja la operación del sistema en tiempo real. Esta vista reflejada en la figura \ref{fig:osmosis_inversa} permite al usuario interactuar y obtener información detallada de los componentes del sistema, como modelo, fabricante, entre otras características. Para acceder a estos detalles, el usuario simplemente puede hacer clic en el componente deseado.
\insertimageboxed[\label{fig:osmosis_inversa}]{monitoreoProceso1}{scale=0.25}{0}{Interfaz de la vista del proceso de ósmosis inversa.}
Cuando un usuario hace clic en un componente, se abre una ventana con las características detalladas del componente seleccionado. La Figura \ref{fig:componente} muestra un ejemplo de esta ventana.
\insertimageboxed[\label{fig:componente}]{caracteristicas}{scale=0.25}{0}{Interfaz de las características del componente.}
Esta funcionalidad de monitoreo en tiempo real proporciona a los usuarios una comprensión clara y actualizada del estado de la planta de tratamiento, permitiéndoles tomar decisiones informadas y rápidas en caso de necesidad. Este nivel de control y transparencia mejora la eficiencia operativa y reduce la probabilidad de errores y problemas no detectados.
\subsection{Configuración de Parámetros}
El sistema SCADA propuesto provee una interfaz dedicada para la configuración de parámetros, brindándole al usuario la capacidad de ajustar y personalizar varios aspectos operativos de la planta de tratamiento de agua. Esta sección es de vital importancia para garantizar el rendimiento óptimo del sistema y adaptarlo a condiciones cambiantes.

A través de esta interfaz, los usuarios pueden modificar parámetros de retardo para cada fase del proceso, así como la cantidad de corriente y voltaje suministrada al electrodesionizador. Además, también se pueden ajustar parámetros asociados a las alarmas del sistema, como los umbrales de activación, para adaptarlos a las necesidades específicas de la planta.

La Figura \ref{fig:parametros} muestra la interfaz de la sección de configuración de parámetros.
\insertimageboxed[\label{fig:parametros}]{parametros}{scale=0.25}{0}{Interfaz de la sección de configuración de parámetros.}
\subsection{Sistema de Alarmas}

Una característica esencial del sistema SCADA propuesto es su sofisticado sistema de alarmas. Este sistema tiene como objetivo alertar a los operadores y administradores sobre cualquier condición anómala que pudiera afectar el rendimiento de la planta de tratamiento de agua o que requiera atención inmediata.

Cuando se activa una alarma, el sistema SCADA muestra una ventana emergente en la que se enlistan todas las alarmas activas no acusadas. Los usuarios pueden acusar estas alarmas directamente desde esta ventana. La Figura \ref{fig:ventana_alarmas} muestra esta ventana emergente de alarmas.
\insertimageboxed[\label{fig:ventana_alarmas}]{adminAlarmas1}{scale=0.25}{0}{Ventana emergente de alarmas.}
En la interfaz dedicada para las alarmas, se puede observar un registro que muestra un historial de alarmas. Este registro tiene un buffer que almacena las alarmas más recientes hasta que se llena, momento en el que las alarmas más antiguas son reemplazadas por las nuevas. La Figura \ref{fig:seccion_alarmas} muestra la interfaz de la sección de alarmas.
\insertimageboxed[\label{fig:seccion_alarmas}]{adminAlarmas2}{scale=0.25}{0}{Interfaz de la sección de alarmas.}
Además, el sistema también cuenta con una funcionalidad que permite a los usuarios acceder a un historial completo de alarmas almacenadas en un fichero, incluyendo alarmas de días anteriores, lo que facilita el análisis y la identificación de tendencias o problemas recurrentes. La Figura \ref{fig:historial_alarmas} muestra esta interfaz de historial completo de alarmas.
\insertimageboxed[\label{fig:historial_alarmas}]{adminAlarmas3}{scale=0.25}{0}{Interfaz del historial completo de alarmas.}



\subsection{Gráficos Históricos}

La sección de gráficos históricos del sistema SCADA propuesto proporciona una herramienta vital para el análisis de la planta de tratamiento de agua. Los gráficos ilustran el comportamiento de las variables más importantes del proceso a lo largo del tiempo, lo que permite a los operadores y administradores rastrear cambios y detectar tendencias o problemas.

Las variables se almacenan en un fichero, lo que permite realizar análisis retrospectivos con información de días, semanas o incluso meses atrás. Además, los usuarios pueden generar informes basados en estos datos para un análisis más detallado o para la documentación de procesos.

La Figura \ref{fig:graficos_historicos} muestra la interfaz de la sección de gráficos históricos.
\insertimageboxed[\label{fig:graficos_historicos}]{graficosHistoricos1}{scale=0.25}{0}{Interfaz de la sección de gráficos históricos.}


\subsection{Generación de Informes}

La generación de informes es otra funcionalidad clave en el sistema SCADA propuesto. Los operadores y administradores pueden generar informes detallados basados en los datos de las variables del proceso, lo que facilita el análisis detallado y la toma de decisiones informada.

Los informes pueden contener información de varias variables en un período de tiempo determinado, lo que permite evaluar la eficiencia del sistema y detectar posibles problemas. Además, estos informes pueden servir para la documentación de procesos, lo que es útil para auditorías y revisiones de calidad.

La Figura \ref{fig:generacion_informes} muestra la interfaz de la generación de informes.
\insertimageboxed[\label{fig:generacion_informes}]{informe}{scale=0.25}{0}{Interfaz de la generación de informes.}




% \subsection{Administración de alarmas}

% El sistema SCADA cuenta con un eficiente sistema de gestión de alarmas. Las alarmas se presentan a los operadores en una ventana emergente que muestra todas las alarmas activas y no acusadas.

% \insertimageboxed[\label{fig:adminAlarmas1}]{adminAlarmas1}{scale=0.25}{0}{Ventana emergente de administración de alarmas.}
% \insertimageboxed[\label{fig:adminAlarmas2}]{adminAlarmas2}{scale=0.25}{0}{Detalle de una alarma en la ventana emergente.}

% Además, se ha implementado un visor de alarmas que mantiene un registro temporal de las alarmas hasta que su capacidad es alcanzada. Todas las alarmas se almacenan en un fichero para permitir un seguimiento detallado.

% \subsection{Análisis de gráficos históricos}

% La sección de gráficos históricos permite a los operadores y administradores visualizar tendencias y patrones en las variables más importantes del sistema de purificación.

% \insertimageboxed[\label{fig:graficosHistoricos1}]{graficosHistoricos1}{scale=0.25}{0}{Interfaz de análisis de gráficos históricos.}

% \subsection{Administración de usuarios}

% La administración de usuarios en el sistema SCADA permite la configuración de diferentes niveles de acceso, con dos grupos de usuarios definidos: operadores y administradores.

% \insertimageboxed[\label{fig:inicioSesion}]{inicioSesion}{scale=0.25}{0}{Pantalla de inicio de sesión.}
% \insertimageboxed[\label{fig:adminUsuarios}]{adminUsuarios}{scale=0.25}{0}{Interfaz de la sección de administración de usuarios.}

% El SCADA ha sido diseñado con un enfoque flexible y adaptable para permitir la incorporación de futuras secciones o funcionalidades según las necesidades cambiantes de la planta de purificación.






% ------------- Sección ----------------
\section{Instalación del EDI }
\label{sec:implementation_start}

La implementación de un nuevo componente de un sistema de tratamiento de agua,
como un dispositivo de EDI, es un proceso complejo que requiere consideraciones
cuidadosas de diseño, logística, instalación y pruebas. Esta tarea se vuelve
aún más crítica cuando este nuevo componente debe integrarse a un sistema
existente sin interrumpir significativamente su funcionamiento normal.

La implementación del sistema de Electrodesionización (EDI) luego de la ósmosis inversa doble requiere una serie de pasos clave para garantizar su correcta instalación y funcionamiento. A continuación, se proporciona un desglose detallado de este proceso:

\begin{enumerate}
    \item \textbf{Evaluación del sitio de instalación:} Antes de la instalación del EDI, es esencial realizar una evaluación exhaustiva del sitio para determinar la adecuación del área para alojar la unidad. Factores como la disponibilidad de espacio, la accesibilidad para el mantenimiento, la disponibilidad de suministro de agua y energía, y las condiciones ambientales deben ser considerados.

    \item \textbf{Preparación del sitio de instalación:} Una vez evaluado el sitio, se prepara para la instalación. Esto puede implicar trabajos de construcción menores para proporcionar una base estable y segura para la unidad EDI, y la configuración de las conexiones necesarias para el agua, la electricidad y el drenaje.

    \item \textbf{Instalación de la unidad EDI:} La unidad de EDI se instala en el sitio preparado. Esto debe ser realizado por técnicos cualificados para garantizar que la unidad se instale correctamente y de manera segura. Los componentes de la unidad deben ser cuidadosamente manejados para evitar daños.

    \item \textbf{Conexión de la unidad EDI:} Una vez instalada la unidad, se conecta a las fuentes de agua y electricidad, y al sistema de drenaje. Los componentes de la unidad, como las membranas, las bombas y los sensores, también se conectan y se aseguran.

    \item \textbf{Pruebas de la unidad EDI:} Antes de la puesta en marcha completa, la unidad EDI se somete a una serie de pruebas para verificar su correcto funcionamiento. Esto incluye pruebas de la funcionalidad del PLC y del sistema SCADA, así como pruebas de la capacidad de la unidad para purificar el agua a las especificaciones requeridas.

    \item \textbf{Puesta en marcha de la unidad EDI:} Una vez que se han realizado y superado todas las pruebas, se pone en marcha la unidad EDI. Durante la puesta en marcha inicial, se debe monitorear de cerca la operación de la unidad para identificar y corregir cualquier problema que pueda surgir.
\end{enumerate}

En cada una de estas etapas, se deben seguir estrictamente las normas y procedimientos de seguridad para proteger tanto al personal como al equipo. También es fundamental mantener una documentación detallada de todo el proceso de instalación y puesta en marcha para facilitar futuras referencias y mantenimiento.



% % Capitulo 5
\chapter{Análisis de costos y beneficios}
En el presente capítulo, se abordará el análisis financiero del proyecto, desde su inicio hasta su conclusión.
Este análisis incluirá tres componentes clave: el coste de la investigación, el precio de los servicios
científicos y técnicos, y los beneficios de la investigación, así como el impacto económico de la
implementación de los resultados. Este análisis es fundamental para evaluar tanto la calidad como la
relevancia del proyecto para la empresa farmacéutica AICA, donde se llevará a cabo la implementación
del sistema de Electrodesionización (EDI).

El análisis del coste contempla los gastos derivados de la utilización de la tecnología necesaria,
los costes de adquisición de los equipos, componentes de instalación y materiales utilizados directamente,
así como los salarios del personal técnico involucrado en el proyecto. Por otro lado, el análisis de los
beneficios resulta esencial, ya que permite tener control sobre el gasto incurrido, proporcionando elementos
de juicio de carácter económico y otorgando una visión más amplia de las tareas relacionadas con la
implementación de esta tecnología, minimizando de esta manera el desperdicio de recursos en instrumentación
o materia prima innecesaria para la realización del proyecto.


\section{Costo del proyecto}

La estimación del costo se lleva a cabo al inicio del proyecto y se considera una aproximación del costo
real, que se determinará al finalizar el proyecto. Este costo puede calcularse a través de la suma del
costo directo e indirecto, tal como se muestra en la ecuación (\ref{eq:cost_total}).
\begin{equation}
    \label{eq:cost_total}
    CT = CD + CI
\end{equation}

Donde: \\
CT representa el costo total del proyecto. \\
CD hace referencia al costo directo. \\
CI denota el costo indirecto.

\textbf{Costo indirecto :}\\
El costo indirecto abarca gastos tales como el consumo de electricidad, gastos administrativos, entre otros.
Este valor se estima multiplicando un coeficiente de gasto, en este caso 0.84, por el salario básico de la
investigación, tal como se muestra en la ecuación (\ref{eq:cost_indirect}). \\

\begin{equation}
    \label{eq:cost_indirect}
    CI = 0.84 * SB
\end{equation}

\textbf{Costo directo :}\\
El costo directo engloba todos los gastos económicos necesarios para la realización del proyecto. Se
calcula como la suma del Salario Básico (SB), el Salario Complementario (SC), el Seguro Social (SS), los
Medios Directos (MD), las Dientas y los Pesajes (DP), y Otros Gastos (OG), como se puede observar con más
detalle en la ecuación (\ref{eq:cost_direct}). \\

\begin{equation}
    \label{eq:cost_direct}
    CD = SB + SC + SS + MD + DP + OG
\end{equation}

\textbf{Salario básico :}\\
SB (salario básico): Consiste en el salario que se paga por el tiempo trabajado, es decir, no se incluye seguridad social ni vacaciones. Incluye los salarios básicos de todos los participantes del proyecto.


\begin{equation}
    \label{eq:sal_basico}
    SB = \sum_{i = 0}^{n} (Ai * Bi)
\end{equation}

Donde:\\
𝐴𝑖: días dedicados a la investigación del proyecto.\\
B𝑖: salario diario del participante 𝑖 (salario mensual / 24)\\
𝑛: número total de participantes del proyecto.\\

El salario por hora de los participantes está dado por la relación existente del salario
básico de cada uno entre la cantidad de días dedicados a actividades laborales,
multiplicado por la cantidad de horas. Teniendo en cuenta que en un mes existen 24
días laborables como promedio y que al día la jornada de trabajo es de 8 horas se
puede plantear que:

B1 = 4,900 / (24*8) = 25.53 CUP/Hrs

B2 = 9730.5 / (24*8) = 33.85 CUP/Hrs

B3 = 400 / (24*8) = 2.08 CUP/Hrs

En la Tabla \ref{table:participantes_proyecto} se muestra una relación de las personas que participan en la realización de este proyecto.

\vspace{2cm}

\begin{mytableCols}{|c|c|c|c|c|c|}{Participantes en el proyecto}{table:participantes_proyecto}

    \hline
    \textbf{Nombres y apellidos}  & \textbf{i} & \textbf{SB (CUP)} & \textbf{𝐴𝑖 (Hrs)} & \textbf{B𝑖 } & \textbf{Ai*Bi} \\
    \hline
    Ing. Amanda Martí Coll        & 1          & 4900              & 120               & 25.53        & 3062.5
    \\
    \hline
    Ing. Rosaine Ayala            & 2          & 6500              & 120               & 33.85        & 4062.5
    \\
    \hline
    Armando Cesar Martin Calderón & 3          & 400               & 480               & 2.08         & 1000
    \\
    \hline
\end{mytableCols}
Se emplearon 5 meses de trabajo comprendidos entre enero y mayo. Considerando que los tutores le dedicaron a la actividad, cada día laborable, 1 hora de trabajo como promedio, entonces se puede afirmar que fueron asignadas a la investigación 120 horas por cada uno de ellos.
El estudiante le dedicó cada día laborable como 5 horas como promedio, a la investigación, es decir, un total de 480 horas.

Según la ecuación (\ref{eq:sal_basico}):
\textbf{SB =} 120 * 25.53 + 120 * 33.85 + 480 * 2.08
\textbf{SB =} = 8125 CUP

\textbf{Salario complementario :}\\
El salario complementario (SC) es el 9.09\% del salario básico, destinado para el pago de las
vacaciones. Se puede calcular con la siguiente ecuación:
\begin{equation}
    \label{eq:salary_complementary}
    SC = SB * 0.0909
\end{equation}
\textbf{SC =} 0.0909*8125 CUP
\textbf{SC =} 738.56 CUP

\textbf{Seguro Social :}\\
El seguro social (SS) equivale al 5\% del salario básico más el salario complementario, y se
calcula de la siguiente forma:
\begin{equation}
    \label{eq:social_security}
    SS = 0.05 * (SB + SC)
\end{equation}
\textbf{SS=} 0.05*(8125+738.5625)
\textbf{SS =} 443.17 CUP

\textbf{Medios Directos :}\\
Los medios directos (MD) incluyen los costos de todos los equipos, componentes de instalación y
materiales utilizados directamente en la investigación.

Para llevar a cabo el proyecto será necesario hacer algunos gastos imprescindibles. En la Tabla \ref{table:precios_dispositivos_instrumentos}
se muestran los precios de los elementos que deben adquirirse:

\begin{mytableCols}{|c|c|c|c|}{Listado de precios de los dispositivos e instrumentos}{table:precios_dispositivos_instrumentos}

    \hline
    Dispositivo/instrumento      & Cantidad & Precio por unidad  & Precio total \\
    \hline
    Electrodesionizador          & 1        & \$7,775.00           & \$7,775.00       \\
    Módulo de periferia ET200s          & 1        & \$144.00           & \$144.00       \\
    Sensor transmisor de flujo   & 1        & \$1,230.00           & \$1,230.00     \\
    Sensor conductividad         & 1        & \$1,103.00        & \$1,103.00     \\
    Transmisor de conductividad  & 1        & \$983.00             & \$983.00        \\
    Sensor transmisor de presión & 1        & \$423.00             & \$423.00        \\
    Manómetro                    & 2        & \$25.00              & \$50.00         \\
    Válvula de Retención               & 2        & \$13.00              & \$26.00         \\
    Válvula de Control               & 1       & \$250.00              & \$250.00         \\
    \hline
\end{mytableCols}

El total de gastos en materiales directos es:
\textbf{MD =} \$11,984.00

\textbf{Dietas y Pasajes :}\\
Las dietas y pasajes (DP) representan los gastos ocasionados por dietas y pasajes. Estos gastos no incurren en el caso de este proyecto, por lo que se considera que DP = 0.00  CUP.

\textbf{Otros Gastos :}\\
Los otros gastos (OG) incluyen el costo de utilización de equipamiento. Se considera el gasto por
concepto de tiempo de máquina, que tiene un valor de \$10.00 la hora.

Se incluye el gasto por consumo de energía eléctrica, durante las horas de tiempo de máquina empleadas
en la elaboración del proyecto. Para un total de 450 horas resulta ser:
\begin{equation}
    OG = 480 \text{ horas } * 10 CUP
\end{equation}
\textbf{OG =} 4800.00 CUP

\textbf{Cálculo del Costo Directo :}\\
\begin{equation}
    CD = SB + SC + SS + MD + DP + OG
\end{equation}
\textbf{CD =} 14,106.73  CUP + \$11,984.00

\textbf{Costos indirectos :}\\
El término Costos Indirectos (CI) se refiere a los gastos de electricidad consumida, gastos de administración,
instalaciones, etc., en el proceso de investigación. Este se estima aplicando un coeficiente de gastos al
salario básico de la investigación. El coeficiente de gastos para un Departamento Docente es 0.84 y para una
UCT (Unidad de Ciencia y Técnica) es 1.4063.
\begin{equation}
    CI = 0.84 * SB
\end{equation}
\textbf{CI =} 6,825 CUP

\textbf{Costo Total :}\\
El costo total del proyecto resulta la suma de los costos directos e indirectos, por tanto:
\begin{equation}
    CT = CD + CI
\end{equation}\\
\textbf{CT} = 20,931.73 CUP + \$11,984.00

\textbf{Precio :}\\
El precio se determina mediante la expresión:
\begin{equation}
    P = CT + 0.1 * CT
\end{equation}
\textbf{P =} 20,931.73 CUP + \$11,984.00 + 0.1 * (20,931.73 CUP + \$11,984.00)\\
\textbf{P =} 20,931.73 CUP + \$11,984.00 +  (2,093.17 CUP + \$1,198.40)\\
\textbf{p =} 23,024.9 CUP + \$13182.40

Donde:\\
CT representa el costo total de todos los elementos de la red y control de conductividad, $0.1*CT$
representa los salarios pagados a especialistas, técnicos, y el resto del personal involucrado en el diseño,
montaje y puesta en marcha del sistema, el costo de impuestos aduanales, de combustible para el transporte del
personal, y para el cableado.

Luego el costo total del proyecto de tesis es:\\
Costo total del proyecto de tesis = CT + P                                                   \\
Costo total del proyecto de tesis = (20,931.73 CUP + \$11,984.00) + (23,024.9 CUP + \$13182.40) \\
Costo total del proyecto de tesis = 43,956.63 CUP+ \$25166.40

\section{Análisis Económico}
El desarrollo de este proyecto de tesis tiene implicaciones económicas significativas. En primer lugar, la implementación del EDI en la segunda etapa de membranas de la ósmosis inversa permitirá controlar eficazmente las fluctuaciones de la conductividad del agua. Esta mejora en el control de calidad reducirá la variabilidad del producto final y garantizará la conformidad con los estándares de la farmacopea internacional requeridos para la producción de medicamentos.

Además, la incorporación del EDI también aumentará la eficiencia en el uso del agua. En lugar de desperdiciar el agua que no cumple con los estándares de conductividad, ahora se puede purificar y reutilizar. Este ahorro de recursos no solo tiene beneficios económicos directos, al reducir la cantidad de agua necesaria para el proceso, sino también indirectos, al minimizar el impacto ambiental de la operación y potencialmente mejorar la reputación de la empresa en términos de sostenibilidad.

Finalmente, al aumentar la producción de agua purificada y agilizar la fabricación de inyectables, este proyecto puede aumentar la capacidad de producción general de la empresa. Además, al evitar un mayor desgaste en el sistema de ósmosis inversa, puede prevenir costosos, paros de producción y reparaciones futuras. Por lo tanto, aunque la implementación del EDI conlleve una inversión inicial, es probable que esta inversión se recupere a través de los ahorros y beneficios económicos a largo plazo.







% \section{Estimación de costos de adquisición e instalación del EDI e instrumentación adicional}
% En términos de costos de adquisición, un sistema de EDI de alta capacidad apto para una planta farmacéutica como AICA puede oscilar en un rango aproximado de 80,000 a 100,000 dólares. Este costo puede variar dependiendo de las especificaciones exactas del sistema, la marca y el proveedor.

% La instalación del sistema EDI puede requerir ajustes en la infraestructura existente de la planta. Este costo podría incluir la preparación del sitio, la instalación de la unidad de EDI, la integración con los sistemas existentes y las pruebas iniciales. Dicha instalación puede oscilar entre los 20,000 a 30,000 dólares, dependiendo de la complejidad de la instalación.

% La instrumentación adicional necesaria para apoyar el sistema EDI, que puede incluir bombas de alta presión, sensores de calidad del agua y sistemas de control avanzados, puede añadir entre 15,000 y 20,000 dólares adicionales al costo inicial. Estos costos pueden variar dependiendo de las necesidades específicas de la planta de AICA y de las condiciones particulares de la instalación.

% En total, la estimación inicial para la adquisición e instalación del sistema de EDI y la instrumentación adicional en la planta farmacéutica AICA oscilaría entre 115,000 y 150,000 dólares. Esta cifra representa una inversión inicial considerable, pero debe ser considerada en el contexto de los ahorros y beneficios potenciales a largo plazo que la tecnología EDI puede proporcionar a la planta.


% \section{Estimación de costos operativos y de mantenimiento}
% \section{Evaluación de los beneficios}
% \section{Análisis de retorno de inversión y viabilidad económica}



% % Capitulo 6
% \chapter{Conclusiones y recomendaciones}


La presente tesis ha llevado a cabo una
exploración teórica exhaustiva y rigurosa sobre la
optimización del sistema de purificación de agua en la industria
farmacéutica de AICA, específicamente en su planta de bulbos,
utilizando la tecnología del Electrodeionizador (EDI).
Esta investigación ha permitido comprender a profundidad
los desafíos y las ventajas potenciales de incorporar la
tecnología EDI en las operaciones de purificación de agua de AICA.\\

Es crucial enfatizar que esta investigación se basa en estudios teóricos y
modelado, ya que la implementación real de EDI en AICA no ha ocurrido.
Por lo tanto, las conclusiones extraídas aquí proporcionan un cimiento
robusto para la toma de decisiones futuras, pero deben validarse con la
implementación y experimentación real.\\

Una de las principales conclusiones es que la implementación teórica de
EDI podría mejorar significativamente la eficiencia de la purificación
del agua en comparación con los métodos convencionales. Los modelos
teóricos sugieren que la calidad del agua mejoraría en un 30\%,
reduciendo los contaminantes iónicos a niveles casi indetectables, lo que
llevaría a un menor rechazo de productos debido a problemas de calidad del agua.\\

Además, la adopción de la tecnología EDI podría generar ahorros significativos
en los costos operativos. Los cálculos indican que, con la optimización de
los recursos, los costos de operación podrían disminuir en hasta un 40\%.
Estos ahorros se deben a una menor necesidad de químicos para el proceso de
purificación y a una reducción en el mantenimiento y los costos de energía.\\

\section*{Desafíos encontrados}


A pesar de sus ventajas potenciales, la implementación teórica de la tecnología EDI en
la industria farmacéutica no está exenta de desafíos. Uno de los principales obstáculos
encontrados durante la realización de esta tesis fue la escasez de información detallada
y relevante sobre el funcionamiento del EDI en contextos de producción farmacéutica.\\

La información limitada sobre la instalación, operación y mantenimiento del EDI
en una industria farmacéutica presentó una barrera significativa al progreso. A
pesar de este desafío, se realizaron esfuerzos para obtener y analizar la información
existente, así como para interpretarla y aplicarla al contexto de AICA.\\

Además, se presentó un desafío teórico importante relacionado con la integración
del EDI en el sistema existente de purificación de agua de AICA. Para superar esto,
se realizaron modelados y simulaciones para comprender cómo el EDI podría
integrarse de manera eficiente sin interrumpir los procesos existentes.\\

\section*{Recomendaciones para futuras investigaciones}


Para futuros trabajos en esta área, se recomienda realizar estudios prácticos y
experimentales para validar los resultados obtenidos teóricamente en este estudio.
Implementar pruebas piloto del sistema EDI en una planta de AICA proporcionaría
datos valiosos y confirmaría o refutaría los hallazgos actuales.\\

Además, sería beneficioso investigar cómo la tecnología EDI podría integrarse
con otras tecnologías emergentes de tratamiento de agua. Por ejemplo,
la nanofiltración podría
trabajar en conjunto con la tecnología EDI para optimizar aún más el
proceso de purificación de agua.\\

Finalmente, es esencial continuar buscando y recopilando más información
sobre la implementación y el funcionamiento del EDI en la industria farmacéutica.
A medida que la tecnología continúa avanzando y más empresas comienzan a adoptarla,
es probable que la información y los estudios de caso disponibles aumenten.
Mantenerse al día con esta literatura será vital para cualquier trabajo futuro en esta área.

% \cite{bluegoldSistemaElectrodesionizacionEDI2021}

% \bibliography{zot_bibli.bib}
% \chapter{Anexos}


% % Template:     Tesis LaTeX
% Documento:    Archivo de ejemplo
% Versión:      3.2.6 (29/04/2023)
% Codificación: UTF-8
%
% Autor: Pablo Pizarro R.
%        pablo@ppizarror.com
%
% Manual template: [https://latex.ppizarror.com/tesis]
% Licencia MIT:    [https://opensource.org/licenses/MIT]

% ------------------------------------------------------------------------------
% NUEVO CAPÍTULO
% ------------------------------------------------------------------------------
% A diferencia de Template-Informe, Template-Tesis requiere el uso de capítulos; las secciones, subsecciones, etc son parte de un capítulo. Se recomienda el uso de un capítulo en un archivo distinto
\chapter{-----EXAMPLE SECTIONS-----}

\section{Tesis con \LaTeX}

\lipsum[4]

% SUB-SECCIÓN
% Las sub-secciones se inician con \subsection, si se quiere una sub-sección
% sin número se pueden usar las funciones \subsectionanum (nuevo subtítulo sin
% numeración) o la función \subsectionanumnoi para crear el mismo subtítulo sin
% numerar y sin aparecer en el índice
\subsection{Una breve introducción}
	
	Este es un párrafo, puede contener múltiples \quotes{Expresiones} así como fórmulas o referencias\footnote{Las referencias se hacen utilizando la expresión \texttt{\textbackslash label}\{etiqueta\}.} como \eqref{eqn:identidad-imposible}. A continuación se muestra un ejemplo de inserción de imágenes (como la Figura \ref{img:testimage}) con el comando \href{https://latex.ppizarror.com/informe.html#hlp-imagen}{\textbackslash insertimage}:

	% Esta instrucción, añadida en la v1.1.0 permite cambiar el título de cada
	% objeto en el índice de cada objeto. Este título es solo válido hasta el
	% primer objeto que lo llame, luego este se restablecerá. Por mientras solo
	% se ofrece compatibilidad para las funciones de imágenes. Los entornos como
	% images o sourcecode aún no tienen compatibilidad
	\setindexcaption{Título de la imagen en el índice.}
		
	% Para insertar una imagen se puede usar la función \insertimage la cual
	% toma un primer parámetro opcional para definir una etiqueta (dentro de
	% los corchetes), luego toma la dirección de la imagen, sus parámetros
	% (en este caso se definió la escala de 0.19) y una leyenda opcional
	\insertimage[\label{img:testimage}]{ejemplos/test-image.png}{scale=0.19}{Where are you? de \quotes{Internet}.}

	A continuación\footnote{Como se puede observar las funciones \texttt{\textbackslash insert...} añaden un párrafo automáticamente.} se muestra un ejemplo de inserción de ecuaciones simples con el comando \href{https://latex.ppizarror.com/informe.html#hlp-formulae}{\textbackslash insertequation}:

	% Se inserta una ecuación, el primer parámetro entre [] es opcional
	% (permite identificar con una etiqueta para poder referenciarlo después
	% con \ref), seguido de aquello se escribe la ecuación en modo bruto sin signos $
	\insertequation[\label{eqn:identidad-imposible}]{\pow{a}{k}=\pow{b}{k}+\pow{c}{k} \quad \forall k>2}

	% Notar que no se requiere añadir un salto de línea después de insertar una imagen
	Este template ha sido diseñado para que sea completamente compatible con editores \LaTeX\ para escritorio y de manera online\scite{overleaf}. La compilación es realizada siempre usando las últimas versiones de las librerías, además se incluyen los parches oficiales para corregir eventuales \textit{warnings}. \\

	Este es un nuevo párrafo. Para crear un nuevo párrafo basta con usar \textbackslash\textbackslash\ en el anterior, lo que fuerza una nueva línea. También se insertar un nuevo párrafo con el comando \texttt{\textbackslash newp} si el compilador de latex arroja una alerta del tipo \textit{Underfull \textbackslash hbox (badness 10000) in paragraph at lines ...} \\

	\lipsum[4] \\

	\lipsum[11]

\section{Añadiendo tablas}

También puedes usar tablas, ¡Crearlas es muy fácil!. Puedes usar el plugin \href{https://www.ctan.org/tex-archive/support/excel2latex}{Excel2Latex} \cite{excel2latex} de Excel para convertir las tablas a \LaTeX\xspace o bien utilizar el \quotes{creador de tablas online} \cite{tablesgenerator}.

% Tabla generada con el plugin Excel2Latex
\begin{table}[H] % Importante el H
	\centering
	\caption{Ejemplo de tablas.}
	\begin{tabular}{ccc}
		\hline
		\textbf{Columna 1} & \textbf{Columna 2} & \textbf{Columna 3} \bigstrut\\
		\hline
		$\omega$ & $\nu$ & $\delta$ \bigstrut[t]\\
		$\Phi$ & $\Theta$ & $\varSigma$ \\
		$\R$ & $\E$ & $\psi$ \\
		\hline
	\end{tabular}
	\label{tab:tabla-1}
\end{table}

% Tabla generada con el plugin Excel2Latex
\enabletablerowcolor
\begin{table}[H]
	\centering
	\caption{Ejemplo de tablas con colores de filas.}
	\begin{tabular}{ccccc}
		\rowcolor[rgb]{ .749,  .749,  .749} \textbf{Valor A} & \textbf{Valor B} & \textbf{Valor C} & \multicolumn{2}{c}{\textbf{Valor Esperado}} \\
		1     & a     & $3x$  & \multicolumn{2}{c}{Cumple} \\
		2     & b     & $6x$  & \multicolumn{2}{c}{No cumple} \\
		3     & c     & $3x+y$ & \multicolumn{2}{c}{Quizás} \\
		4     & d     & $5\sin x$ & \multicolumn{2}{c}{No} \\
		5     & e     & $0$ & \multicolumn{2}{c}{Sí} \\
	\end{tabular}
\end{table}
\disabletablerowcolor
	
\section{Creando citas}

El template por defecto está configurado para trabajar con citas de la librería \href{https://www.ctan.org/pkg/natbib}{natbib}, y se configuró al estilo \textit{ieeetr}. Puedes usar otros estilos cambiando la configuración \texttt{\textbackslash natbibrefstyle} si es que usas natbib. También se da soporte a las librerías \textbf{bibtex} y \textbf{apacite}, para ello puedes cambiar la configuración \texttt{\textbackslash stylecitereferences}. Una completa guía de estilos la puedes consultar en \url{https://latex.ppizarror.com/doc/bibstylescompared.pdf}. \\

A continuación se detallan algunos links de ayuda para el uso de las referencias:

\begin{itemize}
    \item \href{https://www.bibtex.com/bibliography-styles-numeric-square-brackets}{Galería de estilos numéricos por corchetes}
    \item \href{https://www.bibtex.com/bibliography-styles-author-date}{Galería de estilos por autor/fecha}
    \item \href{https://latex.ppizarror.com/res/guia_basica_referencias_mendeley_v3.pdf}{Guía básica referencias Mendeley}
    \item \href{https://latex.ppizarror.com/doc/bibstylescompared.pdf}{Guía completa de estilos}
\end{itemize}


% ------------------------------------------------------------------------------
% NUEVO CAPÍTULO
% ------------------------------------------------------------------------------
\chapter{El desarrollo de la tesis}

\section{Aquí una nueva sección}

\subsection{Haciendo una tesis como un profesional}

	% Se inserta una imagen flotante en la izquierda del documento con
	% \insertimageleft, al igual que las demás funciones, el primer parámetro
	% es opcional, luego viene la ubicación de la imagen, seguido de la escala
	% (un 27% del ancho de página) y por último su leyenda. Para insertar una
	% imagen flotante en la derecha se utiliza \insertimageright usando los
	% mismos parámetros
	\insertimageleft[\label{img:imagen-izquierda}]{ejemplos/test-image-wrap}{0.27}{Apolo flotando a la izquierda.}

	~ \lipsum[6] \newp

	% Párrafos de ejemplo
	~ \lipsum[115]

	% Agrega una ecuación con leyenda
	\insertequationcaptioned[\label{eqn:formulasinsentido}]{\int_{a}^{b} f(x) \dd{x} = \fracnpartial{f(x)}{x}{\eta} \cdotp \textstyle \sum_{x=a}^{b} f(x)\cancelto{1+\frac{\epsilon}{k}}{\bigp{1+\Delta x}}}{Ecuación sin sentido.}

	% Inserta una definición
	\begin{defn}[ver \cite{einstein}]
		Definición definitiva
		$$\frac{d}{dx}\int_a^xf(y)dy=f(x)$$
	\end{defn}

	\lipsum[115]

	% Inserta un subtítulo sin número
	\subsection{Otros párrafos más normales}

	% Párrafos
	\lipsum[7]

	% Se inserta una ecuación larga con el entorno gathered (1 solo número de ecuación)
	\insertgathered[\label{eqn:eqn-larga}]{
		\lpow{\Lambda}{f} = \frac{L\cdot f}{W} \cdot \frac{\pow{\lpow{Q}{e}}{2}}{8 \pow{\pi}{2} \pow{W}{4} g} + \sum_{i=1}^{l} \frac{f \cdot \bigp{M - d}}{l \cdot W} \cdot \underequal{\frac{\pow{\bigp{\lpow{Q}{e}- i\cdot Q}}{2}}{8 \pow{\pi}{2} \pow{W}{4} g}}{\sim \A}\\
			Q_e = 2.5Q \cdot \int_{0}^{e} V(x) \dd{x} + \aasin{\biggp{1+\frac{1}{1-e}}}
	}

	% Nuevo párrafo
	\lipsum[4]

	% Se inserta un multicols, con esto se pueden escribir párrafos en varias columnas
	\begin{multicols}{2}

		% Párrafo 1
		\lipsum[4]

		% Ecuación encerrada en una caja
		\insertequation[]{ \boxed{f(x) = \fracdpartial{u}{t}} }

		% Párrafo 2 del multicols
		\lipsum[2-3]

	\end{multicols}

\subsection{Ejemplos de inserción de código fuente}

	% A continuación se crea una función auxiliar, esta es una herramienta
	% extremadamente importante y muy útil. Esta función de ejemplo toma dos
	% parámetros, uno es el lenguaje del código fuente, el segundo el
	% identificador en el manual
	\newcommand{\insertsrcmanual}[2]{\href{https://latex.ppizarror.com/informe.html?srctype=#1\#hlp-srccode}{#2}}

	El template permite la inserción de los siguientes lenguajes de programación de forma nativa: \insertsrcmanual{abap}{ABAP}, \insertsrcmanual{ada}{Ada}, \insertsrcmanual{assemblerx64}{Assembler x64}, \insertsrcmanual{assemblerx86}{Assembler x86[masm]}, \insertsrcmanual{awk}{Awk}, \insertsrcmanual{bash}{Bash}, \insertsrcmanual{basic}{Basic}, \insertsrcmanual{c}{C}, \insertsrcmanual{caml}{Caml}, \insertsrcmanual{cmake}{CMake}, \insertsrcmanual{cobol}{Cobol}, \insertsrcmanual{cpp}{C++}, \insertsrcmanual{csharp}{C\#}, \insertsrcmanual{css}{CSS}, \insertsrcmanual{csv}{CSV}, \insertsrcmanual{cuda}{CUDA}, \insertsrcmanual{dart}{Dart}, \insertsrcmanual{docker}{Docker}, \insertsrcmanual{elisp}{Elisp}, \insertsrcmanual{elixir}{Elixir}, \insertsrcmanual{erlang}{Erlang}, \insertsrcmanual{fortran}{Fortran}, \insertsrcmanual{fsharp}{F\#}, \insertsrcmanual{glsl}{GLSL}, \insertsrcmanual{gnuplot}{Gnuplot}, \insertsrcmanual{go}{Go}, \insertsrcmanual{haskell}{Haskell}, \insertsrcmanual{html}{HTML}, \insertsrcmanual{ini}{INI}, \insertsrcmanual{java}{Java}, \insertsrcmanual{javascript}{Javascript}, \insertsrcmanual{json}{JSON}, \insertsrcmanual{julia}{Julia}, \insertsrcmanual{kotlin}{Kotlin}, \insertsrcmanual{latex}{LaTeX}, \insertsrcmanual{lisp}{Lisp}, \insertsrcmanual{llvm}{LLVM}, \insertsrcmanual{lua}{Lua}, \insertsrcmanual{make}{Make}, \insertsrcmanual{maple}{Maple}, \insertsrcmanual{mathematica}{Mathematica}, \insertsrcmanual{matlab}{Matlab}, \insertsrcmanual{mercury}{Mercury}, \insertsrcmanual{modula2}{Modula-2}, \insertsrcmanual{objectivec}{Objective-C}, \insertsrcmanual{octave}{Octave}, \insertsrcmanual{opencl}{OpenCL}, \insertsrcmanual{opensees}{OpenSees}, \insertsrcmanual{pascal}{Pascal}, \insertsrcmanual{perl}{Perl}, \insertsrcmanual{php}{PHP}, \insertsrcmanual{plaintext}{Texto plano}, \insertsrcmanual{postscript}{PostScript}, \insertsrcmanual{powershell}{Powershell}, \insertsrcmanual{prolog}{Prolog}, \insertsrcmanual{promela}{Promela}, \insertsrcmanual{pseudocode}{Pseudocódigo}, \insertsrcmanual{python}{Python}, \insertsrcmanual{qsharp}{Q\#}, \insertsrcmanual{r}{R}, \insertsrcmanual{racket}{Racket}, \insertsrcmanual{reil}{Reil}, \insertsrcmanual{ruby}{Ruby}, \insertsrcmanual{rust}{Rust}, \insertsrcmanual{scala}{Scala}, \insertsrcmanual{scheme}{Scheme}, \insertsrcmanual{scilab}{Scilab}, \insertsrcmanual{simula}{Simula}, \insertsrcmanual{sparql}{SPARQL}, \insertsrcmanual{sql}{SQL}, \insertsrcmanual{swift}{Swift}, \insertsrcmanual{tcl}{TCL}, \insertsrcmanual{vbscript}{VBScript}, \insertsrcmanual{verilog}{Verilog}, \insertsrcmanual{vhdl}{VHDL} y \insertsrcmanual{xml}{XML}. \\
		
	Para insertar un código fuente se debe usar el entorno \texttt{sourcecode}, o el entorno \texttt{sourcecodep} si es que se quiere utilizar parámetros adicionales. A continuación se presenta un ejemplo de inserción de código fuente en Python (Código \ref{codigo-python}), Java y Matlab:




% SUB-SECCIÓN
\subsection{Agregando múltiples imágenes}

	El template ofrece el entorno \href{https://latex.ppizarror.com/informe.html#hlp-images}{images} que permite insertar múltiples imágenes de una manera muy sencilla. Para crear imágenes múltiples se deben usar las siguientes instrucciones:



	Obteniendo así:

	\begin{images}{Ejemplo de imagen múltiple.}
		\addimage{ejemplos/test-image}{width=6.5cm}{Ciudad}
		\addimageanum{ejemplos/test-image-wrap}{height=4cm}
		\imagesnewline
		\addimage{ejemplos/test-image}{width=11cm}{Ciudad más grande}
	\end{images}

% Inserta una sección sin número
\clearpage
\sectionanum{Más ejemplos}

% Inserta un subtítulo sin número
\subsectionanum{Listas y Enumeraciones}

	Hacer listas enumeradas con \LaTeX\ es muy fácil con el template\footnote{También puedes revisar el manual de las enumeraciones en \url{https://latex.ppizarror.com/doc/enumitem.pdf}.}, también puedes revisar el manual \cite{template}, para ello debes usar el comando \texttt{\textbackslash begin\{enumerate\}}, cada elemento comienza por \texttt{\textbackslash item}, resultando así:

	\begin{enumerate}
		\item Grecia
		\item Abracadabra
		\item Manzanas
	\end{enumerate}

	También se puede cambiar el tipo de enumeración, se pueden usar letras, números romanos, entre otros. Esto se logra cambiando el \textbf{label} del objeto \texttt{enumerate}. A continuación se muestra un ejemplo usando letras con el estilo \texttt{\textbackslash alph}\footnote{Con \texttt{\textbackslash Alph} las letras aparecen en mayúscula.}, números romanos con \texttt{\textbackslash roman}\footnote{Con \texttt{\textbackslash Roman} los números romanos salen en mayúscula.} o números griegos con \texttt{\textbackslash greek}\footnote{Una característica propia del template, con \texttt{\textbackslash Greek} las letras griegas están escritas en mayúscula.}:

	\begin{multicols}{3}
		\begin{enumeratebf}[label=\alph*) ] % Fuente en negrita
			\item Peras
			\item Manzanas
			\item Naranjas
		\end{enumeratebf}

		\begin{enumerate}[label=\roman*) ]
			\item Rojo
			\item Café
			\item Morado
		\end{enumerate}
		
		\begin{enumerate}[label=\greek*) ]
			\item Matemáticas
			\item Lenguaje
			\item Filosofía
		\end{enumerate}
	\end{multicols}

	Para hacer listas sin numerar con \LaTeX\ hay que usar el comando \texttt{\textbackslash begin\{itemize\}}, cada elemento empieza por \texttt{\textbackslash item}, resultando:

	\begin{multicols}{3}
		\begin{itemize}[label={--}]
			\item Peras
			\item Manzanas
			\item Naranjas
		\end{itemize}

		\begin{enumerate}[label={*}]
			\item Rojo
			\item Café
			\item Morado
		\end{enumerate}

		\begin{itemize}
			\item Árboles
			\item Pasto
			\item Flores
		\end{itemize}
	\end{multicols}


% ------------------------------------------------------------------------------
% NUEVO CAPÍTULO
% ------------------------------------------------------------------------------
\chapter{Conclusiones}

\lipsum[1] \\

\lipsum[2]


% ------------------------------------------------------------------------------
% REFERENCIAS, revisar configuración \stylecitereferences
% ------------------------------------------------------------------------------
\bibliography{library}


% ------------------------------------------------------------------------------
% ANEXO
% Existe adicionalmente el entorno \begin{appendixd} que permite insertar
% \chapter y el entorno \begin{appendixdtitle}[style1] (4 estilos diferentes),
% el cual acepta \chapter y escribe el título de anexos encima
% ------------------------------------------------------------------------------
\begin{appendixs}
	
	\section{Cálculos realizados}

	\subsection{Metodología}
	\lipsum[1-2]

	% Imagen, se numerará automáticamente con la letra del anexo según
	% la configuración \appendixindepobjnum
	\insertimage[\label{img:anexo-2}]{ejemplos/test-image.png}{scale=0.25}{Imagen en anexo.}

	\subsection{Resultados}
	\lipsum[10]

	% Tablas
	\enabletablerowcolor[2] % Activa el color de celda
	\begin{table}[H]
		\begin{threeparttable}
		\centering
		\caption{Tabla de cálculo.}
		\begin{tabular}{cccC{4cm}}
			\hline
			\textbf{Elemento} & $\epsilon_i$ & \textbf{Valor} & \textbf{Descripción} \bigstrut \\
			\hline
			A     & 10    & 3,14$\pi$ & Valor muy interesante\tnote{a} \\
			B     & 20    & 6 & Segundo elemento \\
			C     & 30    & 7 & Tercer elemento\tnote{1} \\
			D     & 150    & 10 & Sin descripción \\
			E     & 0    & 0 & Cero \\
			\hline
			\end{tabular}
		\begin{tablenotes}
			\item[a] Este elemento tiene una descripción debajo de la tabla
			\item[1] Más comentarios
		\end{tablenotes}
		\end{threeparttable}
		\label{tab:anexo-1}
	\end{table}
	\disabletablerowcolor % Desactiva el color de celda

\end{appendixs}
 % Ejemplo, se puede borrar

% FIN DEL DOCUMENTO
\end{document}