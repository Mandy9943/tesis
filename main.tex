% CREACIÓN DEL DOCUMENTO
\documentclass[
	spanish, % Idioma: spanish, english, etc.	
	letterpaper, oneside
]{book}

% INFORMACIÓN DEL DOCUMENTO
\def\documenttitle { Optimización en el sistema de tratamiento de agua de la planta de bulbos en Laboratorios AICA UEB mediante electrodesionización}
\def\documentsubtitle {}
\def\degreetitle {
	Trabajo de Diploma para optar por el título académico de
Ingeniero en Automática
	 
}

\def\universityname {Universidad Tecnológica de La Habana}
\def\universityfaculty {``José Antonio Echeverría``}
\def\universitydepartment {Facultad de Ingeniería Automática y Biomédica}
\def\universitydepartmentimage {departamentos/cujae}
\def\universitydepartmentimagecfg {height=3cm}
\def\universitylocation {La Habana, Cuba}
 
% INTEGRANTES, PROFESORES Y FECHAS
\def\documentauthor {Armando Cesar Martin Calderón}
\def\documentdate {\the\year}
\def\portrait {

	\begin{center}
		\insertimageboxed[]{departamentos/cujae}{scale=1}{0}{}
% 		\vspace{-2cm} ~ \\
% 		\textbf{Universidad Tecnológica de La Habana}\\
% 		\vspace{-1cm} ~ \\
% \textbf{José Antonio Echeverría}\\
% 		\vspace{-1cm} ~ \\
% 		\textcolor{colorCujae}{\large\textbf{cujae}}\\
\vspace{-2cm} ~ \\
		\Large{ \textbf{Facultad de Ingeniería } }\\
		\vspace{-1.5cm} ~ \\
		\Large{ \textbf{Automática y Biomédica}} \\
		\vspace{0.7cm} 
		\Large{ \textbf{\degreetitle}} ~ 
	\LARGE{	\textbf{\documenttitle}}  \\
	\vspace{0.7cm} 
	\Large	{\textbf{Autor}} \\
		\documentauthor \\
		\vspace{0.7cm} 

		\textbf{Tutores} \\
	Ing. Amanda Martí Coll \\
		Ing. Rosaine Ayala Gispert \\
		\vspace{0.7cm} 

		\universitylocation \\
		Junio, 2023
	
	\end{center}
}
\def\abstracttable {
	\begin{tabular}{l}
		
	\end{tabular}
}
 
% IMPORTACIÓN DEL TEMPLATE
\input{template}

% Nuevo entorno de tabla personalizado
\newenvironment{mytable}[3]{% Aceptar dos argumentos: caption y label
\setstretch{1} 
	% \label{#3}\\% Usar el argumento #3 como label
    \renewcommand{\arraystretch}{2} % Ajustar padding en eje y
    \setlength{\tabcolsep}{0.45cm} % Ajustar padding en eje x
	\rowcolors{1}{}{gray!10} % Colores intercalados, ajusta "gray!10" según tu preferencia de color
    \begin{longtable} {|>{\raggedright\arraybackslash}m{#1} |>{\raggedright\arraybackslash}m{#1}|}
    \caption{#2}\label{#3} \\% Usar el argumento #2 como \caption
}{% 

\end{longtable}
\setstretch{\documentinterline} 

}
% Nuevo entorno de tabla personalizado
\newenvironment{mytableCols}[3]{% Aceptar dos argumentos: caption y label
\setstretch{1} 
	% \label{#3}\\% Usar el argumento #3 como label
    \renewcommand{\arraystretch}{2} % Ajustar padding en eje y
    \setlength{\tabcolsep}{0.45cm} % Ajustar padding en eje x
	\rowcolors{1}{}{gray!10} % Colores intercalados, ajusta "gray!10" según tu preferencia de color
    \begin{longtable} {#1}
    \caption{#2}\label{#3} \\% Usar el argumento #2 como caption
}{% 

\end{longtable}
\setstretch{\documentinterline} 

}

\titlespacing*{\section}{0pt}{-8pt}{-5px}
\titlespacing*{\subsection}{0pt}{-8pt}{-5px}
\titlespacing*{\subsubsection}{0pt}{-8pt}{-5px}
\titlespacing*{\subsubsubsection}{0pt}{-8pt}{-5px}

% INICIO DE LAS PÁGINAS
\begin{document}

% PORTADA
\templatePortrait

% CONFIGURACIÓN DE PÁGINA Y ENCABEZADOS
\templatePagecfg

\newcommand{\keywords}[1]{\par\noindent #1}
\newcommand{\abstracttext}[1]{\par #1}

% Declaración de autoría
\newpage
\section*{Declaración de autoría}
Por este medio doy a conocer que soy el único autor de este trabajo y
autorizo a la Facultad de Ingeniería Automática y Biomédica, a la Universidad
Tecnológica de La Habana (CUJAE) y a los Laboratorios Farmacéuticos AICA a
que hagan uso del mismo para futuras inversiones en nuestro país.\\
Como constancia firmo la presente a los 9 días del mes de junio del año
2023.



\vspace{2cm}
\begin{center}
    \rule{6cm}{0.4pt}\\
    \vspace{0.5cm}
    Armando Cesar Martin Calderón \\
    \vspace*{3cm}
    \rule{6cm}{0.4pt}
    \hspace*{2cm}
    \rule{6cm}{0.4pt} \\
    \vspace{0.5cm}
    Tutora:	Ing. Amanda Martí Coll
    \hspace*{2cm}
    Tutora: Ing. Rosaine Ayala Gispert
\end{center}



% DEDICATORIA
\begin{dedicatory}
    A mi querido despertador, por ser el compañero de batalla en las madrugadas de estudio. Tus estridentes alarmas y tus intentos incansables de sacarme de la cama han sido fundamentales para que aproveche al máximo cada minuto y adelante en mi tesis.

    A mi fiel cafetera, por ser la fuente inagotable de energía en mis largas noches de investigación. Tus deliciosas dosis de cafeína han sido el combustible que me ha mantenido despierto y concentrado, incluso cuando la conexión a internet era un obstáculo.

    A mi lista de reproducción "Modo Nocturno", por llenar mis horas de estudio con melodías motivadoras y canciones pegajosas. Tú has sido mi fiel acompañante, amenizando el ambiente y dándome ese impulso extra para seguir adelante.

    A la luz tenue de mi lámpara, por ser mi aliada en las horas nocturnas de lectura y escritura. Tu suave brillo ha creado un ambiente acogedor y tranquilo, permitiéndome sumergirme en el mundo de la investigación y la escritura.

    Y a la noche misma, por brindarme la tranquilidad y la calma necesarias para concentrarme en mi tesis. Aprovechando la conexión más estable en esas horas, pude avanzar significativamente en mi trabajo y superar los desafíos que la conexión diurna presentaba.

    Esta dedicatoria es un homenaje a esos elementos que me han acompañado en las noches de estudio y han sido fundamentales para avanzar en mi tesis. Sin ustedes, mi experiencia de investigación y escritura no habría sido tan memorable ni efectiva.
\end{dedicatory}

% AGRADECIMIENTOS
\begin{acknowledgments}
    Primero y ante todo, quiero expresar mi más profundo agradecimiento a mi familia, quienes siempre han sido mi faro en la vida. A mis padres, por su incondicional amor, apoyo y enseñanzas, que me han guiado hasta este punto en mi vida. A mis hermanas en especial a mi hermana mayor, que ha sido un pilar de apoyo, sabiduría y amor incondicional. Su presencia ha sido esencial en mi camino y me ha inspirado a ser una mejor persona cada día.

    A mis amigos, que se convirtieron en hermanos, gracias por compartir conmigo momentos de risas y lágrimas, por estar a mi lado en los momentos de tensión y alivio, y por ser mi red de apoyo durante este arduo camino. No tengo palabras para expresar cuánto valoro cada uno de ustedes. Mención especial para los tanques de ``Cuestionarios los viernes`` y a los de siempre, a mis hermanos de la Lenin.

    Quiero expresar mi más sincero agradecimiento a mis dos tutoras, Ing. Amanda Martí Coll e Ing. Rosaine Ayala Gispert, quienes han sido mis mentores y guías en este viaje académico. La dedicación y apoyo de la Ing. Rosaine Ayala Gispert durante el proceso en el centro de trabajo han sido invaluables, y la ayuda de la Ing. Amanda Martí Coll en la metodología ha sido crucial para el desarrollo y conclusión de esta investigación. Les estaré eternamente agradecido por su apoyo y confianza en mis habilidades.

    Por último, pero no menos importante, deseo agradecer a todas las personas e instituciones que de alguna manera contribuyeron a la realización de esta investigación, aportando recursos, conocimientos o simplemente un espacio donde reflexionar y crecer.

    Este logro no es solo mío, sino de todos los que me han acompañado en este viaje. Con profundo amor y gratitud, dedico esta tesis a cada uno de ustedes.
\end{acknowledgments}


% Resumen en español
\newpage
\section*{Resumen}
\abstracttext{
    Este trabajo de diploma se centra en la optimización del sistema de tratamiento de agua de la planta de bulbos en Laboratorios AICA UEB, mediante la introducción de la tecnología de electrodesionización (EDI). Actualmente, el tratamiento de agua es crucial en la industria farmacéutica, sin embargo, los métodos convencionales presentan desafíos en cuanto a la eficiencia y la calidad del agua. La implementación de la EDI promete superar estos desafíos proporcionando agua de alta pureza de manera constante.

    El objetivo principal de este estudio es proponer una solución de optimización del sistema existente a través de la implementación de una EDI, la cual implica un análisis detallado de la instrumentación necesaria, desde los sensores hasta los sistemas de control. Además, se propone un esquema general de configuración de EDI.

    Los resultados obtenidos sugieren que la implementación de la EDI no solo mejorará la eficiencia del proceso de tratamiento de agua, sino que también reducirá los costos operativos y de mantenimiento, mientras cumple con los requisitos y regulaciones estrictas aplicables al agua en la industria farmacéutica. En conclusión, este estudio sienta las bases para la implementación de la EDI en la planta de bulbos, promoviendo una mejora significativa en el tratamiento de agua.
}

\section*{Palabras claves}
\keywords{Electrodesionización (EDI), Planta de tratamiento de agua, Industria farmacéutica, AICA, Agua purificada (PW), Agua para inyección (WFI), Conductividad del agua, Ósmosis inversa  }

% Resumen en ingles
\newpage
\section*{Abstract}
\abstracttext{
    This diploma work focuses on the optimization of the water treatment system of the bulb plant at AICA UEB Laboratories by introducing electrodeionization (EDI) technology. Currently, water treatment is crucial in the pharmaceutical industry, however, conventional methods present challenges in terms of efficiency and water quality. The implementation of EDI promises to overcome these challenges by providing consistently high purity water.

    The main objective of this study is to propose a solution to optimize the existing system through the implementation of EDI, which involves a detailed analysis of the necessary instrumentation, from sensors to control systems. In addition, a general EDI configuration scheme is proposed.
    
    The results obtained suggest that the implementation of EDI will not only improve the efficiency of the water treatment process, but also reduce operating and maintenance costs, while complying with the strict requirements and regulations applicable to water in the pharmaceutical industry. In conclusion, this study lays the groundwork for the implementation of EDI in the bulb plant, promoting a significant improvement in water treatment.}

\section*{keywords}
\keywords{
    Electrodesionization (EDI), Water treatment plant,
    Pharmaceutical industry, AICA, Purified water (PW),
    Water for injection (WFI), Water conductivity, Reverse osmosis
}




% TABLA DE CONTENIDOS - ÍNDICE
\templateIndex

% CONFIGURACIONES FINALES
\templateFinalcfg




% ======================= INICIO DEL DOCUMENTO =======================
% % INTRODUCCION
\chapter*{Introducción}
\addcontentsline{toc}{chapter}{Introducción}
La calidad del agua en la industria farmacéutica es de suma importancia,
ya que influye directamente en la calidad y seguridad de los productos
farmacéuticos, como los inyectables. La presente tesis se enfoca en la
implementación de un Electrodesionizador (EDI) en una planta
de tratamiento de agua de la industria farmacéutica, con el objetivo de
mejorar la calidad del agua purificada (PW) y el agua para inyección.
A continuación, se presenta el contexto y la justificación de este
proyecto, así como el problema a resolver, la hipótesis, el objeto
de estudio, el campo de acción, los objetivos generales y específicos,
y la estructura por capítulos.\\

% secciones
\section{Contexto y justificación}
La industria farmacéutica desempeña un papel fundamental en la promoción y protección de la salud pública, ya que proporciona medicamentos y productos farmacéuticos que salvan vidas y mejoran la calidad de vida de millones de personas en todo el mundo. La producción de estos productos requiere la utilización de agua de alta calidad, especialmente en la fabricación de soluciones inyectables y otros medicamentos críticos. La calidad del agua utilizada en los procesos de fabricación de medicamentos es un factor esencial para garantizar la seguridad, eficacia y estabilidad de los productos finales.\\

La planta de tratamiento de agua de la empresa AICA, dedicada a la industria farmacéutica, actualmente utiliza un sistema de ósmosis inversa (OI) de doble etapa para la producción de agua purificada (PW). Sin embargo, la planta enfrenta desafíos en la estabilización de los parámetros de calidad del agua, lo que puede afectar negativamente la producción y la calidad de los medicamentos. Este problema se debe, en parte, a la inestabilidad de la calidad del agua potable proveniente del acueducto y otros factores externos.\\

La implementación de un equipo de Electrodesionización (EDI) como etapa posterior al proceso de OI de doble etapa tiene el potencial de mejorar significativamente la calidad del agua purificada y el agua para inyección, al estabilizar los parámetros de calidad y reducir la conductividad. El EDI es una tecnología de purificación de agua que combina procesos de intercambio iónico y electrodiálisis, eliminando efectivamente las partículas inorgánicas disueltas y reduciendo la concentración de iones en el agua.\\

La justificación para esta investigación radica en la importancia de garantizar la calidad del agua en la industria farmacéutica y la necesidad de encontrar soluciones efectivas y sostenibles para mejorar y estabilizar la calidad del agua en el proceso de producción. La implementación exitosa del EDI en la planta de tratamiento de agua de AICA podría resultar en una producción más eficiente y segura de medicamentos, reduciendo el riesgo de contaminación y garantizando el cumplimiento de los estándares regulatorios y de calidad. Además, la experiencia y el conocimiento adquiridos en este proyecto podrían ser aplicables a otras plantas de tratamiento de agua y procesos industriales, contribuyendo al avance del campo de la ingeniería automática y la optimización de procesos en la industria farmacéutica.\\

\section*{Situación problemática}
La planta de AICA enfrenta inestabilidad en los parámetros de calidad del agua purificada y el agua para inyección debido a la variabilidad en la calidad del agua potable y otros factores. Esta situación afecta la producción y calidad de los productos farmacéuticos.\\
\section*{Problema a resolver}
El problema a resolver es cómo mejorar y estabilizar la calidad del agua purificada y el agua para inyección en la planta de AICA mediante la incorporación de un equipo de Electrodesionización (EDI) y posibles modificaciones en el sistema de control e instrumentación.
\section{ Hipótesis}
La implementación del EDI como etapa posterior al proceso de OI de doble etapa mejorará significativamente la calidad y estabilidad del agua purificada y el agua para inyección en la planta de AICA.
\section{Objeto de estudio}
El objeto de estudio es el proceso de tratamiento de agua en la planta de AICA y la implementación del EDI como una solución para mejorar y estabilizar la calidad del agua.
\textbf{Campo de acción:}\\
El campo de acción se centra en la evaluación y propuesta de
modificación del sistema de tratamiento de agua en la planta de
AICA, incluyendo la implementación del EDI y propuestas
de el sistema de control e instrumentación.
\section*{Objetivo general}
El objetivo general es mejorar y estabilizar la calidad del 
agua purificada y el agua para inyección en la planta de AICA 
mediante la implementación del EDI y propuesta en el sistema de control e instrumentación.
\section{Objetivos específicos}
\begin{enumerate}
    \item Evaluar la situación actual del proceso de tratamiento de agua en la planta de AICA.
    \item Investigar y proponer la implementación del EDI como etapa posterior al proceso de OI de doble etapa.
    \item Analizar los requisitos técnicos, económicos y regulatorios para la implementación del EDI en la planta.
    \item Proponer modificaciones en el sistema de control e instrumentación existente para la integración del EDI.
\end{enumerate}
\textbf{Alcance y limitaciones:}\\
El alcance de esta tesis incluye la evaluación del proceso de tratamiento de agua en la planta de AICA, la propuesta de implementación del EDI y posibles ajustes en el sistema de control e instrumentación existente. Las limitaciones pueden incluir la disponibilidad de información técnica, económica y regulatoria específica, así como restricciones en el acceso a la planta y los equipos involucrados en el proceso.


\section*{Metodología}
Para abordar el problema planteado en esta tesis, se seguirá una metodología estructurada en diversas etapas, que permitirá una aproximación sistemática al objetivo general. Las etapas de la metodología propuesta son las siguientes:
\begin{itemize}
    \item Diagnóstico del proceso actual: En esta etapa se analizará el proceso de tratamiento de agua en la planta de AICA, identificando las variables críticas, inestabilidades y limitaciones en la calidad del agua purificada y el agua para inyección. Se recopilarán y analizarán datos de producción, calidad del agua y rendimiento de los equipos involucrados en el proceso.
    \item Revisión bibliográfica y análisis del estado del arte: Se llevará a cabo una revisión exhaustiva de la literatura científica y técnica relacionada con el tratamiento de agua en la industria farmacéutica, el proceso de OI de doble etapa y la tecnología de EDI. Se buscarán estudios de caso, investigaciones y experiencias previas en la implementación de EDI en plantas similares para identificar las mejores prácticas y lecciones aprendidas.
    \item Propuesta de implementación del EDI: Basándose en el diagnóstico del proceso actual y el análisis del estado del arte, se propondrá la implementación del EDI como etapa posterior al proceso de OI de doble etapa en la planta de AICA. Se definirán los requisitos técnicos, de instrumentación y de control para la integración del EDI en el proceso existente.
    \item Análisis de costos y beneficios: Se llevará a cabo un análisis económico para estimar los costos asociados con la implementación del EDI y las posibles modificaciones en el sistema de control e instrumentación. Además, se evaluarán los beneficios esperados en términos de mejora en la calidad y estabilidad del agua, así como posibles ahorros en el consumo de energía y recursos.
    \item Evaluación de requisitos regulatorios y de cumplimiento: Se investigarán los requisitos legales y regulatorios aplicables a la implementación del EDI en la planta de AICA, así como las normas y estándares de la industria farmacéutica relacionados con el tratamiento de agua y la calidad del agua purificada y el agua para inyección.
    \item Desarrollo de modificaciones en el sistema de control e instrumentación: Basándose en la propuesta de implementación del EDI y los requisitos identificados, se desarrollarán las modificaciones necesarias en el sistema de control e instrumentación existente, incluyendo la actualización del HMI y la programación del PLC.
\end{itemize}







% \section*{Estructura del Contenido}

\begin{itemize}

    \item \textbf{Capítulo 1: Estado del arte y descripción del proceso} \\ Este capítulo presenta un estudio exhaustivo sobre los sistemas de tratamiento de agua en la industria farmacéutica, detallando su importancia, clasificación, regulaciones aplicables, impurezas presentes, variables críticas y las distintas etapas y variantes de su tratamiento. También se discute la evolución histórica y las innovaciones actuales en este campo.

    \item \textbf{Capítulo 2: Introducción y fundamentos de la Electrodesionización (EDI)} \\ En el segundo capítulo se introducen los principios fundamentales de la Electrodesionización (EDI), sus componentes y diseño, los beneficios y desafíos que presenta, así como sus aplicaciones en la industria farmacéutica.

    \item \textbf{Capítulo 3: Análisis de la instrumentación} \\ El tercer capítulo se centra en un levantamiento instrumental completo del sistema de tratamiento de agua. Esto incluye la descripción de diversos sensores y dispositivos, como sensores de conductividad, pH, temperatura, REDOX, flujo, nivel y presión, entre otros. Además, se presenta una propuesta para la implementación de tecnologías de EDI.

    \item \textbf{Capítulo 4: Propuesta de Implementación de EDI} \\ En el cuarto capítulo, se presenta una propuesta completa para la implementación del sistema de EDI. Esta incluye una discusión del sistema de control, una propuesta de SCADA y los detalles de la instalación del EDI.

    \item \textbf{Capítulo 5: Análisis de costos y beneficios} \\ El quinto capítulo presenta un análisis económico completo del proyecto, incluyendo los costos asociados con la implementación de EDI y un análisis detallado de los beneficios esperados.

\end{itemize}


% % % ESTADO DEL ARTE
% % % Capitulo 1
\chapter{Estado del arte y descripción del proceso}
\vspace{-2cm} 
La industria farmacéutica es un sector crítico para la salud y el bienestar de la sociedad, y la calidad del agua utilizada
en los procesos de producción desempeña un papel fundamental en la garantía de la seguridad y eficacia de los productos farmacéuticos. En este capítulo,
se realizará una revisión exhaustiva de la literatura relacionada con los sistemas de tratamiento de agua en la industria farmacéutica, abordando temas
como la importancia del tratamiento de agua, las clasificaciones y requisitos regulatorios, y las tecnologías de tratamiento empleadas \cite{juanantoniodelacuerdaImportanciaAguaIndustria2021}.
 
Esta revisión tiene como objetivo proporcionar un panorama completo del estado actual del conocimiento en este campo, así como identificar las tendencias y enfoques de investigación que podrían dar lugar a mejoras en los sistemas de tratamiento de agua en el futuro. Al comprender en profundidad el contexto y las consideraciones clave en la purificación del agua farmacéutica, se sentarán las bases para una discusión informada sobre la propuesta de incorporar un electrodesionizador (EDI) en el sistema de ósmosis inversa de la planta de AICA, como se detalla en los capítulos posteriores.

\section{Sistemas de tratamiento de agua en la industria farmacéutica}
Los sistemas de tratamiento de agua desempeñan un papel crucial en la industria farmacéutica al 
garantizar la calidad y pureza del agua utilizada en los procesos de fabricación. 
Estos sistemas están diseñados para eliminar impurezas, microorganismos y productos 
químicos indeseables, asegurando que el agua cumpla con los estándares regulatorios
 y las especificaciones de la industria. La implementación adecuada de estos sistemas asegura 
 que el agua utilizada 
en la producción farmacéutica sea segura, confiable y cumpla con los
 requisitos de calidad necesarios para la fabricación de medicamentos.

\subsection{ Importancia del tratamiento de agua en la industria farmacéutica}
El agua es un recurso indispensable en la industria farmacéutica debido a su amplia utilización en múltiples procesos, tales como la producción de medicamentos, la limpieza de equipos, la fabricación de soluciones y reactivos, y la generación de vapor, entre otros. Dada su relevancia, el tratamiento de agua en este sector es de suma importancia para garantizar la calidad, seguridad y eficacia de los productos farmacéuticos. A continuación, se detallan varias razones que explican la importancia del tratamiento de agua en la industria farmacéutica.\\

\textbf{Calidad del producto:} El agua utilizada en la producción de medicamentos debe cumplir con estándares estrictos de calidad y pureza, ya que su presencia en la composición de los productos puede afectar significativamente su estabilidad, potencia y seguridad. Por ejemplo, la presencia de impurezas en el agua, como iones metálicos, microorganismos o productos químicos, puede reaccionar con los ingredientes activos y excipientes de los medicamentos, alterando sus propiedades y generando efectos adversos en los pacientes.\\

\textbf{Regulaciones y normativas:} Las agencias reguladoras de todo el mundo, como la FDA (Administración de Alimentos y Medicamentos de EE. UU.) y la EMA (Agencia Europea de Medicamentos), establecen requisitos rigurosos y específicos en cuanto a la calidad del agua empleada en la producción farmacéutica. Estas regulaciones tienen como objetivo garantizar que el agua utilizada cumpla con ciertos niveles de pureza y seguridad, y que los sistemas de tratamiento de agua sean adecuados y efectivos para garantizar la calidad del producto final.\\

\textbf{Control de contaminación y biofilm:} La proliferación de microorganismos y la formación de biofilm en los sistemas de tratamiento de agua pueden tener consecuencias negativas para la calidad de los productos farmacéuticos. Un tratamiento de agua eficiente debe eliminar o reducir al mínimo la presencia de microorganismos y prevenir la formación de biofilm en las superficies de los equipos y tuberías. De esta manera, se asegura un ambiente adecuado para la producción de medicamentos y se evita la contaminación cruzada.\\

\textbf{Eficiencia en los procesos:} Un sistema de tratamiento de agua eficiente y bien diseñado puede optimizar los procesos de producción y reducir los costos operativos. El uso de tecnologías avanzadas, como la ósmosis inversa y la electrodeionización (EDI), permite obtener agua de alta calidad y pureza, lo que a su vez disminuye la necesidad de tratamientos adicionales y reduce el consumo de reactivos y energía.\\

\textbf{Responsabilidad medioambiental:} La industria farmacéutica tiene una responsabilidad ética y legal de minimizar su impacto ambiental. El tratamiento adecuado del agua permite reducir la cantidad de contaminantes y sustancias químicas liberadas al medio ambiente y optimizar el uso de los recursos hídricos. Además, las tecnologías de tratamiento de agua más avanzadas pueden contribuir a la reducción del consumo energético y la generación de residuos.\\

En resumen, el tratamiento de agua en la industria farmacéutica es fundamental para garantizar la calidad, seguridad y eficacia de los productos, cumplir con las regulaciones y normativas vigentes, controlar la contaminación y la formación de biofilm, optimizar la eficiencia en los procesos y reducir el impacto medioambiental.\\

El tratamiento adecuado del agua en la industria farmacéutica no sólo garantiza que se cumplan los requisitos de calidad y pureza del agua, sino que también contribuye a la prevención de problemas asociados con la presencia de impurezas y contaminantes. Por lo tanto, es fundamental que las empresas farmacéuticas inviertan en tecnologías de tratamiento de agua apropiadas y en la implementación de sistemas de control y monitoreo efectivos.\\ 


\subsection{Tipos y clasificaciones del agua}

El agua es un componente fundamental en la industria farmacéutica, y su calidad y
 pureza son aspectos críticos para garantizar la seguridad y eficacia de los productos.
 Dependiendo de su uso y aplicación, existen diferentes tipos y clasificaciones de agua en la industria farmacéutica. 
 A continuación, se presentan las categorías más comunes \cite{setaphtTratamientosAguaPara}:\\

\textbf{Agua purificada (PW):} Es el tipo básico de agua utilizada en la industria farmacéutica y se obtiene a través de procesos como ósmosis inversa, destilación, intercambio iónico o filtración. La calidad del agua purificada es menor que la del agua para inyección (WFI), pero es adecuada para la fabricación de productos no parenterales y para su uso en procesos de limpieza.

\textbf{Agua para inyección (WFI):} Es un tipo de agua de alta pureza que se utiliza en la fabricación de productos parenterales, es decir, aquellos que se administran por vías como intravenosa, intramuscular o subcutánea. La calidad del WFI es superior a la del agua purificada, y se obtiene mediante procesos de destilación, ósmosis inversa o por una combinación de ambos métodos.

\textbf{Agua altamente purificada (HPW):} Este tipo de agua tiene una calidad intermedia entre el agua purificada y el WFI. Se utiliza en ciertas aplicaciones farmacéuticas donde se requiere un nivel de pureza más elevado que el del agua purificada, pero no se necesita llegar al grado de pureza del WFI.

\textbf{Agua estéril:} Es agua que ha sido sometida a un proceso de esterilización, como la filtración estéril o la autoclave, para eliminar cualquier microorganismo presente. El agua estéril se utiliza en aplicaciones específicas, como en la fabricación de productos estériles o en procesos de limpieza y desinfección que requieren la eliminación de microorganismos.

Cabe destacar que las regulaciones y normativas, como las establecidas por la Farmacopea de Estados Unidos (USP), la Farmacopea Europea (EP) y la Organización Mundial de la Salud (OMS), definen los requisitos de calidad y las especificaciones para cada tipo de agua en la industria farmacéutica. Estas especificaciones incluyen parámetros como la conductividad, el pH, la presencia de sustancias orgánicas, inorgánicas y microbiológicas, entre otros.

\subsection{ Requisitos y regulaciones aplicables al agua}

La calidad del agua utilizada en la industria farmacéutica está sujeta a una serie de requisitos y regulaciones establecidos por diversas entidades y organismos a nivel nacional e internacional. Estas regulaciones aseguran que el agua cumpla con los estándares de calidad necesarios para garantizar la seguridad y eficacia de los productos farmacéuticos. Algunas de las principales regulaciones y requisitos aplicables al agua en la industria farmacéutica incluyen:

\textbf{Farmacopeas:} Las farmacopeas son documentos oficiales que contienen las especificaciones técnicas y
requisitos de calidad para sustancias y productos farmacéuticos, incluidos los diferentes
tipos de agua. Entre las farmacopeas más reconocidas a nivel mundial se encuentran la
Farmacopea de Estados Unidos (USP), la Farmacopea Europea (EP) y la Farmacopea de Japón
(JP). Cada farmacopea establece parámetros específicos de calidad, como la conductividad,
el pH, la presencia de sustancias orgánicas, inorgánicas y microbiológicas, entre otros \cite{farm.veronicamartinezFARMACOPEAS2005}.

\textbf{ Buenas Prácticas de Fabricación (GMP):} Las GMP son normas que establecen los requisitos mínimos que deben cumplir
los procesos de fabricación, control de calidad y distribución de productos farmacéuticos, incluida la gestión del agua.
Estas normas son aplicables a nivel mundial y son emitidas por organismos como la Food and Drug Administration (FDA) en
Estados Unidos, la European Medicines Agency (EMA) en Europa y la Organización Mundial de la Salud (OMS) \cite{ispeGoodManufacturingPractice}.

\textbf{ Directrices y guías técnicas:} Además de las farmacopeas y las GMP, existen directrices y guías técnicas
emitidas por organismos internacionales y nacionales que abordan aspectos específicos relacionados con el agua en
la industria farmacéutica. Estas directrices pueden incluir recomendaciones sobre el diseño y validación de sistemas
de tratamiento de agua, el monitoreo de la calidad del agua y la prevención de la contaminación.

\textbf{ Normativas nacionales y locales:} Cada país puede tener sus propias normativas y requisitos legales
aplicables al agua en la industria farmacéutica. Estas normativas pueden estar en línea con las farmacopeas y
las GMP, pero también pueden incluir requisitos adicionales específicos para cada país o región.


El cumplimiento de estas regulaciones y requisitos garantiza la calidad y seguridad del agua utilizada en la fabricación de productos farmacéuticos y, en última instancia, protege la salud de los pacientes \cite{juanantoniodelacuerdaImportanciaAguaIndustria2021}.

\subsection{Impurezas presentes en el agua}

El agua utilizada en la industria farmacéutica puede contener diversas impurezas, las cuales pueden afectar la calidad, seguridad y eficacia de los productos finales. Estas impurezas pueden clasificarse en tres categorías principales: impurezas inorgánicas, impurezas orgánicas y contaminantes microbiológicos.\\

\textbf{Impurezas inorgánicas: }Incluyen iones metálicos y no metálicos, como calcio, magnesio, sodio, cloruros, sulfatos y silicatos. Estas impurezas pueden afectar la calidad de los productos farmacéuticos al causar cambios en la solubilidad, la estabilidad y la eficacia de los ingredientes activos, así como en la formación de precipitados y la corrosión de equipos y recipientes. Además, algunos iones metálicos, como el hierro, el cobre y el cromo, pueden ser tóxicos y afectar la seguridad de los productos.\\

\textbf{Impurezas orgánicas: }Son compuestos de origen natural o sintético, como ácidos húmicos y fúlvicos, pesticidas, disolventes y productos químicos de desinfección. Las impurezas orgánicas pueden reaccionar con los ingredientes activos y otros excipientes, lo que puede alterar la estabilidad, la eficacia y la liberación de los fármacos. Además, algunos compuestos orgánicos pueden ser tóxicos y afectar la seguridad de los productos farmacéuticos.\\

\textbf{Contaminantes microbiológicos:} Incluyen bacterias, hongos, levaduras, virus y protozoos. La presencia de microorganismos en el agua puede causar la contaminación de los productos farmacéuticos, lo que puede llevar a infecciones y reacciones adversas en los pacientes. Además, algunos microorganismos pueden producir sustancias tóxicas, como endotoxinas y micotoxinas, que pueden afectar la seguridad y eficacia de los productos.\\

El tratamiento adecuado del agua es esencial para eliminar o reducir estas impurezas a niveles aceptables, de acuerdo con las regulaciones y requisitos aplicables en la industria farmacéutica. Un control riguroso de la calidad del agua, así como el uso de tecnologías de purificación adecuadas, como la ósmosis inversa, la desionización y la electrodesionización (EDI), son fundamentales para garantizar la calidad y seguridad de los productos farmacéuticos.\\

\subsection{Variables críticas en la calidad del agua}

El tratamiento y monitoreo de la calidad del agua en la industria farmacéutica requieren un enfoque riguroso y sistemático para garantizar la eliminación efectiva de impurezas y el cumplimiento de los requisitos regulatorios. A continuación, se presentan algunas de las variables críticas que deben considerarse durante el tratamiento y monitoreo del agua:

\textbf{Conductividad eléctrica:} La conductividad eléctrica es una medida de la capacidad del agua para conducir la corriente eléctrica, y
está directamente relacionada con la concentración de iones disueltos en el agua. Un mayor valor de conductividad indica una mayor
concentración de impurezas inorgánicas. El monitoreo de la conductividad es fundamental para evaluar la efectividad de los procesos
de purificación y para asegurar el cumplimiento de los límites establecidos por las regulaciones aplicables \cite{oceanebidaultQueFactoresDeterminan}.

\textbf{Contenido de carbono orgánico total (TOC ):} El TOC es una medida del contenido de carbono en compuestos
orgánicos disueltos en el agua. Un alto nivel de TOC indica una mayor concentración de impurezas orgánicas.
El monitoreo regular del TOC es esencial para garantizar que el agua cumpla con los requisitos de calidad y para evaluar la eficacia
de los procesos de purificación en la eliminación de compuestos orgánicos \cite{oceanebidaultQueFactoresDeterminan}.

\textbf{Conteo microbiano y endotoxinas:} El monitoreo del recuento microbiano y las endotoxinas es fundamental
para controlar la calidad microbiológica del agua y garantizar la seguridad de los productos farmacéuticos.
Los métodos de análisis microbiológico incluyen el recuento en placa, el método de filtración por membrana
y las técnicas de bioluminiscencia. Las endotoxinas, sustancias tóxicas liberadas por bacterias Gram-negativas,
se miden mediante el ensayo de lisado de amebocitos de Limulus (LAL) \cite{oceanebidaultQueFactoresDeterminan}.

\textbf{pH:} El pH es una medida de la acidez o alcalinidad del agua y puede afectar la solubilidad,
la estabilidad y la reactividad de los ingredientes activos y excipientes en los productos farmacéuticos.
El control del pH es esencial para mantener un ambiente adecuado en los sistemas de tratamiento de agua y
garantizar la calidad del agua producida \cite{oceanebidaultQueFactoresDeterminan}.

\textbf{Turbidez:} La turbidez es una medida de la cantidad de partículas en suspensión en el agua,
incluidas partículas inorgánicas, orgánicas y microbiológicas. Un nivel elevado de turbidez puede afectar
la efectividad de los procesos de purificación y el rendimiento de los equipos. La turbidez se mide utilizando
un turbidímetro y se expresa en unidades de turbidez nefelométrica (NTU) \cite{oceanebidaultQueFactoresDeterminan}.

El monitoreo y control de estas variables críticas durante el tratamiento y purificación del agua son fundamentales para garantizar la calidad, seguridad y eficacia de los productos farmacéuticos y cumplir con los requisitos regulatorios aplicables .

\subsection{Evolución histórica de las tecnologías de tratamiento de agua}

La historia del tratamiento de agua en la industria farmacéutica ha experimentado una evolución considerable a lo largo del tiempo. A medida que la industria ha crecido y los requisitos regulatorios han aumentado en complejidad, las tecnologías de tratamiento de agua han seguido mejorando para garantizar la calidad y la seguridad de los productos farmacéuticos. A continuación, se presenta un breve recorrido histórico de las tecnologías de tratamiento de agua en la industria farmacéutica:\\

\textbf{Finales del siglo XIX y principios del siglo XX:}Durante este período, los sistemas de tratamiento de agua se basaban en procesos simples como la sedimentación, la filtración y la desinfección con cloro. Estos métodos eran efectivos para eliminar partículas en suspensión e impurezas microbiológicas, pero no eran capaces de eliminar completamente las impurezas químicas.\\

\textbf{Mitad del siglo XX:}Con el avance de la química y la comprensión de los requisitos de calidad del agua para los productos farmacéuticos, se introdujeron tecnologías más avanzadas de tratamiento de agua, como la desionización y la destilación. La desionización es un proceso que utiliza resinas de intercambio iónico para eliminar iones del agua, mientras que la destilación es un proceso de separación basado en la diferencia de volatilidad entre el agua y las impurezas.\\

\textbf{Décadas de 1960 y 1970:} Durante este período, se desarrolló la tecnología de ósmosis inversa (OI), que utiliza membranas semipermeables para eliminar la mayoría de las impurezas disueltas en el agua, incluidos iones, compuestos orgánicos y partículas en suspensión. La OI ha sido ampliamente adoptada en la industria farmacéutica debido a su eficacia y eficiencia en la producción de agua de alta calidad.\\

\textbf{Década de 1990:} El desarrollo del proceso de desionización electroquímica, también conocido como desionización capacitiva (CDI) o electrodialización reversible (EDR), proporcionó otra opción para el tratamiento de agua en la industria farmacéutica. Estos sistemas utilizan un campo eléctrico para separar y eliminar iones del agua.\\

\textbf{Siglo XXI:} Con el desarrollo de la tecnología de desionización electrodialítica (EDI), se ha logrado combinar las ventajas de la desionización y la ósmosis inversa para producir agua de mayor pureza y a una menor tasa de rechazo. La EDI es una tecnología híbrida que utiliza membranas de intercambio iónico y un campo eléctrico para eliminar iones y otras impurezas del agua. Además, los avances en la instrumentación y el control permiten una monitorización y control en tiempo real de las variables críticas en el tratamiento de agua, lo que mejora aún más la calidad y la eficiencia del proceso.\\


La evolución de las tecnologías de tratamiento de agua en la industria farmacéutica ha sido impulsada por la creciente demanda de productos de alta calidad y la necesidad de cumplir con requisitos regulatorios cada vez más rigrosos. A medida que la industria farmacéutica continúa avanzando, es probable que surjan nuevas tecnologías y enfoques para el tratamiento y monitoreo del agua en el futuro. Algunas áreas de investigación y desarrollo incluyen:\\

\textbf{Nanotecnología:}La aplicación de nanomateriales y nanopartículas en el tratamiento de agua ofrece oportunidades para mejorar la eficiencia de los procesos existentes y desarrollar nuevos enfoques para la eliminación de impurezas. Por ejemplo, las membranas nanocompuestas y las nanopartículas funcionales pueden mejorar la selectividad y la eficiencia de las membranas de ósmosis inversa y EDI.\\

\textbf{Tratamiento biológico:}Los enfoques biológicos, como la utilización de microorganismos para la degradación de contaminantes orgánicos, pueden proporcionar alternativas sostenibles y de bajo costo a las tecnologías convencionales de tratamiento de agua.\\

\textbf{Sistemas avanzados de monitoreo y control:} Los avances en sensores, analítica en línea y tecnologías de control permiten una mejor comprensión y control del proceso de tratamiento de agua en tiempo real. Esto puede llevar a una mayor eficiencia y garantizar una calidad de agua más consistente.\\

\textbf{Integración de sistemas y automatización:} La integración de diferentes tecnologías de tratamiento de agua y la automatización de los sistemas de control pueden mejorar la eficiencia general del proceso y reducir los costos de operación y mantenimiento.\\


En resumen, la evolución histórica de las tecnologías de tratamiento de agua en la industria farmacéutica ha sido impulsada por la necesidad de garantizar la calidad y la seguridad de los productos y cumplir con requisitos regulatorios cada vez más estrictos. A medida que la industria farmacéutica sigue avanzando, es probable que surjan nuevas tecnologías y enfoques para el tratamiento y monitoreo del agua, lo que permitirá seguir mejorando la calidad y la eficiencia de los procesos.

% \subsection{Tecnologías actuales y enfoques de investigación en sistemas de tratamiento de agua para la industria farmacéutica}

\input{tesis/estado_del_arte/revision_de_literatura/tecnologias_actuales/etapas.tex}
\input{tesis/estado_del_arte/revision_de_literatura/tecnologias_actuales/variantes.tex}
\input{tesis/estado_del_arte/revision_de_literatura/tecnologias_actuales/innovaciones.tex}

\section{Descripción del proceso actual}
La planta de AICA cuenta con un proceso integral de tratamiento y purificación de agua para abastecer
a sus instalaciones con agua de alta calidad y pureza. Este proceso es esencial para garantizar
el cumplimiento de las normativas y estándares aplicables en la industria farmacéutica y biotecnológica.
A continuación, se proporcionará una descripción detallada de las distintas etapas y componentes del
proceso actual en la planta de AICA, desde la captación del agua hasta su punto antes de la distribución y uso en las distintas áreas de producción.

Cabe destacar que en el Anexo \ref{sec:anexoA} se encuentra un diagrama
 P\&ID del sistema de ósmosis inversa en la planta de bulbos en cuestión, el cual constituye un 
 elemento clave dentro de este proceso de tratamiento y purificación de agua. 
 El P\&ID brinda una representación gráfica detallada de las distintas etapas 
 y componentes involucrados en el sistema implementado en 
 la planta de AICA.

% Sistema no tecnológico
\subsection*{Sistema Tecnológico y sus plantas de tratamiento}

El Sistema Tecnológico es el área de interés para esta investigación y se compone de dos plantas de tratamiento de agua.
La primera planta se dedica a la producción de ampolletas, mientras que la segunda planta se encarga de la producción de bulbos, esta última es en la que centra el estudio.



% Almacenamiento y bombeo del agua potable
\subsection*{Almacenamiento y bombeo del agua potable}

El Sistema de Tratamiento de Agua de Bulbos en Laboratorios AICA$^+$ se encarga de garantizar la eficiencia y calidad de los
diferentes tipos de aguas farmacéuticas, como el agua purificada y destilada, que se utilizan en la planta de producción de inyectables.
El proceso comienza con el almacenamiento del agua potable procedente del acueducto en dos cisternas con capacidades de 900 y 700 m$^3$ .
Posteriormente, el agua cruda es bombeada a través de las bombas de la estación de hidroneumáticos hacia las líneas de Servicios Generales y
al Sistema de Tratamiento de Agua, que se divide en dos partes: el Sistema No Tecnológico y el Sistema Tecnológico.


\subsection*{Dosificación de hipoclorito de sodio y filtración}

El agua proveniente de la cisterna llega al sistema de pretratamiento de aguas de Bulbo a una presión entre 4 - 5 bar. En la línea de entrada, se
dosifica hipoclorito de sodio al 3\% para desinfectar el agua y reducir la concentración de bacterias y microorganismos. El sistema de dosificación consta de un
tanque de solución de 50 L y una bomba con capacidad de 1.58 l/h, permitiendo una concentración de cloro residual cercana al 1\%.
Un contador de impulsos acoplado a la línea gobierna esta dosificación, enviando una señal a la bomba cada 100 L de agua, equivalente a 1
impulso. Posteriormente, el agua pasa por un filtro CF-60 de 50 micras, fabricado de acero inoxidable AISI 304, que cumple la función de
filtración y actúa como elemento mezclador después de la dosificación de cloro.

\subsection*{Almacenamiento y monitoreo de parámetros del agua}

Una vez filtrada, el agua sale del filtro CF-60 con un flujo que oscila entre 7-8 m$^3$/h y se almacena en el tanque de almacenamiento de agua potable,
TK-60, con capacidad de 3,000 L. Este tanque sirve como depósito de alimentación para los suavizadores. Se han instalado tomas de muestra antes y
después del filtro para monitorear el pH y el cloro residual del agua. Este monitoreo permite verificar la calidad del agua en esta etapa del proceso y
asegurar que los parámetros se encuentren dentro de los límites aceptables antes de continuar con el proceso de purificación.

% Suavizadores

\input{tesis/estado_del_arte/descripcion_proceso/suavizacion/1suavizacion.tex}
\input{tesis/estado_del_arte/descripcion_proceso/purificacion/1purificacion.tex}


% % Capitulo 2
\chapter{Introducción y fundamentos de la Electrodesionización (EDI)}\label{cap:fundamentosEDI}

La electrodesionización (EDI) es una tecnología que combina la electroquímica y la resina de intercambio iónico
para producir agua ultrapura, que es esencial en una variedad de aplicaciones industriales y de laboratorio.
Desde su invención en la década de 1950, la EDI ha evolucionado para convertirse en una opción preferida para
la purificación de agua, especialmente en industrias que requieren altos estándares de pureza, como la farmacéutica,
la de semiconductores, la de energía y la de alimentos y bebidas. La EDI es especialmente útil en aplicaciones donde se
necesita una desionización continua y sin químicos, y donde la conservación de agua es crucial \cite{alvaradoElectrodeionizationPrinciplesStrategies2014}.

El principio de la EDI se basa en la utilización de corriente eléctrica y resinas de intercambio iónico
para eliminar iones y partículas disueltas en el agua. A diferencia de los métodos tradicionales de desionización,
como la desionización por intercambio iónico, la EDI no requiere el uso de químicos para regenerar las resinas de
intercambio iónico. En su lugar, utiliza corriente eléctrica para regenerar las resinas, lo que permite un proceso
de desionización continua. Esta característica no solo elimina la necesidad de manipular y disponer de productos
químicos dañinos, sino que también mejora la eficiencia del proceso de desionización y reduce el consumo de agua \cite{condorchemUltrapureWaterElectrodeionization2019}.

Este capítulo proporciona una introducción y una descripción detallada de los fundamentos de la EDI. Incluye una
discusión sobre los principios básicos de la EDI, los componentes y el diseño de un sistema de EDI, así como los
beneficios y desafíos asociados con la implementación de la tecnología EDI. El capítulo concluye con una discusión
sobre las aplicaciones de la EDI en la industria farmacéutica, resaltando la importancia de la EDI en la producción
de agua ultrapura para aplicaciones farmacéuticas.

\section{Principios de la EDI}

La Electrodesionización (EDI) es una tecnología que combina métodos físicos y químicos para eliminar iones disueltos
del agua. En el corazón de este proceso se encuentra la resina de intercambio iónico, que actúa como un medio para la
extracción de iones y los electrodos que posibilitan el movimiento de estos \cite{condorchemUltrapureWaterElectrodeionization2019}.

En el intercambio iónico, los iones en el agua son atraídos y retenidos por una red de polímeros
cargados conocida como resina de intercambio iónico. Es un proceso dinámico en el que los iones negativos,
denominados aniones, son atraídos hacia los sitios cargados positivamente en la resina, mientras
que los iones positivos, los cationes, son atraídos hacia los sitios con carga negativa.
El resultado de este intercambio iónico es que los iones disueltos en el agua son efectivamente
atrapados y retenidos en la resina, reduciendo así su concentración en el agua \cite{lenntechElectrodeionizationEDI} \cite{condorchemUltrapureWaterElectrodeionization2019}.

En contrapartida, la electrólisis se utiliza para mover activamente los iones. Este proceso
implica la aplicación de una corriente eléctrica a través de una serie de electrodos, que se
encuentran en los extremos de la celda de EDI. Esta corriente eléctrica instiga la migración
de los iones, con los cationes moviéndose hacia el electrodo negativo y los aniones moviéndose
hacia el electrodo positivo \cite{condorchemUltrapureWaterElectrodeionization2019}.

Un dispositivo EDI típico consta de una cámara que alberga una resina de intercambio iónico,
tanto catiónica fuerte como aniónica fuerte. Esta cámara, o celda, está situada entre una membrana
de intercambio catiónico y una membrana de intercambio aniónico, lo que significa que solo
los iones pueden pasar a través de las membranas \cite{alvaradoElectrodeionizationPrinciplesStrategies2014}.

El agua de alimentación entra en este sistema y fluye a través de la resina de intercambio
iónico. Al mismo tiempo, se aplica una corriente continua externa a través de los electrodos.
Esta corriente continua impulsa a los cationes a moverse hacia el cátodo y a los aniones a
moverse hacia el ánodo \cite{alvaradoElectrodeionizationPrinciplesStrategies2014}.

Finalmente, las membranas de intercambio iónico trabajan para eliminar eléctricamente los iones del
agua de entrada y los transfieren al concentrado. De esta forma, el resultado final es un agua
de alta calidad que ha sido eficientemente desionizada.

\insertimageboxed[\label{fig:EDI}]{EDI}{scale=0.8}{0}{Funcionamiento de un electrodesionizador}

La EDI elimina los iones del agua a la vez que las resinas de intercambio iónico que se contiene entre las membranas se
regeneran con una corriente eléctrica. Esta regeneración electroquímica se sirve de un potencial eléctrico para realizar el
transporte iónico y sustituye a la regeneración química de los sistemas convencionales de intercambio iónico, que, como es conocido,
se verifica mediante ácido y sosa. Dentro del compartimento de alimentación, las resinas de intercambio iónico ayudan en el transporte de
los iones al compartimiento concentrado \cite{alvaradoElectrodeionizationPrinciplesStrategies2014}.

Como el agua va disminuyendo en su concentración de iones, se va produciendo la disociación del agua en la interfase de intercambio
catiónico y aniónico, produciéndose un flujo continuo de hidrógeno y ion hidroxilo. Estos iones actúan como regenerante para las resinas
de intercambio iónico presentes en este compartimento y mantiene las resinas a la salida de éste, en un estado de alta regeneración,
necesario para la producción del agua de alta calidad deseada \cite{alvaradoElectrodeionizationPrinciplesStrategies2014}.


\section{Componentes y diseño de la EDI}
La efectividad y eficiencia de un sistema de EDI dependen en gran medida de su diseño y de los componentes utilizados.
En este sentido, existen varios componentes clave en un sistema de EDI que son fundamentales para su operación.

\subsection{Cámara de dilución y concentración}
Las cámaras de dilución y concentración son un componente crítico en el diseño de la EDI. Estas cámaras permiten la separación
física del agua desionizada del agua concentrada con iones. La cámara de dilución es donde el agua purificada se recoge después
de que los iones disueltos son extraídos, mientras que la cámara de concentración es donde se recogen los iones extraídos.
Esta separación es crucial para mantener la pureza del agua desionizada y para garantizar que los iones disueltos no se
reintroduzcan en el agua. Las cámaras de dilución y concentración están separadas por membranas semipermeables, que permiten
el paso de iones pero restringen el flujo de agua \cite{rasSamplingPreconcentrationTechniques2009} \cite{ruiz-jimenezComparisonMultipleCalibration2020}.

\subsection{Resina de intercambio iónico}
La resina de intercambio iónico es otro componente crítico de un sistema de EDI. La resina actúa como un medio para la
atracción y retención de iones disueltos en el agua. La resina es esencialmente una red de polímeros cargados, con sitios
de intercambio iónico que atraen iones de carga opuesta. Las resinas de intercambio iónico vienen en dos tipos principales:
cationes y aniones, que atraen iones negativos y positivos respectivamente. En un sistema de EDI, se utiliza una mezcla de
resinas de intercambio de cationes y aniones para asegurar la extracción de todos los tipos de iones disueltos \cite{nogueraResinasIntercambioIonico}.

\subsection{Electrodo y membrana}
Los electrodos y las membranas son componentes esenciales en la operación de un sistema de EDI. Los electrodos, situados en
los extremos de la celda de EDI, proporcionan el campo eléctrico que impulsa la migración de iones a través de la celda.
Dependiendo de la carga del ion, los iones disueltos son atraídos hacia el electrodo positivo o negativo. Las membranas,
por otro lado, están diseñadas para permitir el paso de iones, pero no de agua. De esta manera, las membranas facilitan el
movimiento de los iones disueltos hacia la cámara de concentración, mientras que el agua purificada se recoge en la cámara
de dilución \cite{lenntechElectrodeionizationEDI}.

\subsection{Fuente de alimentación}
La fuente de alimentación es otro componente crucial de un sistema de EDI. Proporciona la corriente eléctrica necesaria
para el proceso de electrólisis, que impulsa la migración de iones a través de la celda de EDI. La fuente de alimentación
debe ser capaz de suministrar una corriente eléctrica constante y estable para garantizar una operación eficiente del sistema.


% sección

\section{Tecnologías Alternativas y sus Limitaciones}

En la búsqueda de la mejora continua y optimización de la planta de tratamiento de agua,
es importante considerar las diversas tecnologías alternativas disponibles.
Sin embargo, cada tecnología tiene sus propias limitaciones, algunas de las
cuales pueden no adaptarse a las necesidades y condiciones específicas de nuestra planta.
Las siguientes son algunas de las tecnologías que se han evaluado:

\begin{itemize}
      \item \textbf{Reforzamiento de la Ósmosis Inversa (RO):}  Nuestra planta ya cuenta con
            un sistema de ósmosis inversa de dos etapas que cumple con las necesidades
            básicas de la planta. Sin embargo, incluso con un sistema RO de dos etapas,
            todavía existen limitaciones, especialmente en términos de la eliminación de
            ciertos iones y pequeñas moléculas. Los sistemas RO también son susceptibles a
            la acumulación de sarro y biofilm, lo que puede afectar el rendimiento y
            la vida útil de la membrana.

      \item \textbf{Destilación:}  Aunque la destilación puede ofrecer altos niveles de
            purificación, la energía requerida para este proceso es considerable,
            lo que resulta en costos operativos más altos. Además, la destilación
            no elimina eficientemente algunos contaminantes volátiles que pueden ser arrastrados con el vapor.

      \item \textbf{Desionización (DI): } Los sistemas de DI pueden ser eficientes para
            la eliminación de iones de agua, pero su capacidad para eliminar
            partículas no iónicas, gases disueltos y microorganismos es limitada.
            Además, los cartuchos de DI requieren un reemplazo frecuente, lo que
            implica costos adicionales de operación y mantenimiento.

      \item \textbf{Filtración de Carbón Activado:}  Esta tecnología es efectiva para la
            eliminación de cloro y ciertos otros contaminantes, pero su eficacia es
            limitada cuando se trata de la eliminación de sales disueltas y
            algunos contaminantes orgánicos.

\end{itemize}

Teniendo en cuenta las limitaciones y desafíos presentes en estas tecnologías alternativas, y
dadas las necesidades específicas de nuestra planta de tratamiento de agua, es evidente que se
necesita una solución más eficaz y sostenible. En este contexto, la Electrodesionización (EDI)
emerge como una solución potencialmente superior, debido a su capacidad para superar muchas de las
limitaciones de las tecnologías mencionadas anteriormente.

% sección
\section{Beneficios de la EDI}
La Electrodesionización (EDI) ofrece una variedad de ventajas significativas en comparación con otras tecnologías de
purificación de agua. En este apartado, se detallarán estos beneficios de manera exhaustiva, desde la simplicidad operativa
hasta la eficacia en la eliminación de partículas inorgánicas \cite{condorchemUltrapureWaterElectrodeionization2019}.

Empezando con su operación, la EDI destaca por su simplicidad y continuidad. Dado que esta tecnología combina la electrodiálisis
y el intercambio iónico, permite una producción ininterrumpida de agua de alta pureza. Esto significa que no es necesario
interrumpir el proceso para la regeneración de las resinas, como ocurre con otros métodos de desionización. Esta característica
contribuye a mejorar la eficiencia de los procesos productivos y a minimizar el tiempo de inactividad \cite{lenntechElectrodeionizationEDI}.

Un factor crítico que diferencia a la EDI de otros métodos de purificación es la eliminación casi total del uso de productos
químicos en el proceso de regeneración. A diferencia de los sistemas tradicionales de intercambio iónico, la EDI utiliza una
corriente eléctrica para regenerar las resinas de intercambio iónico. Este enfoque no solo elimina la necesidad de manejar
y almacenar productos químicos peligrosos, sino que también reduce los costos operativos y el impacto ambiental asociado con
la eliminación de productos químicos residuales \cite{lenntechElectrodeionizationEDI}.

Desde la perspectiva operativa y de mantenimiento, la EDI también tiene ventajas económicas significativas. Gracias a su
diseño compacto y a la ausencia de partes móviles, el mantenimiento de los sistemas de EDI es relativamente simple y los
riesgos de averías son bajos. Esta característica se traduce en ahorros en los costos de mantenimiento y en la reducción
del tiempo de inactividad del sistema. Adicionalmente, la EDI se caracteriza por su eficiencia energética, lo cual se refleja
en un menor consumo de energía en comparación con otros métodos de purificación de agua, como la destilación \cite{lenntechElectrodeionizationEDI}.

En lo que respecta al impacto ambiental, la EDI se considera una tecnología ecológica. Al no producir efluentes peligrosos y
al eliminar la necesidad de manejo de productos químicos, se reduce significativamente el riesgo de contaminación ambiental.
Asimismo, no requiere la disposición de resinas de intercambio iónico agotadas, lo que minimiza aún más su huella ambiental.

Finalmente, la EDI es altamente efectiva en la eliminación de partículas inorgánicas disueltas en el agua. Con su capacidad
para eliminar hasta el 99,9\% de los iones presentes en el agua, incluyendo cationes y aniones, ofrece un grado de purificación
que supera a la mayoría de los otros métodos disponibles. Esta efectividad la convierte en una solución de purificación de
agua altamente atractiva para una amplia gama de aplicaciones \cite{lenntechElectrodeionizationEDI}.

\section{Desafíos de la EDI}
A pesar de sus numerosos beneficios, la implementación y operación de la EDI también presentan desafíos.

Uno de los principales desafíos es la necesidad de un pretratamiento del agua de alimentación. La EDI requiere agua de alimentación
de baja conductividad, por lo general proporcionada por la ósmosis inversa (RO). Además, el agua de alimentación debe estar libre
de cloro y otras sustancias oxidantes que pueden dañar las resinas de intercambio iónico y las membranas de la EDI. Por lo tanto,
el diseño del pretratamiento del agua es crucial para el rendimiento de la EDI \cite{lenntechElectrodeionizationEDI}.

Finalmente, la EDI requiere un suministro de energía eléctrica constante para su operación. Cualquier fluctuación en el suministro
de energía puede afectar el rendimiento de la EDI y resultar en una calidad de agua inconsistente. Por lo tanto, un suministro de
energía confiable es esencial para la operación de la EDI \cite{lenntechElectrodeionizationEDI}.

\section{Aplicaciones de la EDI en la industria farmacéutica}
En la industria farmacéutica, la pureza y consistencia del agua utilizada en los procesos de producción son de suma importancia.
Cualquier contaminante, ya sea orgánico, inorgánico o microbiológico, puede afectar la calidad del producto final y comprometer la
seguridad del paciente. En este contexto, la EDI ha encontrado un lugar destacado debido a su capacidad para producir agua de alta
pureza de manera confiable y continua \cite{condorchemUltrapureWaterElectrodeionization2019}.

La EDI es comúnmente utilizada en la producción de agua purificada (PW) y agua para inyección (WFI). El agua purificada es utilizada
en una amplia gama de aplicaciones en la industria farmacéutica, como la preparación de soluciones para la producción de productos
farmacéuticos y la limpieza de equipos y envases. El agua para inyección, que requiere un nivel aún mayor de pureza, es utilizada
en la producción de productos parenterales, como soluciones para inyección y productos liofilizados \cite{condorchemUltrapureWaterElectrodeionization2019}.

La EDI se utiliza a menudo en combinación con otros procesos de purificación de agua, como la ósmosis inversa (RO) y la destilación.
En un sistema típico, la RO se utiliza primero para reducir la concentración de sales y otros contaminantes en el agua. Luego, la EDI
se utiliza para eliminar los iones restantes y lograr el nivel de pureza deseado. Finalmente, si se requiere agua para inyección,
el agua producida por la EDI puede ser sometida a destilación para eliminar cualquier contaminante restante y garantizar la esterilidad \cite{condorchemUltrapureWaterElectrodeionization2019}.



% % % % Capitulo 3
\chapter{Análisis de la instrumentación}

La instrumentación, como pieza fundamental de cualquier proceso de ingeniería, es una serie de
elementos que proporcionan el control y la supervisión necesarios para garantizar la eficiencia y la
seguridad de un sistema. El propósito de este capítulo es analizar la instrumentación actualmente
implementada en nuestro sistema de ósmosis inversa, examinando tanto los componentes físicos
como la red y los protocolos de comunicación que permiten su funcionamiento integrado.

Iniciaremos con una mirada detallada a la instrumentación existente, explorando su funcionalidad,
la interrelación entre los componentes y la forma en que cada pieza contribuye al objetivo global del
sistema de tratamiento de agua. Al entender completamente la configuración actual, podremos identificar
las áreas de mejora y explorar las oportunidades para la optimización y el crecimiento.

Finalmente en la siguiente parte del análisis, abordaremos los detalles de la red y los protocolos de
comunicación. Al igual que el sistema circulatorio en un organismo, la red y los protocolos de
comunicación son los que mantienen viva la instrumentación, permitiendo la comunicación y la colaboración
eficaces entre los diferentes elementos. Profundizaremos en cómo funcionan, cómo están configurados y
cómo impactan en la eficiencia general del sistema.

Es importante destacar que aunque hay numerosos instrumentos y equipos en la planta de tratamiento de agua en general,
el foco de este capítulo es proporcionar una descripción y análisis detallado  para el sistema
de ósmosis inversa y que serán esenciales para entender e implementar las mejoras propuestas.


\section{Instrumentación de la planta de ósmosis inversa}


La instrumentación en un sistema de ósmosis inversa es crucial para su funcionamiento eficiente y seguro. Esta sección se centrará en los distintos dispositivos de control que permiten la operación continua y segura de la planta de ósmosis inversa. Hablaremos de las válvulas, que regulan el flujo de agua y sustancias químicas en el sistema, las bombas que impulsan la circulación y tratan el agua, y los sensores que monitorizan las condiciones y parámetros clave, como la conductividad, la temperatura y el pH.\\

Además, discutiremos el controlador lógico programable (PLC), que es el cerebro de la operación. El PLC es responsable de la gestión y el control de todas las señales de entrada y salida de la planta, lo que implica tomar decisiones basadas en las lecturas de los sensores y ajustar los actuadores, como las bombas y las válvulas, para mantener el sistema funcionando de manera óptima.\\

Por último, exploraremos los diferentes módulos que el PLC necesita para interactuar con los demás componentes del sistema. Estos módulos son esenciales para la comunicación y el control eficaz del proceso de ósmosis inversa.\\


\subsection{Sensor de conductividad} \label{sec:sesor_conductividad}

En el proceso de ósmosis inversa, la conductividad es una variable esencial a ser controlada. Los sensores de
conductividad son, por tanto, componentes críticos en la planta, proporcionando datos en tiempo real que informan
sobre la eficiencia del proceso. Específicamente, son capaces de detectar cambios en la concentración de iones
en el agua, lo que puede ser indicativo de un problema con las membranas de ósmosis inversa.

El principio de funcionamiento de estos sensores radica en la medición de la conductividad eléctrica del agua,
que refleja su capacidad para transmitir corriente eléctrica. Esta propiedad está directamente relacionada
con la concentración de iones disueltos en el agua. El sensor aplica un voltaje a dos electrodos situados a una
distancia fija y mide la corriente resultante. Como la conductividad depende del contenido de sales en el agua,
un aumento de la conductividad sugiere una mayor concentración de iones, indicando una posible eficacia reducida
de las membranas de ósmosis inversa.

Los sensores de conductividad como el de la figura \ref{fig:sensorCLS16}, fabricado por Endress+Hauser, es un  ejemplos de este tipo de
instrumentos utilizados en la planta.

\insertimageboxed[\label{fig:sensorCLS16}]{instrumentacion/sensorCLS16}{scale=0.7}{0}{Sensor de conductividad CLS16-3D1A1P}

Este tipo de sensores entre los lugares donde se pueden encontrar, lo tenemos ubicado en el punto de salida de la primera
etapa del sistema de ósmosis inversa (en el flujo de permeado).


\begin{mytable}{6cm}{Datos técnicos del sensor de conductividad CLS16-3D1A1P.}{table:sensorCLS16}

        \hline
        \textbf{Modelo}                        & CLS16-3D1A1P                        \\
        \hline
        \textbf{Material}                      & Acero inoxidable 316 L (DIN 1.4435) \\
        \hline
        \textbf{Acabado}                       & Electropulido Ra < 0,8µm            \\
        \hline
        \textbf{Constante}                     & 0,1 (0,04 / 500 µS/cm)              \\
        \hline
        \textbf{Rango}                         & 0 a 20 µS/cm                        \\
        \hline
        \textbf{Conexiones}                    & 1"½ (38,10 mm) Tri-Clamp            \\
        \hline
        \textbf{Temperatura máxima del fluido} & 120 °C                              \\
        \hline
        \textbf{Fabricante}                    & Endress+Hauser                      \\
        \hline
   
\end{mytable}




% ------------Sensores de pH-----------
\subsection{Sensor de pH} \label{sec:sensor_ph}

La medición y control del pH en el agua tratada es crucial en una planta de ósmosis inversa. Los sensores de pH
desempeñan un papel vital en este aspecto, permitiendo la monitorización constante del pH del agua y facilitando
el control de la dosificación de hidróxido de sodio (NaOH).

Los sensores de pH operan basándose en el principio de medición del potencial electroquímico a través de una
celda compuesta por un electrodo de referencia y un electrodo de medición. La diferencia de potencial entre
estos electrodos está relacionada con el pH del medio acuoso. El electrodo de medición, fabricado generalmente
de vidrio, tiene una propiedad particular de presentar una diferencia de potencial con el agua que se encuentra en contacto, la cual es dependiente del pH.

Un sensor de pH como el CPS 11D-7AA2G de Endress+Hauser mostrado en la figura \ref{fig:sensorCPS} se utiliza
en la planta para controlar el pH después del filtro de 5 micras. La información de este sensor es utilizada
para el control de la dosificación de NaOH, ayudando a mantener el pH dentro de los rangos deseados, lo cual
es esencial para la eficiencia del proceso de ósmosis inversa.

\insertimageboxed[\label{fig:sensorCPS}]{instrumentacion/sensorCPS}{scale=0.8}{0}{Sensor de pH CPS 11D-7AA2G}

A continuación, se presenta la tabla \ref{table:sensorCPS} con las características técnicas principales del sensor de pH CPS 11D-7AA2G:\\



\begin{mytable}{6cm}{Características del sensor de pH CPS 11D-7AA2G.}{table:sensorCPS}
        \hline
        \textbf{Característica}       & \textbf{Descripción}   \\
        \hline
        \textbf{Modelo}               & CPS 11D-7AA2G Memosens \\
        \hline
        \textbf{Material}             & Vidrio                 \\
        \hline
        \textbf{Rango pH}             & 0-12                   \\
        \hline
        \textbf{Rango de temperatura} & -5 a 80°C              \\
        \hline
        \textbf{Longitud de la sonda} & 120 x 12 mm            \\
        \hline
        \textbf{Conector}             & tipo N con PG13,5      \\
        \hline
        \textbf{Fabricante}           & Endress+Hauser         \\
        \hline
  
\end{mytable}

% ------------Sensores de Temperatura-----------
\subsection{Sensor de temperatura} \label{sec:sensor_temp}

Las sondas de temperatura son un componente crítico en cualquier proceso industrial que requiera control preciso
de la temperatura. Son dispositivos que detectan cambios en las condiciones físicas y convierten los datos en
señales eléctricas que pueden ser leídas y monitorizadas. En el contexto de los sistemas de ósmosis inversa,
estas sondas son esenciales para monitorear y mantener las condiciones óptimas de temperatura que permiten la
eficacia del proceso.

El modelo TSPT-6702UAC de Endress+Hauser es un ejemplo de una sonda de temperatura de alta calidad. Funciona bajo
la clase A, que se refiere a su alta precisión y consistencia en la medición de la temperatura.
Este tipo de sondas son generalmente más precisas y estables que las sondas de clase B, lo que
las hace ideales para aplicaciones industriales que requieren mediciones precisas y repetibles.

Estas sondas están ubicadas en puntos clave del proceso de ósmosis inversa, como en la tubería antes de la
entrada de cada etapa de la ósmosis. Aquí, las sondas pueden monitorear continuamente la temperatura del agua,
proporcionando datos vitales que pueden ayudar a prevenir problemas y garantizar que el sistema funcione de manera óptima.


\insertimageboxed[\label{fig:sensor_temperatura}]{instrumentacion/sensor_temperatura}{scale=0.8}{0}{Sensor de temperatura TSPT-6702UAC}

A continuación, se proporcionan algunas de las características específicas de este sensor en la tabla \ref{table:sensor_temperatura}\\


\begin{mytable}{6cm}{Características del sensor de temperatura TSPT-6702UAC .}{table:sensor_temperatura}

        \hline
        \textbf{Característica}         & \textbf{Descripción} \\
        \hline
        \textbf{Modelo}                 & TSPT-6702UAC         \\
        \hline
        \textbf{Tipo}                   & Clase A              \\
        \hline
        \textbf{Precisión típica}       & +/- 0.15°C a 0°C     \\
        \hline
        \textbf{Valor de Alfa}          & 0.00385 °C$^{-1}$    \\
        \hline
        \textbf{Valor de resistencia}   & 100 ohm al 0°C       \\
        \hline
        \textbf{Rango de medición}      & 0°C a 200°C          \\
        \hline
        \textbf{Longitud de los cables} & 102 mm               \\
        \hline
        \textbf{Conexiones}             & ø 6 mm               \\
        \hline
        \textbf{Fabricante}             & Endress+Hauser       \\
        \hline
   
\end{mytable}

% ------------Sensores de Redox-----------
\subsection{Sensor de Redox} \label{sec:sensor_redox}

Los electrodos de Redox son dispositivos que se utilizan para medir el potencial de óxido-reducción (Redox) en una
solución. Su principal utilidad en los procesos industriales, incluido el tratamiento de agua por ósmosis inversa,
es proporcionar información en tiempo real sobre el estado de la solución, lo que permite ajustar los parámetros
del proceso en consecuencia.

El modelo CPS12D-7PA21 MEMOSENS de Endress+Hauser es un electrodo de Redox del tipo Orbisint. Estos electrodos
funcionan generando una diferencia de potencial eléctrico entre el electrodo y la solución a medida que se establece
un equilibrio electroquímico. Esta diferencia de potencial es proporcional al potencial Redox de la solución y puede
ser interpretada por un transmisor de Redox.

El CPS12D-7PA21 MEMOSENS es un electrodo robusto diseñado para resistir condiciones de proceso adversas, como el
contacto con fluidos agresivos, gracias a su construcción de vidrio inastillable. También puede operar en un amplio
rango de temperaturas, lo que lo hace adecuado para una variedad de aplicaciones.

En el sistema de ósmosis inversa en estudio, el electrodo de Redox CPS12D-7PA21 MEMOSENS como el de la figura \ref{fig:sensor_redox}
se sitúa justo después del filtro de 10 micras. Desde aquí, puede enviar sus mediciones a un transmisor de
Redox para su interpretación y uso en el control del proceso.

\insertimageboxed[\label{fig:sensor_redox}]{instrumentacion/sensor_redox}{scale=0.8}{0}{Sensor de Redox CPS12D-7PA21}


Aquí se detallan las características específicas del electrodo de Redox CPS12D-7PA21 MEMOSENS:\\

\begin{mytable}{6cm}{Características del sensor de Redox CPS12D-7PA21.}{table:sensor_redox}

        \hline
        \textbf{Característica}                & \textbf{Descripción}              \\
        \hline
        \textbf{Modelo}                        & CPS12D-7PA21 MEMOSENS             \\
        \hline
        \textbf{Tipo}                          & Orbisint                          \\
        \hline
        \textbf{Material}                      & Vidrio inastillable               \\
        \hline
        \textbf{Rango}                         & +/- 1500 mV                       \\
        \hline
        \textbf{Rango de temperatura}          & -15 a 80°C                        \\
        \hline
        \textbf{Longitud del electrodo}        & 120 mm                            \\
        \hline
        \textbf{Conector}                      & Standard con acoplamiento coaxial \\
        \hline
        \textbf{Temperatura máxima del fluido} & 20°C                              \\
        \hline
        \textbf{Fabricante}                    & Endress+Hauser                    \\
        \hline
 
\end{mytable}


% ------------Sensores de flujo-----------
\subsection{Sensor-Transmisor de flujo} \label{sec:sensor_flujo}

En cualquier proceso industrial, la medición precisa y la transmisión de los datos de flujo son esenciales para
garantizar la eficiencia y el correcto funcionamiento del sistema. En particular, los instrumentos que combinan
ambas funciones, conocidos como medidores de flujo y transmisores, son especialmente valiosos en la industria de
tratamiento de agua, como en los sistemas de ósmosis inversa. Proporcionan mediciones exactas de la tasa de flujo de
líquidos en distintos puntos del proceso y transmiten estos datos en tiempo real para su monitorización y control.

El modelo RAMC05-S4-SS-64S2- E90424*P6/Z de YOKOGAWA es un ejemplo perfecto de un medidor de flujo y transmisor en uno.
Este dispositivo funciona como un rotámetro, y su diseño permite no solo medir el flujo de líquidos sino también transmitir
estos datos para su monitorización remota o automatizada. Su ubicación en la tubería de permeado en la segunda etapa de la
ósmosis es estratégica, ya que permite un control constante y preciso del flujo de permeado en este punto crucial del proceso.

Por otro lado, el modelo DS20 07 YJ de MADDALENA es otro medidor de flujo y transmisor efectivo,  es un medidor de flujo de dial húmedo de chorro múltiple.
Este medidor se encuentra después del
filtro de 10 micras, proporcionando mediciones de flujo esenciales después de esta etapa de filtración.



Estos son los detalles específicos de ambos medidores de flujo y transmisores:\\

\insertimageboxed[\label{fig:sensor_transmisor_flujo}]{instrumentacion/sensor_transmisor_flujo}{scale=0.4}{0}{Sensor-Transmisor de flujo RAMC05-S4-SS-64S2- E90424}


\begin{mytable}{6cm}{Características del rotámetro RAMC05-S4-SS-64S2- E90424.}{table:sensor_transmisor_flujo}
  
        \hline
        \textbf{Característica}         & \textbf{Descripción}           \\
        \hline
        \textbf{Modelo}                 & RAMC05-S4-SS-64S2- E90424*P6/Z \\
        \hline
        \textbf{Tipo}                   & Rotámetro                      \\
        \hline
        \textbf{Material}               & 316 L                          \\
        \hline
        \textbf{Conexiones}             & 2" Triclamp                    \\
        \hline
        \textbf{Rango}                  & 400 a 4000 l/h                 \\
        \hline
        \textbf{Material de la carcasa} & Acero inoxidable               \\
        \hline
        \textbf{Opción}                 & 4-20 mA - 24Vdc                \\
        \hline
        \textbf{Fabricante}             & YOKOGAWA                       \\
        \hline
   
\end{mytable}

\insertimageboxed[\label{fig:sensor_transmisor_flujo2}]{instrumentacion/sensor_transmisor_flujo2}{scale=0.8}{0}{Sensor-Transmisor de flujo DS20 07 YJ}


\begin{mytable}{6cm}{Características del medidor de flujo DS20 07 YJ.}{table:sensor_transmisor_flujo2}
        \hline
        \textbf{Característica} & \textbf{Descripción}                           \\
        \hline
        \textbf{Modelo}         & DS20 07 YJ                                     \\
        \hline
        \textbf{Tipo}           & Multi-jet wet dial (Multi-jet con dial húmedo) \\
        \hline
        \textbf{Material}       & Latón recubierto de epoxi                      \\
        \hline
        \textbf{Conexiones}     & Roscado ø 1"½ gas                              \\
        \hline
        \textbf{Rango}          & 10 m³/h (caudal nominal)                       \\
        \hline
        \textbf{Fabricante}     & MADDALENA                                      \\
        \hline
\end{mytable}



\subsection{Sensor de Nivel}

El control del nivel de agua en el proceso de ósmosis inversa es una variable clave, específicamente en el tanque TK50 donde se almacena el agua pretratada. Para esta tarea esencial, se utiliza el sensor de nivel Liquicap FMI51, un dispositivo de medición de nivel por capacitancia desarrollado por Endress+Hauser, tal como se muestra en la Figura \ref{fig:sensor_nivel}. Este instrumento asegura que el proceso de ósmosis inversa se inicie solo cuando el nivel de agua en el tanque alcanza un punto establecido, lo que contribuye a optimizar la eficiencia del proceso.

El Liquicap FMI51 opera bajo el principio de la capacitancia. En el interior de su varilla sensora, este instrumento cuenta con dos electrodos que generan un campo eléctrico. Cuando el nivel del agua en el tanque varía, las propiedades dieléctricas del espacio entre los electrodos cambian, lo cual se traduce en una variación de la capacitancia. Esta variación es interpretada por el sensor y convertida en una señal de nivel que es utilizada para controlar el proceso.

\insertimageboxed[\label{fig:sensor_nivel}]{instrumentacion/sensor_nivel}{scale=0.4}{0}{Sensor de nivel Liquicap FMI51}


\begin{mytable}{6cm}{Características del Sensor de nivel Liquicap FMI51}{table:sensor_nivel}
        \hline
        \textbf{Fabricante}                       & Endress+Hauser                                                                                      \\
        \hline
        \textbf{Principio de medición}            & Capacitivo                                                                                          \\
        \hline
        \textbf{Rango de temperatura del proceso} & -80°C a +200°C                                                                                      \\
        \hline
        \textbf{Presión del proceso}              & Vacío a 100 bar                                                                                     \\
        \hline
        \textbf{Precisión}                        & Repetibilidad: 0,1\%, Error de linealidad para líquidos conductivos: <0,25\%                        \\
        \hline
        \textbf{Longitud total del sensor}        & 6m                                                                                                  \\
        \hline
        \textbf{Distancia máxima de medición}     & 0.1 a 4.0 m                                                                                         \\
        \hline
        \textbf{Comunicación}                     & 4...20mA HART, PFM                                                                                  \\
        \hline
        \textbf{Certificaciones / Aprobaciones}   & ATEX, FM, CSA, IEC Ex, TIIS, INMETRO, NEPSI, EAC, SIL                                               \\
        \hline
        \textbf{Limitaciones de aplicación}       & Espacio insuficiente hacia el techo, medios cambiantes no conductivos con conductividad < 100 μS/cm \\
        \hline
\end{mytable}



% ------------Sensores de Presión-----------
\subsection{Sensor-Transmisor de Presión} \label{sec:sensor_presion}

Los sensores de presión desempeñan un papel esencial en numerosos procesos industriales,
incluyendo la ósmosis inversa. Estos instrumentos son responsables de medir la presión en diferentes puntos
del sistema y transmitir esa información a un sistema de control para su seguimiento y análisis.

El principio de funcionamiento de estos dispositivos se basa en la aplicación de presión a un diafragma de
metal sensible, que causa su deformación. Esta deformación es detectada por un sensor, que la convierte en
una señal eléctrica. En el caso de los transmisores de presión, esta señal se transmite luego a un sistema de control en forma
de una señal estandarizada (generalmente 4-20 mA), lo que permite un fácil seguimiento y control de la presión en el proceso.

La importancia de estos instrumentos en la ósmosis inversa es notable. Dado que la presión es un factor
crítico en la ósmosis inversa, la capacidad de medir y controlar la presión a través de todo el sistema
es esencial para garantizar un rendimiento óptimo y prevenir posibles problemas, como la sobrepresión
que podría dañar las membranas de ósmosis.

En el sistema de ósmosis inversa estudiado, estos sensores se encuentran
ubicados en la tubería de concentrado en cada etapa de la ósmosis, así como a la entrada de cada etapa
de la ósmosis. Esta disposición permite el monitoreo constante y preciso de la presión, lo que es
vital para la operación eficiente y segura del sistema.

A continuación, se proporcionan las características específicas del sensor-transmisor de presión
modelo PTP31-A1C13S1AF1A fabricado por Endress+Hauser:\\

\insertimageboxed[\label{fig:sensor_transmisor_presion}]{instrumentacion/sensor_transmisor_presion}{scale=0.6}{0}{Sensor-Transmisor de presión PTP31-A1C13S1AF1A}


\begin{mytable}{6cm}{Características del sensor de presión PTP31-A1C13S1AF1A. }{table:sensor_transmisor_presion}
        \hline
        \textbf{Característica}         & \textbf{Descripción}               \\
        \hline
        \textbf{Modelo}                 & PTP31-A1C13S1AF1A                  \\
        \hline
        \textbf{Rango}                  & 0 a 40 bar (calibración 0-20 bar)  \\
        \hline
        \textbf{Pantalla}               & LCD                                \\
        \hline
        \textbf{Alimentación eléctrica} & 12 a 30 Vdc                        \\
        \hline
        \textbf{Salida}                 & Interruptor PNP, 3 hilos + 4-20 mA \\
        \hline
        \textbf{Conexión eléctrica}     & Conector M12 x 1.5                 \\
        \hline
        \textbf{Protección IP}          & IP 65                              \\
        \hline
        \textbf{Diafragma}              & AISI 316 L                         \\
        \hline
        \textbf{Fluido de llenado}      & Aceite de grado alimenticio        \\
        \hline
        \textbf{Conexión del proceso}   & Roscado G½" ISO228 macho           \\
        \hline
        \textbf{Fabricante}             & Endress+Hauser                     \\
        \hline
\end{mytable}




\subsection{Manómetro} \label{sec:indicador_manometro}

Los manómetros son instrumentos de medición de presión esenciales en cualquier proceso industrial, incluyendo el tratamiento de agua por ósmosis inversa. Proporcionan una medida de la presión existente en un punto específico del proceso, permitiendo ajustar y controlar parámetros críticos para garantizar la eficacia del sistema.

Los manómetros de tipo seco, como el modelo P600 de ITEC, funcionan basándose en la flexión de un tubo Bourdon (un tubo delgado y hueco que se curva en forma de C) por la presión del fluido. Al aumentar la presión, el tubo se endereza y este movimiento se traduce a una aguja en la esfera del manómetro para proporcionar una lectura de presión. Su diseño resistente y su facilidad de lectura los hacen idóneos para una amplia gama de aplicaciones industriales.

En el sistema de ósmosis inversa en estudio, los manómetros de tipo P600 se sitúan en puntos estratégicos: en cada filtro (de 10 micras y de 5 micras) y antes de la bomba que impulsa el agua a la segunda etapa de la ósmosis. La correcta monitorización de la presión en estas ubicaciones es vital para garantizar el adecuado funcionamiento del sistema y prevenir posibles problemas, como la sobrepresión que podría dañar las membranas de ósmosis.


\insertimageboxed[\label{fig:manometro}]{instrumentacion/manometro}{scale=0.5}{0}{Manómetro P600}


\begin{table}[H]
    \centering
    \caption{Características del manómetro P600.}
    \label{table:manometro}
    \begin{tabular}{| L{6cm} | L{6cm} |}

        \hline
        \textbf{Modelo}                  & P600                        \\
        \hline
        \textbf{Tipo}                    & Ejecución seca              \\
        \hline
        \textbf{Material}                & Acero inoxidable            \\
        \hline
        \textbf{Rango de presión}        & 0 a 10 bar                  \\
        \hline
        \textbf{Diámetro de la carcasa}  & 63 mm o 200 mm              \\
        \hline
        \textbf{Temperatura del proceso} & 20°C                        \\
        \hline
        \textbf{Conexión del proceso}    & Roscado ¼" gas radial o ½'' \\
        \hline
        \textbf{Fabricante}              & ITEC                        \\
        \hline
    \end{tabular}
\end{table}


\subsection{Indicador de Flujo} \label{sec:indicador_flujo}

Los indicadores de flujo son instrumentos indispensables en cualquier proceso industrial,
incluyendo el tratamiento de agua por ósmosis inversa. Estos dispositivos permiten medir
y controlar la cantidad de líquido que fluye por una tubería, proporcionando datos cruciales
para el funcionamiento correcto y eficiente del sistema.

Los indicadores de flujo de tipo rotámetro y de área variable son particularmente comunes
en la industria. Los rotámetros, como el modelo RAMC02-S4-SS-61S1-T90NNNZ de Yokogawa,
funcionan basándose en la elevación de un flotador en un tubo cónico debido al flujo del
fluido.

En el sistema de ósmosis inversa en estudio, estos indicadores de flujo se encuentran en
ubicaciones clave: como por ejemplo en la tubería de concentrado
de la segunda etapa de la ósmosis,así como en la tubería de permeado de la primera etapa de la ósmosis. Monitorear
el flujo en estas ubicaciones es esencial para garantizar la eficiencia y seguridad del
proceso.

A continuación, se presentan las características específicas de este indicador:\\

\insertimageboxed[\label{fig:indicador_flujo}]{instrumentacion/indicador_flujo}{scale=1.1}{0}{Indicador de flujo RAMC05-S4-SS-64V2-T90}


\begin{table}[H]
    \centering
    \caption{Características del dispositivo RAMC02-S4-SS-61S1-T90NNN*Z.}
    \label{table:indicador_flujo}
    \begin{tabular}{| L{6cm} | L{6cm} |}

        \hline
        \textbf{Modelo}                 & RAMC02-S4-SS-61S1-T90NNN*Z \\
        \hline
        \textbf{Tipo}                   & Rotámetro                  \\
        \hline
        \textbf{Material}               & 316 L                      \\
        \hline
        \textbf{Acabado}                & Decapado y pasivado        \\
        \hline
        \textbf{Conexiones}             & 1" clamp                   \\
        \hline
        \textbf{Rango}                  & 100 a 1000 lt/h            \\
        \hline
        \textbf{Material de la carcasa} & Acero inoxidable           \\
        \hline
        \textbf{Fabricante}             & Yokogawa                   \\
        \hline
    \end{tabular}
\end{table}



\subsection{Transmisores}

En la arquitectura de este sistema de ósmosis inversa, los transmisores son piezas esenciales que funcionan como vínculos de
comunicación entre los sensores o analizadores y el PLC (Controlador Lógico Programable).
Su función principal es transformar las señales eléctricas recibidas de los sensores en una forma que el
PLC pueda interpretar y utilizar para el control y monitorización del proceso. \\

Más allá de esta función de conversión de señales, los transmisores también cuentan con sistemas de alarmas e indicadores
integrados. Estos sistemas permiten detectar y alertar sobre cualquier desviación o
anomalía en los parámetros medidos, permitiendo una respuesta rápida para mantener la eficiencia y seguridad del proceso. \\



\subsection{Equipos de control}

En el vasto y complejo universo de la ingeniería de procesos, los equipos de 
control son los actores silenciosos que juegan un papel crucial en el funcionamiento eficiente y 
efectivo de cualquier sistema de tratamiento. Desde mantener condiciones óptimas hasta permitir ajustes precisos y 
oportunos, estos equipos son la columna vertebral de cualquier proceso industrial, incluyendo el tratamiento de agua 
mediante ósmosis inversa.\\

En esta sección, centraremos nuestro análisis en los distintos equipos de control presentes en nuestro subsistema.
 Examinaremos detenidamente equipos como las bombas y válvulas que conforman 
la instrumentación de este sistema, estudiando su funcionamiento, características y 
ubicación en el proceso. Al hacerlo, esperamos proporcionar una visión clara y completa de la instrumentación 
actual del sistema y destacar su importancia en el mantenimiento de un proceso de ósmosis inversa seguro y eficaz.\\

\subsubsection{Bombas de Alta Presión}

Las bombas de alta presión son elementos fundamentales en el sistema de ósmosis inversa. Son responsables de aplicar la presión necesaria para que se produzca la ósmosis, un aspecto crucial para el adecuado funcionamiento del sistema.\\

En el proceso que estamos analizando, se utilizan bombas centrífugas verticales de múltiples etapas, específicamente del modelo CRN10-7 de la marca GRUNDFOS. Estas bombas son conocidas por su eficiencia y durabilidad, lo que las hace ideales para aplicaciones de alta presión como la ósmosis inversa.\\

El funcionamiento de las bombas centrífugas se basa en la conversión de la energía cinética en energía de presión. El agua entra en la bomba y es impulsada por un impulsor que gira a alta velocidad. Cuando el agua sale del impulsor, su energía cinética se transforma en energía de presión a medida que su velocidad disminuye en la voluta o carcasa de la bomba.\\

Estas bombas están ubicadas antes de cada etapa de la ósmosis, donde su tarea es generar la presión necesaria para forzar el paso del agua a través de la membrana semi-permeable del sistema de ósmosis inversa.\\

A continuación, se presenta una tabla con las características técnicas más relevantes de las bombas de alta presión CRN10-7:\\

\insertimageboxed[\label{fig:bomba_centrifuga}]{instrumentacion/bomba_centrifuga}{scale=1.1}{0}{Bombas centrífuga CRN10-7}


\begin{table}[H]
    \centering
    \caption{Características de la bomba centrífuga vertical multietapa CRN10-7.}
    \label{table:bomba_centrifuga}
    \begin{tabular}{| L{6cm} | L{6cm} |}
        \hline
        \textbf{Modelo} & CRN10-7  \\
        \hline
        \textbf{Tipo} & Centrífuga vertical multietapa  \\
        \hline
        \textbf{Material} & AISI 316  \\
        \hline
        \textbf{Sello} & HUUE (Carburo de Tungsteno / EPDM)  \\
        \hline
        \textbf{Medio} & Agua ablandada  \\
        \hline
        \textbf{Temperatura de trabajo} & 20°C  \\
        \hline
        \textbf{Caudal} & 8000 lt/h  \\
        \hline
        \textbf{Presión de descarga} & 10 bar  \\
        \hline
        \textbf{Diámetro del impulsor} & n.a  \\
        \hline
        \textbf{Puerto de entrada} & 2" Tri-Clamp  \\
        \hline
        \textbf{Puerto de salida} & 2" Tri-Clamp  \\
        \hline
        \textbf{Suministro eléctrico} & 3 x 380V 60 Hz  \\
        \hline
        \textbf{Potencia} & 5,5 kW  \\
        \hline
        \textbf{Amperios} & 10,8  \\
        \hline
        \textbf{RPM} & 3600  \\
        \hline
        \textbf{Opciones} & Base de acero inoxidable  \\
        \hline
        \textbf{Fabricante} & GRUNDFOS  \\
        \hline
    \end{tabular}
\end{table}



\subsubsection{Bombas Dosificadoras}

Las bombas dosificadoras son las encargadas de administrar con precisión pequeñas 
 cantidades de químicos para alterar las características del agua. Estos químicos incluyen 
 agentes como el NaOH y Na2S2O5, que respectivamente alteran el pH y reducen el oxígeno disuelto en el agua.\\

En nuestro sistema, se utilizan dos bombas dosificadoras específicas de la marca PROMINENT: 
los modelos GALA G/L G1005 NPB 200UA 103000 figura \ref{fig:bomba_dosificadora} y G/L 1601 NPB 220UA 103 000 figura \ref{fig:bomba_dosificadora2} . Ambos modelos 
son reconocidos por su precisión y fiabilidad, y utilizan la tecnología de diafragma 
solenoide para garantizar una dosificación exacta.\\

El principio de funcionamiento de estas bombas se basa en la acción de un solenoide que atrae y repele un diafragma, creando un movimiento oscilante. Este movimiento provoca la succión del medio (el químico a dosificar) durante la fase de retracción del diafragma y su posterior expulsión durante la fase de compresión.\\

La bomba GALA G/L G1005 NPB 200UA 103000 se encuentra en el sistema de dosificación bomba-tanque de NaOH, mientras que la bomba G/L 1601 NPB 220UA 103 000 se utiliza en el sistema de dosificación bomba-tanque de Na2S2O5.\\

A continuación, se presentan las características técnicas de cada una de estas bombas dosificadoras:\\

\insertimageboxed[\label{fig:bomba_dosificadora}]{instrumentacion/bomba_dosificadora}{scale=0.8}{0}{Bomba dosificadora G1005}


\begin{table}[H]
    \centering
    \caption{Características de la bomba dosificadora G1005.}
    \label{table:bomba_dosificadora}
    \begin{tabular}{| L{6cm} | L{6cm} |}
        \hline
        \textbf{Modelo} & GALA G/L G1005 NPB 200UA 103000  \\
        \hline
        \textbf{Material} & Plexiglás \\
        \hline
        \textbf{Caudal} & 4,4 lt @ 10 bar \\
        \hline
        \textbf{Voltaje} & 100-230 V / 50-60 Hz \\
        \hline
        \textbf{Protección IP} & 65 \\
        \hline
        \textbf{Potencia} & 12W \\
        \hline
        \textbf{Fabricante} & PROMINENT \\
        \hline
    \end{tabular}
\end{table}

\insertimageboxed[\label{fig:bomba_dosificadora2}]{instrumentacion/bomba_dosificadora}{scale=0.8}{0}{Bomba dosificadora G/L 1601}


\begin{table}[H]
    \centering
    \caption{Características de la bomba dosificadora G/L 1601.}
    \label{table:bomba_dosificadora2}
    \begin{tabular}{| L{6cm} | L{6cm} |}
        \hline
        \textbf{Modelo} & G/L 1601 NPB 220UA  \\
        \hline
        \textbf{Tipo} & Diafragma de solenoide \\
        \hline
        \textbf{Medio} & Solución acuosa de Na2S2O5 \\
        \hline
        \textbf{Materiales} & Cabeza de dosificación: Acrílico, elemento de succión / presión: PVC, sellos: FPM-B, bolas: cerámica \\
        \hline
        \textbf{Caudal} & 1,1 lt/h \\
        \hline
        \textbf{Presión de descarga} & 16 bar \\
        \hline
        \textbf{Suministro eléctrico} & 100-230 V / 50-60 Hz \\
        \hline
        \textbf{Potencia} & 12 W \\
        \hline
        \textbf{Fabricante} & PROMINENT \\
        \hline
    \end{tabular}
\end{table}


\subsubsection{Válvulas de Retención}

Las válvulas de retención o check valves son elementos clave en cualquier sistema de tratamiento de agua o proceso industrial, ya que garantizan la unidireccionalidad del flujo en las tuberías. Su papel es esencial para mantener la seguridad y la eficiencia del sistema, ya que evitan el flujo inverso que podría causar daños en los equipos o interrumpir el proceso.\\

El papel de las válvulas de retención en nuestro sistema de ósmosis inversa es multifacético. Están ubicadas en varios puntos estratégicos a lo largo del proceso, incluyendo, pero no limitándose a, justo después de las bombas de alta presión, donde evitan que el fluido regrese a la bomba en caso de una parada o apagado. También se utilizan en la línea de dosificación de químicos, para asegurar un suministro constante y seguro de los reactivos necesarios para el proceso. Sin embargo, es importante destacar que pueden encontrarse en otros puntos del sistema donde sea necesario evitar el retroceso del flujo.\\

El principio de funcionamiento de las válvulas de retención es relativamente sencillo. Contienen un componente que se mueve libremente y permite el flujo en una dirección, pero bloquea el flujo si intenta moverse en la dirección contraria.\\

Para nuestro sistema, empleamos el modelo de válvula de retención Art. 048 VRTCV2 de RATTI. Este modelo está construido con un cuerpo de acero inoxidable AISI 316L, lo que garantiza su resistencia a la corrosión, y tiene una junta de PTFE.\\

Válvula de Retención 048 VRTCV2\\

\begin{table}[H]
    \centering
    \caption{Características del cuerpo.}
    \label{table:cuerpo}
    \begin{tabular}{| L{6cm} | L{6cm} |}
        \hline
        \textbf{Material del cuerpo} & AISI 316L \\
        \hline
        \textbf{Junta} & PTFE \\
        \hline
        \textbf{Diámetro} & 1½" \\
        \hline
        \textbf{Conexiones} & Abrazadera (clamp) \\
        \hline
        \textbf{Resorte} & Estándar \\
        \hline
        \textbf{Fabricante} & RATTI \\
        \hline
    \end{tabular}
\end{table}


\subsubsection{Válvulas Multiusos}

Las válvulas multiusos son componentes esenciales en cualquier sistema de tratamiento de agua. Actúan como puntos de control, permitiendo o impidiendo el paso de fluidos a través de las tuberías. La capacidad de controlar el flujo de agua y otros líquidos es crucial para el funcionamiento seguro y eficiente de todo el sistema. Son llamadas "multiusos" porque se utilizan en una variedad de aplicaciones dentro del sistema, dependiendo de las necesidades del proceso en particular.\\

En nuestro sistema de ósmosis inversa, las válvulas multiusos se encuentran en varios puntos críticos. Una ubicación importante es en las tuberías de concentrado de la ósmosis, en la línea que va al drenaje o que retorna al tanque de almacenamiento de agua de pretratamiento. Además, se utilizan en la línea de bypass que se encuentra después de la bomba de lavado químico. Estas ubicaciones no son exhaustivas, y es posible encontrar estas válvulas en otros puntos del sistema donde se requiera controlar el flujo de fluido.\\

El principio de funcionamiento de las válvulas multiusos es simple pero efectivo. Cuando la válvula está abierta, permite el flujo de fluido; cuando está cerrada, detiene el flujo.\\

Utilizamos el modelo J4M1G00 de RATTI para nuestras válvulas multiusos. Esta válvula está fabricada con acero inoxidable AISI 316L para una resistencia óptima a la corrosión y durabilidad a largo plazo. Tiene un diámetro de 1" - ¾" y se conecta mediante una conexión Tri-Clamp.\\

\begin{table}[H]
    \centering
    \caption{Características del cuerpo.}
    \label{table:valvula_multiusos}
    \begin{tabular}{| L{6cm} | L{6cm} |}
        \hline
        \textbf{Material del cuerpo} & AISI 316 L \\
        \hline
        \textbf{Diámetro} & 1" - ¾" \\
        \hline
        \textbf{Conexiones} & Tri-Clamp \\
        \hline
        \textbf{Fabricante} & OMAL \\
        \hline
    \end{tabular}
\end{table}

\subsubsection{Válvulas de Control ON/OFF}

En cualquier proceso industrial, las válvulas de control ON/OFF son elementos críticos para la gestión y regulación del flujo de fluidos. Su importancia radica en su habilidad para controlar de manera precisa y eficiente el flujo a través de las tuberías, permitiendo un total paso del fluido o su completa interrupción, según las demandas del sistema.\\

Un ejemplo específico de este tipo de válvulas es el modelo S386FPLY004Y05 de la reconocida empresa OMAL. Esta válvula cuenta con un cuerpo de hierro fundido revestido con EPOXY y EPDM, lo que la hace resistente y duradera. La característica más notable de esta válvula es su actuador neumático de retorno por resorte que, junto con su conjunto de accesorios, garantiza un funcionamiento fiable y una integración eficiente con el sistema de control del proceso.\\

La planta de ósmosis inversa que analizamos está equipada con numerosas válvulas de control ON/OFF, distribuidas estratégicamente en diferentes puntos del sistema. Son componentes indispensables que aseguran la correcta operación del proceso, y debido a su importancia, están presentes en gran cantidad en todas las áreas de la planta. Algunos lugares estratégicos donde podemos encontrar este tipo de válvulas puede ser la línea que precede a la bomba de alta presión y en la línea de concentrado de la primera etapa de ósmosis que retorna al tanque de pretratamiento.\\

Válvula de Control ON/OFF S386FPLY004Y05\\

\begin{table}[H]
    \centering
    \caption{Características del cuerpo.}
    \label{table:valvula_on_off}
    \begin{tabular}{| L{6cm} | L{6cm} |}
        \hline
        \textbf{Material del cuerpo} & Hierro fundido con revestimiento de EPOXY \\
        \hline
        \textbf{Revestimiento} & EPDM \\
        \hline
        \textbf{Vástago y disco} & Acero inoxidable AISI 316 \\
        \hline
        \textbf{Estilo del cuerpo} & "LUG" roscado para brida EN1092-1 \\
        \hline
        \textbf{Tamaño} & DN40 \\
        \hline
        \textbf{Actuador} & Neumático de retorno por resorte N.O. tipo SR30 y tornillos de regulación \\
        \hline
        \textbf{Accesorios} & Indicador de posición del eje KI02PP10, regulador de flujo de aire comprimido KAPR00101, filtro de aire de bronce 9490S001 \\
        \hline
        \textbf{Fabricante} & OMAL \\
        \hline
    \end{tabular}
\end{table}

\subsubsection{Válvulas de Retención de Presión de Inyección}

Las válvulas de retención de presión de inyección juegan un papel importante en el sistema de dosificación, especialmente en procesos industriales que requieren una precisión en la dosificación de ciertos productos químicos. Estas válvulas mantienen una presión constante de salida, evitando fluctuaciones y garantizando una dosificación precisa y estable.\\

En nuestra planta de ósmosis inversa, estas válvulas son vitales en la dosificación precisa de sustancias químicas como NaOH y Na2S2O5. Son componentes esenciales para asegurar la eficacia de las operaciones de dosificación y se encuentran estratégicamente ubicadas en las líneas de dosificación correspondientes. Sin embargo, su presencia no se limita a estas áreas, y se podrían encontrar en otras partes del sistema donde se necesite una dosificación precisa.\\

El modelo MFV-DK de PROMINENT, una empresa reconocida por la fabricación de componentes de alta calidad, es una de las válvulas utilizadas en nuestro sistema. Con un cuerpo de PVDF y un diafragma de PTFE, esta válvula es capaz de mantener una presión de alivio de hasta 16 bar, lo que asegura su capacidad para trabajar bajo condiciones exigentes.\\

Válvula de Retención de Presión de Inyección MFV-DK\\

\begin{table}[H]
    \centering
    \caption{Características del tipo MFV-DK, PVDF.}
    \label{table:valvula_retencion}
    \begin{tabular}{| L{6cm} | L{6cm} |}
        \hline
        \textbf{Tipo} & MFV-DK, PVDF \\
        \hline
        \textbf{Tamaño} & 1 \\
        \hline
        \textbf{Presión de alivio} & 16 bar \\
        \hline
        \textbf{Conector} & 6-12 mm \\
        \hline
        \textbf{Conector de bypass} & 6/4 mm \\
        \hline
        \textbf{Materiales} & Cuerpo de PVDF, diafragma de PTFE, sello de FPM \\
        \hline
        \textbf{Fabricante} & PROMINENT \\
        \hline
    \end{tabular}
\end{table}



\section{Comunicación de la Planta}

La red de comunicación en la planta de ósmosis inversa está
estratificada en tres niveles de automatización, con un sistema
de interconexión que garantiza una rápida y eficiente transmisión
de datos entre los distintos elementos de la red. En cada nivel
de la jerarquía de automatización, se utiliza un protocolo de
comunicación que se adapta mejor a las necesidades
de ese nivel, tal como se detalla en la Figura \ref{fig:comunicacion}.
A continuación, se analizan los protocolos de comunicación empleados
en cada nivel.

\textbf{Nivel de Campo}\\
En el nivel de campo, se encuentran los diversos dispositivos de campo,
como válvulas, bombas y sensores, donde se envían datos al proceso o se recogen del
mismo. Estos dispositivos de campo se conectan a los módulos de
entrada/salida, como los módulos CPX de Festo (especialmente diseñado para válvulas neumáticas) y ET 200s de Siemens,
a través de un bus de campo utilizando el protocolo Profibus DP.
El Profibus DP es un estándar de comunicación industrial de alta
velocidad y bajo retardo, especialmente diseñado para aplicaciones
de control de procesos. Permite un intercambio eficiente y robusto
de datos entre los dispositivos de campo y los módulos de control,
garantizando así un control preciso y en tiempo real del proceso.

\textbf{Nivel de Control}\\
En el nivel de control, los módulos CPX de Festo y ET 200s de
Siemens reciben las señales de los sensores y las transmiten al
autómata programable (PLC) maestro a través del protocolo Profibus DP.
Profibus DP es un estándar de comunicación industrial de alta velocidad
y bajo retardo, especialmente diseñado para aplicaciones de control de
procesos. Gracias a Profibus, el PLC maestro puede comunicarse de manera
eficiente con los módulos de periferia descentralizada,
permitiendo un control preciso y en tiempo real del proceso.

\textbf{Nivel de Supervisión}\\
En el nivel superior de la jerarquía, el PLC se comunica con el sistema SCADA (Control Supervisor y Adquisición de Datos) mediante una interfaz multipunto (MPI) que utiliza el protocolo de comunicación TCP/IP. TCP/IP es un conjunto de protocolos de comunicación de alto nivel que permite la transmisión de datos entre dispositivos en una red de área amplia, como Internet. Esta forma de comunicación permite la visualización y el control del proceso en tiempo real desde el sistema SCADA, proporcionando al operador una interfaz de usuario intuitiva y potente.

\insertimageboxed[\label{fig:comunicacion}]{comunicacion}{scale=0.3}{0}{Procolos de comunicación}






% % % Capitulo 4
\chapter{Propuesta de Implementación de EDI Después de la Ósmosis Inversa Doble}
\label{cap:propuesta_implementacion}

La necesidad de alcanzar niveles más altos de pureza del agua ha impulsado
la evolución y mejora continua de las tecnologías de tratamiento de agua.
Una de estas tecnologías es la Electrodesionización (EDI), que combina
los principios de electrodiálisis y resinas de intercambio iónico para
producir agua de alta pureza. En la industria farmacéutica, donde se
requiere un agua con una calidad excepcional, la implementación de la
tecnología EDI se convierte en un paso esencial después de la ósmosis inversa doble.\\

Este capítulo presentará la propuesta de implementación de un sistema
EDI en la empresa AICA. En primer lugar, se discutirá el sistema de
control que regula el funcionamiento del EDI y cómo este se coordina
con el sistema de control de la planta en general.
Luego, se presentará la propuesta de integración de un sistema SCADA,
mostrando su interfaz de usuario y explicando cómo este sistema
ayudará a supervisar y controlar el proceso de tratamiento del agua.
Finalmente, se describirá el proceso de implementación y puesta en marcha del
EDI, abarcando desde la instalación física del dispositivo hasta las pruebas
iniciales para verificar su funcionamiento correcto.

% ------------- Sección ----------------
\section{Sistema de control}
\label{sec:sistema_control}

En cualquier sistema industrial, el control es un componente crucial. La eficiencia, seguridad y eficacia de un sistema dependen en gran medida de su capacidad para responder a las variables del entorno y ajustar su funcionamiento en consecuencia. El sistema de control es el cerebro de la operación, coordinando y supervisando todos los aspectos del proceso. En el caso de un sistema de Electrodesionización, el control es aún más crítico debido a la complejidad del proceso y la alta calidad del producto final requerido.\\

La programación del Controlador Lógico Programable (PLC, por sus siglas en inglés) es un elemento esencial de este sistema de control. El PLC se encarga de interpretar las señales de entrada de los distintos sensores y actuadores y ejecutar la lógica de control para ajustar las operaciones del sistema de acuerdo a las necesidades. En esta sección, se presentará la programación del PLC en forma de diagrama de flujo para la secuencia principal de funcionamiento del sistema de Electrodesionización.\\

\subsection{Puesta en marcha}

Un diagrama de flujo ofrece una visión clara y concisa de la lógica de control,
facilitando la comprensión y el seguimiento de la secuencia de operaciones.
Esto es especialmente útil para el personal de mantenimiento y operación,
así como para cualquier persona que necesite entender el funcionamiento del sistema.\\

La secuencia operacional del sistema de Electrodesionización se inicia con la activación de la planta de
ósmosis inversa a través de una interfaz de usuario. Este evento de inicio es seguido de un
período de espera hasta que el sensor de nivel determine que el tanque de agua pretratada (TK50A)
ha alcanzado su nivel operativo óptimo (\ref{fig:flujo_1}).\\

Durante este tiempo inicial, las etapas de ósmosis inversa (RO1 y RO2) se encuentran en un estado
latente. RO1 aguarda la señal de nivel correcta del tanque TK50A, mientras que RO2 permanece en un
estado de inactividad.\\

Una vez que el sensor de nivel indica que TK50A ha alcanzado su nivel adecuado, se implementa
un período de confirmación del nivel, que sirve para mitigar el impacto de posibles fluctuaciones
en el nivel del tanque. Esta duración de tiempo se ha establecido típicamente en 60 segundos.\\

A continuación, se inicia el flujo de agua hacia la primera etapa de ósmosis inversa (RO1). Esta
etapa implica una descarga inicial de agua, necesaria debido a las posibles condiciones iniciales
subóptimas del agua que entra en el sistema. Este período de descarga varía dependiendo de la condición
de la membrana de ósmosis, pero suele ser de aproximadamente 120 segundos.\\

Después de este período de descarga, el agua de RO1 es examinada para determinar si cumple con los
parámetros de conductividad requeridos. Si la conductividad no cumple con las especificaciones, RO1
entra en un estado de descarga por alta conductividad y se mantiene en este estado hasta que las
mediciones de conductividad y un período de confirmación de 60 segundos indiquen que se cumplen
los parámetros de conductividad.\\

\insertimageboxed[\label{fig:flujo_1}]{/flujo_OI1}{scale=0.45}{0}{Diagrama de flujo para la RO1 del proceso de producción de PW.}


En cuanto las condiciones de conductividad sean satisfactorias, RO1 cambia a un estado de producción
y, simultáneamente, se inicia la segunda etapa de ósmosis inversa (RO2). Esta segunda etapa, al
igual que RO1, comienza con una descarga inicial (ver Figura \ref{fig:flujo_2}). No obstante, a diferencia de RO1, el agua
descargada por RO2 se devuelve al tanque de agua pretratada (TK50A), combinándose con el agua
de permeado y concentrado. Este período de descarga también está sujeto a las condiciones de
las membranas de ósmosis y dura aproximadamente 120 segundos.\\

Posteriormente, se evalúan los parámetros de conductividad y temperatura en el permeado de RO2.
Si alguno de estos parámetros no cumple con las especificaciones, RO2 entra en un estado de descarga
por parámetros deficientes y se mantiene en este estado hasta que los parámetros medidos y un período
de confirmación de 60 segundos indiquen condiciones aceptables.\\

\insertimageboxed[\label{fig:flujo_2}]{/flujo_OI2}{scale=0.6}{0}{Diagrama de flujo para la RO2 del proceso de producción de PW.}


Una vez que se alcanzan estos criterios, RO2 cambia a un estado de producción. Con ambas etapas de
ósmosis inversa (RO1 y RO2) en producción, el módulo de Electrodesionización (EDI) puede comenzar su
operación con una descarga inicial hacia el tanque de pretratamiento. Esta descarga inicial tiene
una duración de aproximadamente 60 segundos.\\

Posteriormente, se comprueban los parámetros como la conductividad y la presión en el producto del EDI.
Si alguno de estos parámetros no cumple con las especificaciones, el EDI entra en un estado de descarga
por parámetros deficientes y se mantiene en este estado hasta que los parámetros medidos y un período
de confirmación de 60 segundos indiquen condiciones aceptables.\\

Finalmente, una vez que los parámetros de conductividad y presión son óptimos y han pasado 60 segundos
de confirmación, el EDI cambia a un estado de producción, indicando la finalización exitosa de la
secuencia operacional del sistema de Electrodesionización.\\

Con el sistema completo en estado de producción (ver Figura \ref{fig:flujo_3}), el estado
posterior depende del nivel del tanque final. Si el tanque final está
completamente lleno, la ósmosis comienza una circulación conjunta, que dura un
tiempo de alrededor de 10 minutos. Superado este tiempo, se realiza una pausa de tiempo de 60 minutos antes de
comenzar otro ciclo. La ósmosis continúa recirculando y no vuelve a producir
hasta que el tanque de almacenamiento de agua purificada, que distribuye a los
puntos de uso, señale un nivel del 75\% de capacidad.\\

Cada vez que concluye un ciclo de producción y debe comenzar otro, se comprueba
el estado del sensor de nivel bajo del tanque de agua pretratada. Si este
sensor permanece activo (ver Figura \ref{fig:flujo_3}), se lleva a cabo directamente la
descarga inicial de la OI1. De lo contrario, será necesario esperar hasta que
el tanque TK 50A alcance el nivel mínimo necesario para poner el sistema a purificar.\\


\insertimageboxed[\label{fig:flujo_3}]{/flujo_OI3}{scale=0.4}{0}{Diagrama de flujo para el EDI del proceso de producción de PW.}


% ------------- Sección ----------------
\section{Propuesta de SCADA}
\label{sec:scada_proposal}

Los Sistemas de Control y Adquisición de Datos (SCADA) se han convertido en una herramienta fundamental en el ámbito de la automatización industrial, permitiendo la supervisión y control de procesos a gran escala de una manera eficiente y centralizada. Este sistema ofrece ventajas significativas, como la optimización de operaciones, el aumento de la eficiencia, la mejora de la calidad del producto y la prevención de condiciones peligrosas.\\

En el contexto del sistema de purificación de agua de la planta, la implementación de un SCADA proporcionaría una visibilidad en tiempo real del proceso y facilitaría la gestión de alarmas y el control de los componentes clave del sistema, como las membranas de la ósmosis inversa y el dispositivo EDI. Además, un sistema SCADA permitiría el registro de datos, esencial para el análisis de tendencias y la toma de decisiones basada en datos.\\

En la presente sección, se propone una implementación de SCADA para el sistema de purificación de agua. Esta propuesta se centra en proporcionar una interfaz amigable para el usuario y en maximizar el potencial del sistema de control existente. La propuesta también incluye el diseño de las pantallas de visualización y los detalles de cómo los operadores pueden interactuar con el sistema para mantener el proceso funcionando de manera eficiente y segura.\\

\subsection{Implementación de SCADA}

% ------------- Sección ----------------
\section{Instalación del EDI }
\label{sec:implementation_start}

La implementación de un nuevo componente de un sistema de tratamiento de agua,
como un dispositivo de EDI, es un proceso complejo que requiere consideraciones
cuidadosas de diseño, logística, instalación y pruebas. Esta tarea se vuelve
aún más crítica cuando este nuevo componente debe integrarse a un sistema
existente sin interrumpir significativamente su funcionamiento normal.\\

En esta sección, describiremos el proceso de implementación y puesta en marcha del dispositivo EDI propuesto después de la doble ósmosis inversa. Este proceso incluirá los detalles de la instalación física del dispositivo EDI, desde su recepción y montaje hasta su conexión con el sistema existente. También cubriremos las pruebas iniciales que deben realizarse para garantizar que el dispositivo EDI esté operando correctamente y para confirmar que se pueden alcanzar los parámetros deseados de pureza del agua.\\

Además, discutiremos la importancia de la formación del personal que manejará el dispositivo EDI para asegurar una operación segura y eficiente a largo plazo. Este entrenamiento debe incluir el uso del nuevo sistema SCADA propuesto, así como los procedimientos de mantenimiento y resolución de problemas específicos del dispositivo EDI. \\

Finalmente, se abordarán las medidas de seguimiento y evaluación que deben implementarse para garantizar la eficacia y eficiencia del sistema a lo largo del tiempo.\\

\subsection{Proceso de instalación del EDI}

La implementación del sistema de Electrodesionización (EDI) luego de la ósmosis inversa doble requiere una serie de pasos clave para garantizar su correcta instalación y funcionamiento. A continuación, se proporciona un desglose detallado de este proceso:

\begin{enumerate}
    \item \textbf{Evaluación del sitio de instalación:} Antes de la instalación del EDI, es esencial realizar una evaluación exhaustiva del sitio para determinar la adecuación del área para alojar la unidad. Factores como la disponibilidad de espacio, la accesibilidad para el mantenimiento, la disponibilidad de suministro de agua y energía, y las condiciones ambientales deben ser considerados.

    \item \textbf{Preparación del sitio de instalación:} Una vez evaluado el sitio, se prepara para la instalación. Esto puede implicar trabajos de construcción menores para proporcionar una base estable y segura para la unidad EDI, y la configuración de las conexiones necesarias para el agua, la electricidad y el drenaje.

    \item \textbf{Instalación de la unidad EDI:} La unidad de EDI se instala en el sitio preparado. Esto debe ser realizado por técnicos cualificados para garantizar que la unidad se instale correctamente y de manera segura. Los componentes de la unidad deben ser cuidadosamente manejados para evitar daños.

    \item \textbf{Conexión de la unidad EDI:} Una vez instalada la unidad, se conecta a las fuentes de agua y electricidad, y al sistema de drenaje. Los componentes de la unidad, como las membranas, las bombas y los sensores, también se conectan y se aseguran.

    \item \textbf{Pruebas de la unidad EDI:} Antes de la puesta en marcha completa, la unidad EDI se somete a una serie de pruebas para verificar su correcto funcionamiento. Esto incluye pruebas de la funcionalidad del PLC y del sistema SCADA, así como pruebas de la capacidad de la unidad para purificar el agua a las especificaciones requeridas.

    \item \textbf{Puesta en marcha de la unidad EDI:} Una vez que se han realizado y superado todas las pruebas, se pone en marcha la unidad EDI. Durante la puesta en marcha inicial, se debe monitorear de cerca la operación de la unidad para identificar y corregir cualquier problema que pueda surgir.
\end{enumerate}

En cada una de estas etapas, se deben seguir estrictamente las normas y procedimientos de seguridad para proteger tanto al personal como al equipo. También es fundamental mantener una documentación detallada de todo el proceso de instalación y puesta en marcha para facilitar futuras referencias y mantenimiento.



% % % Capitulo 5
\chapter{Análisis de costos y beneficios}
En el presente capítulo, se abordará el análisis financiero del proyecto, desde su inicio hasta su conclusión.
Este análisis incluirá tres componentes clave: el coste de la investigación, el precio de los servicios
científicos y técnicos, y los beneficios de la investigación, así como el impacto económico de la
implementación de los resultados. Este análisis es fundamental para evaluar tanto la calidad como la
relevancia del proyecto para la empresa farmacéutica AICA, donde se llevará a cabo la implementación
del sistema de Electrodesionización (EDI).

El análisis del coste contempla los gastos derivados de la utilización de la tecnología necesaria,
los costes de adquisición de los equipos, componentes de instalación y materiales utilizados directamente,
así como los salarios del personal técnico involucrado en el proyecto. Por otro lado, el análisis de los
beneficios resulta esencial, ya que permite tener control sobre el gasto incurrido, proporcionando elementos
de juicio de carácter económico y otorgando una visión más amplia de las tareas relacionadas con la
implementación de esta tecnología, minimizando de esta manera el desperdicio de recursos en instrumentación
o materia prima innecesaria para la realización del proyecto.

El objetivo de este análisis económico es proveer una visión detallada y objetiva de los costes asociados con
la implementación de un sistema de EDI en la industria farmacéutica AICA, permitiendo de este modo una
planificación y gestión eficiente de los recursos disponibles.


\section{Costo del proyecto}

La estimación del costo se lleva a cabo al inicio del proyecto y se considera una aproximación del costo
real, que se determinará al finalizar el proyecto. Este costo puede calcularse a través de la suma del
costo directo e indirecto, tal como se muestra en la ecuación (\ref{eq:cost_total}). \\

\begin{equation}
    \label{eq:cost_total}
    CT = CD + CI
\end{equation}

Donde: \\
CT representa el costo total del proyecto. \\
CD hace referencia al costo directo. \\
CI denota el costo indirecto.

\subsection{Costo indirecto}

El costo indirecto abarca gastos tales como el consumo de electricidad, gastos administrativos, entre otros.
Este valor se estima multiplicando un coeficiente de gasto, en este caso 0.84, por el salario básico de la
investigación, tal como se muestra en la ecuación (\ref{eq:cost_indirect}). \\

\begin{equation}
    \label{eq:cost_indirect}
    CI = 0.84 * SB
\end{equation}

\subsection{Costo directo}

El costo directo engloba todos los gastos económicos necesarios para la realización del proyecto. Se
calcula como la suma del Salario Básico (SB), el Salario Complementario (SC), el Seguro Social (SS), los
Medios Directos (MD), las Dientas y los Pesajes (DP), y Otros Gastos (OG), como se puede observar con más
detalle en la ecuación (\ref{eq:cost_direct}). \\

\begin{equation}
    \label{eq:cost_direct}
    CD = SB + SC + SS + MD + DP + OG
\end{equation}

\subsection{Salario básico}

SB (salario básico): Consiste en el salario que se paga por el tiempo trabajado, es decir, no se incluye seguridad social ni vacaciones. Incluye los salarios básicos de todos los participantes del proyecto.


\begin{equation}
    \label{eq:sal_basico}
    SB = \sum_{i = 0}^{n} (Ai * Bi)
\end{equation}

donde:

𝐴𝑖: días dedicados a la investigación del proyecto.

B𝑖: salario diario del participante 𝑖 (salario mensual / 24)

𝑛: número total de participantes del proyecto.

El salario por hora de los participantes está dado por la relación existente del salario
básico de cada uno entre la cantidad de días dedicados a actividades laborales,
multiplicado por la cantidad de horas. Teniendo en cuenta que en un mes existen 24
días laborables como promedio y que al día la jornada de trabajo es de 8 horas se
puede plantear que:\\

B1 = 4954 / (24*8) = 25.80 CUP/Hrs

B2 = 9730.5 / (24*8) = 50.68 CUP/Hrs

B3 = 400 / (24*8) = 2.08 CUP/Hrs\\

En la Tabla \ref{table:participantes_proyecto} se muestra una relación de las personas que participan en la realización de este proyecto. \\


\begin{table}[H]
    \caption{Participantes en el proyecto}
    \label{table:participantes_proyecto}

    \begin{tabular}{|c|c|c|c|c|}
        \hline
        \textbf{Nombres y apellidos}  & \textbf{SB (CUP)} & \textbf{𝐴𝑖 (Hrs)} & \textbf{B𝑖 (CUP/Hrs)} & \textbf{Ai*Bi} \\
        \hline
        Ing. Amanda Martí Coll        & 4900              & 120               & 25.80                 & 3096           \\
        Ing. Rosaine Ayala            & 6500              & 120               & 25.80                 & 3096           \\
        Armando Cesar Martin Calderón & 4900              & 600               & 25.80                 & 3096           \\

        \hline
    \end{tabular}
\end{table}


Se emplearon 5 meses de trabajo comprendidos entre enero y mayo. Considerando que los tutores le dedicaron a la actividad, cada día laborable, 1 hora de trabajo como promedio, entonces se puede afirmar que fueron asignadas a la investigación 120 horas por cada uno de ellos.
El estudiante le dedicó cada día laborable como 5 horas como promedio, a la investigación, es decir, un total de 600 horas.
Según la ecuación (\ref{eq:sal_basico}):\\

\begin{equation}
    \label{eq:salary_basico_total}
    SB = 120 * 25.80 + 120 * 50.68 + 600 * 2.08 = 10425.6 CUP
\end{equation}


\subsection{Salario complementario}

El salario complementario (SC) es el 9.09\% del salario básico, destinado para el pago de las
vacaciones. Se puede calcular con la siguiente ecuación:

\begin{equation}
    \label{eq:salary_complementary}
    SC = SB * 0.0909
\end{equation}
\begin{equation}
    \label{eq:salary_complementary_total}
    SC=0.0909*10425.60=947.69 CUP
\end{equation}

\subsection{Seguro Social}

El seguro social (SS) equivale al 5\% del salario básico más el salario complementario, y se
calcula de la siguiente forma:

\begin{equation}
    \label{eq:social_security}
    SS = 0.05 * (SB + SC)
\end{equation}
\begin{equation}
    \label{eq:social_security_total}
    SS=0.05*(10425.60+947.69)=1137.33 CUP
\end{equation}

\subsection{Medios Directos}

Los medios directos (MD) incluyen los costos de todos los equipos, componentes de instalación y
materiales utilizados directamente en la investigación.


Para llevar a cabo el proyecto será necesario hacer algunos gastos imprescindibles. En la Tabla 4.1
se muestran los precios de los elementos que deben adquirirse:

\begin{table}[h]
    \caption{Listado de precios de los dispositivos e instrumentos necesarios para la elaboración del proyecto}
    \begin{tabular}{|c|c|c|c|}
        \hline
        Dispositivo/instrumento/otros      & Cantidad & Precio por unidad & Precio total \\
        \hline
        Electrodesionizador                & 1        & 10379.4€          & 10379.4€     \\
        Sensor transmisor de temperatura   & 1        & 175.10€           & 175.10€      \\
        Sensor transmisor de flujo         & 1        & 1732.00€          & 1732.00€     \\
        Sensor transmisor de conductividad & 1        & 661.25€           & 661.25€      \\
        Sensor transmisor de presión       & 2        & 691.85€           & 1393.7€      \\
        \hline
    \end{tabular}
\end{table}

El total de gastos en materiales directos es:

\begin{equation}
    MD = 15753.15€ + \$217694.5 MN
\end{equation}

\subsection{Dietas y Pasajes}

Las dietas y pasajes (DP) representan los gastos ocasionados por dietas y pasajes.

\subsection{Otros Gastos}

Los otros gastos (OG) incluyen el costo de utilización de equipamiento. Se considera el gasto por
concepto de tiempo de máquina, que tiene un valor de \$10.00 la hora.

Se incluye el gasto por consumo de energía eléctrica, durante las horas de tiempo de máquina empleadas
en la elaboración del proyecto. Para un total de 450 horas resulta ser:

\begin{equation}
    OG = 450 \text{ horas } * \$10MN = \$4500.00MN
\end{equation}





\subsection{Cálculo del Costo Directo}

\begin{equation}
    CD = SB + SC + SS + MD + DP + OG = \$228564.42 MN + 15753.15€
\end{equation}

\subsection{Costos indirectos}

El término Costos Indirectos (CI) se refiere a los gastos de electricidad consumida, gastos de administración,
instalaciones, etc., en el proceso de investigación. Este se estima aplicando un coeficiente de gastos al
salario básico de la investigación. El coeficiente de gastos para un Departamento Docente es 0.84 y para una
UCT (Unidad de Ciencia y Técnica) es 1.4063.

\begin{equation}
    CI = 0.84 * SB = \$6468 MN
\end{equation}

\subsection{Costo Total}

El costo total del proyecto resulta la suma de los costos directos e indirectos, por tanto:

\begin{equation}
    CT = CD + CI = \$235032.42MN + 15753.15€
\end{equation}

\subsection{Precio}

El precio se determina mediante la expresión:

\begin{equation}
    P = CT + 0.1 * CT
\end{equation}

Donde: $CT$ representa el costo total de todos los elementos de la red y control de conductividad, $0.1*CT$
representa los salarios pagados a especialistas, técnicos, y el resto del personal involucrado en el diseño,
montaje y puesta en marcha del sistema, el costo de impuestos aduanales, de combustible para el transporte del
personal, y para el cableado.

El costo total es de $CT = \$235032.42MN + 15753.15€$

\begin{align*}
    P & = ( \$235032.42MN + 15753.15€) + 0.1*( \$235032.42MN + 15753.15€) \\
    P & = \$258535.66MN + 17328.46€
\end{align*}

Luego el costo total del proyecto de tesis es:

\begin{align*}
    \text{Costo total del proyecto de tesis} & = CT + P                                                    \\
    \text{Costo total del proyecto de tesis} & = (\$235032.42MN + 15753.15€) + (\$258535.66MN + 17328.46€) \\
    \text{Costo total del proyecto de tesis} & = \$493568.08MN + 33081.61€
\end{align*}


\section{Viabilidad del proyecto}

El proyecto es viable teniendo en cuenta los siguientes aspectos:

\begin{itemize}
    \item \textbf{Medio ambiente:} Este proyecto es respetuoso con el medio ambiente. No hay elementos o procesos
          que generen contaminación o residuos dañinos. Además, se ha hecho un esfuerzo consciente por minimizar el uso
          de recursos y maximizar la eficiencia en todas las etapas del proyecto. Se espera que este proyecto tenga un
          impacto ambiental positivo o neutral en su ejecución a corto y largo plazo.

    \item \textbf{Jurídico:} El proyecto se desarrolla completamente dentro del marco legal existente y cumple con
          todas las normas y leyes pertinentes. En particular, se han tomado medidas para garantizar que todas las
          actividades estén en línea con las regulaciones y directrices de la industria farmacéutica. El proyecto también
          mantiene un compromiso de adherirse a cualquier cambio o actualización futura en la legislación relevante.

    \item \textbf{Económico y financiero:} Desde una perspectiva económica, el proyecto es viable. El centro tiene
          los recursos financieros necesarios para financiar completamente el proyecto. Además, se espera que el proyecto
          sea rentable y genere un retorno de la inversión a largo plazo. El apoyo intelectual de la institución docente
          también contribuye a la viabilidad económica del proyecto, ya que proporciona acceso a expertos y recursos
          académicos.

    \item \textbf{Técnico:} La entidad dispone de los recursos técnicos necesarios para llevar a cabo el proyecto.
          Esto incluye la disponibilidad de equipos, tecnología y personal cualificado. El personal tiene las habilidades
          y la experiencia necesarias para implementar y gestionar el proyecto eficazmente. Además, se disponen de las
          instalaciones y el equipamiento necesarios para desarrollar todas las etapas del proyecto sin problemas.
\end{itemize}








% \section{Estimación de costos de adquisición e instalación del EDI e instrumentación adicional}
% En términos de costos de adquisición, un sistema de EDI de alta capacidad apto para una planta farmacéutica como AICA puede oscilar en un rango aproximado de 80,000 a 100,000 dólares. Este costo puede variar dependiendo de las especificaciones exactas del sistema, la marca y el proveedor.

% La instalación del sistema EDI puede requerir ajustes en la infraestructura existente de la planta. Este costo podría incluir la preparación del sitio, la instalación de la unidad de EDI, la integración con los sistemas existentes y las pruebas iniciales. Dicha instalación puede oscilar entre los 20,000 a 30,000 dólares, dependiendo de la complejidad de la instalación.

% La instrumentación adicional necesaria para apoyar el sistema EDI, que puede incluir bombas de alta presión, sensores de calidad del agua y sistemas de control avanzados, puede añadir entre 15,000 y 20,000 dólares adicionales al costo inicial. Estos costos pueden variar dependiendo de las necesidades específicas de la planta de AICA y de las condiciones particulares de la instalación.

% En total, la estimación inicial para la adquisición e instalación del sistema de EDI y la instrumentación adicional en la planta farmacéutica AICA oscilaría entre 115,000 y 150,000 dólares. Esta cifra representa una inversión inicial considerable, pero debe ser considerada en el contexto de los ahorros y beneficios potenciales a largo plazo que la tecnología EDI puede proporcionar a la planta.


% \section{Estimación de costos operativos y de mantenimiento}
% \section{Evaluación de los beneficios}
% \section{Análisis de retorno de inversión y viabilidad económica}



% % % Capitulo 6
\chapter{Conclusiones y recomendaciones}


La presente tesis ha llevado a cabo una
exploración teórica exhaustiva y rigurosa sobre la
optimización del sistema de purificación de agua en la industria
farmacéutica de AICA, específicamente en su planta de bulbos,
utilizando la tecnología del Electrodesionizador (EDI).
Esta investigación ha permitido comprender a profundidad
los desafíos y las ventajas potenciales de incorporar la
tecnología EDI en las operaciones de purificación de agua de AICA.

Es crucial enfatizar que esta investigación se basa en estudios teóricos y
modelado, ya que la implementación real de EDI en AICA no ha ocurrido.
Por lo tanto, las conclusiones extraídas aquí proporcionan un cimiento
robusto para la toma de decisiones futuras, pero deben validarse con la
implementación y experimentación real.

Una de las principales conclusiones es que la implementación teórica de
EDI podría mejorar significativamente la eficiencia de la purificación
del agua en comparación con los métodos convencionales. Los modelos
teóricos sugieren que la calidad del agua mejoraría en un 30\%,
reduciendo los contaminantes iónicos a niveles casi indetectables, lo que
llevaría a un menor rechazo de productos debido a problemas de calidad del agua.

Además, la adopción de la tecnología EDI podría generar ahorros significativos
en los costos operativos. Los cálculos indican que, con la optimización de
los recursos, los costos de operación podrían disminuir en hasta un 40\%.
Estos ahorros se deben a una menor necesidad de químicos para el proceso de
purificación y a una reducción en el mantenimiento y los costos de energía.

\section*{Desafíos encontrados}


A pesar de sus ventajas potenciales, la implementación teórica de la tecnología EDI en
la industria farmacéutica no está exenta de desafíos. Uno de los principales obstáculos
encontrados durante la realización de esta tesis fue la escasez de información detallada
y relevante sobre el funcionamiento del EDI en contextos de producción farmacéutica.

La información limitada sobre la instalación, operación y mantenimiento del EDI
en una industria farmacéutica presentó una barrera significativa al progreso. A
pesar de este desafío, se realizaron esfuerzos para obtener y analizar la información
existente, así como para interpretarla y aplicarla al contexto de AICA.

Además, se presentó un desafío teórico importante relacionado con la integración
del EDI en el sistema existente de purificación de agua de AICA. Para superar esto,
se realizaron modelados y simulaciones para comprender cómo el EDI podría
integrarse de manera eficiente sin interrumpir los procesos existentes.

\section*{Recomendaciones para futuras investigaciones}


Para futuros trabajos en esta área, se recomienda realizar estudios prácticos y
experimentales para validar los resultados obtenidos teóricamente en este estudio.
Implementar pruebas piloto del sistema EDI en una planta de AICA proporcionaría
datos valiosos y confirmaría o refutaría los hallazgos actuales.

Además, sería beneficioso investigar cómo la tecnología EDI podría integrarse
con otras tecnologías emergentes de tratamiento de agua. Por ejemplo,
la nanofiltración podría
trabajar en conjunto con la tecnología EDI para optimizar aún más el
proceso de purificación de agua.

Finalmente, es esencial continuar buscando y recopilando más información
sobre la implementación y el funcionamiento del EDI en la industria farmacéutica.
A medida que la tecnología continúa avanzando y más empresas comienzan a adoptarla,
es probable que la información y los estudios de caso disponibles aumenten.
Mantenerse al día con esta literatura será vital para cualquier trabajo futuro en esta área.

% Recomendaciones
\chapter*{Recomendaciones}
\phantomsection
\addcontentsline{toc}{chapter}{Recomendaciones}
Para futuros trabajos en esta área, se recomienda realizar estudios prácticos y
experimentales para validar los resultados obtenidos teóricamente en este estudio.
Implementar pruebas piloto del sistema EDI en una planta de AICA proporcionaría
datos valiosos y confirmaría o refutaría los hallazgos actuales.

Además, sería beneficioso investigar cómo la tecnología EDI podría integrarse
con otras tecnologías emergentes de tratamiento de agua. Por ejemplo,
la nanofiltración podría
trabajar en conjunto con la tecnología EDI para optimizar aún más el
proceso de purificación de agua.

Finalmente, es esencial continuar buscando y recopilando más información
sobre la implementación y el funcionamiento del EDI en la industria farmacéutica.
A medida que la tecnología continúa avanzando y más empresas comienzan a adoptarla,
es probable que la información y los estudios de caso disponibles aumenten.
Mantenerse al día con esta literatura será vital para cualquier trabajo futuro en esta área.




\phantomsection
\bibliography{zot_bibli.bib}
\phantomsection
\chapter*{Glosario}
\addcontentsline{toc}{chapter}{Glosario}

\large{Lista de siglas y abreviaturas}

\begin{itemize}
    \item \textbf{EDI} : Electrodesionización o Electrodesionizador.
    \item \textbf{RO} : Ósmosis Inversa.
    \item \textbf{PW} : Agua Purificada.
    \item \textbf{WFI} : Agua para inyección.
    \item \textbf{P\&ID} : Diagrama de tuberías e instrumentación.
    \item \textbf{PLC} : Controlador lógico programable.
    \item \textbf{DC} : Corriente directa.
\end{itemize}

\phantomsection

\begin{appendixs}

    \fontsize{10}{12}\selectfont



    \section{Requisitos de agua de alimentación para los módulos IP-LX}
    \renewcommand{\arraystretch}{1} % Ajustar padding en eje y
    \begin{longtable}{|>{\raggedright\arraybackslash}m{8cm} |>{\raggedright\arraybackslash}m{8cm}|}
        \toprule
        \textbf{Parámetro}                                                  & \textbf{Valor}              \\
        \midrule
        Fuente de agua de alimentación                                      & permeado de ósmosis inversa \\
        \hline
        Conductividad del agua de alimentación equivalente, incluyendo CO2* & < 40 µS/cm                  \\
        \hline
        Silica (SiO2)                                                       & < 1 ppm                     \\
        \hline
        Hierro, Manganeso, Sulfuro                                          & < 0.01 ppm                  \\
        \hline
        Cloro total                                                         & < 0.02 ppm como Cl2         \\
        \hline
        Dureza total                                                        & < 1.0 ppm como CaCO3        \\
        \hline
        Materia orgánica disuelta (TOC)                                     & < 0.5 ppm                   \\
        \hline
        Rango de pH de operación                                            & 4 – 11                      \\
        \hline
        Temperatura de operación                                            & 41 - 113 °F (5 – 45 °C)     \\
        \hline
        Presión de entrada                                                  & <100 psi (7 bar)            \\
        \bottomrule
    \end{longtable}
    \section{Diagrama P\&ID del sistema de ósmosis inversa}
    \insertimageboxed[\label{fig:P\&ID_OSMOSIS}]{P&ID_OSMOSIS}{scale=0.2}{0}{}
    \section{Conexionado de los módulos IP-LX}
    \insertimageboxed[\label{fig:edi_conexionado}]{edi_conexionado}{scale=0.7}{0}{}

\end{appendixs}



% FIN DEL DOCUMENTO
\end{document}