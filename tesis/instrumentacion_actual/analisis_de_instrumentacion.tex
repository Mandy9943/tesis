\chapter{Análisis de la instrumentación}

La instrumentación, como pieza fundamental de cualquier proceso de ingeniería, es una serie de
elementos que proporcionan el control y la supervisión necesarios para garantizar la eficiencia y la
seguridad de un sistema. El propósito de este capítulo es analizar la instrumentación actualmente
implementada en nuestro sistema de ósmosis inversa, examinando tanto los componentes físicos
como la red y los protocolos de comunicación que permiten su funcionamiento integrado.

Iniciaremos con una mirada detallada a la instrumentación existente, explorando su funcionalidad,
la interrelación entre los componentes y la forma en que cada pieza contribuye al objetivo global del
sistema de tratamiento de agua. Al entender completamente la configuración actual, podremos identificar
las áreas de mejora y explorar las oportunidades para la optimización y el crecimiento.

Finalmente en la siguiente parte del análisis, abordaremos los detalles de la red y los protocolos de
comunicación. Al igual que el sistema circulatorio en un organismo, la red y los protocolos de
comunicación son los que mantienen viva la instrumentación, permitiendo la comunicación y la colaboración
eficaces entre los diferentes elementos. Profundizaremos en cómo funcionan, cómo están configurados y
cómo impactan en la eficiencia general del sistema.

Es importante destacar que aunque hay numerosos instrumentos y equipos en la planta de tratamiento de agua en general,
el foco de este capítulo es proporcionar una descripción y análisis detallado  para el sistema
de ósmosis inversa y que serán esenciales para entender e implementar las mejoras propuestas.


\section{Instrumentación de la planta de ósmosis inversa}


La instrumentación en un sistema de ósmosis inversa es crucial para su funcionamiento eficiente y seguro. Esta sección se centrará en los distintos dispositivos de control que permiten la operación continua y segura de la planta de ósmosis inversa. Hablaremos de las válvulas, que regulan el flujo de agua y sustancias químicas en el sistema, las bombas que impulsan la circulación y tratan el agua, y los sensores que monitorizan las condiciones y parámetros clave, como la conductividad, la temperatura y el pH.\\

Además, discutiremos el controlador lógico programable (PLC), que es el cerebro de la operación. El PLC es responsable de la gestión y el control de todas las señales de entrada y salida de la planta, lo que implica tomar decisiones basadas en las lecturas de los sensores y ajustar los actuadores, como las bombas y las válvulas, para mantener el sistema funcionando de manera óptima.\\

Por último, exploraremos los diferentes módulos que el PLC necesita para interactuar con los demás componentes del sistema. Estos módulos son esenciales para la comunicación y el control eficaz del proceso de ósmosis inversa.\\


\subsection{Sensor de conductividad} \label{sec:sesor_conductividad}

En el proceso de ósmosis inversa, la conductividad es una variable esencial a ser controlada. Los sensores de
conductividad son, por tanto, componentes críticos en la planta, proporcionando datos en tiempo real que informan
sobre la eficiencia del proceso. Específicamente, son capaces de detectar cambios en la concentración de iones
en el agua, lo que puede ser indicativo de un problema con las membranas de ósmosis inversa.

El principio de funcionamiento de estos sensores radica en la medición de la conductividad eléctrica del agua,
que refleja su capacidad para transmitir corriente eléctrica. Esta propiedad está directamente relacionada
con la concentración de iones disueltos en el agua. El sensor aplica un voltaje a dos electrodos situados a una
distancia fija y mide la corriente resultante. Como la conductividad depende del contenido de sales en el agua,
un aumento de la conductividad sugiere una mayor concentración de iones, indicando una posible eficacia reducida
de las membranas de ósmosis inversa.

Los sensores de conductividad como el de la figura \ref{fig:sensorCLS16}, fabricado por Endress+Hauser, es un  ejemplos de este tipo de
instrumentos utilizados en la planta.

\insertimageboxed[\label{fig:sensorCLS16}]{instrumentacion/sensorCLS16}{scale=0.7}{0}{Sensor de conductividad CLS16-3D1A1P}

Este tipo de sensores entre los lugares donde se pueden encontrar, lo tenemos ubicado en el punto de salida de la primera
etapa del sistema de ósmosis inversa (en el flujo de permeado).


\begin{mytable}{6cm}{Datos técnicos del sensor de conductividad CLS16-3D1A1P.}{table:sensorCLS16}

        \hline
        \textbf{Modelo}                        & CLS16-3D1A1P                        \\
        \hline
        \textbf{Material}                      & Acero inoxidable 316 L (DIN 1.4435) \\
        \hline
        \textbf{Acabado}                       & Electropulido Ra < 0,8µm            \\
        \hline
        \textbf{Constante}                     & 0,1 (0,04 / 500 µS/cm)              \\
        \hline
        \textbf{Rango}                         & 0 a 20 µS/cm                        \\
        \hline
        \textbf{Conexiones}                    & 1"½ (38,10 mm) Tri-Clamp            \\
        \hline
        \textbf{Temperatura máxima del fluido} & 120 °C                              \\
        \hline
        \textbf{Fabricante}                    & Endress+Hauser                      \\
        \hline
   
\end{mytable}




% ------------Sensores de pH-----------
\subsection{Sensor de pH} \label{sec:sensor_ph}

La medición y control del pH en el agua tratada es crucial en una planta de ósmosis inversa. Los sensores de pH
desempeñan un papel vital en este aspecto, permitiendo la monitorización constante del pH del agua y facilitando
el control de la dosificación de hidróxido de sodio (NaOH).

Los sensores de pH operan basándose en el principio de medición del potencial electroquímico a través de una
celda compuesta por un electrodo de referencia y un electrodo de medición. La diferencia de potencial entre
estos electrodos está relacionada con el pH del medio acuoso. El electrodo de medición, fabricado generalmente
de vidrio, tiene una propiedad particular de presentar una diferencia de potencial con el agua que se encuentra en contacto, la cual es dependiente del pH.

Un sensor de pH como el CPS 11D-7AA2G de Endress+Hauser mostrado en la figura \ref{fig:sensorCPS} se utiliza
en la planta para controlar el pH después del filtro de 5 micras. La información de este sensor es utilizada
para el control de la dosificación de NaOH, ayudando a mantener el pH dentro de los rangos deseados, lo cual
es esencial para la eficiencia del proceso de ósmosis inversa.

\insertimageboxed[\label{fig:sensorCPS}]{instrumentacion/sensorCPS}{scale=0.8}{0}{Sensor de pH CPS 11D-7AA2G}

A continuación, se presenta la tabla \ref{table:sensorCPS} con las características técnicas principales del sensor de pH CPS 11D-7AA2G:\\



\begin{mytable}{6cm}{Características del sensor de pH CPS 11D-7AA2G.}{table:sensorCPS}
        \hline
        \textbf{Característica}       & \textbf{Descripción}   \\
        \hline
        \textbf{Modelo}               & CPS 11D-7AA2G Memosens \\
        \hline
        \textbf{Material}             & Vidrio                 \\
        \hline
        \textbf{Rango pH}             & 0-12                   \\
        \hline
        \textbf{Rango de temperatura} & -5 a 80°C              \\
        \hline
        \textbf{Longitud de la sonda} & 120 x 12 mm            \\
        \hline
        \textbf{Conector}             & tipo N con PG13,5      \\
        \hline
        \textbf{Fabricante}           & Endress+Hauser         \\
        \hline
  
\end{mytable}

% ------------Sensores de Temperatura-----------
\subsection{Sensor de temperatura} \label{sec:sensor_temp}

Las sondas de temperatura son un componente crítico en cualquier proceso industrial que requiera control preciso
de la temperatura. Son dispositivos que detectan cambios en las condiciones físicas y convierten los datos en
señales eléctricas que pueden ser leídas y monitorizadas. En el contexto de los sistemas de ósmosis inversa,
estas sondas son esenciales para monitorear y mantener las condiciones óptimas de temperatura que permiten la
eficacia del proceso.

El modelo TSPT-6702UAC de Endress+Hauser es un ejemplo de una sonda de temperatura de alta calidad. Funciona bajo
la clase A, que se refiere a su alta precisión y consistencia en la medición de la temperatura.
Este tipo de sondas son generalmente más precisas y estables que las sondas de clase B, lo que
las hace ideales para aplicaciones industriales que requieren mediciones precisas y repetibles.

Estas sondas están ubicadas en puntos clave del proceso de ósmosis inversa, como en la tubería antes de la
entrada de cada etapa de la ósmosis. Aquí, las sondas pueden monitorear continuamente la temperatura del agua,
proporcionando datos vitales que pueden ayudar a prevenir problemas y garantizar que el sistema funcione de manera óptima.


\insertimageboxed[\label{fig:sensor_temperatura}]{instrumentacion/sensor_temperatura}{scale=0.8}{0}{Sensor de temperatura TSPT-6702UAC}

A continuación, se proporcionan algunas de las características específicas de este sensor en la tabla \ref{table:sensor_temperatura}\\


\begin{mytable}{6cm}{Características del sensor de temperatura TSPT-6702UAC .}{table:sensor_temperatura}

        \hline
        \textbf{Característica}         & \textbf{Descripción} \\
        \hline
        \textbf{Modelo}                 & TSPT-6702UAC         \\
        \hline
        \textbf{Tipo}                   & Clase A              \\
        \hline
        \textbf{Precisión típica}       & +/- 0.15°C a 0°C     \\
        \hline
        \textbf{Valor de Alfa}          & 0.00385 °C$^{-1}$    \\
        \hline
        \textbf{Valor de resistencia}   & 100 ohm al 0°C       \\
        \hline
        \textbf{Rango de medición}      & 0°C a 200°C          \\
        \hline
        \textbf{Longitud de los cables} & 102 mm               \\
        \hline
        \textbf{Conexiones}             & ø 6 mm               \\
        \hline
        \textbf{Fabricante}             & Endress+Hauser       \\
        \hline
   
\end{mytable}

% ------------Sensores de Redox-----------
\subsection{Sensor de Redox} \label{sec:sensor_redox}

Los electrodos de Redox son dispositivos que se utilizan para medir el potencial de óxido-reducción (Redox) en una
solución. Su principal utilidad en los procesos industriales, incluido el tratamiento de agua por ósmosis inversa,
es proporcionar información en tiempo real sobre el estado de la solución, lo que permite ajustar los parámetros
del proceso en consecuencia.

El modelo CPS12D-7PA21 MEMOSENS de Endress+Hauser es un electrodo de Redox del tipo Orbisint. Estos electrodos
funcionan generando una diferencia de potencial eléctrico entre el electrodo y la solución a medida que se establece
un equilibrio electroquímico. Esta diferencia de potencial es proporcional al potencial Redox de la solución y puede
ser interpretada por un transmisor de Redox.

El CPS12D-7PA21 MEMOSENS es un electrodo robusto diseñado para resistir condiciones de proceso adversas, como el
contacto con fluidos agresivos, gracias a su construcción de vidrio inastillable. También puede operar en un amplio
rango de temperaturas, lo que lo hace adecuado para una variedad de aplicaciones.

En el sistema de ósmosis inversa en estudio, el electrodo de Redox CPS12D-7PA21 MEMOSENS como el de la figura \ref{fig:sensor_redox}
se sitúa justo después del filtro de 10 micras. Desde aquí, puede enviar sus mediciones a un transmisor de
Redox para su interpretación y uso en el control del proceso.

\insertimageboxed[\label{fig:sensor_redox}]{instrumentacion/sensor_redox}{scale=0.8}{0}{Sensor de Redox CPS12D-7PA21}


Aquí se detallan las características específicas del electrodo de Redox CPS12D-7PA21 MEMOSENS:\\

\begin{mytable}{6cm}{Características del sensor de Redox CPS12D-7PA21.}{table:sensor_redox}

        \hline
        \textbf{Característica}                & \textbf{Descripción}              \\
        \hline
        \textbf{Modelo}                        & CPS12D-7PA21 MEMOSENS             \\
        \hline
        \textbf{Tipo}                          & Orbisint                          \\
        \hline
        \textbf{Material}                      & Vidrio inastillable               \\
        \hline
        \textbf{Rango}                         & +/- 1500 mV                       \\
        \hline
        \textbf{Rango de temperatura}          & -15 a 80°C                        \\
        \hline
        \textbf{Longitud del electrodo}        & 120 mm                            \\
        \hline
        \textbf{Conector}                      & Standard con acoplamiento coaxial \\
        \hline
        \textbf{Temperatura máxima del fluido} & 20°C                              \\
        \hline
        \textbf{Fabricante}                    & Endress+Hauser                    \\
        \hline
 
\end{mytable}


% ------------Sensores de flujo-----------
\subsection{Sensor-Transmisor de flujo} \label{sec:sensor_flujo}

En cualquier proceso industrial, la medición precisa y la transmisión de los datos de flujo son esenciales para
garantizar la eficiencia y el correcto funcionamiento del sistema. En particular, los instrumentos que combinan
ambas funciones, conocidos como medidores de flujo y transmisores, son especialmente valiosos en la industria de
tratamiento de agua, como en los sistemas de ósmosis inversa. Proporcionan mediciones exactas de la tasa de flujo de
líquidos en distintos puntos del proceso y transmiten estos datos en tiempo real para su monitorización y control.

El modelo RAMC05-S4-SS-64S2- E90424*P6/Z de YOKOGAWA es un ejemplo perfecto de un medidor de flujo y transmisor en uno.
Este dispositivo funciona como un rotámetro, y su diseño permite no solo medir el flujo de líquidos sino también transmitir
estos datos para su monitorización remota o automatizada. Su ubicación en la tubería de permeado en la segunda etapa de la
ósmosis es estratégica, ya que permite un control constante y preciso del flujo de permeado en este punto crucial del proceso.

Por otro lado, el modelo DS20 07 YJ de MADDALENA es otro medidor de flujo y transmisor efectivo,  es un medidor de flujo de dial húmedo de chorro múltiple.
Este medidor se encuentra después del
filtro de 10 micras, proporcionando mediciones de flujo esenciales después de esta etapa de filtración.



Estos son los detalles específicos de ambos medidores de flujo y transmisores:\\

\insertimageboxed[\label{fig:sensor_transmisor_flujo}]{instrumentacion/sensor_transmisor_flujo}{scale=0.4}{0}{Sensor-Transmisor de flujo RAMC05-S4-SS-64S2- E90424}


\begin{mytable}{6cm}{Características del rotámetro RAMC05-S4-SS-64S2- E90424.}{table:sensor_transmisor_flujo}
  
        \hline
        \textbf{Característica}         & \textbf{Descripción}           \\
        \hline
        \textbf{Modelo}                 & RAMC05-S4-SS-64S2- E90424*P6/Z \\
        \hline
        \textbf{Tipo}                   & Rotámetro                      \\
        \hline
        \textbf{Material}               & 316 L                          \\
        \hline
        \textbf{Conexiones}             & 2" Triclamp                    \\
        \hline
        \textbf{Rango}                  & 400 a 4000 l/h                 \\
        \hline
        \textbf{Material de la carcasa} & Acero inoxidable               \\
        \hline
        \textbf{Opción}                 & 4-20 mA - 24Vdc                \\
        \hline
        \textbf{Fabricante}             & YOKOGAWA                       \\
        \hline
   
\end{mytable}

\insertimageboxed[\label{fig:sensor_transmisor_flujo2}]{instrumentacion/sensor_transmisor_flujo2}{scale=0.8}{0}{Sensor-Transmisor de flujo DS20 07 YJ}


\begin{mytable}{6cm}{Características del medidor de flujo DS20 07 YJ.}{table:sensor_transmisor_flujo2}
        \hline
        \textbf{Característica} & \textbf{Descripción}                           \\
        \hline
        \textbf{Modelo}         & DS20 07 YJ                                     \\
        \hline
        \textbf{Tipo}           & Multi-jet wet dial (Multi-jet con dial húmedo) \\
        \hline
        \textbf{Material}       & Latón recubierto de epoxi                      \\
        \hline
        \textbf{Conexiones}     & Roscado ø 1"½ gas                              \\
        \hline
        \textbf{Rango}          & 10 m³/h (caudal nominal)                       \\
        \hline
        \textbf{Fabricante}     & MADDALENA                                      \\
        \hline
\end{mytable}



\subsection{Sensor de Nivel}

El control del nivel de agua en el proceso de ósmosis inversa es una variable clave, específicamente en el tanque TK50 donde se almacena el agua pretratada. Para esta tarea esencial, se utiliza el sensor de nivel Liquicap FMI51, un dispositivo de medición de nivel por capacitancia desarrollado por Endress+Hauser, tal como se muestra en la Figura \ref{fig:sensor_nivel}. Este instrumento asegura que el proceso de ósmosis inversa se inicie solo cuando el nivel de agua en el tanque alcanza un punto establecido, lo que contribuye a optimizar la eficiencia del proceso.

El Liquicap FMI51 opera bajo el principio de la capacitancia. En el interior de su varilla sensora, este instrumento cuenta con dos electrodos que generan un campo eléctrico. Cuando el nivel del agua en el tanque varía, las propiedades dieléctricas del espacio entre los electrodos cambian, lo cual se traduce en una variación de la capacitancia. Esta variación es interpretada por el sensor y convertida en una señal de nivel que es utilizada para controlar el proceso.

\insertimageboxed[\label{fig:sensor_nivel}]{instrumentacion/sensor_nivel}{scale=0.4}{0}{Sensor de nivel Liquicap FMI51}


\begin{mytable}{6cm}{Características del Sensor de nivel Liquicap FMI51}{table:sensor_nivel}
        \hline
        \textbf{Fabricante}                       & Endress+Hauser                                                                                      \\
        \hline
        \textbf{Principio de medición}            & Capacitivo                                                                                          \\
        \hline
        \textbf{Rango de temperatura del proceso} & -80°C a +200°C                                                                                      \\
        \hline
        \textbf{Presión del proceso}              & Vacío a 100 bar                                                                                     \\
        \hline
        \textbf{Precisión}                        & Repetibilidad: 0,1\%, Error de linealidad para líquidos conductivos: <0,25\%                        \\
        \hline
        \textbf{Longitud total del sensor}        & 6m                                                                                                  \\
        \hline
        \textbf{Distancia máxima de medición}     & 0.1 a 4.0 m                                                                                         \\
        \hline
        \textbf{Comunicación}                     & 4...20mA HART, PFM                                                                                  \\
        \hline
        \textbf{Certificaciones / Aprobaciones}   & ATEX, FM, CSA, IEC Ex, TIIS, INMETRO, NEPSI, EAC, SIL                                               \\
        \hline
        \textbf{Limitaciones de aplicación}       & Espacio insuficiente hacia el techo, medios cambiantes no conductivos con conductividad < 100 μS/cm \\
        \hline
\end{mytable}



% ------------Sensores de Presión-----------
\subsection{Sensor-Transmisor de Presión} \label{sec:sensor_presion}

Los sensores de presión desempeñan un papel esencial en numerosos procesos industriales,
incluyendo la ósmosis inversa. Estos instrumentos son responsables de medir la presión en diferentes puntos
del sistema y transmitir esa información a un sistema de control para su seguimiento y análisis.

El principio de funcionamiento de estos dispositivos se basa en la aplicación de presión a un diafragma de
metal sensible, que causa su deformación. Esta deformación es detectada por un sensor, que la convierte en
una señal eléctrica. En el caso de los transmisores de presión, esta señal se transmite luego a un sistema de control en forma
de una señal estandarizada (generalmente 4-20 mA), lo que permite un fácil seguimiento y control de la presión en el proceso.

La importancia de estos instrumentos en la ósmosis inversa es notable. Dado que la presión es un factor
crítico en la ósmosis inversa, la capacidad de medir y controlar la presión a través de todo el sistema
es esencial para garantizar un rendimiento óptimo y prevenir posibles problemas, como la sobrepresión
que podría dañar las membranas de ósmosis.

En el sistema de ósmosis inversa estudiado, estos sensores se encuentran
ubicados en la tubería de concentrado en cada etapa de la ósmosis, así como a la entrada de cada etapa
de la ósmosis. Esta disposición permite el monitoreo constante y preciso de la presión, lo que es
vital para la operación eficiente y segura del sistema.

A continuación, se proporcionan las características específicas del sensor-transmisor de presión
modelo PTP31-A1C13S1AF1A fabricado por Endress+Hauser:\\

\insertimageboxed[\label{fig:sensor_transmisor_presion}]{instrumentacion/sensor_transmisor_presion}{scale=0.6}{0}{Sensor-Transmisor de presión PTP31-A1C13S1AF1A}


\begin{mytable}{6cm}{Características del sensor de presión PTP31-A1C13S1AF1A. }{table:sensor_transmisor_presion}
        \hline
        \textbf{Característica}         & \textbf{Descripción}               \\
        \hline
        \textbf{Modelo}                 & PTP31-A1C13S1AF1A                  \\
        \hline
        \textbf{Rango}                  & 0 a 40 bar (calibración 0-20 bar)  \\
        \hline
        \textbf{Pantalla}               & LCD                                \\
        \hline
        \textbf{Alimentación eléctrica} & 12 a 30 Vdc                        \\
        \hline
        \textbf{Salida}                 & Interruptor PNP, 3 hilos + 4-20 mA \\
        \hline
        \textbf{Conexión eléctrica}     & Conector M12 x 1.5                 \\
        \hline
        \textbf{Protección IP}          & IP 65                              \\
        \hline
        \textbf{Diafragma}              & AISI 316 L                         \\
        \hline
        \textbf{Fluido de llenado}      & Aceite de grado alimenticio        \\
        \hline
        \textbf{Conexión del proceso}   & Roscado G½" ISO228 macho           \\
        \hline
        \textbf{Fabricante}             & Endress+Hauser                     \\
        \hline
\end{mytable}




\subsection{Manómetro} \label{sec:indicador_manometro}

Los manómetros son instrumentos de medición de presión esenciales en cualquier proceso industrial, incluyendo el tratamiento de agua por ósmosis inversa. Proporcionan una medida de la presión existente en un punto específico del proceso, permitiendo ajustar y controlar parámetros críticos para garantizar la eficacia del sistema.

Los manómetros de tipo seco, como el modelo P600 de ITEC, funcionan basándose en la flexión de un tubo Bourdon (un tubo delgado y hueco que se curva en forma de C) por la presión del fluido. Al aumentar la presión, el tubo se endereza y este movimiento se traduce a una aguja en la esfera del manómetro para proporcionar una lectura de presión. Su diseño resistente y su facilidad de lectura los hacen idóneos para una amplia gama de aplicaciones industriales.

En el sistema de ósmosis inversa en estudio, los manómetros de tipo P600 se sitúan en puntos estratégicos: en cada filtro (de 10 micras y de 5 micras) y antes de la bomba que impulsa el agua a la segunda etapa de la ósmosis. La correcta monitorización de la presión en estas ubicaciones es vital para garantizar el adecuado funcionamiento del sistema y prevenir posibles problemas, como la sobrepresión que podría dañar las membranas de ósmosis.


\insertimageboxed[\label{fig:manometro}]{instrumentacion/manometro}{scale=0.5}{0}{Manómetro P600}


\begin{table}[H]
    \centering
    \caption{Características del manómetro P600.}
    \label{table:manometro}
    \begin{tabular}{| L{6cm} | L{6cm} |}

        \hline
        \textbf{Modelo}                  & P600                        \\
        \hline
        \textbf{Tipo}                    & Ejecución seca              \\
        \hline
        \textbf{Material}                & Acero inoxidable            \\
        \hline
        \textbf{Rango de presión}        & 0 a 10 bar                  \\
        \hline
        \textbf{Diámetro de la carcasa}  & 63 mm o 200 mm              \\
        \hline
        \textbf{Temperatura del proceso} & 20°C                        \\
        \hline
        \textbf{Conexión del proceso}    & Roscado ¼" gas radial o ½'' \\
        \hline
        \textbf{Fabricante}              & ITEC                        \\
        \hline
    \end{tabular}
\end{table}


\subsection{Indicador de Flujo} \label{sec:indicador_flujo}

Los indicadores de flujo son instrumentos indispensables en cualquier proceso industrial,
incluyendo el tratamiento de agua por ósmosis inversa. Estos dispositivos permiten medir
y controlar la cantidad de líquido que fluye por una tubería, proporcionando datos cruciales
para el funcionamiento correcto y eficiente del sistema.

Los indicadores de flujo de tipo rotámetro y de área variable son particularmente comunes
en la industria. Los rotámetros, como el modelo RAMC02-S4-SS-61S1-T90NNNZ de Yokogawa,
funcionan basándose en la elevación de un flotador en un tubo cónico debido al flujo del
fluido.

En el sistema de ósmosis inversa en estudio, estos indicadores de flujo se encuentran en
ubicaciones clave: como por ejemplo en la tubería de concentrado
de la segunda etapa de la ósmosis,así como en la tubería de permeado de la primera etapa de la ósmosis. Monitorear
el flujo en estas ubicaciones es esencial para garantizar la eficiencia y seguridad del
proceso.

A continuación, se presentan las características específicas de este indicador:\\

\insertimageboxed[\label{fig:indicador_flujo}]{instrumentacion/indicador_flujo}{scale=1.1}{0}{Indicador de flujo RAMC05-S4-SS-64V2-T90}


\begin{table}[H]
    \centering
    \caption{Características del dispositivo RAMC02-S4-SS-61S1-T90NNN*Z.}
    \label{table:indicador_flujo}
    \begin{tabular}{| L{6cm} | L{6cm} |}

        \hline
        \textbf{Modelo}                 & RAMC02-S4-SS-61S1-T90NNN*Z \\
        \hline
        \textbf{Tipo}                   & Rotámetro                  \\
        \hline
        \textbf{Material}               & 316 L                      \\
        \hline
        \textbf{Acabado}                & Decapado y pasivado        \\
        \hline
        \textbf{Conexiones}             & 1" clamp                   \\
        \hline
        \textbf{Rango}                  & 100 a 1000 lt/h            \\
        \hline
        \textbf{Material de la carcasa} & Acero inoxidable           \\
        \hline
        \textbf{Fabricante}             & Yokogawa                   \\
        \hline
    \end{tabular}
\end{table}



\subsection{Transmisores}

En la arquitectura de este sistema de ósmosis inversa, los transmisores son piezas esenciales que funcionan como vínculos de
comunicación entre los sensores o analizadores y el PLC (Controlador Lógico Programable).
Su función principal es transformar las señales eléctricas recibidas de los sensores en una forma que el
PLC pueda interpretar y utilizar para el control y monitorización del proceso. \\

Más allá de esta función de conversión de señales, los transmisores también cuentan con sistemas de alarmas e indicadores
integrados. Estos sistemas permiten detectar y alertar sobre cualquier desviación o
anomalía en los parámetros medidos, permitiendo una respuesta rápida para mantener la eficiencia y seguridad del proceso. \\



\subsection{Equipos de control}

En el vasto y complejo universo de la ingeniería de procesos, los equipos de 
control son los actores silenciosos que juegan un papel crucial en el funcionamiento eficiente y 
efectivo de cualquier sistema de tratamiento. Desde mantener condiciones óptimas hasta permitir ajustes precisos y 
oportunos, estos equipos son la columna vertebral de cualquier proceso industrial, incluyendo el tratamiento de agua 
mediante ósmosis inversa.\\

En esta sección, centraremos nuestro análisis en los distintos equipos de control presentes en nuestro subsistema.
 Examinaremos detenidamente equipos como las bombas y válvulas que conforman 
la instrumentación de este sistema, estudiando su funcionamiento, características y 
ubicación en el proceso. Al hacerlo, esperamos proporcionar una visión clara y completa de la instrumentación 
actual del sistema y destacar su importancia en el mantenimiento de un proceso de ósmosis inversa seguro y eficaz.\\

\subsubsection{Bombas de Alta Presión}

Las bombas de alta presión son elementos fundamentales en el sistema de ósmosis inversa. Son responsables de aplicar la presión necesaria para que se produzca la ósmosis, un aspecto crucial para el adecuado funcionamiento del sistema.\\

En el proceso que estamos analizando, se utilizan bombas centrífugas verticales de múltiples etapas, específicamente del modelo CRN10-7 de la marca GRUNDFOS. Estas bombas son conocidas por su eficiencia y durabilidad, lo que las hace ideales para aplicaciones de alta presión como la ósmosis inversa.\\

El funcionamiento de las bombas centrífugas se basa en la conversión de la energía cinética en energía de presión. El agua entra en la bomba y es impulsada por un impulsor que gira a alta velocidad. Cuando el agua sale del impulsor, su energía cinética se transforma en energía de presión a medida que su velocidad disminuye en la voluta o carcasa de la bomba.\\

Estas bombas están ubicadas antes de cada etapa de la ósmosis, donde su tarea es generar la presión necesaria para forzar el paso del agua a través de la membrana semi-permeable del sistema de ósmosis inversa.\\

A continuación, se presenta una tabla con las características técnicas más relevantes de las bombas de alta presión CRN10-7:\\

\insertimageboxed[\label{fig:bomba_centrifuga}]{instrumentacion/bomba_centrifuga}{scale=1.1}{0}{Bombas centrífuga CRN10-7}


\begin{table}[H]
    \centering
    \caption{Características de la bomba centrífuga vertical multietapa CRN10-7.}
    \label{table:bomba_centrifuga}
    \begin{tabular}{| L{6cm} | L{6cm} |}
        \hline
        \textbf{Modelo} & CRN10-7  \\
        \hline
        \textbf{Tipo} & Centrífuga vertical multietapa  \\
        \hline
        \textbf{Material} & AISI 316  \\
        \hline
        \textbf{Sello} & HUUE (Carburo de Tungsteno / EPDM)  \\
        \hline
        \textbf{Medio} & Agua ablandada  \\
        \hline
        \textbf{Temperatura de trabajo} & 20°C  \\
        \hline
        \textbf{Caudal} & 8000 lt/h  \\
        \hline
        \textbf{Presión de descarga} & 10 bar  \\
        \hline
        \textbf{Diámetro del impulsor} & n.a  \\
        \hline
        \textbf{Puerto de entrada} & 2" Tri-Clamp  \\
        \hline
        \textbf{Puerto de salida} & 2" Tri-Clamp  \\
        \hline
        \textbf{Suministro eléctrico} & 3 x 380V 60 Hz  \\
        \hline
        \textbf{Potencia} & 5,5 kW  \\
        \hline
        \textbf{Amperios} & 10,8  \\
        \hline
        \textbf{RPM} & 3600  \\
        \hline
        \textbf{Opciones} & Base de acero inoxidable  \\
        \hline
        \textbf{Fabricante} & GRUNDFOS  \\
        \hline
    \end{tabular}
\end{table}



\subsubsection{Bombas Dosificadoras}

Las bombas dosificadoras son las encargadas de administrar con precisión pequeñas 
 cantidades de químicos para alterar las características del agua. Estos químicos incluyen 
 agentes como el NaOH y Na2S2O5, que respectivamente alteran el pH y reducen el oxígeno disuelto en el agua.\\

En nuestro sistema, se utilizan dos bombas dosificadoras específicas de la marca PROMINENT: 
los modelos GALA G/L G1005 NPB 200UA 103000 figura \ref{fig:bomba_dosificadora} y G/L 1601 NPB 220UA 103 000 figura \ref{fig:bomba_dosificadora2} . Ambos modelos 
son reconocidos por su precisión y fiabilidad, y utilizan la tecnología de diafragma 
solenoide para garantizar una dosificación exacta.\\

El principio de funcionamiento de estas bombas se basa en la acción de un solenoide que atrae y repele un diafragma, creando un movimiento oscilante. Este movimiento provoca la succión del medio (el químico a dosificar) durante la fase de retracción del diafragma y su posterior expulsión durante la fase de compresión.\\

La bomba GALA G/L G1005 NPB 200UA 103000 se encuentra en el sistema de dosificación bomba-tanque de NaOH, mientras que la bomba G/L 1601 NPB 220UA 103 000 se utiliza en el sistema de dosificación bomba-tanque de Na2S2O5.\\

A continuación, se presentan las características técnicas de cada una de estas bombas dosificadoras:\\

\insertimageboxed[\label{fig:bomba_dosificadora}]{instrumentacion/bomba_dosificadora}{scale=0.8}{0}{Bomba dosificadora G1005}


\begin{table}[H]
    \centering
    \caption{Características de la bomba dosificadora G1005.}
    \label{table:bomba_dosificadora}
    \begin{tabular}{| L{6cm} | L{6cm} |}
        \hline
        \textbf{Modelo} & GALA G/L G1005 NPB 200UA 103000  \\
        \hline
        \textbf{Material} & Plexiglás \\
        \hline
        \textbf{Caudal} & 4,4 lt @ 10 bar \\
        \hline
        \textbf{Voltaje} & 100-230 V / 50-60 Hz \\
        \hline
        \textbf{Protección IP} & 65 \\
        \hline
        \textbf{Potencia} & 12W \\
        \hline
        \textbf{Fabricante} & PROMINENT \\
        \hline
    \end{tabular}
\end{table}

\insertimageboxed[\label{fig:bomba_dosificadora2}]{instrumentacion/bomba_dosificadora}{scale=0.8}{0}{Bomba dosificadora G/L 1601}


\begin{table}[H]
    \centering
    \caption{Características de la bomba dosificadora G/L 1601.}
    \label{table:bomba_dosificadora2}
    \begin{tabular}{| L{6cm} | L{6cm} |}
        \hline
        \textbf{Modelo} & G/L 1601 NPB 220UA  \\
        \hline
        \textbf{Tipo} & Diafragma de solenoide \\
        \hline
        \textbf{Medio} & Solución acuosa de Na2S2O5 \\
        \hline
        \textbf{Materiales} & Cabeza de dosificación: Acrílico, elemento de succión / presión: PVC, sellos: FPM-B, bolas: cerámica \\
        \hline
        \textbf{Caudal} & 1,1 lt/h \\
        \hline
        \textbf{Presión de descarga} & 16 bar \\
        \hline
        \textbf{Suministro eléctrico} & 100-230 V / 50-60 Hz \\
        \hline
        \textbf{Potencia} & 12 W \\
        \hline
        \textbf{Fabricante} & PROMINENT \\
        \hline
    \end{tabular}
\end{table}


\subsubsection{Válvulas de Retención}

Las válvulas de retención o check valves son elementos clave en cualquier sistema de tratamiento de agua o proceso industrial, ya que garantizan la unidireccionalidad del flujo en las tuberías. Su papel es esencial para mantener la seguridad y la eficiencia del sistema, ya que evitan el flujo inverso que podría causar daños en los equipos o interrumpir el proceso.\\

El papel de las válvulas de retención en nuestro sistema de ósmosis inversa es multifacético. Están ubicadas en varios puntos estratégicos a lo largo del proceso, incluyendo, pero no limitándose a, justo después de las bombas de alta presión, donde evitan que el fluido regrese a la bomba en caso de una parada o apagado. También se utilizan en la línea de dosificación de químicos, para asegurar un suministro constante y seguro de los reactivos necesarios para el proceso. Sin embargo, es importante destacar que pueden encontrarse en otros puntos del sistema donde sea necesario evitar el retroceso del flujo.\\

El principio de funcionamiento de las válvulas de retención es relativamente sencillo. Contienen un componente que se mueve libremente y permite el flujo en una dirección, pero bloquea el flujo si intenta moverse en la dirección contraria.\\

Para nuestro sistema, empleamos el modelo de válvula de retención Art. 048 VRTCV2 de RATTI. Este modelo está construido con un cuerpo de acero inoxidable AISI 316L, lo que garantiza su resistencia a la corrosión, y tiene una junta de PTFE.\\

Válvula de Retención 048 VRTCV2\\

\begin{table}[H]
    \centering
    \caption{Características del cuerpo.}
    \label{table:cuerpo}
    \begin{tabular}{| L{6cm} | L{6cm} |}
        \hline
        \textbf{Material del cuerpo} & AISI 316L \\
        \hline
        \textbf{Junta} & PTFE \\
        \hline
        \textbf{Diámetro} & 1½" \\
        \hline
        \textbf{Conexiones} & Abrazadera (clamp) \\
        \hline
        \textbf{Resorte} & Estándar \\
        \hline
        \textbf{Fabricante} & RATTI \\
        \hline
    \end{tabular}
\end{table}


\subsubsection{Válvulas Multiusos}

Las válvulas multiusos son componentes esenciales en cualquier sistema de tratamiento de agua. Actúan como puntos de control, permitiendo o impidiendo el paso de fluidos a través de las tuberías. La capacidad de controlar el flujo de agua y otros líquidos es crucial para el funcionamiento seguro y eficiente de todo el sistema. Son llamadas "multiusos" porque se utilizan en una variedad de aplicaciones dentro del sistema, dependiendo de las necesidades del proceso en particular.\\

En nuestro sistema de ósmosis inversa, las válvulas multiusos se encuentran en varios puntos críticos. Una ubicación importante es en las tuberías de concentrado de la ósmosis, en la línea que va al drenaje o que retorna al tanque de almacenamiento de agua de pretratamiento. Además, se utilizan en la línea de bypass que se encuentra después de la bomba de lavado químico. Estas ubicaciones no son exhaustivas, y es posible encontrar estas válvulas en otros puntos del sistema donde se requiera controlar el flujo de fluido.\\

El principio de funcionamiento de las válvulas multiusos es simple pero efectivo. Cuando la válvula está abierta, permite el flujo de fluido; cuando está cerrada, detiene el flujo.\\

Utilizamos el modelo J4M1G00 de RATTI para nuestras válvulas multiusos. Esta válvula está fabricada con acero inoxidable AISI 316L para una resistencia óptima a la corrosión y durabilidad a largo plazo. Tiene un diámetro de 1" - ¾" y se conecta mediante una conexión Tri-Clamp.\\

\begin{table}[H]
    \centering
    \caption{Características del cuerpo.}
    \label{table:valvula_multiusos}
    \begin{tabular}{| L{6cm} | L{6cm} |}
        \hline
        \textbf{Material del cuerpo} & AISI 316 L \\
        \hline
        \textbf{Diámetro} & 1" - ¾" \\
        \hline
        \textbf{Conexiones} & Tri-Clamp \\
        \hline
        \textbf{Fabricante} & OMAL \\
        \hline
    \end{tabular}
\end{table}

\subsubsection{Válvulas de Control ON/OFF}

En cualquier proceso industrial, las válvulas de control ON/OFF son elementos críticos para la gestión y regulación del flujo de fluidos. Su importancia radica en su habilidad para controlar de manera precisa y eficiente el flujo a través de las tuberías, permitiendo un total paso del fluido o su completa interrupción, según las demandas del sistema.\\

Un ejemplo específico de este tipo de válvulas es el modelo S386FPLY004Y05 de la reconocida empresa OMAL. Esta válvula cuenta con un cuerpo de hierro fundido revestido con EPOXY y EPDM, lo que la hace resistente y duradera. La característica más notable de esta válvula es su actuador neumático de retorno por resorte que, junto con su conjunto de accesorios, garantiza un funcionamiento fiable y una integración eficiente con el sistema de control del proceso.\\

La planta de ósmosis inversa que analizamos está equipada con numerosas válvulas de control ON/OFF, distribuidas estratégicamente en diferentes puntos del sistema. Son componentes indispensables que aseguran la correcta operación del proceso, y debido a su importancia, están presentes en gran cantidad en todas las áreas de la planta. Algunos lugares estratégicos donde podemos encontrar este tipo de válvulas puede ser la línea que precede a la bomba de alta presión y en la línea de concentrado de la primera etapa de ósmosis que retorna al tanque de pretratamiento.\\

Válvula de Control ON/OFF S386FPLY004Y05\\

\begin{table}[H]
    \centering
    \caption{Características del cuerpo.}
    \label{table:valvula_on_off}
    \begin{tabular}{| L{6cm} | L{6cm} |}
        \hline
        \textbf{Material del cuerpo} & Hierro fundido con revestimiento de EPOXY \\
        \hline
        \textbf{Revestimiento} & EPDM \\
        \hline
        \textbf{Vástago y disco} & Acero inoxidable AISI 316 \\
        \hline
        \textbf{Estilo del cuerpo} & "LUG" roscado para brida EN1092-1 \\
        \hline
        \textbf{Tamaño} & DN40 \\
        \hline
        \textbf{Actuador} & Neumático de retorno por resorte N.O. tipo SR30 y tornillos de regulación \\
        \hline
        \textbf{Accesorios} & Indicador de posición del eje KI02PP10, regulador de flujo de aire comprimido KAPR00101, filtro de aire de bronce 9490S001 \\
        \hline
        \textbf{Fabricante} & OMAL \\
        \hline
    \end{tabular}
\end{table}

\subsubsection{Válvulas de Retención de Presión de Inyección}

Las válvulas de retención de presión de inyección juegan un papel importante en el sistema de dosificación, especialmente en procesos industriales que requieren una precisión en la dosificación de ciertos productos químicos. Estas válvulas mantienen una presión constante de salida, evitando fluctuaciones y garantizando una dosificación precisa y estable.\\

En nuestra planta de ósmosis inversa, estas válvulas son vitales en la dosificación precisa de sustancias químicas como NaOH y Na2S2O5. Son componentes esenciales para asegurar la eficacia de las operaciones de dosificación y se encuentran estratégicamente ubicadas en las líneas de dosificación correspondientes. Sin embargo, su presencia no se limita a estas áreas, y se podrían encontrar en otras partes del sistema donde se necesite una dosificación precisa.\\

El modelo MFV-DK de PROMINENT, una empresa reconocida por la fabricación de componentes de alta calidad, es una de las válvulas utilizadas en nuestro sistema. Con un cuerpo de PVDF y un diafragma de PTFE, esta válvula es capaz de mantener una presión de alivio de hasta 16 bar, lo que asegura su capacidad para trabajar bajo condiciones exigentes.\\

Válvula de Retención de Presión de Inyección MFV-DK\\

\begin{table}[H]
    \centering
    \caption{Características del tipo MFV-DK, PVDF.}
    \label{table:valvula_retencion}
    \begin{tabular}{| L{6cm} | L{6cm} |}
        \hline
        \textbf{Tipo} & MFV-DK, PVDF \\
        \hline
        \textbf{Tamaño} & 1 \\
        \hline
        \textbf{Presión de alivio} & 16 bar \\
        \hline
        \textbf{Conector} & 6-12 mm \\
        \hline
        \textbf{Conector de bypass} & 6/4 mm \\
        \hline
        \textbf{Materiales} & Cuerpo de PVDF, diafragma de PTFE, sello de FPM \\
        \hline
        \textbf{Fabricante} & PROMINENT \\
        \hline
    \end{tabular}
\end{table}



\section{Comunicación de la Planta}

La red de comunicación en la planta de ósmosis inversa está
estratificada en tres niveles de automatización, con un sistema
de interconexión que garantiza una rápida y eficiente transmisión
de datos entre los distintos elementos de la red. En cada nivel
de la jerarquía de automatización, se utiliza un protocolo de
comunicación que se adapta mejor a las necesidades
de ese nivel, tal como se detalla en la Figura \ref{fig:comunicacion}.
A continuación, se analizan los protocolos de comunicación empleados
en cada nivel.

\textbf{Nivel de Campo}\\
En el nivel de campo, se encuentran los diversos dispositivos de campo,
como válvulas, bombas y sensores, donde se envían datos al proceso o se recogen del
mismo. Estos dispositivos de campo se conectan a los módulos de
entrada/salida, como los módulos CPX de Festo (especialmente diseñado para válvulas neumáticas) y ET 200s de Siemens,
a través de un bus de campo utilizando el protocolo Profibus DP.
El Profibus DP es un estándar de comunicación industrial de alta
velocidad y bajo retardo, especialmente diseñado para aplicaciones
de control de procesos. Permite un intercambio eficiente y robusto
de datos entre los dispositivos de campo y los módulos de control,
garantizando así un control preciso y en tiempo real del proceso.

\textbf{Nivel de Control}\\
En el nivel de control, los módulos CPX de Festo y ET 200s de
Siemens reciben las señales de los sensores y las transmiten al
autómata programable (PLC) maestro a través del protocolo Profibus DP.
Profibus DP es un estándar de comunicación industrial de alta velocidad
y bajo retardo, especialmente diseñado para aplicaciones de control de
procesos. Gracias a Profibus, el PLC maestro puede comunicarse de manera
eficiente con los módulos de periferia descentralizada,
permitiendo un control preciso y en tiempo real del proceso.

\textbf{Nivel de Supervisión}\\
En el nivel superior de la jerarquía, el PLC se comunica con el sistema SCADA (Control Supervisor y Adquisición de Datos) mediante una interfaz multipunto (MPI) que utiliza el protocolo de comunicación TCP/IP. TCP/IP es un conjunto de protocolos de comunicación de alto nivel que permite la transmisión de datos entre dispositivos en una red de área amplia, como Internet. Esta forma de comunicación permite la visualización y el control del proceso en tiempo real desde el sistema SCADA, proporcionando al operador una interfaz de usuario intuitiva y potente.

\insertimageboxed[\label{fig:comunicacion}]{comunicacion}{scale=0.3}{0}{Procolos de comunicación}


