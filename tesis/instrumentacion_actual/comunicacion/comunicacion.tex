\section{Comunicación de la planta}

Para entender completamente la operación de la planta de bulbos de AICA, es esencial entender los protocolos de comunicación
en uso y cómo facilitan la eficiencia de la planta. La comunicación en un ambiente industrial como este no es una cuestión
trivial, ya que cada nivel de la pirámide de automatización requiere protocolos específicos para garantizar la correcta
transmisión de información.

\begin{itemize}
      \item \textbf{Nivel de dispositivos de campo:} Este nivel comprende los dispositivos de campo, incluyendo sensores y
            actuadores. Estos dispositivos son responsables de adquirir datos a través de señales de campo de tipo intensidad
            (4-20 mA) o tensión (0-10, 24V DC), y ejecutar acciones físicas dentro del sistema. Se comunican con módulos de
            periferia descentralizada, como el ET200S de Siemens, y módulos eléctricos como el CPX de Festo a través de Profibus DP.

            Profibus DP es un protocolo de comunicación industrial eficiente y rentable que proporciona altas velocidades
            de transmisión de datos. Los módulos de periferia descentralizada ET 200S y los módulos eléctricos CPX de Festo
            actúan como esclavos en la configuración Profibus. Reciben las señales de los dispositivos de campo y las
            procesan para su posterior transmisión.


      \item \textbf{Nivel de control:} En el siguiente nivel de la jerarquía, los módulos de periferia descentralizada
            ET 200s y los módulos eléctricos CPX de Festo se comunican con un PLC maestro. Los módulos transmiten la
            información recolectada de los dispositivos de campo al PLC maestro a través de Profibus DP, lo que facilita
            un intercambio de datos rápido y eficiente.

            El PLC maestro procesa estos datos, tomando decisiones que influirán en el desarrollo del proceso. La
            configuración concebida para esta red de comunicación garantiza una comunicación eficiente entre el
            PLC maestro y los módulos que se le subordinan.

      \item \textbf{Nivel de supervisión:} En el nivel más alto de la jerarquía de control, encontramos el
            sistema SCADA (Supervisory Control And Data Acquisition). Este sistema se encarga de supervisar y
            controlar todo el proceso, recopilando información del PLC maestro. Para facilitar esta comunicación,
            se utiliza el protocolo OPC (OLE for Process Control).

            OPC es un estándar de comunicación en la industria de la automatización que permite la interacción
            fluida entre los sistemas SCADA y los PLC. Este protocolo, basado en la tecnología Microsoft OLE/COM,
            permite que el sistema SCADA reciba datos en tiempo real del PLC maestro, proporcionando una visión
            integral del proceso para su supervisión y control.


\end{itemize}
