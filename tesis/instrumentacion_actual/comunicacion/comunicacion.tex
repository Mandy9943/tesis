\section{Comunicación de la Planta}

La red de comunicación en la planta de ósmosis inversa está
estratificada en tres niveles de automatización, con un sistema
de interconexión que garantiza una rápida y eficiente transmisión
de datos entre los distintos elementos de la red. En cada nivel
de la jerarquía de automatización, se utiliza un protocolo de
comunicación que se adapta mejor a las necesidades
de ese nivel, tal como se detalla en la Figura \ref{fig:comunicacion}.
A continuación, se analizan los protocolos de comunicación empleados
en cada nivel.

\textbf{Nivel de Campo}\\
En este nivel, se encuentran
diversos dispositivos esenciales para el proceso. Los sensores se conectan a
los módulos ET 200s de Siemens mediante el protocolo de comunicación 4-20 mA
(4-20 miliamperios), que permite la transmisión de señales analógicas de corriente.
Esta conexión permite recopilar datos precisos del proceso. Por otro lado,
las válvulas se comunican con el módulo de comunicación CPX de Festo a través
del protocolo Profibus DP. El CPX de Festo es un dispositivo eléctrico multifuncional
que actúa como interfaz entre las válvulas y el sistema de control.
El protocolo Profibus DP es un estándar de comunicación industrial de alta
velocidad y baja latencia, diseñado específicamente para aplicaciones de control
de procesos. 

\textbf{Nivel de Control}\\
En el nivel de control, los módulos CPX de Festo y ET 200s de
Siemens reciben las señales de los sensores y las transmiten al
autómata programable (PLC) maestro a través del protocolo Profibus DP. Gracias a Profibus, el PLC maestro puede comunicarse de manera
eficiente con los módulos de periferia descentralizada,
permitiendo un control preciso y en tiempo real del proceso. 


\textbf{Nivel de Supervisión}\\
En el nivel superior de la jerarquía, el PLC se comunica con el sistema SCADA 
mediante una interfaz multipunto (MPI) que utiliza el protocolo de comunicación TCP/IP.
 TCP/IP es un conjunto de protocolos de comunicación de alto nivel que permite la 
 transmisión de datos entre dispositivos en una red de área amplia, como Internet.
  El protocolo TCP/IP es el protocolo de comunicación más utilizado en Internet y 
  se basa en el modelo de referencia OSI de siete capas. El protocolo TCP/IP se 
  utiliza en el nivel de supervisión porque permite la transmisión de datos a 
  través de redes de área amplia, lo que permite la supervisión remota de la planta.

\insertimageboxed[\label{fig:comunicacion}]{comunicacion}{scale=0.3}{0}{Protocolos de comunicación}
