\subsection{Sensor-transmisor de temperatura}

En los sensores de resistencia RTD (ver Figura \ref{fig:sensor_temperatura}), la resistencia
eléctrica varía al cambiar la temperatura. Son aptos para la medición de
temperaturas entre 200 °C y 600 °C y se caracterizan por su alta precisión de
medición y estabilidad a largo plazo. El elemento utilizado como resistencia más
común es la Pt100. Por norma, los sensores de resistencia RTD de
Endress+Hauser cumplen los requisitos de precisión de clase A según IEC
60751.\\

\subsubsection*{Principio de funcionamiento}

Es un detector de temperatura resistivo, es decir, un sensor de
temperatura basado en la variación de la resistencia de un conductor con
coeficiente positivo. Consiste en una película delgada de platino en una película
de plástico. Su resistencia varía con la temperatura, pasar corriente a través de
la RTD genera una caída de voltaje en ella. Al medir este voltaje se puede
determinar su resistencia y a su vez la temperatura pues, la relación entre la
resistencia y la temperatura es relativamente lineal.\\

Este instrumento se utiliza en múltiples ocasiones en este proceso de
purificación de agua, se localiza: en la entrada de agua de ambos grupos de
membranas, con el objetivo de determinar si la temperatura del flujo que desea
procesarse se encuentra en parámetros; también puede encontrarse en la salida
de concentrado de la segunda etapa de membranas.\\

\insertimageboxed[\label{fig:sensor_temperatura}]{instrumentacion/sensor_temperatura}{scale=0.5}{0}{Sensor transmisor de temperatura RTD}


\renewcommand{\arraystretch}{2}
\begin{table}[H]
    \centering
    \caption{Datos técnicos del sensor transmisor de temperatura RTD.}
    \label{table:sensor_temperatura}
    \begin{tabular}{| L{5cm} | L{5cm} |}
        \hline
        \textbf{Fabricante} & Endress+Hauser  \\
        \hline
        \textbf{Tipo de sensor} & RTD con conector TSPT-602UXX de 3 hilos  \\
        \hline
        \textbf{Rango de medición} & 0 a 200°C  \\
        \hline
        \textbf{Material} & AISI 316 L  \\
        \hline
        \textbf{Máxima temperature de operación} & 250 °C  \\
        \hline
        \textbf{Alpha} & 0.00385 °C-1  \\
        \hline
        \textbf{Salida} & 4 a 20 mA  \\
        \hline
   
    \end{tabular}
\end{table}

