\subsection{Sensor transmisor - indicador de conductividad}
El sensor transmisor-indicador de conductividad utilizado (ver Figura 2.10)
pertenece al fabricante ENDRESS + HAUSER y es a la vez indicador de
conductividad y temperatura, pues posee sensor de temperatura integrado. Este
transmisor es de célula conductiva, con dos señales de salida 0/4-20mA (señal
primaria más señal secundaria) y alimentación de 230VCA. Este tipo de
dispositivos son compatibles con reactivos biológicos lo cual ha sido probado y
certificado por la USP.\\

\subsubsection*{Principio de funcionamiento}
Dos electrodos son introducidos en el medio que desea medirse. Se aplica
un voltaje de alterna a dichos electrodos para generar una corriente en dicho
medio. La resistencia o en este caso su valor inverso, la conductancia (G) es
calculada con la ley de Ohm. La conductancia específica se determina usando
la constante k, la cual depende de la geometría de cada sensor.Por último, este
valor de lectura en $\mu S$ /cm se convierte en una señal de 4-20 mA\\

El sensor indicador-transmisor de conductividad se encuentra en las
tuberías de permeado de cada una de las etapas de purificación que posee la
ósmosis inversa, lo que permite determinar la calidad del agua que se ha
procesado.


\insertimageboxed[\label{fig:sensor_conductividad}]{instrumentacion/sensor_conductividad}{scale=0.5}{0}{Sensor y el indicador-transmisor de conductividad}


\renewcommand{\arraystretch}{2}
\begin{table}[H]
    \centering
    \caption{Datos técnicos del sensor y el indicador-transmisor de conductividad.}
    \label{table:sensor_conductividad}
    \begin{tabular}{| L{5cm} | L{5cm} |}
        \hline
        \textbf{Fabricante} & Endress+Hauser  \\
        \hline
        \textbf{Instrumento} & Liquiline M CM42  \\
        \hline
        \textbf{Celda} & Condumax CLS16  \\
        \hline
        \textbf{Rango de medición} & 0.04 a 500 µS/cm \\
        \hline
        \textbf{Temperatura del medio de medición} & -5 a 150 °C  \\
        \hline
        \textbf{Protección} & IP68  \\
        \hline
    \end{tabular}
\end{table}
