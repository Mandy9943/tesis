
\subsection{Transmisores}

En la arquitectura de este sistema de ósmosis inversa, los transmisores son piezas esenciales que funcionan como vínculos de
comunicación entre los sensores o analizadores y el PLC (Controlador Lógico Programable).
Su función principal es transformar las señales eléctricas recibidas de los sensores en una forma que el
PLC pueda interpretar y utilizar para el control y monitorización del proceso.

Más allá de esta función de conversión de señales, los transmisores también cuentan con sistemas de alarmas e indicadores
integrados. Estos sistemas permiten detectar y alertar sobre cualquier desviación o
anomalía en los parámetros medidos, permitiendo una respuesta rápida para mantener la eficiencia y seguridad del proceso.


