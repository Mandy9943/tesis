
\subsection{Transmisores}

En la arquitectura de este sistema de ósmosis inversa, los transmisores son piezas esenciales que funcionan como vínculos de
comunicación entre los sensores o analizadores y el PLC (Controlador Lógico Programable). 
Su función principal es transformar las señales eléctricas recibidas de los sensores en una forma que el 
PLC pueda interpretar y utilizar para el control y monitorización del proceso. \\

Más allá de esta función de conversión de señales, los transmisores también cuentan con sistemas de alarmas e indicadores 
integrados. Estos sistemas permiten detectar y alertar sobre cualquier desviación o
anomalía en los parámetros medidos, permitiendo una respuesta rápida para mantener la eficiencia y seguridad del proceso. \\

\subsubsection{Transmisores de Conductividad }

Los transmisores de conductividad son elementos vitales en diversos procesos industriales, incluyendo la ósmosis inversa. Aunque estos dispositivos no miden directamente la conductividad, desempeñan un papel esencial en la interpretación y transmisión de las mediciones de conductividad realizadas por un sensor.\\

El principio de funcionamiento de estos dispositivos es bastante sencillo pero fundamental. Un sensor de conductividad mide la capacidad de un medio (en este caso, el agua) para conducir la corriente eléctrica. Esta señal eléctrica es luego enviada al transmisor, que la convierte en una señal normalizada, típicamente de 4-20 mA, que puede ser fácilmente interpretada por otros dispositivos o sistemas de control.\\

El modelo CLM223-CD8110 de Endress+Hauser, es un equipo que, además de convertir y transmitir la señal de conductividad, también presenta una funcionalidad de alarma para valores altos de conductividad o errores del sistema. Además, este transmisor puede proporcionar una señal de temperatura, lo que aumenta su utilidad en el control de procesos.\\

En el sistema de ósmosis inversa en estudio, el transmisor de conductividad CLM223-CD8110 se sitúa a continuación de los sensores de conductividad, que se encuentran en el permeado a la salida de cada etapa de la ósmosis. De este modo, el transmisor juega un papel esencial en la monitorización y control de la pureza del agua.\\

Aquí se detallan las características específicas del transmisor de conductividad CLM223-CD8110:\\

\begin{table}[H]
    \centering
    \caption{Características del dispositivo CLM223-CD8110.}
    \label{table:dispositivoCLM223}
    \begin{tabular}{| L{5cm} | L{5cm} |}
        
        \hline
        \textbf{Modelo} & CLM223-CD8110  \\
        \hline
        \textbf{Rango} & 0 a 20µS/cm  \\
        \hline
        \textbf{Salida} & 2 x 4 a 20 mA (conductividad y temperatura)  \\
        \hline
        \textbf{Alarmas} & 2 relés x alta conductividad + error del sistema  \\
        \hline
        \textbf{Voltaje} & 24 V ac/dc  \\
        \hline
        \textbf{Opciones} & transmisor de temperatura  \\
        \hline
        \textbf{Fabricante} & Endress+Hauser  \\
        \hline
    \end{tabular}
\end{table}



\subsubsection{Transmisores de pH y REDOX }

Los transmisores de pH y REDOX son instrumentos clave en el análisis y control de procesos químicos e industriales, especialmente en sistemas como el tratamiento de agua por ósmosis inversa. Proporcionan mediciones precisas y fiables de dos parámetros críticos: el pH, que es una medida de la acidez o alcalinidad de una solución, y el potencial de reducción-oxidación (REDOX), que indica la capacidad de una solución para ganar o perder electrones.\\

El principio de medición de estos transmisores es potenciométrico. En el caso del pH, este método utiliza una celda de medición de pH que consiste en un electrodo de vidrio y un electrodo de referencia. La diferencia de potencial entre estos dos electrodos varía con el pH de la solución. En cuanto a la medición REDOX, se utiliza un electrodo REDOX en lugar del electrodo de vidrio, el cual genera una señal eléctrica que varía con el potencial REDOX de la solución.\\

En el sistema de ósmosis inversa en estudio, el modelo CPM 223-MR8010 Memosens de Endress+Hauser se utiliza para transmitir datos de pH y REDOX desde las ubicaciones de medición hasta los sistemas de control o monitorización. Este dispositivo se encuentra después del analizador de Redox, que está ubicado después del filtro de 10 micras, y después de los sensores de pH, que están ubicados luego del filtro de 5 micras. Esta ubicación es esencial para controlar y ajustar el proceso de ósmosis inversa en función de las condiciones del agua.\\

Las características específicas del modelo CPM 223-MR8010 Memosens de Endress+Hauser son las siguientes:\\

\begin{table}[H]
    \centering
    \caption{Características del Transmisor de pH y REDOX.}
    \label{table:dispositivoCPM223}
    \begin{tabular}{| L{5cm} | L{5cm} |}
        \hline
        \textbf{Característica} & \textbf{Descripción}  \\
        \hline
        \textbf{Modelo} & CPM 223-MR8010 Memosens  \\
        \hline
      
        \textbf{Diseño} & Transmisor de pH/ORP en caja de panel (96x96 mm)  \\
        \hline
        \textbf{Alarmas} & 2 relay + error de sistema  \\
        \hline
        \textbf{Dimensiones} & 96 mm x 96 mm x 146 mm (profundidad de montaje)  \\
        \hline
        \textbf{Protección de ingreso} & IP65  \\
        \hline
        \textbf{Entrada} & Transmisor de un solo canal  \\
        \hline
        \textbf{Salida / comunicación} & 0/4-20 mA, Hart, Profibus  \\
        \hline
       
        \textbf{Fabricante} & Endress+Hauser  \\
        \hline
    \end{tabular}
\end{table}
