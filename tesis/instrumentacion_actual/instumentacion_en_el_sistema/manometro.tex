\subsection{Indicador de presión}
Los manómetros (ver Figura \ref{fig:manometro}) se emplean generalmente para reflejar
la presión en las tuberías de entrada y salida de las membranas de la ósmosis
inversa, para así comprobar el buen funcionamiento de las bombas que impulsan
el agua en el proceso, en la planta estudiada, se ubica previo a la bomba
presurizadora de la segunda etapa de membranas.\\

\subsubsection*{Principio de funcionamiento}

Los manómetros Bourdon son óptimos para la medición de presión
relativa desde 0.6 hasta 7 bar. Debido a su tecnología mecánica no necesitan
energía auxiliar. Los muelles Bourdon consisten en tubos curvados en arco de
sección oval. A medida que se aplica presión al interior del tubo, éste tiende a
enderezarse. El trayecto del movimiento se transmite a un mecanismo y es la
medida de presión que se indica mediante una aguja.\\

\insertimageboxed[\label{fig:manometro}]{instrumentacion/manometro}{scale=0.5}{0}{Indicador depresión}


\renewcommand{\arraystretch}{2}
\begin{table}[H]
    \centering
    \caption{Datos técnicos del manómetro.}
    \label{table:manometro}
    \begin{tabular}{| L{5cm} | L{5cm} |}
        \hline
        \textbf{Fabricante} & ITEC  \\
        \hline
        \textbf{Sistema de medición} & Tubo de Bourdon  \\
        \hline
        \textbf{Modelo} & P600  \\
        \hline
        \textbf{Rango de medición} & 0 a 16 bar  \\
        \hline
        \textbf{Temperatura del medio de medición} & 0 a 200°C  \\
        \hline
        \textbf{Tip de protección} & P 65-EN60529/IEC 529  \\
        \hline
        \textbf{Diámetro} & 100mm  \\
        \hline
     
  
    \end{tabular}
\end{table}
