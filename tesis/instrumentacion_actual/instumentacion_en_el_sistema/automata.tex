\subsection{ Autómata Programable}
La planta de agua cuenta con un panel principal (CB1) que contiene un
SCADA y un PLC maestro, el cual, a través de una red Profibus, se conecta con
todos los subsistemas de la planta de tratamiento de agua entre los cuales se
incluye la ósmosis inversa.\\

El autómata programable empleado es el S7-300 perteneciente a
Siemens, de procedencia alemana, estos controladores ahorran espacio de
instalación y presentan un diseño modular. El fabricante ofrece una gama amplia
de productos de este tipo para el desempeño de tareas de bajo, mediano y/o
gran alcance. Los autómatas de Siemens se caracterizan por su robustez,
durabilidad y optimización en las tareas de control; además sus módulos pueden
emplearse para expandir el sistema de manera centralizada o para crear
estructuras descentralizadas de acuerdo con la tarea en cuestión.\\

El control de las variables, medios técnicos e instrumentos es fundamental
para el correcto funcionamiento de la ósmosis, para garantizarlo, se emplea en
la planta la CPU 315-2DP (ver Figura 2.11). Se empelan además módulos
externos como el módulo de periferia descentralizada ET200S (esclavo), que
recibe todas las señales correspondientes a la ósmosis y el módulo CPX de
Festo (esclavo) .\\

\insertimageboxed[\label{fig:automata}]{instrumentacion/automata}{scale=0.5}{0}{PLC S7-300 CPU 315-2DP}


\renewcommand{\arraystretch}{2}
\begin{table}[H]
    \centering
    \caption{Datos técnicos del autómata programable CPU 315-2DP}
    \label{table:automata}
    \begin{tabular}{| L{5cm} | L{5cm} |}
        \hline
        \textbf{Fabricante} & Siemens  \\
        \hline
        \textbf{CPU} & 315-2DP  \\
        \hline
        \textbf{Serie} & 6ES7V- 315-2AH14-0AB0  \\
        \hline
        \textbf{Unidad} & Modular \\
        \hline
        \textbf{Alimentación} & 24V DC  \\
        \hline
        \textbf{Memoria central} & 256 kbytes  \\
        \hline
        \textbf{Puerto de comunicación} & 1, RS485  \\
        \hline
        \textbf{Rango de voltaje permisible} & 19.2 - 28.8 V  \\
        \hline
        \textbf{Corriente de entrada} & 850 mA  \\
        \hline
    \end{tabular}
\end{table}


% \subsubsection{Módulo de periferia descentralizada ET 200s, interfaz IM 151-1 BASIC}

% La periferia descentralizada, también conocida como distribuida o de E/S
% remotas, consiste en implementar la señal de E/S próxima a los sensores,
% instrumentos y actuadores del sistema de control, reduciendo el cableado y por
% ello la masificación de canalizaciones. Este sistema de E/S (ver Figura 2.12) ha
% sido diseñado para montaje en caja de distribución y sus bornes de inserción
% rápida garantizan un cableado cómodo que pude soltarse fácilmente gracias a la
% nueva disposición de mecanismos de apertura por resortes. Permite desenchufar
% y sustituir varios módulos simultáneamente .\\

% \insertimageboxed[\label{fig:modulo_periferia}]{instrumentacion/modulo_periferia}{scale=0.5}{0}{Módulo de periferia descentralizada ET 200s, interfaz IM 151-1 BASIC}


% \renewcommand{\arraystretch}{2}
% \begin{table}[H]
%     \centering
%     \caption{Datos del  módulo de periferia descentralizada.}
%     \label{table:modulo_periferia}
%     \begin{tabular}{| L{5cm} | L{5cm} |}
%         \hline
%         \textbf{Fabricante} & Endress+Hauser  \\
%         \hline
%         \textbf{Instrumento} & Liquiline M CM42  \\
%         \hline
%         \textbf{Celda} & Condumax CLS16  \\
%         \hline
%         \textbf{Rango de medición} & 0.04 a 500 µS/cm \\
%         \hline
%         \textbf{Temperatura del medio de medición} & -5 a 150 °C  \\
%         \hline
%         \textbf{Protección} & IP68  \\
%         \hline
%     \end{tabular}
% \end{table}


% \subsubsection{Módulo eléctrico CPX}

% El terminal eléctrico CPX que se observa en la Figura 2.13, es un sistema
% periférico modular para terminales de válvulas a los cuales se les ha puesto
% especial cuidado en la adaptabilidad del terminal a las más diversas aplicaciones.
% Su estructura modular como unidad remota de entradas/salidas permite la
% configuración individual del número de válvulas, entradas y salidas adicionales
% en función de cada aplicación.\\

% \insertimageboxed[\label{fig:modulo_periferia}]{instrumentacion/modulo_periferia}{scale=0.5}{0}{Módulo de periferia descentralizada ET 200s, interfaz IM 151-1 BASIC}


% \renewcommand{\arraystretch}{2}
% \begin{table}[H]
%     \centering
%     \caption{Datos del  módulo de periferia descentralizada.}
%     \label{table:modulo_periferia}
%     \begin{tabular}{| L{5cm} | L{5cm} |}
%         \hline
%         \textbf{Fabricante} & Endress+Hauser  \\
%         \hline
%         \textbf{Instrumento} & Liquiline M CM42  \\
%         \hline
%         \textbf{Celda} & Condumax CLS16  \\
%         \hline
%         \textbf{Rango de medición} & 0.04 a 500 µS/cm \\
%         \hline
%         \textbf{Temperatura del medio de medición} & -5 a 150 °C  \\
%         \hline
%         \textbf{Protección} & IP68  \\
%         \hline
%     \end{tabular}
% \end{table}



