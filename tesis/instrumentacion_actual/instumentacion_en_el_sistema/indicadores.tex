
\subsection{Indicadores}

El control efectivo y eficiente de cualquier proceso industrial requiere no sólo la recopilación precisa de datos por parte de los sensores, sino también la capacidad de visualizar y entender rápidamente estos datos. Aquí es donde entran en juego los indicadores.\\

Los indicadores son dispositivos esenciales que proporcionan una visualización en tiempo real de los parámetros clave del proceso, permitiendo a los operadores y técnicos monitorear el estado del proceso y tomar decisiones informadas. En el sistema de tratamiento de agua por ósmosis inversa que estamos examinando, los indicadores desempeñan un papel fundamental en el monitoreo de variables críticas como la presión y el flujo.\\

\subsubsection{Manómetros}

Los manómetros son instrumentos de medición de presión esenciales en cualquier proceso industrial, incluyendo el tratamiento de agua por ósmosis inversa. Proporcionan una medida de la presión existente en un punto específico del proceso, permitiendo ajustar y controlar parámetros críticos para garantizar la eficacia del sistema.\\

Los manómetros de tipo seco, como el modelo P600 de ITEC, funcionan basándose en la flexión de un tubo Bourdon (un tubo delgado y hueco que se curva en forma de C) por la presión del fluido. Al aumentar la presión, el tubo se endereza y este movimiento se traduce a una aguja en la esfera del manómetro para proporcionar una lectura de presión. Su diseño resistente y su facilidad de lectura los hacen idóneos para una amplia gama de aplicaciones industriales.\\

En el sistema de ósmosis inversa en estudio, los manómetros de tipo P600 se sitúan en puntos estratégicos: en cada filtro (de 10 micras y de 5 micras) y antes de la bomba que impulsa el agua a la segunda etapa de la ósmosis. La correcta monitorización de la presión en estas ubicaciones es vital para garantizar el adecuado funcionamiento del sistema y prevenir posibles problemas, como la sobrepresión que podría dañar las membranas de ósmosis.\\

Aquí se detallan las características específicas del manómetro P600:\\

\begin{table}[H]
    \centering
    \caption{Características del dispositivo P600.}
    \label{table:dispositivoP600}
    \begin{tabular}{| L{5cm} | L{5cm} |}
        \hline
        \textbf{Característica} & \textbf{Descripción}  \\
        \hline
        \textbf{Modelo} & P600  \\
        \hline
        \textbf{Tipo} & Ejecución seca  \\
        \hline
        \textbf{Material} & Acero inoxidable  \\
        \hline
        \textbf{Rango de presión} & 0 a 10 bar  \\
        \hline
        \textbf{Diámetro de la carcasa} & 63 mm o 200 mm  \\
        \hline
        \textbf{Temperatura del proceso} & 20°C  \\
        \hline
        \textbf{Conexión del proceso} & Roscado ¼" gas radial o ½''  \\
        \hline
    \end{tabular}
\end{table}


\subsubsection{Indicadores de Flujo}

Los indicadores de flujo son instrumentos indispensables en cualquier proceso industrial, incluyendo el tratamiento de agua por ósmosis inversa. Estos dispositivos permiten medir y controlar la cantidad de líquido que fluye por una tubería, proporcionando datos cruciales para el funcionamiento correcto y eficiente del sistema.\\

Los indicadores de flujo de tipo rotámetro y de área variable son particularmente comunes en la industria. Los rotámetros, como el modelo RAMC02-S4-SS-61S1-T90NNNZ de Yokogawa, funcionan basándose en la elevación de un flotador en un tubo cónico debido al flujo del fluido. Por otro lado, los indicadores de flujo de área variable, como el modelo RAMC05-S4-SS-64V2-T90NNNZ de Yokogawa, operan mediante la variación del área de la sección transversal de la tubería en función del flujo.\\

En el sistema de ósmosis inversa en estudio, estos indicadores de flujo se encuentran en ubicaciones clave: el modelo RAMC02-S4-SS-61S1-T90NNNZ está en la tubería de concentrado en la segunda etapa de la ósmosis, mientras que el modelo RAMC05-S4-SS-64V2-T90NNNZ se sitúa en la tubería de permeado en la primera etapa de la ósmosis. Monitorear y controlar el flujo en estas ubicaciones es esencial para garantizar la eficiencia y seguridad del proceso.\\

A continuación, se presentan las características específicas de cada uno de estos indicadores de flujo:\\

\begin{table}[H]
    \centering
    \caption{Características del dispositivo RAMC02-S4-SS-61S1-T90NNN*Z.}
    \label{table:dispositivoRAMC02}
    \begin{tabular}{| L{5cm} | L{5cm} |}
        \hline
        \textbf{Característica} & \textbf{Descripción}  \\
        \hline
        \textbf{Modelo} & RAMC02-S4-SS-61S1-T90NNN*Z  \\
        \hline
        \textbf{Tipo} & Rotámetro  \\
        \hline
        \textbf{Material} & 316 L  \\
        \hline
        \textbf{Acabado} & Decapado y pasivado  \\
        \hline
        \textbf{Conexiones} & 1" clamp  \\
        \hline
        \textbf{Rango} & 100 a 1000 lt/h  \\
        \hline
        \textbf{Material de la carcasa} & Acero inoxidable  \\
        \hline
        \textbf{Fabricante} & Yokogawa  \\
        \hline
    \end{tabular}
\end{table}


\begin{table}[H]
    \centering
    \caption{Características del dispositivo RAMC05-S4-SS-64V2-T90NNN*Z.}
    \label{table:dispositivoRAMC05}
    \begin{tabular}{| L{5cm} | L{5cm} |}
        \hline
        \textbf{Característica} & \textbf{Descripción}  \\
        \hline
        \textbf{Modelo} & RAMC05-S4-SS-64V2-T90NNN*Z  \\
        \hline
        \textbf{Tipo} & Área variable  \\
        \hline
        \textbf{Material} & AISI 316 L DIN 1.4404  \\
        \hline
        \textbf{Superficies en contacto con el fluido} & Decapado y pasivado  \\
        \hline
        \textbf{Conexiones} & 2" Triclamp  \\
        \hline
        \textbf{Material de la carcasa} & Acero inoxidable  \\
        \hline
        \textbf{Rango} & 600 a 6000 l/h  \\
        \hline
        \textbf{Temperatura máxima} & 20°C  \\
        \hline
        \textbf{Fabricante} & Yokogawa  \\
        \hline
    \end{tabular}
\end{table}
