\subsection{Sensor de PH}

El valor del pH se usa como unidad de medición del nivel de acidez o
alcalinidad de un producto líquido conociendo la concentración de iones H+ en
una solución.\\

En procesos industriales como la producción de medicamentos,
procesamiento de alimentos envasados y la purificación de agua, los cuales son
muy sensibles al pH, se utiliza en ocasiones el potenciómetro como método de
control de los procesos y calidad .\\

El electrodo tiene integrado un sensor de temperatura de tipo PT100,
requiere solo un cable para la conexión y montaje. Mide de manera continua y
exacta el valor de temperatura compensado con el pH.\\

\subsubsection*{Principio de funcionamiento}

El elemento sensible al pH de los electrodos de vidrio es una bombilla de
vidrio que suministra un potencial electroquímico que depende del valor de pH
del producto. Este potencial se genera porque los pequeños iones H+ penetran
a través de la capa exterior de la membrana mientras los iones con carga
negativa más grandes permanecen en la solución .\\

El electrodo de referencia tiene un potencial conocido, constante y
estable, no así, el electrodo de trabajo, donde el potencial que se desarrolla
depende de la proporción de la concentración de los iones H+ presentes en la
solución que se está analizando, así como también de la temperatura en la que
esté.\\

Este sensor está ubicado luego del filtro de 5 micras (5 µm) y antes de las
válvulas que permiten en paso de agua a las membranas, de modo que la
medición se lleve a cabo, el PID procese el valor medido y la bomba dosificadora
de sosa suministre lo necesario en correspondencia al nivel del alcalinidad o
acidez que tenga el agua.\\

El sensor empleado en la planta de tratamiento de agua se observa en la
Figura \ref{fig:sensor_ph} y sus características técnicas en la Tabla \ref{table:sensor_ph}.

\insertimageboxed[\label{fig:sensor_ph}]{instrumentacion/sensor_ph}{scale=0.5}{0}{Sensor de ph}


\renewcommand{\arraystretch}{2}
\begin{table}[H]
    \centering
    \caption{Datos técnicos del sensor de pH.}
    \label{table:sensor_ph}
    \begin{tabular}{| L{5cm} | L{5cm} |}
        \hline
        \textbf{Fabricante} & Endress + Hauser  \\
        \hline
        \textbf{Rango de medición} & 0-14  \\
        \hline
        \textbf{Temperatura de operación} & -15 a 130 °C  \\
        \hline
        \textbf{Salida} & 4-24 mA  \\
        \hline
        \textbf{Material del cuerpo del sensor} & Cristal  \\
        \hline
        \textbf{Longitud del cable} & 1 m  \\
        \hline
        \textbf{Presión de operación} & ≤ 6 bar  \\
        \hline
        \textbf{Protección} & IP68  \\
        \hline
        \textbf{Tamaño} & ≤ 225 mm  \\
        \hline
    \end{tabular}
\end{table}