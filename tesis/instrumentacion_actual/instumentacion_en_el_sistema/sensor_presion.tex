\subsection{Sensor-indicador-transmisor de presión}

Este tipo de instrumentos permiten la medición y monitoreo de presiones
absolutas y relativas en procesos higiénicos; a través de células metálicas o
piezoeléctricos. El tipo de medición puede seleccionarse entre presión absoluta
y presión relativa. \\

El sensor instalado en la planta se muestra a continuación en la Figura
\ref{fig:sensor_presion}, así como sus características técnicas en la Tabla \ref{table:sensor_presion}.


\subsubsection*{Principio de funcionamiento}

La presión del proceso actúa sobre el diafragma del sensor de cerámica,
el cambio de capacitancia depende de la presión del sensor cerámico. Un
microprocesador evalúa la señal y conmuta la salida o salidas del valor medido
correspondiente. El sensor de cerámica es un sensor seco, es decir, no se
necesita líquido de llenado para la transmisión de presión. Esto significa que el
sensor puede soportar completamente un vacío. Se logra una durabilidad
extremadamente alta.\\

Los transmisores de presión se localizan a la entrada de las membranas
para medir la presión del flujo que les llega, así se comprueba el estado de
trabajo de las bombas de alta presión y se mantiene el cuidado de las
membranas; pueden encontrarse también en el flujo de concentrado de cada una
de las etapas de membranas.\\ 

\insertimageboxed[\label{fig:sensor_presion}]{instrumentacion/sensor_presion}{scale=0.5}{0}{Sensor-indicador-transmisor de presión}


\renewcommand{\arraystretch}{2}
\begin{table}[H]
    \centering
    \caption{Datos técnicos del sensor de pH.}
    \label{table:sensor_presion}
    \begin{tabular}{| L{5cm} | L{5cm} |}
        \hline
        \textbf{Fabricante} & Endress + Hauser  \\
        \hline
        \textbf{Modelo} & Cerephant PTP35  \\
        \hline
        \textbf{Tensión de alimentación} & 12-30 V DC  \\
        \hline
        \textbf{Rango de medición} & 0-40 bar  \\
        \hline
        \textbf{Temperatura de trabajo} & -40°C a 100°C- o hasta 135 °C durante para un tiempo máximo de 1 hora  \\
        \hline
        \textbf{Salida} & Digital/ 4-20 mA/ 3 cables  \\
        \hline
        \textbf{Indicación} & Display 7 segmentos de 4 dígitos  \\
        \hline
        \textbf{Material} & AISI 316 L  \\
        \hline
  
    \end{tabular}
\end{table}

