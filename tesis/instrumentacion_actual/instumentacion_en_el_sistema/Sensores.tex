\subsection{Sensor de conductividad} \label{sec:sesor_conductividad}

En el proceso de ósmosis inversa, la conductividad es una variable esencial a ser controlada. Los sensores de
conductividad son, por tanto, componentes críticos en la planta, proporcionando datos en tiempo real que informan
sobre la eficiencia del proceso. Específicamente, son capaces de detectar cambios en la concentración de iones
en el agua, lo que puede ser indicativo de un problema con las membranas de ósmosis inversa.

El principio de funcionamiento de estos sensores radica en la medición de la conductividad eléctrica del agua,
que refleja su capacidad para transmitir corriente eléctrica. Esta propiedad está directamente relacionada
con la concentración de iones disueltos en el agua. El sensor aplica un voltaje a dos electrodos situados a una
distancia fija y mide la corriente resultante. Como la conductividad depende del contenido de sales en el agua,
un aumento de la conductividad sugiere una mayor concentración de iones, indicando una posible eficacia reducida
de las membranas de ósmosis inversa.

Los sensores de conductividad como el de la figura \ref{fig:sensorCLS16}, fabricado por Endress+Hauser, es un  ejemplos de este tipo de
instrumentos utilizados en la planta.

\insertimageboxed[\label{fig:sensorCLS16}]{instrumentacion/sensorCLS16}{scale=0.7}{0}{Sensor de conductividad CLS16-3D1A1P}

Este tipo de sensores entre los lugares donde se pueden encontrar, lo tenemos ubicado en el punto de salida de la primera
etapa del sistema de ósmosis inversa (en el flujo de permeado).


\begin{mytable}{6cm}{Datos técnicos del sensor de conductividad CLS16-3D1A1P.}{table:sensorCLS16}

        \hline
        \textbf{Modelo}                        & CLS16-3D1A1P                        \\
        \hline
        \textbf{Material}                      & Acero inoxidable 316 L (DIN 1.4435) \\
        \hline
        \textbf{Acabado}                       & Electropulido Ra < 0,8µm            \\
        \hline
        \textbf{Constante}                     & 0,1 (0,04 / 500 µS/cm)              \\
        \hline
        \textbf{Rango}                         & 0 a 20 µS/cm                        \\
        \hline
        \textbf{Conexiones}                    & 1"½ (38,10 mm) Tri-Clamp            \\
        \hline
        \textbf{Temperatura máxima del fluido} & 120 °C                              \\
        \hline
        \textbf{Fabricante}                    & Endress+Hauser                      \\
        \hline
   
\end{mytable}




% ------------Sensores de pH-----------
\subsection{Sensor de pH} \label{sec:sensor_ph}

La medición y control del pH en el agua tratada es crucial en una planta de ósmosis inversa. Los sensores de pH
desempeñan un papel vital en este aspecto, permitiendo la monitorización constante del pH del agua y facilitando
el control de la dosificación de hidróxido de sodio (NaOH).

Los sensores de pH operan basándose en el principio de medición del potencial electroquímico a través de una
celda compuesta por un electrodo de referencia y un electrodo de medición. La diferencia de potencial entre
estos electrodos está relacionada con el pH del medio acuoso. El electrodo de medición, fabricado generalmente
de vidrio, tiene una propiedad particular de presentar una diferencia de potencial con el agua que se encuentra en contacto, la cual es dependiente del pH.

Un sensor de pH como el CPS 11D-7AA2G de Endress+Hauser mostrado en la figura \ref{fig:sensorCPS} se utiliza
en la planta para controlar el pH después del filtro de 5 micras. La información de este sensor es utilizada
para el control de la dosificación de NaOH, ayudando a mantener el pH dentro de los rangos deseados, lo cual
es esencial para la eficiencia del proceso de ósmosis inversa.

\insertimageboxed[\label{fig:sensorCPS}]{instrumentacion/sensorCPS}{scale=0.8}{0}{Sensor de pH CPS 11D-7AA2G}

A continuación, se presenta la tabla \ref{table:sensorCPS} con las características técnicas principales del sensor de pH CPS 11D-7AA2G:\\



\begin{mytable}{6cm}{Características del sensor de pH CPS 11D-7AA2G.}{table:sensorCPS}
        \hline
        \textbf{Característica}       & \textbf{Descripción}   \\
        \hline
        \textbf{Modelo}               & CPS 11D-7AA2G Memosens \\
        \hline
        \textbf{Material}             & Vidrio                 \\
        \hline
        \textbf{Rango pH}             & 0-12                   \\
        \hline
        \textbf{Rango de temperatura} & -5 a 80°C              \\
        \hline
        \textbf{Longitud de la sonda} & 120 x 12 mm            \\
        \hline
        \textbf{Conector}             & tipo N con PG13,5      \\
        \hline
        \textbf{Fabricante}           & Endress+Hauser         \\
        \hline
  
\end{mytable}

% ------------Sensores de Temperatura-----------
\subsection{Sensor de temperatura} \label{sec:sensor_temp}

Las sondas de temperatura son un componente crítico en cualquier proceso industrial que requiera control preciso
de la temperatura. Son dispositivos que detectan cambios en las condiciones físicas y convierten los datos en
señales eléctricas que pueden ser leídas y monitorizadas. En el contexto de los sistemas de ósmosis inversa,
estas sondas son esenciales para monitorear y mantener las condiciones óptimas de temperatura que permiten la
eficacia del proceso.

El modelo TSPT-6702UAC de Endress+Hauser es un ejemplo de una sonda de temperatura de alta calidad. Funciona bajo
la clase A, que se refiere a su alta precisión y consistencia en la medición de la temperatura.
Este tipo de sondas son generalmente más precisas y estables que las sondas de clase B, lo que
las hace ideales para aplicaciones industriales que requieren mediciones precisas y repetibles.

Estas sondas están ubicadas en puntos clave del proceso de ósmosis inversa, como en la tubería antes de la
entrada de cada etapa de la ósmosis. Aquí, las sondas pueden monitorear continuamente la temperatura del agua,
proporcionando datos vitales que pueden ayudar a prevenir problemas y garantizar que el sistema funcione de manera óptima.


\insertimageboxed[\label{fig:sensor_temperatura}]{instrumentacion/sensor_temperatura}{scale=0.8}{0}{Sensor de temperatura TSPT-6702UAC}

A continuación, se proporcionan algunas de las características específicas de este sensor en la tabla \ref{table:sensor_temperatura}\\


\begin{mytable}{6cm}{Características del sensor de temperatura TSPT-6702UAC .}{table:sensor_temperatura}

        \hline
        \textbf{Característica}         & \textbf{Descripción} \\
        \hline
        \textbf{Modelo}                 & TSPT-6702UAC         \\
        \hline
        \textbf{Tipo}                   & Clase A              \\
        \hline
        \textbf{Precisión típica}       & +/- 0.15°C a 0°C     \\
        \hline
        \textbf{Valor de Alfa}          & 0.00385 °C$^{-1}$    \\
        \hline
        \textbf{Valor de resistencia}   & 100 ohm al 0°C       \\
        \hline
        \textbf{Rango de medición}      & 0°C a 200°C          \\
        \hline
        \textbf{Longitud de los cables} & 102 mm               \\
        \hline
        \textbf{Conexiones}             & ø 6 mm               \\
        \hline
        \textbf{Fabricante}             & Endress+Hauser       \\
        \hline
   
\end{mytable}

% ------------Sensores de Redox-----------
\subsection{Sensor de Redox} \label{sec:sensor_redox}

Los electrodos de Redox son dispositivos que se utilizan para medir el potencial de óxido-reducción (Redox) en una
solución. Su principal utilidad en los procesos industriales, incluido el tratamiento de agua por ósmosis inversa,
es proporcionar información en tiempo real sobre el estado de la solución, lo que permite ajustar los parámetros
del proceso en consecuencia.

El modelo CPS12D-7PA21 MEMOSENS de Endress+Hauser es un electrodo de Redox del tipo Orbisint. Estos electrodos
funcionan generando una diferencia de potencial eléctrico entre el electrodo y la solución a medida que se establece
un equilibrio electroquímico. Esta diferencia de potencial es proporcional al potencial Redox de la solución y puede
ser interpretada por un transmisor de Redox.

El CPS12D-7PA21 MEMOSENS es un electrodo robusto diseñado para resistir condiciones de proceso adversas, como el
contacto con fluidos agresivos, gracias a su construcción de vidrio inastillable. También puede operar en un amplio
rango de temperaturas, lo que lo hace adecuado para una variedad de aplicaciones.

En el sistema de ósmosis inversa en estudio, el electrodo de Redox CPS12D-7PA21 MEMOSENS como el de la figura \ref{fig:sensor_redox}
se sitúa justo después del filtro de 10 micras. Desde aquí, puede enviar sus mediciones a un transmisor de
Redox para su interpretación y uso en el control del proceso.

\insertimageboxed[\label{fig:sensor_redox}]{instrumentacion/sensor_redox}{scale=0.8}{0}{Sensor de Redox CPS12D-7PA21}


Aquí se detallan las características específicas del electrodo de Redox CPS12D-7PA21 MEMOSENS:\\

\begin{mytable}{6cm}{Características del sensor de Redox CPS12D-7PA21.}{table:sensor_redox}

        \hline
        \textbf{Característica}                & \textbf{Descripción}              \\
        \hline
        \textbf{Modelo}                        & CPS12D-7PA21 MEMOSENS             \\
        \hline
        \textbf{Tipo}                          & Orbisint                          \\
        \hline
        \textbf{Material}                      & Vidrio inastillable               \\
        \hline
        \textbf{Rango}                         & +/- 1500 mV                       \\
        \hline
        \textbf{Rango de temperatura}          & -15 a 80°C                        \\
        \hline
        \textbf{Longitud del electrodo}        & 120 mm                            \\
        \hline
        \textbf{Conector}                      & Standard con acoplamiento coaxial \\
        \hline
        \textbf{Temperatura máxima del fluido} & 20°C                              \\
        \hline
        \textbf{Fabricante}                    & Endress+Hauser                    \\
        \hline
 
\end{mytable}


% ------------Sensores de flujo-----------
\subsection{Sensor-Transmisor de flujo} \label{sec:sensor_flujo}

En cualquier proceso industrial, la medición precisa y la transmisión de los datos de flujo son esenciales para
garantizar la eficiencia y el correcto funcionamiento del sistema. En particular, los instrumentos que combinan
ambas funciones, conocidos como medidores de flujo y transmisores, son especialmente valiosos en la industria de
tratamiento de agua, como en los sistemas de ósmosis inversa. Proporcionan mediciones exactas de la tasa de flujo de
líquidos en distintos puntos del proceso y transmiten estos datos en tiempo real para su monitorización y control.

El modelo RAMC05-S4-SS-64S2- E90424*P6/Z de YOKOGAWA es un ejemplo perfecto de un medidor de flujo y transmisor en uno.
Este dispositivo funciona como un rotámetro, y su diseño permite no solo medir el flujo de líquidos sino también transmitir
estos datos para su monitorización remota o automatizada. Su ubicación en la tubería de permeado en la segunda etapa de la
ósmosis es estratégica, ya que permite un control constante y preciso del flujo de permeado en este punto crucial del proceso.

Por otro lado, el modelo DS20 07 YJ de MADDALENA es otro medidor de flujo y transmisor efectivo,  es un medidor de flujo de dial húmedo de chorro múltiple.
Este medidor se encuentra después del
filtro de 10 micras, proporcionando mediciones de flujo esenciales después de esta etapa de filtración.



Estos son los detalles específicos de ambos medidores de flujo y transmisores:\\

\insertimageboxed[\label{fig:sensor_transmisor_flujo}]{instrumentacion/sensor_transmisor_flujo}{scale=0.4}{0}{Sensor-Transmisor de flujo RAMC05-S4-SS-64S2- E90424}


\begin{mytable}{6cm}{Características del rotámetro RAMC05-S4-SS-64S2- E90424.}{table:sensor_transmisor_flujo}
  
        \hline
        \textbf{Característica}         & \textbf{Descripción}           \\
        \hline
        \textbf{Modelo}                 & RAMC05-S4-SS-64S2- E90424*P6/Z \\
        \hline
        \textbf{Tipo}                   & Rotámetro                      \\
        \hline
        \textbf{Material}               & 316 L                          \\
        \hline
        \textbf{Conexiones}             & 2" Triclamp                    \\
        \hline
        \textbf{Rango}                  & 400 a 4000 l/h                 \\
        \hline
        \textbf{Material de la carcasa} & Acero inoxidable               \\
        \hline
        \textbf{Opción}                 & 4-20 mA - 24Vdc                \\
        \hline
        \textbf{Fabricante}             & YOKOGAWA                       \\
        \hline
   
\end{mytable}

\insertimageboxed[\label{fig:sensor_transmisor_flujo2}]{instrumentacion/sensor_transmisor_flujo2}{scale=0.8}{0}{Sensor-Transmisor de flujo DS20 07 YJ}


\begin{mytable}{6cm}{Características del medidor de flujo DS20 07 YJ.}{table:sensor_transmisor_flujo2}
        \hline
        \textbf{Característica} & \textbf{Descripción}                           \\
        \hline
        \textbf{Modelo}         & DS20 07 YJ                                     \\
        \hline
        \textbf{Tipo}           & Multi-jet wet dial (Multi-jet con dial húmedo) \\
        \hline
        \textbf{Material}       & Latón recubierto de epoxi                      \\
        \hline
        \textbf{Conexiones}     & Roscado ø 1"½ gas                              \\
        \hline
        \textbf{Rango}          & 10 m³/h (caudal nominal)                       \\
        \hline
        \textbf{Fabricante}     & MADDALENA                                      \\
        \hline
\end{mytable}



\subsection{Sensor de Nivel}

El control del nivel de agua en el proceso de ósmosis inversa es una variable clave, específicamente en el tanque TK50 donde se almacena el agua pretratada. Para esta tarea esencial, se utiliza el sensor de nivel Liquicap FMI51, un dispositivo de medición de nivel por capacitancia desarrollado por Endress+Hauser, tal como se muestra en la Figura \ref{fig:sensor_nivel}. Este instrumento asegura que el proceso de ósmosis inversa se inicie solo cuando el nivel de agua en el tanque alcanza un punto establecido, lo que contribuye a optimizar la eficiencia del proceso.

El Liquicap FMI51 opera bajo el principio de la capacitancia. En el interior de su varilla sensora, este instrumento cuenta con dos electrodos que generan un campo eléctrico. Cuando el nivel del agua en el tanque varía, las propiedades dieléctricas del espacio entre los electrodos cambian, lo cual se traduce en una variación de la capacitancia. Esta variación es interpretada por el sensor y convertida en una señal de nivel que es utilizada para controlar el proceso.

\insertimageboxed[\label{fig:sensor_nivel}]{instrumentacion/sensor_nivel}{scale=0.4}{0}{Sensor de nivel Liquicap FMI51}


\begin{mytable}{6cm}{Características del Sensor de nivel Liquicap FMI51}{table:sensor_nivel}
        \hline
        \textbf{Fabricante}                       & Endress+Hauser                                                                                      \\
        \hline
        \textbf{Principio de medición}            & Capacitivo                                                                                          \\
        \hline
        \textbf{Rango de temperatura del proceso} & -80°C a +200°C                                                                                      \\
        \hline
        \textbf{Presión del proceso}              & Vacío a 100 bar                                                                                     \\
        \hline
        \textbf{Precisión}                        & Repetibilidad: 0,1\%, Error de linealidad para líquidos conductivos: <0,25\%                        \\
        \hline
        \textbf{Longitud total del sensor}        & 6m                                                                                                  \\
        \hline
        \textbf{Distancia máxima de medición}     & 0.1 a 4.0 m                                                                                         \\
        \hline
        \textbf{Comunicación}                     & 4...20mA HART, PFM                                                                                  \\
        \hline
        \textbf{Certificaciones / Aprobaciones}   & ATEX, FM, CSA, IEC Ex, TIIS, INMETRO, NEPSI, EAC, SIL                                               \\
        \hline
        \textbf{Limitaciones de aplicación}       & Espacio insuficiente hacia el techo, medios cambiantes no conductivos con conductividad < 100 μS/cm \\
        \hline
\end{mytable}



% ------------Sensores de Presión-----------
\subsection{Sensor-Transmisor de Presión} \label{sec:sensor_presion}

Los sensores de presión desempeñan un papel esencial en numerosos procesos industriales,
incluyendo la ósmosis inversa. Estos instrumentos son responsables de medir la presión en diferentes puntos
del sistema y transmitir esa información a un sistema de control para su seguimiento y análisis.

El principio de funcionamiento de estos dispositivos se basa en la aplicación de presión a un diafragma de
metal sensible, que causa su deformación. Esta deformación es detectada por un sensor, que la convierte en
una señal eléctrica. En el caso de los transmisores de presión, esta señal se transmite luego a un sistema de control en forma
de una señal estandarizada (generalmente 4-20 mA), lo que permite un fácil seguimiento y control de la presión en el proceso.

La importancia de estos instrumentos en la ósmosis inversa es notable. Dado que la presión es un factor
crítico en la ósmosis inversa, la capacidad de medir y controlar la presión a través de todo el sistema
es esencial para garantizar un rendimiento óptimo y prevenir posibles problemas, como la sobrepresión
que podría dañar las membranas de ósmosis.

En el sistema de ósmosis inversa estudiado, estos sensores se encuentran
ubicados en la tubería de concentrado en cada etapa de la ósmosis, así como a la entrada de cada etapa
de la ósmosis. Esta disposición permite el monitoreo constante y preciso de la presión, lo que es
vital para la operación eficiente y segura del sistema.

A continuación, se proporcionan las características específicas del sensor-transmisor de presión
modelo PTP31-A1C13S1AF1A fabricado por Endress+Hauser:\\

\insertimageboxed[\label{fig:sensor_transmisor_presion}]{instrumentacion/sensor_transmisor_presion}{scale=0.6}{0}{Sensor-Transmisor de presión PTP31-A1C13S1AF1A}


\begin{mytable}{6cm}{Características del sensor de presión PTP31-A1C13S1AF1A. }{table:sensor_transmisor_presion}
        \hline
        \textbf{Característica}         & \textbf{Descripción}               \\
        \hline
        \textbf{Modelo}                 & PTP31-A1C13S1AF1A                  \\
        \hline
        \textbf{Rango}                  & 0 a 40 bar (calibración 0-20 bar)  \\
        \hline
        \textbf{Pantalla}               & LCD                                \\
        \hline
        \textbf{Alimentación eléctrica} & 12 a 30 Vdc                        \\
        \hline
        \textbf{Salida}                 & Interruptor PNP, 3 hilos + 4-20 mA \\
        \hline
        \textbf{Conexión eléctrica}     & Conector M12 x 1.5                 \\
        \hline
        \textbf{Protección IP}          & IP 65                              \\
        \hline
        \textbf{Diafragma}              & AISI 316 L                         \\
        \hline
        \textbf{Fluido de llenado}      & Aceite de grado alimenticio        \\
        \hline
        \textbf{Conexión del proceso}   & Roscado G½" ISO228 macho           \\
        \hline
        \textbf{Fabricante}             & Endress+Hauser                     \\
        \hline
\end{mytable}

