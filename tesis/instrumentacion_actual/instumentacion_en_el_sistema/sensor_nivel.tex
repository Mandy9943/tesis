\subsection{Sensor de Nivel}

El sensor Liquiphant T FTL31 (ver Figura \ref{fig:sensor_nivel}) es un detector de nivel para
líquidos. Está diseñado para aplicaciones en todo tipo de industrias. Se utiliza
para prevención contra sobrellenados o la protección de la bomba en seco en
los sistemas de limpieza. Con la función IO-Link, la configuración de parámetros
se puede hacer fácilmente. Es un interruptor de nivel para líquidos y es utilizado
en tanques, contenedores y tuberías, además, ofrece un punto de conmutación
exacto independiente de las propiedades cambiantes del producto.\\

\subsubsection*{Principio de funcionamiento}

El sensor de horquilla vibrante se excita a su frecuencia de resonancia. El
accionamiento se realiza piezoeléctricamente. La frecuencia de oscilación
cambia cuando la horquilla entra en contacto con el producto. El cambio se
analiza y se transfiere en una señal de conmutación.\\

\insertimageboxed[\label{fig:sensor_nivel}]{instrumentacion/sensor_nivel}{scale=0.5}{0}{Sensor de nivel}

\renewcommand{\arraystretch}{2}
\begin{table}[H]
    \centering
    \caption{Datos técnicos delsensor de nivel.}
    \label{table:sensor_nivel}
    \begin{tabular}{| L{5cm} | L{5cm} |}
        \hline
        \textbf{Fabricante} & Endress + Hauser  \\
        \hline
        \textbf{Tipo de sensor} & Digital (On-OFF)  \\
        \hline
        \textbf{Modelo} & Liquiphant T FTL20H  \\
        \hline
        \textbf{Temperatura de operación} & -40 a 150 °C  \\
        \hline
        \textbf{Tipo de conexión al proceso} & Rosca  \\
        \hline
        \textbf{Material del cuerpo del sensor} & Acero inoxidable 316L  \\
        \hline
        \textbf{Presión de operación} & ≤ -1 a 40 bar  \\
        \hline
        \textbf{Protección} & IP68  \\
        \hline
   
    \end{tabular}
\end{table}
