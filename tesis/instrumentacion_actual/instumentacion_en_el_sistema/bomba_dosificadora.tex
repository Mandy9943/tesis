\subsection{Bomba dosificadora}
Las bombas dosificadoras de metabisulfito de sodio (Na$_2$S$_2$O$_5$) y sosa cáustica (NaOH), como la que se refleja en la Figura \ref{fig:bomba_dosidicadora} tienen iguales características técnicas, solamente varían los medios en donde trabajan.

\insertimageboxed[\label{fig:bomba_dosidicadora}]{instrumentacion/bomba_dosificadora}{scale=0.5}{0}{Bomba dosificadora metabisufito de sodio}

\subsubsection*{Principio de funcionamiento}

La dosificación se lleva a cabo de la siguiente manera: el diafragma de
dosificación se fuerza en el extremo líquido; la presión en el lado del líquido hace
que la válvula de succión se cierre y el producto químico fluya fuera del lado del
líquido a través de la válvula de descarga. Seguidamente, el diafragma de
dosificación se fuerza hacia atrás, fuera de la unidad de bombeo. El vacío en el
lado líquido hace que la válvula de descarga se cierre y el producto químico
fresco fluye hacia la válvula de succión en el lado líquido, concluyendo así un
ciclo operativo.\\

Estas bombas se encuentran ubicadas en dos sitios diferentes: en la
entrada del agua al proceso luego del pretratamiento, en el caso de la
dosificación de metabisulfito de sodio. La bomba encargada de dosificar la sosa
cáustica se localiza después del tanque de almacenamiento de agua pretratada
y antes de la bomba de lavado químico. Su funcionamiento se lleva a cabo a
través de un controlador PID que se encuentra implementado y se configura
tanto de manera automática como manual. Las características técnicas de las
bombas dosificadoras se recogen en la Tabla \ref{table:bomba_dosidicadora}.

\renewcommand{\arraystretch}{2}
\begin{table}[H]
    \centering
    \caption{Datos técnicos de la bomba dosificadora.}
    \label{table:bomba_dosidicadora}
    \begin{tabular}{| L{5cm} | L{5cm} |}
        \hline
        \textbf{Fabricante} & ProMinent  \\
        \hline
        \textbf{Modelo} & gamma/L  \\
        \hline
        \textbf{Temperatura del químico dosificado} & -10 a +35 °C  \\
        \hline
        \textbf{Temperatura ambiente de operación} & -10 a +45 °C  \\
        \hline
        \textbf{Tensión de alimentación} & 100-230 V ± 10\%, 50-60 Hz  \\
        \hline
        \multicolumn{2}{|L{10cm}|}{Exclusivamente para la dosificación de líquidos por tanto no es admisible para la dosificación de gases o sólidos} \\
        \hline
    \end{tabular}
\end{table}