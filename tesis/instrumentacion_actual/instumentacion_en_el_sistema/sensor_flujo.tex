\subsection{Manómetro}
Un flotador se guía de forma concéntrica al tubo metálico cónico. La
posición de este flotador se transmite magnéticamente al indicador. El rotámetro
de tubo corto es empleado para medir los caudales de líquidos y gases y tiene
especial aplicación en medios turbios, opacos o agresivos. El instrumento (ver
Figura 2.8) se monta sobre una tubería vertical con dirección de flujo hacia arriba.
Los indicadores pueden variar sin influencia de la presión.\\

\subsubsection*{Principio de funcionamiento}

El principio de funcionamiento de este tipo de caudalímetros se basa en
un rotor helicoidal que gira libremente en el interior de un tubo cilíndrico. El
líquido de operación empuja las palas del rotor haciendo que giren a una
velocidad proporcional al caudal circulante. Una bobina de inducción (pickup)
montada exteriormente capta el giro de las palas de la hélice y genera una señal
eléctrica que, tratada por los diferentes sistemas electrónicos, proporciona valor
de caudal instantáneo, volumen total o parcial, salidas digitales y analógicas o
dosificación.\\

\insertimageboxed[\label{fig:sensor_flujo}]{instrumentacion/sensor_flujo}{scale=0.5}{0}{Sensor de flujo}


\renewcommand{\arraystretch}{2}
\begin{table}[H]
    \centering
    \caption{Datos técnicos del sensor de flujo.}
    \label{table:sensor_flujo}
    \begin{tabular}{| L{5cm} | L{5cm} |}
        \hline
        \textbf{Fabricante} & ROTA Yokogawa  \\
        \hline
        \textbf{Modelo} & RAMC  \\
        \hline
        \textbf{Material} & 316L  \\
        \hline
        \textbf{Presión de operación} & ≤ 700 bar  \\
        \hline
        \textbf{Tensión de alimentación} & 230 v AC +10\%/-15\%, 50/60 Hz  \\
        \hline
        \textbf{Salida} & 4 a 20 mA  \\
        \hline
        \textbf{Indicador} & Posee un display 7 segmentos de 8 dígitos con caracteres de 6 mm  \\
        \hline
     
  
    \end{tabular}
\end{table}

Los caudalímetros se encuentran ubicados en las tuberías de rechazo de
la ósmosis inversa, con el propósito de medir el flujo de concentrado y determinar
el agua que no logra procesarse; lo que permite estimar el posible estado de las
membranas.\\