\subsection{Transmisor de Conductividad } \label{sec:transmisor_conductividad}

Los transmisores de conductividad son elementos vitales en diversos procesos industriales, incluyendo la ósmosis inversa. Aunque estos dispositivos no miden directamente la conductividad, desempeñan un papel esencial en la interpretación y transmisión de las mediciones de conductividad realizadas por un sensor.

El modelo CLM223-CD8110 de Endress+Hauser, es un equipo que, además de convertir y transmitir la señal de conductividad, también presenta
una funcionalidad de alarma para valores altos de conductividad o errores del sistema. Además, este transmisor puede proporcionar una señal
de temperatura, lo que aumenta su utilidad en el control de procesos.

En el sistema de ósmosis inversa en estudio, los transmisores de conductividad como el CLM223-CD8110 se sitúan a continuación
de los sensores de conductividad, que usualmente se encuentran en el permeado a la salida de cada etapa de la ósmosis. De este modo, el
transmisor juega un papel esencial en la monitorización y control de la pureza del agua.

Aquí se detallan las características específicas del transmisor de conductividad CLM223-CD8110:

\insertimageboxed[\label{fig:transmisor_conductividad}]{instrumentacion/transmisor_conductividad}{scale=1.1}{0}{Transmisor CLM223-CD8110}


\begin{mytable}{6cm}{Características del transmisor CLM223-CD8110.}{table:transmisor_conductividad}
        \hline
        \textbf{Modelo}     & CLM223-CD8110                                    \\
        \hline
        \textbf{Rango}      & 0 a 20µS/cm                                      \\
        \hline
        \textbf{Salida}     & 2 x 4 a 20 mA (conductividad y temperatura)      \\
        \hline
        \textbf{Alarmas}    & 2 relés x alta conductividad + error del sistema \\
        \hline
        \textbf{Voltaje}    & 24 V ac/dc                                       \\
        \hline
        \textbf{Opciones}   & transmisor de temperatura                        \\
        \hline
        \textbf{Fabricante} & Endress+Hauser                                   \\
        \hline
\end{mytable}



\subsection{Transmisores de pH y REDOX } \label{sec:transmisor_ph_redox}

Los transmisores de pH y REDOX son instrumentos clave en el análisis y control de procesos químicos e industriales, especialmente en sistemas como el tratamiento de agua por ósmosis inversa. Proporcionan mediciones precisas y fiables de dos parámetros críticos: el pH, que es una medida de la acidez o alcalinidad de una solución, y el potencial de reducción-oxidación (REDOX), que indica la capacidad de una solución para ganar o perder electrones.

En el sistema de ósmosis inversa en estudio, el modelo CPM 223-MR8010 Memosens de Endress+Hauser se utiliza para transmitir datos de pH y REDOX
desde las ubicaciones de medición hasta los sistemas de control o monitorización. Estos dispositivos se encuentran después de los analizadores de Redox,
que están ubicados después del filtro de 10 micras, y después de los sensores de pH, que están ubicados luego del filtro de 5 micras. Esta ubicación
es esencial para controlar y ajustar el proceso de ósmosis inversa en función de las condiciones del agua.

Las características específicas del modelo CPM 223-MR8010 Memosens de Endress+Hauser son las siguientes:

\insertimageboxed[\label{fig:transmisor_pH}]{instrumentacion/transmisor_pH}{scale=1.1}{0}{Transmisor de pH y Redox CPM 223-MR8010}


\begin{mytable}{6cm}{Características del transmisor de pH y REDOX.}{table:transmisor_pH}
        \hline
        \textbf{Modelo}                & CPM 223-MR8010 Memosens                          \\
        \hline

        \textbf{Diseño}                & Transmisor de pH/ORP en caja de panel (96x96 mm) \\
        \hline
        \textbf{Alarmas}               & 2 relay + error de sistema                       \\
        \hline
        \textbf{Dimensiones}           & 96 mm x 96 mm x 146 mm (profundidad de montaje)  \\
        \hline
        \textbf{Protección de ingreso} & IP65                                             \\
        \hline
        \textbf{Entrada}               & Transmisor de un solo canal                      \\
        \hline
        \textbf{Salida / comunicación} & 0/4-20 mA, Hart, Profibus                        \\
        \hline

        \textbf{Fabricante}            & Endress+Hauser                                   \\
        \hline
\end{mytable}



\subsection{Bomba de Alta Presión}

Las bombas de alta presión son elementos fundamentales en el sistema de ósmosis inversa. Son responsables de aplicar la presión necesaria para que se produzca la ósmosis, un aspecto crucial para el adecuado funcionamiento del sistema.

En el proceso que estamos analizando, se utilizan bombas centrífugas verticales de múltiples etapas, específicamente del modelo CRN10-7 de la marca GRUNDFOS. Estas bombas son conocidas por su eficiencia y durabilidad, lo que las hace ideales para aplicaciones de alta presión como la ósmosis inversa.

El funcionamiento de las bombas centrífugas se basa en la conversión de la energía cinética en energía de presión. El agua entra en la bomba y es impulsada por un impulsor que gira a alta velocidad. Cuando el agua sale del impulsor, su energía cinética se transforma en energía de presión a medida que su velocidad disminuye en la voluta o carcasa de la bomba.

Estas bombas están ubicadas antes de cada etapa de la ósmosis, donde su tarea es generar la presión necesaria para forzar el paso del agua a través de la membrana semi-permeable del sistema de ósmosis inversa.


\insertimageboxed[\label{fig:bomba_centrifuga}]{instrumentacion/bomba_centrifuga}{scale=0.9}{0}{Bombas centrífuga CRN10-7}

A continuación, se presenta una tabla con las características técnicas más relevantes de las bombas de alta presión CRN10-7:

\begin{mytable}{6cm}{Características de la bomba centrífuga vertical multietapa CRN10-7.}{table:bomba_centrifuga}
        \hline
        \textbf{Modelo}                 & CRN10-7                            \\
        \hline
        \textbf{Tipo}                   & Centrífuga vertical multietapa     \\
        \hline
        \textbf{Material}               & AISI 316                           \\
        \hline
        \textbf{Medio}                  & Agua ablandada                     \\
        \hline
        \textbf{Temperatura de trabajo} & 20°C                               \\
        \hline
        \textbf{Caudal}                 & 8000 l/h                           \\
        \hline
        \textbf{Presión de descarga}    & 10 bar                             \\
        \hline
        \textbf{Puerto de entrada}      & 2" Tri-Clamp                       \\
        \hline
        \textbf{Puerto de salida}       & 2" Tri-Clamp                       \\
        \hline
        \textbf{Suministro eléctrico}   & 3 x 380V 60 Hz                     \\
        \hline
        \textbf{Potencia}               & 5,5 kW                             \\
        \hline
        \textbf{Amperios}               & 10,8                               \\
        \hline
        \textbf{RPM}                    & 3600                               \\
        \hline
        \textbf{Fabricante}             & GRUNDFOS                           \\
        \hline
\end{mytable}



\subsection{Bomba Dosificadoras}

Las bombas dosificadoras son las encargadas de administrar con precisión pequeñas
cantidades de químicos para alterar las características del agua. Estos químicos incluyen
agentes como el NaOH y Na$_2$S$_2$O$_5$, que respectivamente alteran el pH y reducen el oxígeno disuelto en el agua.

En nuestro sistema, se utilizan bombas dosificadoras de la marca PROMINENT, modelo GALA G/L G1005 NPB 200UA 103000
figura \ref{fig:bomba_dosificadora}. Este modelo
es reconocido por su precisión y fiabilidad, y utilizan la tecnología de diafragma
solenoide para garantizar una dosificación exacta.

El principio de funcionamiento de estas bombas se basa en la acción de un solenoide que atrae y repele un diafragma, creando un movimiento oscilante. Este movimiento provoca la succión del medio (el químico a dosificar) durante la fase de retracción del diafragma y su posterior expulsión durante la fase de compresión.

Las bombas GALA G/L G1005 NPB 200UA 103000 se encuentran en el sistema de dosificación bomba-tanque de NaOH, así como en el sistema de dosificación bomba-tanque de Na$_2$S$_2$O$_5$.


\insertimageboxed[\label{fig:bomba_dosificadora}]{instrumentacion/bomba_dosificadora}{scale=0.8}{0}{Bomba dosificadora G1005}

A continuación, se presentan las características técnicas estas bombas dosificadoras:

\begin{mytable}{6cm}{Características de la bomba dosificadora G1005.}{table:bomba_dosificadora}
        \hline
        \textbf{Modelo}        & GALA G/L G1005 NPB 200UA 103000 \\
        \hline
        \textbf{Material}      & Plexiglás                       \\
        \hline
        \textbf{Caudal}        & 4,4 l a 10 bar                  \\
        \hline
        \textbf{Voltaje}       & 100-230 V / 50-60 Hz            \\
        \hline
        \textbf{Protección IP} & 65                              \\
        \hline
        \textbf{Potencia}      & 12W                             \\
        \hline
        \textbf{Fabricante}    & PROMINENT                       \\
        \hline
\end{mytable}

\subsection{Válvula de Retención} \label{sec:valvula_retencion}

Las válvulas de retención o check valves son elementos clave en cualquier sistema de tratamiento de agua o proceso industrial, ya que garantizan la unidireccionalidad del flujo en las tuberías. Su papel es esencial para mantener la seguridad y la eficiencia del sistema, ya que evitan el flujo inverso que podría causar daños en los equipos o interrumpir el proceso.

El papel de las válvulas de retención en nuestro sistema de ósmosis inversa es multifacético. Están ubicadas en varios puntos estratégicos a lo largo del proceso, incluyendo, pero no limitándose a, justo después de las bombas de alta presión, donde evitan que el fluido regrese a la bomba en caso de una parada o apagado. También se utilizan en la línea de dosificación de químicos, para asegurar un suministro constante y seguro de los reactivos necesarios para el proceso. Sin embargo, es importante destacar que pueden encontrarse en otros puntos del sistema donde sea necesario evitar el retroceso del flujo.

El principio de funcionamiento de las válvulas de retención es relativamente sencillo. Contienen un componente que se mueve libremente y permite el flujo en una dirección, pero bloquea el flujo si intenta moverse en la dirección contraria.

En el sistema se emplea el modelo de válvula de retención Art. 048 VRTCV2 de RATTI, figura \ref{fig:valvula_retencion}. Este modelo está construido con un cuerpo de acero inoxidable AISI 316L, lo que garantiza su resistencia a la corrosión, y tiene una junta de PTFE.

\insertimageboxed[\label{fig:valvula_retencion}]{instrumentacion/valvula_retencion}{scale=0.8}{0}{Válvula de retención 048 VRTCV2}


\begin{mytable}{6cm}{Características de la Válvula de retención 048 VRTCV2.}{table:valvula_retencion}
        \hline
        \textbf{Modelo } & Art. 048 VRTCV2         \\
        \hline
        \textbf{Material del cuerpo} & AISI 316L          \\
        \hline
        \textbf{Junta}               & PTFE               \\
        \hline
        \textbf{Diámetro}            & 1½"                \\
        \hline
        \textbf{Conexiones}          & Abrazadera (clamp) \\
        \hline
        \textbf{Resorte}             & Estándar           \\
        \hline
        \textbf{Fabricante}          & RATTI              \\
        \hline
\end{mytable}


% \subsection{Válvulas de operación} \label{sec:valvula_multi}

% Las válvulas de operación son componentes esenciales en cualquier sistema de tratamiento de agua. Estas permiten o impiden el paso de fluidos a través de las tuberías. Mediante estas válvulas el sistema es capaz de adoptar otras configuraciones, además proporcionan un sistema de seguridad para cuando falla el control automático.

% La válvula neumática de accionamiento manual ARES con dispositivo de bloqueo juega un papel crucial en el funcionamiento de la planta. Este componente de alta resistencia facilita la gestión precisa del flujo de diferentes medios operativos en las líneas de proceso. El versátil diseño de la válvula permite su instalación en una variedad de configuraciones, contribuyendo a la eficiencia y al control efectivo del sistema.

% En particular, la válvula puede estar situada en la tubería de desagüe del concentrado después de la primera etapa de ósmosis inversa. Esta ubicación es estratégica, ya que permite el control del flujo de desechos y contribuye al funcionamiento óptimo del proceso de purificación. Además, otra posible ubicación para la válvula es la tubería de retorno del concentrado después de la segunda etapa de ósmosis inversa, donde su presencia asegura un manejo adecuado del concentrado y evita posibles problemas de contrapresión.

% El principio de operación de la válvula neumática ARES se basa en el control manual de la presión del aire. Esta característica permite un ajuste sutil y personalizado de la válvula para permitir el paso de los medios operativos a diferentes presiones, lo que garantiza un control eficiente y preciso del flujo de fluidos a través de la planta.


% \insertimageboxed[\label{fig:valvula_operacion}]{instrumentacion/valvula_operacion}{scale=0.35}{0}{Válvula neumática de accionamiento manual ARES}

% \begin{table}[H]
%     \centering
%     \caption{Características de la válvula de accionamiento manual.}
%     \label{table:valvula_operacion}
%     \begin{tabular}{| L{6cm} | L{6cm} |}
%         \hline
%         \textbf{Material del cuerpo}    & A351-CF8M (316 S.S.)                                 \\
%         \hline
%         \textbf{Posibilidad de montaje} & En todas las posiciones: vertical, plana o inclinada \\
%         \hline
%         \textbf{Rango disponible}       & De DN 10 a DN 50                                     \\
%         \hline
%         \textbf{Presión}                & De 0 a 25 bar                                        \\
%         \hline
%         \textbf{Temperatura}            & De –10°C a 180°C                                     \\
%         \hline
%         \textbf{Viscosidad máxima}      & 600 cst (mm$^2$ / s)                                 \\
%         \hline
%     \end{tabular}
% \end{table}


\subsection{Válvula de Control} \label{sec:valvula_OnOff}

En cualquier proceso industrial, las válvulas de control son elementos críticos para la gestión y regulación del flujo de fluidos.
Su importancia radica en su habilidad para controlar de manera precisa y eficiente el flujo a través de las tuberías, permitiendo un
total paso del fluido o su completa interrupción, según las demandas del sistema.

El siguiente componente que se analizará es la válvula de asiento en ángulo ARES con conexiones especiales. Esta pieza juega un papel vital en el control y dirección del flujo de fluidos a lo largo del sistema. Gracias a su diseño robusto y capacidades de alta resistencia, es una parte indispensable de la operación total de la planta.

% La válvula puede encontrarse estratégicamente ubicada en la línea que precede a la bomba de alta presión. Esta ubicación es vital ya que la válvula de asiento en ángulo facilita el control de flujo preciso necesario antes de que el fluido entre en la bomba de alta presión, lo que permite que la bomba funcione a una eficiencia óptima. Otra ubicación crítica para la válvula es en la línea de concentrado que retorna del tanque de pretratamiento después de la primera etapa de ósmosis inversa. Aquí, la válvula permite controlar eficientemente el flujo de fluido concentrado, evitando posibles problemas de contrapresión.

% La válvula ARES opera bajo el principio de control de fluido, donde el fluido de control puede ser aire comprimido, lubricado o seco, gas o medios neutros. Esta característica ofrece un gran nivel de control y precisión en la gestión del flujo de fluidos en diversas condiciones de operación.

\insertimageboxed[\label{fig:valvula_on_off}]{instrumentacion/valvula_on_off}{scale=0.4}{0}{Válvula de asiento en ángulo ARES.}

\begin{mytable}{6cm}{Características de la válvula de asiento en ángulo ARES con conexiones especiales.}{table:ares_angle_valve}

        \hline
        \textbf{Material del cuerpo}    & AISI 316 L                                                                                                                                                       \\
        \hline
        \textbf{Posibilidad de montaje} & En todas las posiciones: vertical, plana o inclinada                                                                                                             \\
        \hline
        \textbf{Rango disponible}       & De DN 15 a DN 50 en las versiones de doble efecto, retorno por resorte N.C. desde arriba y debajo del enchufe, retorno por resorte N.O. desde debajo del enchufe \\
        \hline
        \textbf{Medios operativos}      & Aire, agua, alcohol, petróleo, soluciones salinas, vapor, etc. (siempre que sean compatibles con AISI 316L o PTFE)                                               \\
        \hline
        \textbf{Presión}                & De 0 a 16 bar                                                                                                                                                    \\
        \hline
        \textbf{Temperatura}            & De –10°C a 180°C                                                                                                                                                 \\
        \hline
        \textbf{Viscosidad máxima}      & 600 cst (mm2/s)                                                                                                                                                  \\
        \hline
        \hline
        \textbf{Fabricante } & OMAL        \\
        \hline
\end{mytable}


\subsection{Válvula de contrapresión}

Las válvulas de contrapresión juegan un papel importante en el sistema de dosificación, especialmente en procesos industriales
que requieren una precisión en la dosificación de ciertos productos químicos. Estas válvulas mantienen una presión constante
de salida, evitando fluctuaciones y garantizando una dosificación precisa y estable.

En nuestra planta de ósmosis inversa, estas válvulas son vitales en la dosificación precisa de sustancias químicas como
NaOH y Na$_2$S$_2$O$_5$. Son componentes esenciales para asegurar la eficacia de las operaciones de dosificación y
se encuentran estratégicamente ubicadas en las líneas de dosificación correspondientes. Sin embargo, su presencia
no se limita a estas áreas, y se podrían encontrar en otras partes del sistema donde se necesite una dosificación
precisa.

El tipo MFV-DK de PROMINENT mostrada en la figura \ref{fig:valvula_retencion_presion}, una empresa reconocida por la fabricación de componentes de alta
calidad, es una de las válvulas utilizadas en nuestro sistema. Con un cuerpo de PVDF y un diafragma
de PTFE, esta válvula es capaz de mantener una presión de alivio de hasta 16 bar, lo que asegura su
capacidad para trabajar bajo condiciones exigentes.

\insertimageboxed[\label{fig:valvula_retencion_presion}]{instrumentacion/valvula_retencio_presion}{scale=0.6}{0}{Válvula de contrapresión}


\begin{mytable}{6cm}{Características de la válvula de contrapresión.}{table:valvula_retencion_presion}
        \hline
        \textbf{Tipo}               & MFV-DK, PVDF                                    \\
        \hline
        \textbf{Tamaño}             & 1                                               \\
        \hline
        \textbf{Presión de alivio}  & 16 bar                                          \\
        \hline
        \textbf{Conector}           & 6-12 mm                                         \\
        \hline
        \textbf{Conector de bypass} & 6/4 mm                                          \\
        \hline
        \textbf{Materiales}         & Cuerpo de PVDF, diafragma de PTFE, sello de FPM \\
        \hline
        \textbf{Fabricante}         & PROMINENT                                       \\
        \hline
\end{mytable}
