\subsection{Equipos de control}

En el vasto y complejo universo de la ingeniería de procesos, los equipos de 
control son los actores silenciosos que juegan un papel crucial en el funcionamiento eficiente y 
efectivo de cualquier sistema de tratamiento. Desde mantener condiciones óptimas hasta permitir ajustes precisos y 
oportunos, estos equipos son la columna vertebral de cualquier proceso industrial, incluyendo el tratamiento de agua 
mediante ósmosis inversa.\\

En esta sección, centraremos nuestro análisis en los distintos equipos de control presentes en nuestro subsistema.
 Examinaremos detenidamente equipos como las bombas y válvulas que conforman 
la instrumentación de este sistema, estudiando su funcionamiento, características y 
ubicación en el proceso. Al hacerlo, esperamos proporcionar una visión clara y completa de la instrumentación 
actual del sistema y destacar su importancia en el mantenimiento de un proceso de ósmosis inversa seguro y eficaz.\\

\subsubsection{Bombas de Alta Presión}

Las bombas de alta presión son elementos fundamentales en el sistema de ósmosis inversa. Son responsables de aplicar la presión necesaria para que se produzca la ósmosis, un aspecto crucial para el adecuado funcionamiento del sistema.\\

En el proceso que estamos analizando, se utilizan bombas centrífugas verticales de múltiples etapas, específicamente del modelo CRN10-7 de la marca GRUNDFOS. Estas bombas son conocidas por su eficiencia y durabilidad, lo que las hace ideales para aplicaciones de alta presión como la ósmosis inversa.\\

El funcionamiento de las bombas centrífugas se basa en la conversión de la energía cinética en energía de presión. El agua entra en la bomba y es impulsada por un impulsor que gira a alta velocidad. Cuando el agua sale del impulsor, su energía cinética se transforma en energía de presión a medida que su velocidad disminuye en la voluta o carcasa de la bomba.\\

Estas bombas están ubicadas antes de cada etapa de la ósmosis, donde su tarea es generar la presión necesaria para forzar el paso del agua a través de la membrana semi-permeable del sistema de ósmosis inversa.\\

A continuación, se presenta una tabla con las características técnicas más relevantes de las bombas de alta presión CRN10-7:\\

\insertimageboxed[\label{fig:bomba_centrifuga}]{instrumentacion/bomba_centrifuga}{scale=1.1}{0}{Bombas centrífuga CRN10-7}


\begin{table}[H]
    \centering
    \caption{Características de la bomba centrífuga vertical multietapa CRN10-7.}
    \label{table:bomba_centrifuga}
    \begin{tabular}{| L{6cm} | L{6cm} |}
        \hline
        \textbf{Modelo} & CRN10-7  \\
        \hline
        \textbf{Tipo} & Centrífuga vertical multietapa  \\
        \hline
        \textbf{Material} & AISI 316  \\
        \hline
        \textbf{Sello} & HUUE (Carburo de Tungsteno / EPDM)  \\
        \hline
        \textbf{Medio} & Agua ablandada  \\
        \hline
        \textbf{Temperatura de trabajo} & 20°C  \\
        \hline
        \textbf{Caudal} & 8000 lt/h  \\
        \hline
        \textbf{Presión de descarga} & 10 bar  \\
        \hline
        \textbf{Diámetro del impulsor} & n.a  \\
        \hline
        \textbf{Puerto de entrada} & 2" Tri-Clamp  \\
        \hline
        \textbf{Puerto de salida} & 2" Tri-Clamp  \\
        \hline
        \textbf{Suministro eléctrico} & 3 x 380V 60 Hz  \\
        \hline
        \textbf{Potencia} & 5,5 kW  \\
        \hline
        \textbf{Amperios} & 10,8  \\
        \hline
        \textbf{RPM} & 3600  \\
        \hline
        \textbf{Opciones} & Base de acero inoxidable  \\
        \hline
        \textbf{Fabricante} & GRUNDFOS  \\
        \hline
    \end{tabular}
\end{table}



\subsubsection{Bombas Dosificadoras}

Las bombas dosificadoras son las encargadas de administrar con precisión pequeñas 
 cantidades de químicos para alterar las características del agua. Estos químicos incluyen 
 agentes como el NaOH y Na2S2O5, que respectivamente alteran el pH y reducen el oxígeno disuelto en el agua.\\

En nuestro sistema, se utilizan dos bombas dosificadoras específicas de la marca PROMINENT: 
los modelos GALA G/L G1005 NPB 200UA 103000 figura \ref{fig:bomba_dosificadora} y G/L 1601 NPB 220UA 103 000 figura \ref{fig:bomba_dosificadora2} . Ambos modelos 
son reconocidos por su precisión y fiabilidad, y utilizan la tecnología de diafragma 
solenoide para garantizar una dosificación exacta.\\

El principio de funcionamiento de estas bombas se basa en la acción de un solenoide que atrae y repele un diafragma, creando un movimiento oscilante. Este movimiento provoca la succión del medio (el químico a dosificar) durante la fase de retracción del diafragma y su posterior expulsión durante la fase de compresión.\\

La bomba GALA G/L G1005 NPB 200UA 103000 se encuentra en el sistema de dosificación bomba-tanque de NaOH, mientras que la bomba G/L 1601 NPB 220UA 103 000 se utiliza en el sistema de dosificación bomba-tanque de Na2S2O5.\\

A continuación, se presentan las características técnicas de cada una de estas bombas dosificadoras:\\

\insertimageboxed[\label{fig:bomba_dosificadora}]{instrumentacion/bomba_dosificadora}{scale=0.8}{0}{Bomba dosificadora G1005}


\begin{table}[H]
    \centering
    \caption{Características de la bomba dosificadora G1005.}
    \label{table:bomba_dosificadora}
    \begin{tabular}{| L{6cm} | L{6cm} |}
        \hline
        \textbf{Modelo} & GALA G/L G1005 NPB 200UA 103000  \\
        \hline
        \textbf{Material} & Plexiglás \\
        \hline
        \textbf{Caudal} & 4,4 lt @ 10 bar \\
        \hline
        \textbf{Voltaje} & 100-230 V / 50-60 Hz \\
        \hline
        \textbf{Protección IP} & 65 \\
        \hline
        \textbf{Potencia} & 12W \\
        \hline
        \textbf{Fabricante} & PROMINENT \\
        \hline
    \end{tabular}
\end{table}

\insertimageboxed[\label{fig:bomba_dosificadora2}]{instrumentacion/bomba_dosificadora}{scale=0.8}{0}{Bomba dosificadora G/L 1601}


\begin{table}[H]
    \centering
    \caption{Características de la bomba dosificadora G/L 1601.}
    \label{table:bomba_dosificadora2}
    \begin{tabular}{| L{6cm} | L{6cm} |}
        \hline
        \textbf{Modelo} & G/L 1601 NPB 220UA  \\
        \hline
        \textbf{Tipo} & Diafragma de solenoide \\
        \hline
        \textbf{Medio} & Solución acuosa de Na2S2O5 \\
        \hline
        \textbf{Materiales} & Cabeza de dosificación: Acrílico, elemento de succión / presión: PVC, sellos: FPM-B, bolas: cerámica \\
        \hline
        \textbf{Caudal} & 1,1 lt/h \\
        \hline
        \textbf{Presión de descarga} & 16 bar \\
        \hline
        \textbf{Suministro eléctrico} & 100-230 V / 50-60 Hz \\
        \hline
        \textbf{Potencia} & 12 W \\
        \hline
        \textbf{Fabricante} & PROMINENT \\
        \hline
    \end{tabular}
\end{table}


\subsubsection{Válvulas de Retención} \label{sec:valvula_retencion}

Las válvulas de retención o check valves son elementos clave en cualquier sistema de tratamiento de agua o proceso industrial, ya que garantizan la unidireccionalidad del flujo en las tuberías. Su papel es esencial para mantener la seguridad y la eficiencia del sistema, ya que evitan el flujo inverso que podría causar daños en los equipos o interrumpir el proceso.\\

El papel de las válvulas de retención en nuestro sistema de ósmosis inversa es multifacético. Están ubicadas en varios puntos estratégicos a lo largo del proceso, incluyendo, pero no limitándose a, justo después de las bombas de alta presión, donde evitan que el fluido regrese a la bomba en caso de una parada o apagado. También se utilizan en la línea de dosificación de químicos, para asegurar un suministro constante y seguro de los reactivos necesarios para el proceso. Sin embargo, es importante destacar que pueden encontrarse en otros puntos del sistema donde sea necesario evitar el retroceso del flujo.\\

El principio de funcionamiento de las válvulas de retención es relativamente sencillo. Contienen un componente que se mueve libremente y permite el flujo en una dirección, pero bloquea el flujo si intenta moverse en la dirección contraria.\\

Para nuestro sistema, empleamos el modelo de válvula de retención Art. 048 VRTCV2 de RATTI. Este modelo está construido con un cuerpo de acero inoxidable AISI 316L, lo que garantiza su resistencia a la corrosión, y tiene una junta de PTFE.\\

Válvula de Retención 048 VRTCV2\\

\begin{table}[H]
    \centering
    \caption{Características del cuerpo.}
    \label{table:cuerpo}
    \begin{tabular}{| L{6cm} | L{6cm} |}
        \hline
        \textbf{Material del cuerpo} & AISI 316L \\
        \hline
        \textbf{Junta} & PTFE \\
        \hline
        \textbf{Diámetro} & 1½" \\
        \hline
        \textbf{Conexiones} & Abrazadera (clamp) \\
        \hline
        \textbf{Resorte} & Estándar \\
        \hline
        \textbf{Fabricante} & RATTI \\
        \hline
    \end{tabular}
\end{table}


\subsubsection{Válvulas Multiusos} \label{sec:valvula_multi}

Las válvulas multiusos son componentes esenciales en cualquier sistema de tratamiento de agua. Actúan como puntos de control, permitiendo o impidiendo el paso de fluidos a través de las tuberías. La capacidad de controlar el flujo de agua y otros líquidos es crucial para el funcionamiento seguro y eficiente de todo el sistema. Son llamadas "multiusos" porque se utilizan en una variedad de aplicaciones dentro del sistema, dependiendo de las necesidades del proceso en particular.\\

En nuestro sistema de ósmosis inversa, las válvulas multiusos se encuentran en varios puntos críticos. Una ubicación importante es en las tuberías de concentrado de la ósmosis, en la línea que va al drenaje o que retorna al tanque de almacenamiento de agua de pretratamiento. Además, se utilizan en la línea de bypass que se encuentra después de la bomba de lavado químico. Estas ubicaciones no son exhaustivas, y es posible encontrar estas válvulas en otros puntos del sistema donde se requiera controlar el flujo de fluido.\\

El principio de funcionamiento de las válvulas multiusos es simple pero efectivo. Cuando la válvula está abierta, permite el flujo de fluido; cuando está cerrada, detiene el flujo.\\

Utilizamos el modelo J4M1G00 de RATTI para nuestras válvulas multiusos. Esta válvula está fabricada con acero inoxidable AISI 316L para una resistencia óptima a la corrosión y durabilidad a largo plazo. Tiene un diámetro de 1" - ¾" y se conecta mediante una conexión Tri-Clamp.\\

\begin{table}[H]
    \centering
    \caption{Características del cuerpo.}
    \label{table:valvula_multiusos}
    \begin{tabular}{| L{6cm} | L{6cm} |}
        \hline
        \textbf{Material del cuerpo} & AISI 316 L \\
        \hline
        \textbf{Diámetro} & 1" - ¾" \\
        \hline
        \textbf{Conexiones} & Tri-Clamp \\
        \hline
        \textbf{Fabricante} & OMAL \\
        \hline
    \end{tabular}
\end{table}

\subsubsection{Válvulas de Control ON/OFF} \label{sec:valvula_OnOff}

En cualquier proceso industrial, las válvulas de control ON/OFF son elementos críticos para la gestión y regulación del flujo de fluidos. Su importancia radica en su habilidad para controlar de manera precisa y eficiente el flujo a través de las tuberías, permitiendo un total paso del fluido o su completa interrupción, según las demandas del sistema.\\

Un ejemplo específico de este tipo de válvulas es el modelo S386FPLY004Y05 de la reconocida empresa OMAL. Esta válvula cuenta con un cuerpo de hierro fundido revestido con EPOXY y EPDM, lo que la hace resistente y duradera. La característica más notable de esta válvula es su actuador neumático de retorno por resorte que, junto con su conjunto de accesorios, garantiza un funcionamiento fiable y una integración eficiente con el sistema de control del proceso.\\

La planta de ósmosis inversa que analizamos está equipada con numerosas válvulas de control ON/OFF, distribuidas estratégicamente en diferentes puntos del sistema. Son componentes indispensables que aseguran la correcta operación del proceso, y debido a su importancia, están presentes en gran cantidad en todas las áreas de la planta. Algunos lugares estratégicos donde podemos encontrar este tipo de válvulas puede ser la línea que precede a la bomba de alta presión y en la línea de concentrado de la primera etapa de ósmosis que retorna al tanque de pretratamiento.\\

Válvula de Control ON/OFF S386FPLY004Y05\\

\begin{table}[H]
    \centering
    \caption{Características del cuerpo.}
    \label{table:valvula_on_off}
    \begin{tabular}{| L{6cm} | L{6cm} |}
        \hline
        \textbf{Material del cuerpo} & Hierro fundido con revestimiento de EPOXY \\
        \hline
        \textbf{Revestimiento} & EPDM \\
        \hline
        \textbf{Vástago y disco} & Acero inoxidable AISI 316 \\
        \hline
        \textbf{Estilo del cuerpo} & "LUG" roscado para brida EN1092-1 \\
        \hline
        \textbf{Tamaño} & DN40 \\
        \hline
        \textbf{Actuador} & Neumático de retorno por resorte N.O. tipo SR30 y tornillos de regulación \\
        \hline
        \textbf{Accesorios} & Indicador de posición del eje KI02PP10, regulador de flujo de aire comprimido KAPR00101, filtro de aire de bronce 9490S001 \\
        \hline
        \textbf{Fabricante} & OMAL \\
        \hline
    \end{tabular}
\end{table}

\subsubsection{Válvulas de Retención de Presión de Inyección}

Las válvulas de retención de presión de inyección juegan un papel importante en el sistema de dosificación, especialmente en procesos industriales que requieren una precisión en la dosificación de ciertos productos químicos. Estas válvulas mantienen una presión constante de salida, evitando fluctuaciones y garantizando una dosificación precisa y estable.\\

En nuestra planta de ósmosis inversa, estas válvulas son vitales en la dosificación precisa de sustancias químicas como NaOH y Na2S2O5. Son componentes esenciales para asegurar la eficacia de las operaciones de dosificación y se encuentran estratégicamente ubicadas en las líneas de dosificación correspondientes. Sin embargo, su presencia no se limita a estas áreas, y se podrían encontrar en otras partes del sistema donde se necesite una dosificación precisa.\\

El modelo MFV-DK de PROMINENT, una empresa reconocida por la fabricación de componentes de alta calidad, es una de las válvulas utilizadas en nuestro sistema. Con un cuerpo de PVDF y un diafragma de PTFE, esta válvula es capaz de mantener una presión de alivio de hasta 16 bar, lo que asegura su capacidad para trabajar bajo condiciones exigentes.\\

Válvula de Retención de Presión de Inyección MFV-DK\\

\begin{table}[H]
    \centering
    \caption{Características del tipo MFV-DK, PVDF.}
    \label{table:valvula_retencion}
    \begin{tabular}{| L{6cm} | L{6cm} |}
        \hline
        \textbf{Tipo} & MFV-DK, PVDF \\
        \hline
        \textbf{Tamaño} & 1 \\
        \hline
        \textbf{Presión de alivio} & 16 bar \\
        \hline
        \textbf{Conector} & 6-12 mm \\
        \hline
        \textbf{Conector de bypass} & 6/4 mm \\
        \hline
        \textbf{Materiales} & Cuerpo de PVDF, diafragma de PTFE, sello de FPM \\
        \hline
        \textbf{Fabricante} & PROMINENT \\
        \hline
    \end{tabular}
\end{table}
