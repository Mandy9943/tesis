\chapter{Análisis de costos y beneficios}
En el presente capítulo, se abordará el análisis financiero del proyecto, desde su inicio hasta su conclusión.
Este análisis incluirá tres componentes clave: el coste de la investigación, el precio de los servicios
científicos y técnicos, y los beneficios de la investigación, así como el impacto económico de la
implementación de los resultados. Este análisis es fundamental para evaluar tanto la calidad como la
relevancia del proyecto para la empresa farmacéutica AICA, donde se llevará a cabo la implementación
del sistema de Electrodesionización (EDI).

El análisis del coste contempla los gastos derivados de la utilización de la tecnología necesaria,
los costes de adquisición de los equipos, componentes de instalación y materiales utilizados directamente,
así como los salarios del personal técnico involucrado en el proyecto. Por otro lado, el análisis de los
beneficios resulta esencial, ya que permite tener control sobre el gasto incurrido, proporcionando elementos
de juicio de carácter económico y otorgando una visión más amplia de las tareas relacionadas con la
implementación de esta tecnología, minimizando de esta manera el desperdicio de recursos en instrumentación
o materia prima innecesaria para la realización del proyecto.

El objetivo de este análisis económico es proveer una visión detallada y objetiva de los costes asociados con
la implementación de un sistema de EDI en la industria farmacéutica AICA, permitiendo de este modo una
planificación y gestión eficiente de los recursos disponibles.


\section{Costo del proyecto}

La estimación del costo se lleva a cabo al inicio del proyecto y se considera una aproximación del costo
real, que se determinará al finalizar el proyecto. Este costo puede calcularse a través de la suma del
costo directo e indirecto, tal como se muestra en la ecuación (\ref{eq:cost_total}). \\

\begin{equation}
    \label{eq:cost_total}
    CT = CD + CI
\end{equation}

Donde: \\
CT representa el costo total del proyecto. \\
CD hace referencia al costo directo. \\
CI denota el costo indirecto.

\subsection{Costo indirecto}

El costo indirecto abarca gastos tales como el consumo de electricidad, gastos administrativos, entre otros.
Este valor se estima multiplicando un coeficiente de gasto, en este caso 0.84, por el salario básico de la
investigación, tal como se muestra en la ecuación (\ref{eq:cost_indirect}). \\

\begin{equation}
    \label{eq:cost_indirect}
    CI = 0.84 * SB
\end{equation}

\subsection{Costo directo}

El costo directo engloba todos los gastos económicos necesarios para la realización del proyecto. Se
calcula como la suma del Salario Básico (SB), el Salario Complementario (SC), el Seguro Social (SS), los
Medios Directos (MD), las Dientas y los Pesajes (DP), y Otros Gastos (OG), como se puede observar con más
detalle en la ecuación (\ref{eq:cost_direct}). \\

\begin{equation}
    \label{eq:cost_direct}
    CD = SB + SC + SS + MD + DP + OG
\end{equation}

\subsection{Salario básico}

SB (salario básico): Consiste en el salario que se paga por el tiempo trabajado, es decir, no se incluye seguridad social ni vacaciones. Incluye los salarios básicos de todos los participantes del proyecto.


\begin{equation}
    \label{eq:sal_basico}
    SB = \sum_{i = 0}^{n} (Ai * Bi)
\end{equation}

donde:

𝐴𝑖: días dedicados a la investigación del proyecto.

B𝑖: salario diario del participante 𝑖 (salario mensual / 24)

𝑛: número total de participantes del proyecto.

El salario por hora de los participantes está dado por la relación existente del salario
básico de cada uno entre la cantidad de días dedicados a actividades laborales,
multiplicado por la cantidad de horas. Teniendo en cuenta que en un mes existen 24
días laborables como promedio y que al día la jornada de trabajo es de 8 horas se
puede plantear que:\\

B1 = 4954 / (24*8) = 25.80 CUP/Hrs

B2 = 9730.5 / (24*8) = 50.68 CUP/Hrs

B3 = 400 / (24*8) = 2.08 CUP/Hrs\\

En la Tabla \ref{table:participantes_proyecto} se muestra una relación de las personas que participan en la realización de este proyecto. \\


\begin{table}[H]
    \caption{Participantes en el proyecto}
    \label{table:participantes_proyecto}

    \begin{tabular}{|c|c|c|c|c|}
        \hline
        \textbf{Nombres y apellidos}  & \textbf{SB (CUP)} & \textbf{𝐴𝑖 (Hrs)} & \textbf{B𝑖 (CUP/Hrs)} & \textbf{Ai*Bi} \\
        \hline
        Ing. Amanda Martí Coll        & 4900              & 120               & 25.80                 & 3096           \\
        Ing. Rosaine Ayala            & 6500              & 120               & 25.80                 & 3096           \\
        Armando Cesar Martin Calderón & 4900              & 600               & 25.80                 & 3096           \\

        \hline
    \end{tabular}
\end{table}


Se emplearon 5 meses de trabajo comprendidos entre enero y mayo. Considerando que los tutores le dedicaron a la actividad, cada día laborable, 1 hora de trabajo como promedio, entonces se puede afirmar que fueron asignadas a la investigación 120 horas por cada uno de ellos.
El estudiante le dedicó cada día laborable como 5 horas como promedio, a la investigación, es decir, un total de 600 horas.
Según la ecuación (\ref{eq:sal_basico}):\\

\begin{equation}
    \label{eq:salary_basico_total}
    SB = 120 * 25.80 + 120 * 50.68 + 600 * 2.08 = 10425.6 CUP
\end{equation}


\subsection{Salario complementario}

El salario complementario (SC) es el 9.09\% del salario básico, destinado para el pago de las
vacaciones. Se puede calcular con la siguiente ecuación:

\begin{equation}
    \label{eq:salary_complementary}
    SC = SB * 0.0909
\end{equation}
\begin{equation}
    \label{eq:salary_complementary_total}
    SC=0.0909*10425.60=947.69 CUP
\end{equation}

\subsection{Seguro Social}

El seguro social (SS) equivale al 5\% del salario básico más el salario complementario, y se
calcula de la siguiente forma:

\begin{equation}
    \label{eq:social_security}
    SS = 0.05 * (SB + SC)
\end{equation}
\begin{equation}
    \label{eq:social_security_total}
    SS=0.05*(10425.60+947.69)=1137.33 CUP
\end{equation}

\subsection{Medios Directos}

Los medios directos (MD) incluyen los costos de todos los equipos, componentes de instalación y
materiales utilizados directamente en la investigación.


Para llevar a cabo el proyecto será necesario hacer algunos gastos imprescindibles. En la Tabla 4.1
se muestran los precios de los elementos que deben adquirirse:

\begin{table}[h]
    \caption{Listado de precios de los dispositivos e instrumentos necesarios para la elaboración del proyecto}
    \begin{tabular}{|c|c|c|c|}
        \hline
        Dispositivo/instrumento/otros      & Cantidad & Precio por unidad & Precio total \\
        \hline
        Electrodesionizador                & 1        & 10379.4€          & 10379.4€     \\
        Sensor transmisor de temperatura   & 1        & 175.10€           & 175.10€      \\
        Sensor transmisor de flujo         & 1        & 1732.00€          & 1732.00€     \\
        Sensor transmisor de conductividad & 1        & 661.25€           & 661.25€      \\
        Sensor transmisor de presión       & 2        & 691.85€           & 1393.7€      \\
        \hline
    \end{tabular}
\end{table}

El total de gastos en materiales directos es:

\begin{equation}
    MD = 15753.15€ + \$217694.5 MN
\end{equation}

\subsection{Dietas y Pasajes}

Las dietas y pasajes (DP) representan los gastos ocasionados por dietas y pasajes.

\subsection{Otros Gastos}

Los otros gastos (OG) incluyen el costo de utilización de equipamiento. Se considera el gasto por
concepto de tiempo de máquina, que tiene un valor de \$10.00 la hora.

Se incluye el gasto por consumo de energía eléctrica, durante las horas de tiempo de máquina empleadas
en la elaboración del proyecto. Para un total de 450 horas resulta ser:

\begin{equation}
    OG = 450 \text{ horas } * \$10MN = \$4500.00MN
\end{equation}





\subsection{Cálculo del Costo Directo}

\begin{equation}
    CD = SB + SC + SS + MD + DP + OG = \$228564.42 MN + 15753.15€
\end{equation}

\subsection{Costos indirectos}

El término Costos Indirectos (CI) se refiere a los gastos de electricidad consumida, gastos de administración,
instalaciones, etc., en el proceso de investigación. Este se estima aplicando un coeficiente de gastos al
salario básico de la investigación. El coeficiente de gastos para un Departamento Docente es 0.84 y para una
UCT (Unidad de Ciencia y Técnica) es 1.4063.

\begin{equation}
    CI = 0.84 * SB = \$6468 MN
\end{equation}

\subsection{Costo Total}

El costo total del proyecto resulta la suma de los costos directos e indirectos, por tanto:

\begin{equation}
    CT = CD + CI = \$235032.42MN + 15753.15€
\end{equation}

\subsection{Precio}

El precio se determina mediante la expresión:

\begin{equation}
    P = CT + 0.1 * CT
\end{equation}

Donde: $CT$ representa el costo total de todos los elementos de la red y control de conductividad, $0.1*CT$
representa los salarios pagados a especialistas, técnicos, y el resto del personal involucrado en el diseño,
montaje y puesta en marcha del sistema, el costo de impuestos aduanales, de combustible para el transporte del
personal, y para el cableado.

El costo total es de $CT = \$235032.42MN + 15753.15€$

\begin{align*}
    P & = ( \$235032.42MN + 15753.15€) + 0.1*( \$235032.42MN + 15753.15€) \\
    P & = \$258535.66MN + 17328.46€
\end{align*}

Luego el costo total del proyecto de tesis es:

\begin{align*}
    \text{Costo total del proyecto de tesis} & = CT + P                                                    \\
    \text{Costo total del proyecto de tesis} & = (\$235032.42MN + 15753.15€) + (\$258535.66MN + 17328.46€) \\
    \text{Costo total del proyecto de tesis} & = \$493568.08MN + 33081.61€
\end{align*}


\section{Viabilidad del proyecto}

El proyecto es viable teniendo en cuenta los siguientes aspectos:

\begin{itemize}
    \item \textbf{Medio ambiente:} Este proyecto es respetuoso con el medio ambiente. No hay elementos o procesos
          que generen contaminación o residuos dañinos. Además, se ha hecho un esfuerzo consciente por minimizar el uso
          de recursos y maximizar la eficiencia en todas las etapas del proyecto. Se espera que este proyecto tenga un
          impacto ambiental positivo o neutral en su ejecución a corto y largo plazo.

    \item \textbf{Jurídico:} El proyecto se desarrolla completamente dentro del marco legal existente y cumple con
          todas las normas y leyes pertinentes. En particular, se han tomado medidas para garantizar que todas las
          actividades estén en línea con las regulaciones y directrices de la industria farmacéutica. El proyecto también
          mantiene un compromiso de adherirse a cualquier cambio o actualización futura en la legislación relevante.

    \item \textbf{Económico y financiero:} Desde una perspectiva económica, el proyecto es viable. El centro tiene
          los recursos financieros necesarios para financiar completamente el proyecto. Además, se espera que el proyecto
          sea rentable y genere un retorno de la inversión a largo plazo. El apoyo intelectual de la institución docente
          también contribuye a la viabilidad económica del proyecto, ya que proporciona acceso a expertos y recursos
          académicos.

    \item \textbf{Técnico:} La entidad dispone de los recursos técnicos necesarios para llevar a cabo el proyecto.
          Esto incluye la disponibilidad de equipos, tecnología y personal cualificado. El personal tiene las habilidades
          y la experiencia necesarias para implementar y gestionar el proyecto eficazmente. Además, se disponen de las
          instalaciones y el equipamiento necesarios para desarrollar todas las etapas del proyecto sin problemas.
\end{itemize}








% \section{Estimación de costos de adquisición e instalación del EDI e instrumentación adicional}
% En términos de costos de adquisición, un sistema de EDI de alta capacidad apto para una planta farmacéutica como AICA puede oscilar en un rango aproximado de 80,000 a 100,000 dólares. Este costo puede variar dependiendo de las especificaciones exactas del sistema, la marca y el proveedor.

% La instalación del sistema EDI puede requerir ajustes en la infraestructura existente de la planta. Este costo podría incluir la preparación del sitio, la instalación de la unidad de EDI, la integración con los sistemas existentes y las pruebas iniciales. Dicha instalación puede oscilar entre los 20,000 a 30,000 dólares, dependiendo de la complejidad de la instalación.

% La instrumentación adicional necesaria para apoyar el sistema EDI, que puede incluir bombas de alta presión, sensores de calidad del agua y sistemas de control avanzados, puede añadir entre 15,000 y 20,000 dólares adicionales al costo inicial. Estos costos pueden variar dependiendo de las necesidades específicas de la planta de AICA y de las condiciones particulares de la instalación.

% En total, la estimación inicial para la adquisición e instalación del sistema de EDI y la instrumentación adicional en la planta farmacéutica AICA oscilaría entre 115,000 y 150,000 dólares. Esta cifra representa una inversión inicial considerable, pero debe ser considerada en el contexto de los ahorros y beneficios potenciales a largo plazo que la tecnología EDI puede proporcionar a la planta.


% \section{Estimación de costos operativos y de mantenimiento}
% \section{Evaluación de los beneficios}
% \section{Análisis de retorno de inversión y viabilidad económica}