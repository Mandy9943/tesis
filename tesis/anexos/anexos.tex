
\begin{appendixs}
    \fontsize{10}{12}\selectfont
    \section{Requisitos de agua de alimentación para los módulos IP-LX}
    \renewcommand{\arraystretch}{1} % Ajustar padding en eje y
    \begin{longtable}{|>{\raggedright\arraybackslash}m{8cm} |>{\raggedright\arraybackslash}m{8cm}|}
        \toprule
        \textbf{Parámetro}                                                  & \textbf{Valor}              \\
        \midrule
        Fuente de agua de alimentación                                      & permeado de ósmosis inversa \\
        \hline
        Conductividad del agua de alimentación equivalente, incluyendo CO2* & < 40 µS/cm                  \\
        \hline
        Silica (SiO2)                                                       & < 1 ppm                     \\
        \hline
        Hierro, Manganeso, Sulfuro                                          & < 0.01 ppm                  \\
        \hline
        Cloro total                                                         & < 0.02 ppm como Cl2         \\
        \hline
        Dureza total                                                        & < 1.0 ppm como CaCO3        \\
        \hline
        Materia orgánica disuelta (TOC)                                     & < 0.5 ppm                   \\
        \hline
        Rango de pH de operación                                            & 4 – 11                      \\
        \hline
        Temperatura de operación                                            & 41 - 113 °F (5 – 45 °C)     \\
        \hline
        Presión de entrada                                                  & <100 psi (7 bar)            \\
        \bottomrule
    \end{longtable}
    \section{Diagrama P\&ID del sistema de ósmosis inversa}
    \insertimageboxed[\label{fig:P\&ID_OSMOSIS}]{P&ID_OSMOSIS}{scale=0.2}{0}{}
    \section{Conexionado de los módulos IP-LX}
    \insertimageboxed[\label{fig:edi_conexionado}]{edi_conexionado}{scale=0.7}{0}{}

\end{appendixs}