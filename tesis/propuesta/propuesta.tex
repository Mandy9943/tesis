\chapter{Propuesta de Implementación de EDI Después de la Ósmosis Inversa Doble}
\label{cap:propuesta_implementacion}

La necesidad de alcanzar niveles más altos de pureza del agua ha impulsado
la evolución y mejora continua de las tecnologías de tratamiento de agua.
Una de estas tecnologías es la Electrodesionización (EDI), que combina
los principios de electrodiálisis y resinas de intercambio iónico para
producir agua de alta pureza. En la industria farmacéutica, donde se
requiere un agua con una calidad excepcional, la implementación de la
tecnología EDI se convierte en un paso esencial después de la ósmosis inversa doble.

Este capítulo presentará la propuesta de implementación de un sistema
EDI en la empresa AICA. En primer lugar, se discutirá el sistema de
control que regula el funcionamiento del EDI y cómo este se coordina
con el sistema de control de la planta en general.
Luego, se presentará la propuesta de integración de un sistema SCADA,
mostrando su interfaz de usuario y explicando cómo este sistema
ayudará a supervisar y controlar el proceso de tratamiento del agua.
Finalmente, se describirá el proceso de implementación y puesta en marcha del
EDI, abarcando desde la instalación física del dispositivo hasta las pruebas
iniciales para verificar su funcionamiento correcto.

% ------------- Sección ----------------
\section{Sistema de control}
\label{sec:sistema_control}

En cualquier sistema industrial, el control es un componente crucial. La eficiencia, seguridad y eficacia de un sistema dependen en gran medida de su capacidad para responder a las variables del entorno y ajustar su funcionamiento en consecuencia. El sistema de control es el cerebro de la operación, coordinando y supervisando todos los aspectos del proceso. En el caso de un sistema de Electrodesionización, el control es aún más crítico debido a la complejidad del proceso y la alta calidad del producto final requerido.

La programación del Controlador Lógico Programable (PLC, por sus siglas en inglés) es un elemento esencial de este sistema de control. El PLC se encarga de interpretar las señales de entrada de los distintos sensores y actuadores y ejecutar la lógica de control para ajustar las operaciones del sistema de acuerdo a las necesidades. En esta sección, se presentará la programación del PLC en forma de diagrama de flujo para la secuencia principal de funcionamiento del sistema de Electrodesionización.

\subsection{Puesta en marcha}

Un diagrama de flujo ofrece una visión clara y concisa de la lógica de control,
facilitando la comprensión y el seguimiento de la secuencia de operaciones.
Esto es especialmente útil para el personal de mantenimiento y operación,
así como para cualquier persona que necesite entender el funcionamiento del sistema.

La secuencia operacional del sistema de Electrodesionización se inicia con la activación de la planta de
ósmosis inversa a través de una interfaz de usuario. Este evento de inicio es seguido de un
período de espera hasta que el sensor de nivel determine que el tanque de agua pretratada (TK50A)
ha alcanzado su nivel operativo óptimo (\ref{fig:flujo_1}).

Durante este tiempo inicial, las etapas de ósmosis inversa (RO1 y RO2) se encuentran en un estado
latente. RO1 aguarda la señal de nivel correcta del tanque TK50A, mientras que RO2 permanece en un
estado de inactividad.

Una vez que el sensor de nivel indica que TK50A ha alcanzado su nivel adecuado, se implementa
un período de confirmación del nivel, que sirve para mitigar el impacto de posibles fluctuaciones
en el nivel del tanque. Esta duración de tiempo se ha establecido típicamente en 60 segundos.

A continuación, se inicia el flujo de agua hacia la primera etapa de ósmosis inversa (RO1). Esta
etapa implica una descarga inicial de agua, necesaria debido a las posibles condiciones iniciales
subóptimas del agua que entra en el sistema. Este período de descarga varía dependiendo de la condición
de la membrana de ósmosis, pero suele ser de aproximadamente 120 segundos.

Después de este período de descarga, el agua de RO1 es examinada para determinar si cumple con los
parámetros de conductividad requeridos. Si la conductividad no cumple con las especificaciones, RO1
entra en un estado de descarga por alta conductividad y se mantiene en este estado hasta que las
mediciones de conductividad y un período de confirmación de 60 segundos indiquen que se cumplen
los parámetros de conductividad.

\insertimageboxed[\label{fig:flujo_1}]{/flujo_OI1}{scale=0.45}{0}{Diagrama de flujo para la RO1 del proceso de producción de PW.}


En cuanto las condiciones de conductividad sean satisfactorias, RO1 cambia a un estado de producción
y, simultáneamente, se inicia la segunda etapa de ósmosis inversa (RO2). Esta segunda etapa, al
igual que RO1, comienza con una descarga inicial (ver Figura \ref{fig:flujo_2}). No obstante, a diferencia de RO1, el agua
descargada por RO2 se devuelve al tanque de agua pretratada (TK50A), combinándose con el agua
de permeado y concentrado. Este período de descarga también está sujeto a las condiciones de
las membranas de ósmosis y dura aproximadamente 120 segundos.

Posteriormente, se evalúan los parámetros de conductividad y temperatura en el permeado de RO2.
Si alguno de estos parámetros no cumple con las especificaciones, RO2 entra en un estado de descarga
por parámetros deficientes y se mantiene en este estado hasta que los parámetros medidos y un período
de confirmación de 60 segundos indiquen condiciones aceptables.

\insertimageboxed[\label{fig:flujo_2}]{/flujo_OI2}{scale=0.6}{0}{Diagrama de flujo para la RO2 del proceso de producción de PW.}


Una vez que se alcanzan estos criterios, RO2 cambia a un estado de producción. Con ambas etapas de
ósmosis inversa (RO1 y RO2) en producción, el módulo de Electrodesionización (EDI) puede comenzar su
operación con una descarga inicial hacia el tanque de pretratamiento. Esta descarga inicial tiene
una duración de aproximadamente 60 segundos.

Posteriormente, se comprueban los parámetros como la conductividad y la presión en el producto del EDI.
Si alguno de estos parámetros no cumple con las especificaciones, el EDI entra en un estado de descarga
por parámetros deficientes y se mantiene en este estado hasta que los parámetros medidos y un período
de confirmación de 60 segundos indiquen condiciones aceptables.

Finalmente, una vez que los parámetros de conductividad y presión son óptimos y han pasado 60 segundos
de confirmación, el EDI cambia a un estado de producción, indicando la finalización exitosa de la
secuencia operacional del sistema de Electrodesionización.

Con el sistema completo en estado de producción (ver Figura \ref{fig:flujo_3}), el estado
posterior depende del nivel del tanque final. Si el tanque final está
completamente lleno, la ósmosis comienza una circulación conjunta, que dura un
tiempo de alrededor de 10 minutos. Superado este tiempo, se realiza una pausa de tiempo de 60 minutos antes de
comenzar otro ciclo. La ósmosis continúa recirculando y no vuelve a producir
hasta que el tanque de almacenamiento de agua purificada, que distribuye a los
puntos de uso, señale un nivel del 75\% de capacidad.

Cada vez que concluye un ciclo de producción y debe comenzar otro, se comprueba
el estado del sensor de nivel bajo del tanque de agua pretratada. Si este
sensor permanece activo (ver Figura \ref{fig:flujo_3}), se lleva a cabo directamente la
descarga inicial de la OI1. De lo contrario, será necesario esperar hasta que
el tanque TK 50A alcance el nivel mínimo necesario para poner el sistema a purificar.


\insertimageboxed[\label{fig:flujo_3}]{/flujo_OI3}{scale=0.4}{0}{Diagrama de flujo para el EDI del proceso de producción de PW.}


% ------------- Sección ----------------
\section{Propuesta de SCADA}
\label{sec:scada_proposal}

Los Sistemas de Control y Adquisición de Datos (SCADA) se han convertido en una herramienta fundamental en el ámbito de la automatización industrial, permitiendo la supervisión y control de procesos a gran escala de una manera eficiente y centralizada. Este sistema ofrece ventajas significativas, como la optimización de operaciones, el aumento de la eficiencia, la mejora de la calidad del producto y la prevención de condiciones peligrosas.

En el contexto del sistema de purificación de agua de la planta, la implementación de un SCADA proporcionaría una visibilidad en tiempo real del proceso y facilitaría la gestión de alarmas y el control de los componentes clave del sistema, como las membranas de la ósmosis inversa y el dispositivo EDI. Además, un sistema SCADA permitiría el registro de datos, esencial para el análisis de tendencias y la toma de decisiones basada en datos.

El SCADA propuesto para la optimización de la purificación de agua en la industria farmacéutica AICA se ha desarrollado en el entorno de TIA Portal. Este sistema está diseñado para proporcionar un monitoreo en tiempo real del proceso de ósmosis inversa, además de ofrecer una interfaz de usuario intuitiva e interactiva para los operadores.

\insertimageboxed[\label{fig:vistaGeneral}]{vistaGeneral}{scale=0.3}{0}{Vista general del sistema SCADA.}

El SCADA se estructura en varias secciones dedicadas a diferentes aspectos del proceso de purificación de agua. A continuación, se describen detalladamente cada una de estas secciones.

\subsection{Monitoreo del proceso}

El monitoreo en tiempo real del proceso de ósmosis inversa es una función esencial del sistema SCADA propuesto. Esta característica proporciona una comprensión inmediata y detallada del estado del proceso, lo que permite a los operadores tomar decisiones informadas y oportunistas.

\insertimageboxed[\label{fig:monitoreoProceso1}]{monitoreoProceso1}{scale=0.3}{0}{Interfaz de la sección de monitoreo del proceso.}

Como se puede observar en la Figura \ref{fig:monitoreoProceso1}, la interfaz de usuario presenta una representación gráfica del proceso de ósmosis inversa, con indicadores en tiempo real de las principales variables del sistema. Cada componente del sistema es monitoreado y su estado se visualiza claramente, permitiendo a los operadores identificar rápidamente cualquier desviación del rendimiento óptimo.

La sección de monitoreo del proceso es, por lo tanto, una herramienta valiosa para asegurar la eficiencia y efectividad de la planta de purificación de agua.

\subsection{Administración de alarmas}

La administración de alarmas es una funcionalidad crucial en cualquier sistema SCADA, y en este caso, se ha prestado especial atención a su diseño y ejecución. Un sistema de alarmas eficiente permite a los operadores reaccionar rápidamente ante cualquier eventualidad o desviación en el sistema de purificación.

\insertimageboxed[\label{fig:adminAlarmas1}]{adminAlarmas1}{scale=0.3}{0}{Ventana emergente de administración de alarmas.}

En la Figura \ref{fig:adminAlarmas1}, se puede observar la interfaz de la ventana emergente de alarmas. Cuando se activa una alarma, esta ventana aparece automáticamente, presentando las alarmas activas y no acusadas. Cada alarma se muestra con información detallada, incluyendo la hora de activación, la descripción del problema y el componente del sistema afectado.


Los operadores pueden acusar las alarmas directamente desde esta ventana emergente. Al acusar una alarma, se reconoce formalmente la situación de alerta y se inicia la resolución del problema.

Además de la ventana emergente, el sistema SCADA propuesto cuenta con un visor de alarmas. Este visor mantiene un registro de las alarmas recientes hasta que su capacidad de buffer es alcanzada.

\insertimageboxed[\label{fig:adminAlarmas3}]{adminAlarmas2}{scale=0.3}{0}{Visor de alarmas con capacidad de buffer.}

La Figura \ref{fig:adminAlarmas3} muestra el visor de alarmas. Todas las alarmas se almacenan en un fichero, permitiendo un seguimiento detallado de los eventos y una revisión histórica de las situaciones de alarma. Esto puede ser útil para el análisis de las tendencias de las alarmas y la identificación de patrones de fallo recurrentes.

\subsection{Análisis de gráficos históricos}

Una de las fortalezas del sistema SCADA propuesto radica en su capacidad para el análisis de gráficos históricos. Esta característica proporciona una visión valiosa de las tendencias y patrones de las variables más importantes del sistema de purificación de agua.

\insertimageboxed[\label{fig:graficosHistoricos1}]{graficosHistoricos1}{scale=0.3}{0}{Interfaz de análisis de gráficos históricos.}

La Figura \ref{fig:graficosHistoricos1} muestra la interfaz del sistema para el análisis de gráficos históricos. Aquí, los operadores pueden seleccionar y visualizar gráficos de diferentes variables clave, permitiéndoles examinar su comportamiento y evolución a lo largo del tiempo. Esto puede ser útil para identificar cambios sutiles en el rendimiento del sistema, detectar tendencias emergentes y realizar diagnósticos predictivos.

Además, los datos históricos de estas variables se almacenan en un fichero, lo que ofrece la posibilidad de realizar análisis retrospectivos de larga duración. Esta capacidad de examinar los datos históricos puede ser invaluable para entender los cambios a largo plazo en el rendimiento del sistema y para tomar decisiones informadas sobre futuras optimizaciones y mejoras.

\subsection{Administración de usuarios}

La administración de usuarios es una característica esencial de cualquier sistema SCADA, proporcionando tanto un medio para controlar el acceso al sistema como una manera de personalizar la experiencia del usuario de acuerdo a su rol y responsabilidades.

\insertimageboxed[\label{fig:adminUsuarios1}]{inicioSesion}{scale=0.3}{0}{Interfaz de inicio de sesión del sistema SCADA.}

Como se muestra en la Figura \ref{fig:adminUsuarios1}, los usuarios deben autenticarse antes de poder interactuar con el sistema. Esto garantiza que solo las personas autorizadas puedan acceder y realizar operaciones en el sistema SCADA. Existen dos grupos de usuarios: los operadores y los administradores, cada uno con diferentes niveles de acceso y control.

\insertimageboxed[\label{fig:adminUsuarios2}]{adminUsuarios}{scale=0.3}{0}{Interfaz de la sección de administración de usuarios.}

La Figura \ref{fig:adminUsuarios2} muestra la interfaz de la sección de administración de usuarios. Aquí, los administradores pueden gestionar los perfiles de usuario, asignar roles y establecer privilegios de acceso. Los administradores tienen acceso total a todas las partes del sistema, mientras que los operadores tienen acceso restringido, limitándose a ciertas funcionalidades de acuerdo a sus responsabilidades.

Este diseño no solo garantiza la seguridad del sistema, sino que también permite una gestión eficiente del personal y una distribución adecuada de las responsabilidades dentro de la planta de purificación de agua.


% \subsection{Administración de alarmas}

% El sistema SCADA cuenta con un eficiente sistema de gestión de alarmas. Las alarmas se presentan a los operadores en una ventana emergente que muestra todas las alarmas activas y no acusadas.

% \insertimageboxed[\label{fig:adminAlarmas1}]{adminAlarmas1}{scale=0.3}{0}{Ventana emergente de administración de alarmas.}
% \insertimageboxed[\label{fig:adminAlarmas2}]{adminAlarmas2}{scale=0.3}{0}{Detalle de una alarma en la ventana emergente.}

% Además, se ha implementado un visor de alarmas que mantiene un registro temporal de las alarmas hasta que su capacidad es alcanzada. Todas las alarmas se almacenan en un fichero para permitir un seguimiento detallado.

% \subsection{Análisis de gráficos históricos}

% La sección de gráficos históricos permite a los operadores y administradores visualizar tendencias y patrones en las variables más importantes del sistema de purificación.

% \insertimageboxed[\label{fig:graficosHistoricos1}]{graficosHistoricos1}{scale=0.3}{0}{Interfaz de análisis de gráficos históricos.}

% \subsection{Administración de usuarios}

% La administración de usuarios en el sistema SCADA permite la configuración de diferentes niveles de acceso, con dos grupos de usuarios definidos: operadores y administradores.

% \insertimageboxed[\label{fig:inicioSesion}]{inicioSesion}{scale=0.3}{0}{Pantalla de inicio de sesión.}
% \insertimageboxed[\label{fig:adminUsuarios}]{adminUsuarios}{scale=0.3}{0}{Interfaz de la sección de administración de usuarios.}

% El SCADA ha sido diseñado con un enfoque flexible y adaptable para permitir la incorporación de futuras secciones o funcionalidades según las necesidades cambiantes de la planta de purificación.






% ------------- Sección ----------------
\section{Instalación del EDI }
\label{sec:implementation_start}

La implementación de un nuevo componente de un sistema de tratamiento de agua,
como un dispositivo de EDI, es un proceso complejo que requiere consideraciones
cuidadosas de diseño, logística, instalación y pruebas. Esta tarea se vuelve
aún más crítica cuando este nuevo componente debe integrarse a un sistema
existente sin interrumpir significativamente su funcionamiento normal.

En esta sección, describiremos el proceso de implementación y puesta en marcha del dispositivo EDI propuesto después de la doble ósmosis inversa. Este proceso incluirá los detalles de la instalación física del dispositivo EDI, desde su recepción y montaje hasta su conexión con el sistema existente. También cubriremos las pruebas iniciales que deben realizarse para garantizar que el dispositivo EDI esté operando correctamente y para confirmar que se pueden alcanzar los parámetros deseados de pureza del agua.

Además, discutiremos la importancia de la formación del personal que manejará el dispositivo EDI para asegurar una operación segura y eficiente a largo plazo. Este entrenamiento debe incluir el uso del nuevo sistema SCADA propuesto, así como los procedimientos de mantenimiento y resolución de problemas específicos del dispositivo EDI. \\

Finalmente, se abordarán las medidas de seguimiento y evaluación que deben implementarse para garantizar la eficacia y eficiencia del sistema a lo largo del tiempo.

La implementación del sistema de Electrodesionización (EDI) luego de la ósmosis inversa doble requiere una serie de pasos clave para garantizar su correcta instalación y funcionamiento. A continuación, se proporciona un desglose detallado de este proceso:

\begin{enumerate}
    \item \textbf{Evaluación del sitio de instalación:} Antes de la instalación del EDI, es esencial realizar una evaluación exhaustiva del sitio para determinar la adecuación del área para alojar la unidad. Factores como la disponibilidad de espacio, la accesibilidad para el mantenimiento, la disponibilidad de suministro de agua y energía, y las condiciones ambientales deben ser considerados.

    \item \textbf{Preparación del sitio de instalación:} Una vez evaluado el sitio, se prepara para la instalación. Esto puede implicar trabajos de construcción menores para proporcionar una base estable y segura para la unidad EDI, y la configuración de las conexiones necesarias para el agua, la electricidad y el drenaje.

    \item \textbf{Instalación de la unidad EDI:} La unidad de EDI se instala en el sitio preparado. Esto debe ser realizado por técnicos cualificados para garantizar que la unidad se instale correctamente y de manera segura. Los componentes de la unidad deben ser cuidadosamente manejados para evitar daños.

    \item \textbf{Conexión de la unidad EDI:} Una vez instalada la unidad, se conecta a las fuentes de agua y electricidad, y al sistema de drenaje. Los componentes de la unidad, como las membranas, las bombas y los sensores, también se conectan y se aseguran.

    \item \textbf{Pruebas de la unidad EDI:} Antes de la puesta en marcha completa, la unidad EDI se somete a una serie de pruebas para verificar su correcto funcionamiento. Esto incluye pruebas de la funcionalidad del PLC y del sistema SCADA, así como pruebas de la capacidad de la unidad para purificar el agua a las especificaciones requeridas.

    \item \textbf{Puesta en marcha de la unidad EDI:} Una vez que se han realizado y superado todas las pruebas, se pone en marcha la unidad EDI. Durante la puesta en marcha inicial, se debe monitorear de cerca la operación de la unidad para identificar y corregir cualquier problema que pueda surgir.
\end{enumerate}

En cada una de estas etapas, se deben seguir estrictamente las normas y procedimientos de seguridad para proteger tanto al personal como al equipo. También es fundamental mantener una documentación detallada de todo el proceso de instalación y puesta en marcha para facilitar futuras referencias y mantenimiento.
