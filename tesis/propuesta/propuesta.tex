\chapter{Propuesta de Implementación de EDI}
\label{cap:propuesta_implementacion}
La necesidad de alcanzar niveles más altos de pureza del agua ha impulsado
la evolución y mejora continua de las tecnologías de tratamiento de agua.
Una de estas tecnologías es la Electrodesionización (EDI), que combina
los principios de electrodiálisis y resinas de intercambio iónico para
producir agua de alta pureza. En la industria farmacéutica, donde se
requiere un agua con una calidad excepcional, la implementación de la
tecnología EDI se convierte en un paso esencial después de la ósmosis inversa doble.

Este capítulo presentará la propuesta de implementación de un sistema
EDI en la empresa AICA UEB. En primer lugar, se discutirá el sistema de
control que regula el funcionamiento del EDI y cómo este se coordina
con el sistema de control de la planta en general.
Luego, se presentará la propuesta de integración de un sistema SCADA,
mostrando su interfaz de usuario y explicando cómo este sistema
ayudará a supervisar y controlar el proceso de tratamiento del agua.
Finalmente, se describirá el proceso de implementación y puesta en marcha del
EDI, abarcando desde la instalación física del dispositivo hasta las pruebas
iniciales para verificar su funcionamiento correcto.

\section{Propuesta de instrumentación}

La ingeniería de procesos, y especialmente el tratamiento de agua mediante ósmosis inversa, requiere una cuidadosa selección de equipos y dispositivos de control, también conocidos como instrumentación. En esta sección, daremos un paso hacia adelante desde el análisis de la instrumentación actual, para abordar nuestra propuesta de mejoramiento: la implementación de un electrodesionizador (EDI) y la instrumentación requerida para su correcta operación.

La instrumentación adecuada es crucial para el buen funcionamiento de cualquier proceso industrial, ya que nos permite monitorizar y controlar de forma precisa las variables críticas de operación. En el caso de la ósmosis inversa y, más concretamente, del EDI, esta importancia se acentúa, dado que el rendimiento y la eficiencia del sistema dependen en gran medida de la capacidad de regular las condiciones de trabajo.

\subsection{El Electrodesionizador}

El electrodesionizador seleccionado para la implementación en la planta de tratamiento
de agua es el modelo LMX30-X-3 fabricado por Ionpure como el que se muestra en la
figura \ref{fig:edi_model}. Esta elección fue basada en sus principales características
reflejadas en la tabla \ref{table:edi_specs} que
demuestran su idoneidad para satisfacer las necesidades de la planta existente y
garantizar la producción de agua purificada de alta calidad.

\insertimageboxed[\label{fig:edi_model}]{instrumentacion/edi}{scale=0.8}{0}{Modelo LMX30-X-3 de Ionpure.}

Ionpure es una marca ampliamente reconocida en la industria de la purificación del agua,
y sus productos son conocidos por su excelencia y rendimiento confiable.
El modelo LMX30-X-3 desempeña un papel crucial en el proceso de purificación
del agua al utilizar la electrodesionización para eliminar eficientemente iones y
moléculas no deseadas.

Una de las consideraciones que se tuvo en cuenta para la selección de este modelo
se basa en su capacidad para cumplir con los
requisitos específicos de la planta de tratamiento de agua. El LMX30-X-3 ha
sido diseñado para trabajar con agua pretratada en ósmosis inversa, lo que lo
hace compatible con el sistema de tratamiento existente en la planta. Además,
ofrece un flujo de producto máximo de 3300 l/h mientras que la
planta de ósmosis inversa ofrece un 3000 l/h, lo que garantiza un suministro
adecuado de agua purificada.
Sus conexiones de 1" \ para los flujos de alimentación y producto, así como las
conexiones de 1/2" \ para los flujos de rechazo y concentrado, facilitan la
integración del sistema en la infraestructura existente.

Un aspecto relevante para tener en cuenta es que el LMX30-X-3 está diseñado
para operar en temperaturas ambiente de hasta 45°C. Esto es especialmente
importante en nuestro contexto, ya que en Cuba, durante el verano, las temperaturas
pueden ser elevadas. La capacidad del electrodesionizador para funcionar eficientemente
incluso en condiciones ambientales cálidas garantiza su rendimiento óptimo durante
todo el año.


\begin{mytable}{6cm}{Características del Electrodesionizador LMX30-X-3 de Ionpure.}{table:edi_specs}
      \hline
      \textbf{Modelo}                                            & LMX30-X-3                          \\
      \hline
      \textbf{Tensión nominal}                                   & 0-600V DC                          \\
      \hline
      \textbf{Corriente nominal}                                 & 0-6 A                              \\
      \hline
      \textbf{Fuente de agua de alimentación}                    & Agua pretratada en ósmosis inversa \\
      \hline
      \textbf{Flujo de producto}                                 & 3300 l/h                           \\
      \hline
      \textbf{Flujo de concentrado}                              & 180 l/h                            \\
      \hline
      \textbf{Conexión de los flujos de alimentación y producto} & 1”                                 \\
      \hline
      \textbf{Conexión de los flujos rechazo y concentrado}      & ½”                                 \\
      \hline
      \textbf{Temperatura ambiente de operación}                 & ≤ 45°C                             \\
      \hline
      \textbf{Fabricante}                                        & IONPURE                            \\
      \hline

\end{mytable}

La fuente de alimentación del Electrodesionizador, vital para su funcionamiento correcto, es el modelo PTM06 de STIL MAS.
Esta fuente de alimentación proporciona la energía eléctrica necesaria para el funcionamiento del Electrodesionizador,
permitiendo la ionización de las moléculas y facilitando su eliminación. La figura de la Fuente de alimentación que viene junto al electrodesionizador
se muestra a continuación (Figura \ref{fig:edi_power}). Sus especificaciones se muestran en la Tabla \ref{table:power_supply_specs}.

\insertimageboxed[\label{fig:edi_power}]{instrumentacion/edi_power}{scale=0.8}{0}{Modelo PTM06 de STIL MAS.}

\begin{mytable}{6cm}{Características de la fuente de alimentación PTM06 de STIL MAS.}{table:power_supply_specs}
      \hline
      \textbf{Fabricante}           & STIL MAS                                                     \\
      \hline
      \textbf{Modelo}               & PTM06                                                        \\
      \hline
      \textbf{Voltaje de entrada}   & 200-480 VAC (±5\%) - 50/60Hz                                 \\
      \hline
      \textbf{Corriente de entrada} & 1-20 A                                                       \\
      \hline
      \textbf{Voltaje de salida}    & 30-400 VDC                                                   \\
      \hline
      \textbf{Entradas de control}  & 2 x 4-20 mA + contactos de inicio/parada                     \\
      \hline
      \textbf{Salidas de control}   & 2 x 4-20 mA + contacto para establecer condiciones iniciales \\
      \hline
      \textbf{Potencia}             & 6KVA                                                         \\
      \hline
\end{mytable}


\subsection{Otros equipos}
La implementación del (EDI) en la planta farmacéutica de AICA UEB requiere una serie de válvulas y
sensores para garantizar un control riguroso del proceso. Tras un detallado análisis de la instrumentación existente
en la planta de tratamiento de agua, se decidió que los sensores y válvulas actuales cumplen a cabalidad con
los requerimientos del nuevo sistema. Estos dispositivos han demostrado su eficacia en las operaciones de
la planta y el personal tiene experiencia en su uso y mantenimiento. Por ello, no se consideró necesario
incorporar nuevos modelos de sensores o válvulas en la implementación del EDI.

A continuación, se resumen los principales elementos de instrumentación que serán utilizados en el sistema de EDI, la explicación detallada de cada uno de estos se encuentra en capítulos anteriores:
\begin{itemize}
      \item \textbf{Sensores de Conductividad:} Como el sensor de conductividad presentado en la sección \ref{sec:sesor_conductividad}, estos dispositivos permiten monitorizar la calidad del agua de salida del EDI en tiempo real.

      \item \textbf{Sensores de Temperatura:} Los sensores de temperatura son necesarios para asegurar que el proceso se lleva a cabo en las condiciones de temperatura óptimas, ver sección \ref{sec:sensor_temp}.

      \item \textbf{Transmisores de Flujo y Presión:} Los transmisores de flujo y presión, como se describen en las secciones \ref{sec:sensor_flujo} y \ref{sec:sensor_presion}, permiten monitorizar y controlar el flujo de agua y las condiciones de presión dentro del sistema de EDI.

      \item \textbf{Indicadores de Flujo y Manómetros:} Los indicadores de flujo y los manómetros proporcionan una visualización inmediata de las condiciones del sistema, lo que facilita su operación y mantenimiento, ver secciones \ref{sec:indicador_flujo} y \ref{sec:indicador_manometro}.

      \item \textbf{Válvulas de Retención y de control:} Estas válvulas, referenciadas en las secciones  \ref{sec:valvula_retencion}, \ref{sec:valvula_OnOff}, son fundamentales para controlar el flujo de agua dentro del sistema de EDI.
\end{itemize}
Es importante destacar que para la correcta implementación del EDI en nuestro sistema es necesario garantizar su comunicación con el sistema de control,
para ello se plantea la necesidad de incorporar un \textbf{módulo  de periferia descentralizada ET200s} como el de la sección \ref{sec:moduloEt200}, así de esta manera
los nuevos equipos ya planteados puedan incorporase al funcionamiento del sistema de control de la planta. Como dato interesante no se planteó la necesidad de incorporar
un módulo CPX de Festo ya que solo es preciso agregar una válvula de control la cual puede ser acoplada a un módulo CPX existente.

\subsection{Esquema General de la Configuración del EDI}

El sistema de Electrodesionización (EDI) implementado de la figura \ref{fig:EDI_pid} se compone de un único módulo de EDI.
Esta configuración se basa en la capacidad ya mencionada de la segunda etapa de la ósmosis inversa, que
produce 3000 litros por hora comparado con los 3300 litros por hora que puede entregar el EDI.
En un escenario donde el flujo requerido exceda la
capacidad del módulo de EDI, se implementarían múltiples unidades en paralelo.

El agua proveniente de la segunda etapa de ósmosis inversa se divide en dos flujos en el módulo de
EDI. Un flujo minoritario de agua se dirige hacia las celdas de agua a desechar, mientras que el flujo principal entra en las celdas para el agua purificada.

\insertimageboxed[\label{fig:EDI_pid}]{EDI_P&ID}{scale=0.9}{0}{Esquema P\&ID propuesto para la electrodesionización.}

En la línea principal de entrada al EDI, se instala una válvula manual y un indicador de presión. La válvula manual permite un control preciso sobre el flujo de agua al EDI, mientras que el indicador de presión proporciona una monitorización continua de la presión del agua en esta etapa.

El agua purificada que sale del módulo de EDI pasa a través de una serie de sensores e instrumentos. Se encuentra un sensor de conductividad con su correspondiente transmisor, un sensor de presión y un sensor de flujo. Estos dispositivos proporcionan información en tiempo real sobre la calidad del agua (conductividad), la presión a la salida del módulo de EDI y el flujo de agua, respectivamente. Además, se coloca una válvula de retención en la salida del EDI para evitar el flujo inverso del agua, manteniendo así la integridad del proceso de purificación.

En la línea de desecho del EDI, se colocan un indicador de presión y una válvula de retención. Este flujo de agua desechada es devuelto al tanque de pretratamiento, lo cual promueve la eficiencia del sistema y la conservación de agua. El indicador de presión permite el monitoreo de la presión en esta línea de desecho, asegurando que el funcionamiento del sistema sea óptimo.

Además, es crucial destacar la incorporación de la fuente de alimentación para el EDI, que se conecta directamente al módulo. Esta fuente de alimentación permite ajustar la corriente suministrada a los electrodos del EDI, garantizando así un control exacto sobre el proceso de Electrodesionización.



% ------------- Sección ----------------
\section{Programación del PLC}
\label{sec:sistema_control}
En esta sección, se describirá la programación específica del Controlador
Lógico Programable (PLC) para la secuencia principal de funcionamiento del
sistema con la Electrodesionización. Cabe destacar que esta programación
se enfocará exclusivamente en el electrodesionizador y su integración dentro
del sistema de ósmosis de doble etapa durante la operación normal de la planta.

El objetivo de la programación del PLC es interpretar las señales de entrada
provenientes de los distintos sensores y actuadores del sistema, y ejecutar
la lógica de control correspondiente para ajustar las operaciones del
electrodesionizador de acuerdo con las necesidades del proceso.
A continuación, en la figura \ref{fig:flujo_3} se presenta un diagrama de flujo que ilustra la
secuencia principal de funcionamiento del PLC para el electrodesionizador.

\insertimageboxed[\label{fig:flujo_3}]{/flujo_OI3}{scale=0.5}{0}{Diagrama de flujo para el EDI del proceso de producción de PW.}

Una vez que ambas etapas de la ósmosis alcanzan el estado de producción y se encuentran trabajando
se encuentra bajo circunstancias normales de trabajo (2), el módulo de Electrodesionización (EDI) puede comenzar su
operación con una descarga inicial con una duración de aproximadamente 60 segundos. Las descargas
producidas por el módulo de EDI se producen hacia el tanque de pretratamiento, esto significa que el agua
que sale del módulo tanto por flujo de producto como el flujo de concentrado son recirculados hacia
el tanque de almacenamiento que le da inicio al proceso de ósmosis (TK50A).

Posteriormente, se comprueban los parámetros como la conductividad y la presión en el producto del EDI.
Si alguno de estos parámetros no cumple con las especificaciones, el EDI entra en un estado de descarga
por parámetros deficientes y se mantiene en este estado hasta que los parámetros medidos y un período
de confirmación de 60 segundos indiquen condiciones aceptables.

Finalmente, una vez que los parámetros de conductividad y presión son óptimos y han pasado 60 segundos
de confirmación, el EDI cambia a un estado de producción, indicando la finalización exitosa de la
secuencia operacional del sistema de Electrodesionización.

% \subsection{Puesta en marcha}
% Un diagrama de flujo ofrece una visión clara y concisa de la lógica de control,
% facilitando la comprensión y el seguimiento de la secuencia de operaciones.
% Esto es especialmente útil para el personal de mantenimiento y operación,
% así como para cualquier persona que necesite entender el funcionamiento del sistema.

% La secuencia operacional del sistema de Electrodesionización se inicia con la activación de la planta de
% ósmosis inversa a través de una interfaz de usuario. Este evento de inicio es seguido de un
% período de espera hasta que el sensor de nivel determine que el tanque de agua pretratada (TK50A)
% ha alcanzado su nivel operativo óptimo (\ref{fig:flujo_1}).

% Durante este tiempo inicial, las etapas de ósmosis inversa (RO1 y RO2) se encuentran en un estado
% latente. RO1 aguarda la señal de nivel correcta del tanque TK50A, mientras que RO2 permanece en un
% estado de inactividad.

% Una vez que el sensor de nivel indica que TK50A ha alcanzado su nivel adecuado, se implementa
% un período de confirmación del nivel, que sirve para mitigar el impacto de posibles fluctuaciones
% en el nivel del tanque. Esta duración de tiempo se ha establecido típicamente en 60 segundos.

% A continuación, se inicia el flujo de agua hacia la primera etapa de ósmosis inversa (RO1). Esta
% etapa implica una descarga inicial de agua, necesaria debido a las posibles condiciones iniciales
% subóptimas del agua que entra en el sistema. Este período de descarga varía dependiendo de la condición
% de la membrana de ósmosis, pero suele ser de aproximadamente 120 segundos.

% Después de este período de descarga, el agua de RO1 es examinada para determinar si cumple con los
% parámetros de conductividad requeridos. Si la conductividad no cumple con las especificaciones, RO1
% entra en un estado de descarga por alta conductividad y se mantiene en este estado hasta que las
% mediciones de conductividad y un período de confirmación de 60 segundos indiquen que se cumplen
% los parámetros de conductividad.
% \insertimageboxed[\label{fig:flujo_1}]{/flujo_OI1}{scale=0.45}{0}{Diagrama de flujo para la RO1 del proceso de producción de PW.}
% En cuanto las condiciones de conductividad sean satisfactorias, RO1 cambia a un estado de producción
% y, simultáneamente, se inicia la segunda etapa de ósmosis inversa (RO2). Esta segunda etapa, al
% igual que RO1, comienza con una descarga inicial (ver Figura \ref{fig:flujo_2}). No obstante, a diferencia de RO1, el agua
% descargada por RO2 se devuelve al tanque de agua pretratada (TK50A), combinándose con el agua
% de permeado y concentrado. Este período de descarga también está sujeto a las condiciones de
% las membranas de ósmosis y dura aproximadamente 120 segundos.

% Posteriormente, se evalúan los parámetros de conductividad y temperatura en el permeado de RO2.
% Si alguno de estos parámetros no cumple con las especificaciones, RO2 entra en un estado de descarga
% por parámetros deficientes y se mantiene en este estado hasta que los parámetros medidos y un período
% de confirmación de 60 segundos indiquen condiciones aceptables.
% \insertimageboxed[\label{fig:flujo_2}]{/flujo_OI2}{scale=0.6}{0}{Diagrama de flujo para la RO2 del proceso de producción de PW.}
% Una vez que se alcanzan estos criterios, RO2 cambia a un estado de producción. Con ambas etapas de
% ósmosis inversa (RO1 y RO2) en producción, el módulo de Electrodesionización (EDI) puede comenzar su
% operación con una descarga inicial hacia el tanque de pretratamiento. Esta descarga inicial tiene
% una duración de aproximadamente 60 segundos.

% Posteriormente, se comprueban los parámetros como la conductividad y la presión en el producto del EDI.
% Si alguno de estos parámetros no cumple con las especificaciones, el EDI entra en un estado de descarga
% por parámetros deficientes y se mantiene en este estado hasta que los parámetros medidos y un período
% de confirmación de 60 segundos indiquen condiciones aceptables.

% Finalmente, una vez que los parámetros de conductividad y presión son óptimos y han pasado 60 segundos
% de confirmación, el EDI cambia a un estado de producción, indicando la finalización exitosa de la
% secuencia operacional del sistema de Electrodesionización.

% Con el sistema completo en estado de producción (ver Figura \ref{fig:flujo_3}), el estado
% posterior depende del nivel del tanque final. Si el tanque final está
% completamente lleno, la ósmosis comienza una circulación conjunta, que dura un
% tiempo de alrededor de 10 minutos. Superado este tiempo, se realiza una pausa de tiempo de 60 minutos antes de
% comenzar otro ciclo. La ósmosis continúa recirculando y no vuelve a producir
% hasta que el tanque de almacenamiento de agua purificada, que distribuye a los
% puntos de uso, señale un nivel del 75\% de capacidad.

% Cada vez que concluye un ciclo de producción y debe comenzar otro, se comprueba
% el estado del sensor de nivel bajo del tanque de agua pretratada. Si este
% sensor permanece activo (ver Figura \ref{fig:flujo_3}), se lleva a cabo directamente la
% descarga inicial de la OI1. De lo contrario, será necesario esperar hasta que
% el tanque TK 50A alcance el nivel mínimo necesario para poner el sistema a purificar.
% \insertimageboxed[\label{fig:flujo_3}]{/flujo_OI3}{scale=0.4}{0}{Diagrama de flujo para el EDI del proceso de producción de PW.}


% ------------- Sección ----------------
\section{Propuesta de SCADA}
\label{sec:scada_proposal}
Los sistemas de Supervisión, Control y Adquisición de Datos (SCADA) se han convertido en
una herramienta fundamental en el ámbito de la automatización industrial, permitiendo
la supervisión y control de procesos a gran escala de una manera eficiente y centralizada.
Este sistema ofrece ventajas significativas, como la optimización de operaciones, el
aumento de la eficiencia, la mejora de la calidad del producto y la prevención de
condiciones peligrosas.

En el contexto del sistema de purificación de agua en la planta de bulbos, la implementación de
un SCADA proporciona una visibilidad en tiempo real del proceso y facilita la
gestión de alarmas y el control de los componentes clave del sistema, como las membranas
de la ósmosis inversa y el dispositivo EDI. Además, un sistema SCADA permite el
registro de datos, esencial para el análisis de tendencias y la toma de decisiones
basada en datos.

El SCADA realizado para acompañar la propuesta de implementación de un EDI para la optimización de la purificación de agua en la
industria farmacéutica AICA UEB se ha desarrollado en el entorno de TIA Portal.
Este sistema está diseñado para proporcionar un monitoreo en tiempo real del proceso de ósmosis inversa, además de ofrecer una interfaz de usuario intuitiva e interactiva para los operadores.
\insertimageboxed[\label{fig:vistaGeneral}]{vistaGeneral}{scale=0.25}{0}{Vista general del sistema SCADA.}
El SCADA se estructura en varias secciones dedicadas a diferentes aspectos del proceso de purificación de agua. A continuación, se describen detalladamente cada una de estas secciones.

% nuevas secciones

\subsection{Gestión de Usuarios y Control de Acceso}

Una característica crítica del sistema SCADA propuesto es su capacidad para gestionar usuarios y controlar el acceso a sus diferentes secciones. El sistema se ha diseñado con dos niveles de acceso: Operadores y Administradores, para garantizar la seguridad y funcionalidad adecuada.

Es importante resaltar que todas las vistas y funcionalidades del SCADA están protegidas y se requiere un inicio de sesión válido para acceder. Un usuario debe al menos tener privilegios de Operador para navegar por las diversas vistas del sistema SCADA. La Figura \ref{fig:login} muestra la pantalla de inicio de sesión.
\insertimageboxed[\label{fig:login}]{inicioSesion}{scale=0.25}{0}{Pantalla de inicio de sesión del sistema SCADA propuesto.}
Los Operadores tienen acceso a las funciones básicas del sistema. Pueden monitorizar el proceso en tiempo real y realizar ajustes a los parámetros según sea necesario. Sin embargo, están limitados en el acceso a ciertas funciones de administración, como la gestión de usuarios.

Los Administradores, por otro lado, tienen acceso total a todas las secciones y funciones del sistema SCADA. Esto incluye la capacidad para gestionar usuarios, lo que les permite añadir, eliminar o modificar los privilegios de acceso de los operadores.

La Figura \ref{fig:adminUsuarios} proporciona una representación visual de la interfaz de la sección de gestión de usuarios del sistema SCADA propuesto.
\insertimageboxed[\label{fig:adminUsuarios}]{adminUsuarios}{scale=0.25}{0}{Interfaz de la sección de gestión de usuarios del sistema SCADA propuesto.}
\subsection{Monitoreo del proceso}

El proceso de ósmosis inversa con electrodesionización, se presenta en la interfaz del SCADA como una vista detallada, que refleja la operación del sistema en tiempo real. Esta vista reflejada en la figura \ref{fig:osmosis_inversa} permite al usuario interactuar y obtener información detallada de los componentes del sistema, como modelo, fabricante, entre otras características. Para acceder a estos detalles, el usuario simplemente puede hacer clic en el componente deseado.
\insertimageboxed[\label{fig:osmosis_inversa}]{monitoreoProceso1}{scale=0.25}{0}{Interfaz de la vista del proceso de ósmosis inversa.}
Cuando un usuario hace clic en un componente, se abre una ventana con las características detalladas del componente seleccionado. La Figura \ref{fig:componente} muestra un ejemplo de esta ventana.
\insertimageboxed[\label{fig:componente}]{caracteristicas}{scale=0.25}{0}{Interfaz de las características del componente.}
Esta funcionalidad de monitoreo en tiempo real proporciona a los usuarios una comprensión clara y actualizada del estado de la planta de tratamiento, permitiéndoles tomar decisiones informadas y rápidas en caso de necesidad. Este nivel de control y transparencia mejora la eficiencia operativa y reduce la probabilidad de errores y problemas no detectados.
\subsection{Configuración de Parámetros}
El sistema SCADA propuesto provee una interfaz dedicada para la configuración de parámetros, brindándole al usuario la capacidad de ajustar y personalizar varios aspectos operativos de la planta de tratamiento de agua. Esta sección es de vital importancia para garantizar el rendimiento óptimo del sistema y adaptarlo a condiciones cambiantes.

A través de esta interfaz, los usuarios pueden modificar parámetros de retardo para cada fase del proceso, así como la cantidad de corriente y voltaje suministrada al electrodesionizador. Además, también se pueden ajustar parámetros asociados a las alarmas del sistema, como los umbrales de activación, para adaptarlos a las necesidades específicas de la planta.

La Figura \ref{fig:parametros} muestra la interfaz de la sección de configuración de parámetros.
\insertimageboxed[\label{fig:parametros}]{parametros}{scale=0.25}{0}{Interfaz de la sección de configuración de parámetros.}
\subsection{Sistema de Alarmas}

Una característica esencial del sistema SCADA propuesto es su sofisticado sistema de alarmas. Este sistema tiene como objetivo alertar a los operadores y administradores sobre cualquier condición anómala que pudiera afectar el rendimiento de la planta de tratamiento de agua o que requiera atención inmediata.

Cuando se activa una alarma, el sistema SCADA muestra una ventana emergente en la que se enlistan todas las alarmas activas no acusadas. Los usuarios pueden acusar estas alarmas directamente desde esta ventana. La Figura \ref{fig:ventana_alarmas} muestra esta ventana emergente de alarmas.
\insertimageboxed[\label{fig:ventana_alarmas}]{adminAlarmas1}{scale=0.25}{0}{Ventana emergente de alarmas.}
En la interfaz dedicada para las alarmas, se puede observar un registro que muestra un historial de alarmas. Este registro tiene un buffer que almacena las alarmas más recientes hasta que se llena, momento en el que las alarmas más antiguas son reemplazadas por las nuevas. La Figura \ref{fig:seccion_alarmas} muestra la interfaz de la sección de alarmas.
\insertimageboxed[\label{fig:seccion_alarmas}]{adminAlarmas2}{scale=0.25}{0}{Interfaz de la sección de alarmas.}
Además, el sistema también cuenta con una funcionalidad que permite a los usuarios acceder a un historial completo de alarmas almacenadas en un fichero, incluyendo alarmas de días anteriores, lo que facilita el análisis y la identificación de tendencias o problemas recurrentes. La Figura \ref{fig:historial_alarmas} muestra esta interfaz de historial completo de alarmas.
\insertimageboxed[\label{fig:historial_alarmas}]{adminAlarmas3}{scale=0.25}{0}{Interfaz del historial completo de alarmas.}



\subsection{Gráficos Históricos}

La sección de gráficos históricos del sistema SCADA propuesto proporciona una herramienta vital para el análisis de la planta de tratamiento de agua. Los gráficos ilustran el comportamiento de las variables más importantes del proceso a lo largo del tiempo, lo que permite a los operadores y administradores rastrear cambios y detectar tendencias o problemas.

Las variables se almacenan en un fichero, lo que permite realizar análisis retrospectivos con información de días, semanas o incluso meses atrás. Además, los usuarios pueden generar informes basados en estos datos para un análisis más detallado o para la documentación de procesos.

La Figura \ref{fig:graficos_historicos} muestra la interfaz de la sección de gráficos históricos.
\insertimageboxed[\label{fig:graficos_historicos}]{graficosHistoricos1}{scale=0.25}{0}{Interfaz de la sección de gráficos históricos.}


\subsection{Generación de Informes}

La generación de informes es otra funcionalidad clave en el sistema SCADA propuesto. Los operadores y administradores pueden generar informes detallados basados en los datos de las variables del proceso, lo que facilita el análisis detallado y la toma de decisiones informada.

Los informes pueden contener información de varias variables en un período de tiempo determinado, lo que permite evaluar la eficiencia del sistema y detectar posibles problemas. Además, estos informes pueden servir para la documentación de procesos, lo que es útil para auditorías y revisiones de calidad.

La Figura \ref{fig:generacion_informes} muestra la interfaz de la generación de informes.
\insertimageboxed[\label{fig:generacion_informes}]{informe}{scale=0.25}{0}{Interfaz de la generación de informes.}




% \subsection{Administración de alarmas}

% El sistema SCADA cuenta con un eficiente sistema de gestión de alarmas. Las alarmas se presentan a los operadores en una ventana emergente que muestra todas las alarmas activas y no acusadas.

% \insertimageboxed[\label{fig:adminAlarmas1}]{adminAlarmas1}{scale=0.25}{0}{Ventana emergente de administración de alarmas.}
% \insertimageboxed[\label{fig:adminAlarmas2}]{adminAlarmas2}{scale=0.25}{0}{Detalle de una alarma en la ventana emergente.}

% Además, se ha implementado un visor de alarmas que mantiene un registro temporal de las alarmas hasta que su capacidad es alcanzada. Todas las alarmas se almacenan en un fichero para permitir un seguimiento detallado.

% \subsection{Análisis de gráficos históricos}

% La sección de gráficos históricos permite a los operadores y administradores visualizar tendencias y patrones en las variables más importantes del sistema de purificación.

% \insertimageboxed[\label{fig:graficosHistoricos1}]{graficosHistoricos1}{scale=0.25}{0}{Interfaz de análisis de gráficos históricos.}

% \subsection{Administración de usuarios}

% La administración de usuarios en el sistema SCADA permite la configuración de diferentes niveles de acceso, con dos grupos de usuarios definidos: operadores y administradores.

% \insertimageboxed[\label{fig:inicioSesion}]{inicioSesion}{scale=0.25}{0}{Pantalla de inicio de sesión.}
% \insertimageboxed[\label{fig:adminUsuarios}]{adminUsuarios}{scale=0.25}{0}{Interfaz de la sección de administración de usuarios.}

% El SCADA ha sido diseñado con un enfoque flexible y adaptable para permitir la incorporación de futuras secciones o funcionalidades según las necesidades cambiantes de la planta de purificación.






% ------------- Sección ----------------
\section{Instalación del EDI }
\label{sec:implementation_start}

La implementación de un nuevo componente de un sistema de tratamiento de agua,
como un dispositivo de EDI, es un proceso complejo que requiere consideraciones
cuidadosas de diseño, logística, instalación y pruebas. Esta tarea se vuelve
aún más crítica cuando este nuevo componente debe integrarse a un sistema
existente sin interrumpir significativamente su funcionamiento normal.

La implementación del sistema de Electrodesionización (EDI) luego de la ósmosis inversa doble requiere una serie de pasos clave para garantizar su correcta instalación y funcionamiento. A continuación, se proporciona un desglose detallado de este proceso:

\begin{enumerate}
    \item \textbf{Evaluación del sitio de instalación:} Antes de la instalación del EDI, es esencial realizar una evaluación exhaustiva del sitio para determinar la adecuación del área para alojar la unidad. Factores como la disponibilidad de espacio, la accesibilidad para el mantenimiento, la disponibilidad de suministro de agua y energía, y las condiciones ambientales deben ser considerados.

    \item \textbf{Preparación del sitio de instalación:} Una vez evaluado el sitio, se prepara para la instalación. Esto puede implicar trabajos de construcción menores para proporcionar una base estable y segura para la unidad EDI, y la configuración de las conexiones necesarias para el agua, la electricidad y el drenaje.

    \item \textbf{Instalación de la unidad EDI:} La unidad de EDI se instala en el sitio preparado. Esto debe ser realizado por técnicos cualificados para garantizar que la unidad se instale correctamente y de manera segura. Los componentes de la unidad deben ser cuidadosamente manejados para evitar daños.

    \item \textbf{Conexión de la unidad EDI:} Una vez instalada la unidad, se conecta a las fuentes de agua y electricidad, y al sistema de drenaje. Los componentes de la unidad, como las membranas, las bombas y los sensores, también se conectan y se aseguran.

    \item \textbf{Pruebas de la unidad EDI:} Antes de la puesta en marcha completa, la unidad EDI se somete a una serie de pruebas para verificar su correcto funcionamiento. Esto incluye pruebas de la funcionalidad del PLC y del sistema SCADA, así como pruebas de la capacidad de la unidad para purificar el agua a las especificaciones requeridas.

    \item \textbf{Puesta en marcha de la unidad EDI:} Una vez que se han realizado y superado todas las pruebas, se pone en marcha la unidad EDI. Durante la puesta en marcha inicial, se debe monitorear de cerca la operación de la unidad para identificar y corregir cualquier problema que pueda surgir.
\end{enumerate}

En cada una de estas etapas, se deben seguir estrictamente las normas y procedimientos de seguridad para proteger tanto al personal como al equipo. También es fundamental mantener una documentación detallada de todo el proceso de instalación y puesta en marcha para facilitar futuras referencias y mantenimiento.
