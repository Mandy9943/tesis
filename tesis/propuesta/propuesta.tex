\chapter{Propuesta de Implementación de EDI Después de la Ósmosis Inversa Doble}
\label{cap:propuesta_implementacion}

La necesidad de alcanzar niveles más altos de pureza del agua ha impulsado
la evolución y mejora continua de las tecnologías de tratamiento de agua.
Una de estas tecnologías es la Electrodeionización (EDI), que combina
los principios de electrodiálisis y resinas de intercambio iónico para
producir agua de alta pureza. En la industria farmacéutica, donde se
requiere un agua con una calidad excepcional, la implementación de la
tecnología EDI se convierte en un paso esencial después de la ósmosis inversa doble.\\

Este capítulo presentará la propuesta de implementación de un sistema
EDI en la empresa AICA. En primer lugar, se discutirá el sistema de
control que regula el funcionamiento del EDI y cómo este se coordina
con el sistema de control de la planta en general.
Luego, se presentará la propuesta de integración de un sistema SCADA,
mostrando su interfaz de usuario y explicando cómo este sistema
ayudará a supervisar y controlar el proceso de tratamiento del agua.
Finalmente, se describirá el proceso de implementación y puesta en marcha del
EDI, abarcando desde la instalación física del dispositivo hasta las pruebas
iniciales para verificar su funcionamiento correcto.

% ------------- Sección ----------------
\section{Sistema de control}
\label{sec:sistema_control}

En cualquier sistema industrial, el control es un componente crucial. La eficiencia, seguridad y eficacia de un sistema dependen en gran medida de su capacidad para responder a las variables del entorno y ajustar su funcionamiento en consecuencia. El sistema de control es el cerebro de la operación, coordinando y supervisando todos los aspectos del proceso. En el caso de un sistema de Electrodeionización, el control es aún más crítico debido a la complejidad del proceso y la alta calidad del producto final requerido.\\

La programación del Controlador Lógico Programable (PLC, por sus siglas en inglés) es un elemento esencial de este sistema de control. El PLC se encarga de interpretar las señales de entrada de los distintos sensores y actuadores y ejecutar la lógica de control para ajustar las operaciones del sistema de acuerdo a las necesidades. En esta sección, se presentará la programación del PLC en forma de diagrama de flujo para la secuencia principal de funcionamiento del sistema de Electrodeionización.\\

\subsection{Puesta en marcha}

Un diagrama de flujo ofrece una visión clara y concisa de la lógica de control,
facilitando la comprensión y el seguimiento de la secuencia de operaciones.
Esto es especialmente útil para el personal de mantenimiento y operación,
así como para cualquier persona que necesite entender el funcionamiento del sistema.\\

El control del proceso de producción de agua purificada es esencial y se
lleva a cabo mediante la programación del PLC, que se representa mediante
el diagrama de flujo en las Figuras 3.1-3.4. El proceso de producción
comienza solo después de que se presiona el botón de inicio (K5, Figura 3.1).
Si no se presiona el botón, el sistema de ósmosis inversa permanece en un
estado de parada. Una vez que se presiona el botón de inicio, el sistema
espera una señal (K10) del sensor de nivel (LT 50-20), que indica el nivel mínimo que debe tener
del tanque de almacenamiento TK 50A para su operación con una señal positiva. Luego, hay un breve período de espera
(K11) para confirmar la señal del sensor de nivel, lo que evita errores de señalización
debido a fluctuaciones en el nivel de agua en el tanque.\\


A continuación, se ponen en marcha las bombas: la dosificadora de sosa cáustica
(PD67-10), la de lavado químico (P50-2A) y la presurizadora de la línea (P50A).
La válvula de lavado químico XV 50-1A se abre para permitir el paso a las membranas,
realizando un barrido que envía las posibles sales incrustadas en la superficie de las
membranas al alcantarillado.\\

El proceso inicial de filtración tiene una duración variable (K12) que depende del
estado de la ósmosis en general. Con la explotación continua, ha sido necesario
aumentar este tiempo, ya que varía debido al desgaste de las membranas.
Los operadores establecen el valor de la demora requerida para este proceso desde el SCADA.\\

Una vez finalizada la filtración inicial en la OI1, se tiene en cuenta el valor de la
conductividad del flujo de permeado. Si esta conductividad es alta, tiene lugar
la descarga al drenaje por alta conductividad en esta primera etapa. Se continúa monitoreando
el valor de la conductividad y si se ajusta a los parámetros, se espera un tiempo
(K13) durante el cual el valor de la conductividad debe ser lo suficientemente bajo.
Pero si se supera el valor de K13 y no persiste el valor de esta variable, el sistema
entra en fase de alarma.\\

Con la producción de agua de la OI1 iniciada (ver Figura 3.2), tiene lugar el filtrado
inicial de la OI2 mientras el EDI permanece en la condición de espera.
La descarga inicial de la OI2 se realiza con el agua producida por la primera etapa.
Para permitir el flujo correcto hacia la OI2 se enciende la bomba
presurizadora P50B. En el concentrado se hace circular
hacia el tanque de agua pretratada.\\

Una vez transcurrido el tiempo (T14) de la descarga inicial de la OI2, se
comprueban los valores de temperatura y conductividad. Al igual que en la etapa
anterior, se espera la estabilización de estos parámetros (K15) y la OI2
comienza a producir en conjunto con la OI1. Si los tiempos establecidos en
el SCADA son superados, el sistema entra en fase de alarma.\\

Con ambos grupos de membranas (OI1 y OI2) en producción (ver Figura 3.2),
el EDI comienza su descarga inicial, que dura un tiempo T16. Durante este
tiempo, se abre la válvula XV 50-15B, que permite el paso del agua al
dispositivo. El agua producida por la OI1 y la OI2 se utiliza para arrastrar
los residuos del ciclo de trabajo anterior y se envía al tanque de pretratamiento,
con el objetivo de no desecharla. Para ello, se abre la válvula XV 50-27C.\\

Una vez superado el tiempo T16, se comprueba el estado de la temperatura y la
conductividad. Si estos valores no cumplen con los valores prefijados, se realiza
la descarga por parámetros incorrectos durante un tiempo T17. Si estos parámetros
no se estabilizan en el rango debido y en el tiempo establecido, la ósmosis activa
el estado de alarma.\\

Cuando la conductividad y la temperatura han alcanzado valores estables, puede
comenzar la producción del EDI, que tiene una duración de valor T18, con la
ósmosis completa en producción (ver Figura 3.3).\\

Con el sistema completo en estado de producción (ver Figura 3.4), el estado
posterior depende del nivel del tanque final. Si el tanque final está
completamente lleno, la ósmosis comienza una circulación conjunta, que dura un
tiempo T19. Superado este tiempo, se realiza una pausa de tiempo T20 antes de
comenzar otro ciclo. La ósmosis continúa recirculando y no vuelve a producir
hasta que el tanque de almacenamiento de agua purificada, que distribuye a los
puntos de uso, señale un nivel medio de capacidad.\\

Cada vez que concluye un ciclo de producción y debe comenzar otro, se comprueba
el estado del sensor de nivel bajo del tanque de agua pretratada. Si este
sensor permanece activo (ver Figura 3.1), se lleva a cabo directamente la
descarga inicial de la OI1. De lo contrario, será necesario esperar hasta que
el tanque TK 50A alcance el nivel mínimo necesario para poner el sistema a purificar.\\




% ------------- Sección ----------------
\section{Propuesta de SCADA}
\label{sec:scada_proposal}

Los Sistemas de Control y Adquisición de Datos (SCADA) se han convertido en una herramienta fundamental en el ámbito de la automatización industrial, permitiendo la supervisión y control de procesos a gran escala de una manera eficiente y centralizada. Este sistema ofrece ventajas significativas, como la optimización de operaciones, el aumento de la eficiencia, la mejora de la calidad del producto y la prevención de condiciones peligrosas.\\

En el contexto del sistema de purificación de agua de la planta, la implementación de un SCADA proporcionaría una visibilidad en tiempo real del proceso y facilitaría la gestión de alarmas y el control de los componentes clave del sistema, como las membranas de la ósmosis inversa y el dispositivo EDI. Además, un sistema SCADA permitiría el registro de datos, esencial para el análisis de tendencias y la toma de decisiones basada en datos.\\

En la presente sección, se propone una implementación de SCADA para el sistema de purificación de agua. Esta propuesta se centra en proporcionar una interfaz amigable para el usuario y en maximizar el potencial del sistema de control existente. La propuesta también incluye el diseño de las pantallas de visualización y los detalles de cómo los operadores pueden interactuar con el sistema para mantener el proceso funcionando de manera eficiente y segura.\\

\subsection{Implementación de SCADA}

% ------------- Sección ----------------
\section{Instalación del EDI }
\label{sec:implementation_start}

La implementación de un nuevo componente de un sistema de tratamiento de agua,
como un dispositivo de EDI, es un proceso complejo que requiere consideraciones
cuidadosas de diseño, logística, instalación y pruebas. Esta tarea se vuelve
aún más crítica cuando este nuevo componente debe integrarse a un sistema
existente sin interrumpir significativamente su funcionamiento normal.\\

En esta sección, describiremos el proceso de implementación y puesta en marcha del dispositivo EDI propuesto después de la doble ósmosis inversa. Este proceso incluirá los detalles de la instalación física del dispositivo EDI, desde su recepción y montaje hasta su conexión con el sistema existente. También cubriremos las pruebas iniciales que deben realizarse para garantizar que el dispositivo EDI esté operando correctamente y para confirmar que se pueden alcanzar los parámetros deseados de pureza del agua.\\

Además, discutiremos la importancia de la formación del personal que manejará el dispositivo EDI para asegurar una operación segura y eficiente a largo plazo. Este entrenamiento debe incluir el uso del nuevo sistema SCADA propuesto, así como los procedimientos de mantenimiento y resolución de problemas específicos del dispositivo EDI. \\

Finalmente, se abordarán las medidas de seguimiento y evaluación que deben implementarse para garantizar la eficacia y eficiencia del sistema a lo largo del tiempo.\\

\subsection{Proceso de instalación del EDI}

La implementación del sistema de Electrodeionización (EDI) luego de la ósmosis inversa doble requiere una serie de pasos clave para garantizar su correcta instalación y funcionamiento. A continuación, se proporciona un desglose detallado de este proceso:

\begin{enumerate}
    \item \textbf{Evaluación del sitio de instalación:} Antes de la instalación del EDI, es esencial realizar una evaluación exhaustiva del sitio para determinar la adecuación del área para alojar la unidad. Factores como la disponibilidad de espacio, la accesibilidad para el mantenimiento, la disponibilidad de suministro de agua y energía, y las condiciones ambientales deben ser considerados.

    \item \textbf{Preparación del sitio de instalación:} Una vez evaluado el sitio, se prepara para la instalación. Esto puede implicar trabajos de construcción menores para proporcionar una base estable y segura para la unidad EDI, y la configuración de las conexiones necesarias para el agua, la electricidad y el drenaje.

    \item \textbf{Instalación de la unidad EDI:} La unidad de EDI se instala en el sitio preparado. Esto debe ser realizado por técnicos cualificados para garantizar que la unidad se instale correctamente y de manera segura. Los componentes de la unidad deben ser cuidadosamente manejados para evitar daños.

    \item \textbf{Conexión de la unidad EDI:} Una vez instalada la unidad, se conecta a las fuentes de agua y electricidad, y al sistema de drenaje. Los componentes de la unidad, como las membranas, las bombas y los sensores, también se conectan y se aseguran.

    \item \textbf{Pruebas de la unidad EDI:} Antes de la puesta en marcha completa, la unidad EDI se somete a una serie de pruebas para verificar su correcto funcionamiento. Esto incluye pruebas de la funcionalidad del PLC y del sistema SCADA, así como pruebas de la capacidad de la unidad para purificar el agua a las especificaciones requeridas.

    \item \textbf{Puesta en marcha de la unidad EDI:} Una vez que se han realizado y superado todas las pruebas, se pone en marcha la unidad EDI. Durante la puesta en marcha inicial, se debe monitorear de cerca la operación de la unidad para identificar y corregir cualquier problema que pueda surgir.
\end{enumerate}

En cada una de estas etapas, se deben seguir estrictamente las normas y procedimientos de seguridad para proteger tanto al personal como al equipo. También es fundamental mantener una documentación detallada de todo el proceso de instalación y puesta en marcha para facilitar futuras referencias y mantenimiento.
