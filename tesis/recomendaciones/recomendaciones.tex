\chapter*{Recomendaciones}
\phantomsection
\addcontentsline{toc}{chapter}{Recomendaciones}
\vspace{-2cm} 
Para futuros trabajos en esta área, se recomienda realizar estudios prácticos y
experimentales para validar los resultados obtenidos teóricamente en este estudio.
Implementar pruebas piloto del sistema EDI en una planta de AICA proporcionaría
datos valiosos y confirmaría o refutaría los hallazgos actuales.

Además, sería beneficioso investigar cómo la tecnología EDI podría integrarse
con otras tecnologías emergentes de tratamiento de agua. Por ejemplo,
la nanofiltración podría
trabajar en conjunto con la tecnología EDI para optimizar aún más el
proceso de purificación de agua.

Finalmente, es esencial continuar buscando y recopilando más información
sobre la implementación y el funcionamiento del EDI en la industria farmacéutica.
A medida que la tecnología continúa avanzando y más empresas comienzan a adoptarla,
es probable que la información y los estudios de caso disponibles aumenten.
Mantenerse al día con esta literatura será vital para cualquier trabajo futuro en esta área.