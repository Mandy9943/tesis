\chapter{Estado del arte y descripción del proceso}
\vspace{-2cm} 
La industria farmacéutica es un sector crítico para la salud y el bienestar de la sociedad, y la calidad del agua utilizada
en los procesos de producción desempeña un papel fundamental en la garantía de la seguridad y eficacia de los productos farmacéuticos. En este capítulo,
se realizará una revisión exhaustiva de la literatura relacionada con los sistemas de tratamiento de agua en la industria farmacéutica, abordando temas
como la importancia del tratamiento de agua, las clasificaciones y requisitos regulatorios, y las tecnologías de tratamiento empleadas \cite{juanantoniodelacuerdaImportanciaAguaIndustria2021}.
 
Esta revisión tiene como objetivo proporcionar un panorama completo del estado actual del conocimiento en este campo, así como identificar las tendencias y enfoques de investigación que podrían dar lugar a mejoras en los sistemas de tratamiento de agua en el futuro. Al comprender en profundidad el contexto y las consideraciones clave en la purificación del agua farmacéutica, se sentarán las bases para una discusión informada sobre la propuesta de incorporar un electrodesionizador (EDI) en el sistema de ósmosis inversa de la planta de AICA, como se detalla en los capítulos posteriores.

\section{Sistemas de tratamiento de agua en la industria farmacéutica}
Los sistemas de tratamiento de agua desempeñan un papel crucial en la industria farmacéutica al 
garantizar la calidad y pureza del agua utilizada en los procesos de fabricación. 
Estos sistemas están diseñados para eliminar impurezas, microorganismos y productos 
químicos indeseables, asegurando que el agua cumpla con los estándares regulatorios
 y las especificaciones de la industria. La implementación adecuada de estos sistemas asegura 
 que el agua utilizada 
en la producción farmacéutica sea segura, confiable y cumpla con los
 requisitos de calidad necesarios para la fabricación de medicamentos.

\subsection{ Importancia del tratamiento de agua en la industria farmacéutica}
El agua es un recurso indispensable en la industria farmacéutica debido a su amplia utilización en múltiples procesos, tales como la producción de medicamentos, la limpieza de equipos, la fabricación de soluciones y reactivos, y la generación de vapor, entre otros. Dada su relevancia, el tratamiento de agua en este sector es de suma importancia para garantizar la calidad, seguridad y eficacia de los productos farmacéuticos. A continuación, se detallan varias razones que explican la importancia del tratamiento de agua en la industria farmacéutica.\\

\textbf{Calidad del producto:} El agua utilizada en la producción de medicamentos debe cumplir con estándares estrictos de calidad y pureza, ya que su presencia en la composición de los productos puede afectar significativamente su estabilidad, potencia y seguridad. Por ejemplo, la presencia de impurezas en el agua, como iones metálicos, microorganismos o productos químicos, puede reaccionar con los ingredientes activos y excipientes de los medicamentos, alterando sus propiedades y generando efectos adversos en los pacientes.\\

\textbf{Regulaciones y normativas:} Las agencias reguladoras de todo el mundo, como la FDA (Administración de Alimentos y Medicamentos de EE. UU.) y la EMA (Agencia Europea de Medicamentos), establecen requisitos rigurosos y específicos en cuanto a la calidad del agua empleada en la producción farmacéutica. Estas regulaciones tienen como objetivo garantizar que el agua utilizada cumpla con ciertos niveles de pureza y seguridad, y que los sistemas de tratamiento de agua sean adecuados y efectivos para garantizar la calidad del producto final.\\

\textbf{Control de contaminación y biofilm:} La proliferación de microorganismos y la formación de biofilm en los sistemas de tratamiento de agua pueden tener consecuencias negativas para la calidad de los productos farmacéuticos. Un tratamiento de agua eficiente debe eliminar o reducir al mínimo la presencia de microorganismos y prevenir la formación de biofilm en las superficies de los equipos y tuberías. De esta manera, se asegura un ambiente adecuado para la producción de medicamentos y se evita la contaminación cruzada.\\

\textbf{Eficiencia en los procesos:} Un sistema de tratamiento de agua eficiente y bien diseñado puede optimizar los procesos de producción y reducir los costos operativos. El uso de tecnologías avanzadas, como la ósmosis inversa y la electrodeionización (EDI), permite obtener agua de alta calidad y pureza, lo que a su vez disminuye la necesidad de tratamientos adicionales y reduce el consumo de reactivos y energía.\\

\textbf{Responsabilidad medioambiental:} La industria farmacéutica tiene una responsabilidad ética y legal de minimizar su impacto ambiental. El tratamiento adecuado del agua permite reducir la cantidad de contaminantes y sustancias químicas liberadas al medio ambiente y optimizar el uso de los recursos hídricos. Además, las tecnologías de tratamiento de agua más avanzadas pueden contribuir a la reducción del consumo energético y la generación de residuos.\\

En resumen, el tratamiento de agua en la industria farmacéutica es fundamental para garantizar la calidad, seguridad y eficacia de los productos, cumplir con las regulaciones y normativas vigentes, controlar la contaminación y la formación de biofilm, optimizar la eficiencia en los procesos y reducir el impacto medioambiental.\\

El tratamiento adecuado del agua en la industria farmacéutica no sólo garantiza que se cumplan los requisitos de calidad y pureza del agua, sino que también contribuye a la prevención de problemas asociados con la presencia de impurezas y contaminantes. Por lo tanto, es fundamental que las empresas farmacéuticas inviertan en tecnologías de tratamiento de agua apropiadas y en la implementación de sistemas de control y monitoreo efectivos.\\ 


\subsection{Tipos y clasificaciones del agua}

El agua es un componente fundamental en la industria farmacéutica, y su calidad y
 pureza son aspectos críticos para garantizar la seguridad y eficacia de los productos.
 Dependiendo de su uso y aplicación, existen diferentes tipos y clasificaciones de agua en la industria farmacéutica. 
 A continuación, se presentan las categorías más comunes \cite{setaphtTratamientosAguaPara}:\\

\textbf{Agua purificada (PW):} Es el tipo básico de agua utilizada en la industria farmacéutica y se obtiene a través de procesos como ósmosis inversa, destilación, intercambio iónico o filtración. La calidad del agua purificada es menor que la del agua para inyección (WFI), pero es adecuada para la fabricación de productos no parenterales y para su uso en procesos de limpieza.

\textbf{Agua para inyección (WFI):} Es un tipo de agua de alta pureza que se utiliza en la fabricación de productos parenterales, es decir, aquellos que se administran por vías como intravenosa, intramuscular o subcutánea. La calidad del WFI es superior a la del agua purificada, y se obtiene mediante procesos de destilación, ósmosis inversa o por una combinación de ambos métodos.

\textbf{Agua altamente purificada (HPW):} Este tipo de agua tiene una calidad intermedia entre el agua purificada y el WFI. Se utiliza en ciertas aplicaciones farmacéuticas donde se requiere un nivel de pureza más elevado que el del agua purificada, pero no se necesita llegar al grado de pureza del WFI.

\textbf{Agua estéril:} Es agua que ha sido sometida a un proceso de esterilización, como la filtración estéril o la autoclave, para eliminar cualquier microorganismo presente. El agua estéril se utiliza en aplicaciones específicas, como en la fabricación de productos estériles o en procesos de limpieza y desinfección que requieren la eliminación de microorganismos.

Cabe destacar que las regulaciones y normativas, como las establecidas por la Farmacopea de Estados Unidos (USP), la Farmacopea Europea (EP) y la Organización Mundial de la Salud (OMS), definen los requisitos de calidad y las especificaciones para cada tipo de agua en la industria farmacéutica. Estas especificaciones incluyen parámetros como la conductividad, el pH, la presencia de sustancias orgánicas, inorgánicas y microbiológicas, entre otros.

\subsection{ Requisitos y regulaciones aplicables al agua}

La calidad del agua utilizada en la industria farmacéutica está sujeta a una serie de requisitos y regulaciones establecidos por diversas entidades y organismos a nivel nacional e internacional. Estas regulaciones aseguran que el agua cumpla con los estándares de calidad necesarios para garantizar la seguridad y eficacia de los productos farmacéuticos. Algunas de las principales regulaciones y requisitos aplicables al agua en la industria farmacéutica incluyen:

\textbf{Farmacopeas:} Las farmacopeas son documentos oficiales que contienen las especificaciones técnicas y
requisitos de calidad para sustancias y productos farmacéuticos, incluidos los diferentes
tipos de agua. Entre las farmacopeas más reconocidas a nivel mundial se encuentran la
Farmacopea de Estados Unidos (USP), la Farmacopea Europea (EP) y la Farmacopea de Japón
(JP). Cada farmacopea establece parámetros específicos de calidad, como la conductividad,
el pH, la presencia de sustancias orgánicas, inorgánicas y microbiológicas, entre otros \cite{farm.veronicamartinezFARMACOPEAS2005}.

\textbf{ Buenas Prácticas de Fabricación (GMP):} Las GMP son normas que establecen los requisitos mínimos que deben cumplir
los procesos de fabricación, control de calidad y distribución de productos farmacéuticos, incluida la gestión del agua.
Estas normas son aplicables a nivel mundial y son emitidas por organismos como la Food and Drug Administration (FDA) en
Estados Unidos, la European Medicines Agency (EMA) en Europa y la Organización Mundial de la Salud (OMS) \cite{ispeGoodManufacturingPractice}.

\textbf{ Directrices y guías técnicas:} Además de las farmacopeas y las GMP, existen directrices y guías técnicas
emitidas por organismos internacionales y nacionales que abordan aspectos específicos relacionados con el agua en
la industria farmacéutica. Estas directrices pueden incluir recomendaciones sobre el diseño y validación de sistemas
de tratamiento de agua, el monitoreo de la calidad del agua y la prevención de la contaminación.

\textbf{ Normativas nacionales y locales:} Cada país puede tener sus propias normativas y requisitos legales
aplicables al agua en la industria farmacéutica. Estas normativas pueden estar en línea con las farmacopeas y
las GMP, pero también pueden incluir requisitos adicionales específicos para cada país o región.


El cumplimiento de estas regulaciones y requisitos garantiza la calidad y seguridad del agua utilizada en la fabricación de productos farmacéuticos y, en última instancia, protege la salud de los pacientes \cite{juanantoniodelacuerdaImportanciaAguaIndustria2021}.

\subsection{Impurezas presentes en el agua}

El agua utilizada en la industria farmacéutica puede contener diversas impurezas, las cuales pueden afectar la calidad, seguridad y eficacia de los productos finales. Estas impurezas pueden clasificarse en tres categorías principales: impurezas inorgánicas, impurezas orgánicas y contaminantes microbiológicos.\\

\textbf{Impurezas inorgánicas: }Incluyen iones metálicos y no metálicos, como calcio, magnesio, sodio, cloruros, sulfatos y silicatos. Estas impurezas pueden afectar la calidad de los productos farmacéuticos al causar cambios en la solubilidad, la estabilidad y la eficacia de los ingredientes activos, así como en la formación de precipitados y la corrosión de equipos y recipientes. Además, algunos iones metálicos, como el hierro, el cobre y el cromo, pueden ser tóxicos y afectar la seguridad de los productos.\\

\textbf{Impurezas orgánicas: }Son compuestos de origen natural o sintético, como ácidos húmicos y fúlvicos, pesticidas, disolventes y productos químicos de desinfección. Las impurezas orgánicas pueden reaccionar con los ingredientes activos y otros excipientes, lo que puede alterar la estabilidad, la eficacia y la liberación de los fármacos. Además, algunos compuestos orgánicos pueden ser tóxicos y afectar la seguridad de los productos farmacéuticos.\\

\textbf{Contaminantes microbiológicos:} Incluyen bacterias, hongos, levaduras, virus y protozoos. La presencia de microorganismos en el agua puede causar la contaminación de los productos farmacéuticos, lo que puede llevar a infecciones y reacciones adversas en los pacientes. Además, algunos microorganismos pueden producir sustancias tóxicas, como endotoxinas y micotoxinas, que pueden afectar la seguridad y eficacia de los productos.\\

El tratamiento adecuado del agua es esencial para eliminar o reducir estas impurezas a niveles aceptables, de acuerdo con las regulaciones y requisitos aplicables en la industria farmacéutica. Un control riguroso de la calidad del agua, así como el uso de tecnologías de purificación adecuadas, como la ósmosis inversa, la desionización y la electrodesionización (EDI), son fundamentales para garantizar la calidad y seguridad de los productos farmacéuticos.\\

\subsection{Variables críticas en la calidad del agua}

El tratamiento y monitoreo de la calidad del agua en la industria farmacéutica requieren un enfoque riguroso y sistemático para garantizar la eliminación efectiva de impurezas y el cumplimiento de los requisitos regulatorios. A continuación, se presentan algunas de las variables críticas que deben considerarse durante el tratamiento y monitoreo del agua:

\textbf{Conductividad eléctrica:} La conductividad eléctrica es una medida de la capacidad del agua para conducir la corriente eléctrica, y
está directamente relacionada con la concentración de iones disueltos en el agua. Un mayor valor de conductividad indica una mayor
concentración de impurezas inorgánicas. El monitoreo de la conductividad es fundamental para evaluar la efectividad de los procesos
de purificación y para asegurar el cumplimiento de los límites establecidos por las regulaciones aplicables \cite{oceanebidaultQueFactoresDeterminan}.

\textbf{Contenido de carbono orgánico total (TOC ):} El TOC es una medida del contenido de carbono en compuestos
orgánicos disueltos en el agua. Un alto nivel de TOC indica una mayor concentración de impurezas orgánicas.
El monitoreo regular del TOC es esencial para garantizar que el agua cumpla con los requisitos de calidad y para evaluar la eficacia
de los procesos de purificación en la eliminación de compuestos orgánicos \cite{oceanebidaultQueFactoresDeterminan}.

\textbf{Conteo microbiano y endotoxinas:} El monitoreo del recuento microbiano y las endotoxinas es fundamental
para controlar la calidad microbiológica del agua y garantizar la seguridad de los productos farmacéuticos.
Los métodos de análisis microbiológico incluyen el recuento en placa, el método de filtración por membrana
y las técnicas de bioluminiscencia. Las endotoxinas, sustancias tóxicas liberadas por bacterias Gram-negativas,
se miden mediante el ensayo de lisado de amebocitos de Limulus (LAL) \cite{oceanebidaultQueFactoresDeterminan}.

\textbf{pH:} El pH es una medida de la acidez o alcalinidad del agua y puede afectar la solubilidad,
la estabilidad y la reactividad de los ingredientes activos y excipientes en los productos farmacéuticos.
El control del pH es esencial para mantener un ambiente adecuado en los sistemas de tratamiento de agua y
garantizar la calidad del agua producida \cite{oceanebidaultQueFactoresDeterminan}.

\textbf{Turbidez:} La turbidez es una medida de la cantidad de partículas en suspensión en el agua,
incluidas partículas inorgánicas, orgánicas y microbiológicas. Un nivel elevado de turbidez puede afectar
la efectividad de los procesos de purificación y el rendimiento de los equipos. La turbidez se mide utilizando
un turbidímetro y se expresa en unidades de turbidez nefelométrica (NTU) \cite{oceanebidaultQueFactoresDeterminan}.

El monitoreo y control de estas variables críticas durante el tratamiento y purificación del agua son fundamentales para garantizar la calidad, seguridad y eficacia de los productos farmacéuticos y cumplir con los requisitos regulatorios aplicables .

\subsection{Evolución histórica de las tecnologías de tratamiento de agua}

La historia del tratamiento de agua en la industria farmacéutica ha experimentado una evolución considerable a lo largo del tiempo. A medida que la industria ha crecido y los requisitos regulatorios han aumentado en complejidad, las tecnologías de tratamiento de agua han seguido mejorando para garantizar la calidad y la seguridad de los productos farmacéuticos. A continuación, se presenta un breve recorrido histórico de las tecnologías de tratamiento de agua en la industria farmacéutica:\\

\textbf{Finales del siglo XIX y principios del siglo XX:}Durante este período, los sistemas de tratamiento de agua se basaban en procesos simples como la sedimentación, la filtración y la desinfección con cloro. Estos métodos eran efectivos para eliminar partículas en suspensión e impurezas microbiológicas, pero no eran capaces de eliminar completamente las impurezas químicas.\\

\textbf{Mitad del siglo XX:}Con el avance de la química y la comprensión de los requisitos de calidad del agua para los productos farmacéuticos, se introdujeron tecnologías más avanzadas de tratamiento de agua, como la desionización y la destilación. La desionización es un proceso que utiliza resinas de intercambio iónico para eliminar iones del agua, mientras que la destilación es un proceso de separación basado en la diferencia de volatilidad entre el agua y las impurezas.\\

\textbf{Décadas de 1960 y 1970:} Durante este período, se desarrolló la tecnología de ósmosis inversa (OI), que utiliza membranas semipermeables para eliminar la mayoría de las impurezas disueltas en el agua, incluidos iones, compuestos orgánicos y partículas en suspensión. La OI ha sido ampliamente adoptada en la industria farmacéutica debido a su eficacia y eficiencia en la producción de agua de alta calidad.\\

\textbf{Década de 1990:} El desarrollo del proceso de desionización electroquímica, también conocido como desionización capacitiva (CDI) o electrodialización reversible (EDR), proporcionó otra opción para el tratamiento de agua en la industria farmacéutica. Estos sistemas utilizan un campo eléctrico para separar y eliminar iones del agua.\\

\textbf{Siglo XXI:} Con el desarrollo de la tecnología de desionización electrodialítica (EDI), se ha logrado combinar las ventajas de la desionización y la ósmosis inversa para producir agua de mayor pureza y a una menor tasa de rechazo. La EDI es una tecnología híbrida que utiliza membranas de intercambio iónico y un campo eléctrico para eliminar iones y otras impurezas del agua. Además, los avances en la instrumentación y el control permiten una monitorización y control en tiempo real de las variables críticas en el tratamiento de agua, lo que mejora aún más la calidad y la eficiencia del proceso.\\


La evolución de las tecnologías de tratamiento de agua en la industria farmacéutica ha sido impulsada por la creciente demanda de productos de alta calidad y la necesidad de cumplir con requisitos regulatorios cada vez más rigrosos. A medida que la industria farmacéutica continúa avanzando, es probable que surjan nuevas tecnologías y enfoques para el tratamiento y monitoreo del agua en el futuro. Algunas áreas de investigación y desarrollo incluyen:\\

\textbf{Nanotecnología:}La aplicación de nanomateriales y nanopartículas en el tratamiento de agua ofrece oportunidades para mejorar la eficiencia de los procesos existentes y desarrollar nuevos enfoques para la eliminación de impurezas. Por ejemplo, las membranas nanocompuestas y las nanopartículas funcionales pueden mejorar la selectividad y la eficiencia de las membranas de ósmosis inversa y EDI.\\

\textbf{Tratamiento biológico:}Los enfoques biológicos, como la utilización de microorganismos para la degradación de contaminantes orgánicos, pueden proporcionar alternativas sostenibles y de bajo costo a las tecnologías convencionales de tratamiento de agua.\\

\textbf{Sistemas avanzados de monitoreo y control:} Los avances en sensores, analítica en línea y tecnologías de control permiten una mejor comprensión y control del proceso de tratamiento de agua en tiempo real. Esto puede llevar a una mayor eficiencia y garantizar una calidad de agua más consistente.\\

\textbf{Integración de sistemas y automatización:} La integración de diferentes tecnologías de tratamiento de agua y la automatización de los sistemas de control pueden mejorar la eficiencia general del proceso y reducir los costos de operación y mantenimiento.\\


En resumen, la evolución histórica de las tecnologías de tratamiento de agua en la industria farmacéutica ha sido impulsada por la necesidad de garantizar la calidad y la seguridad de los productos y cumplir con requisitos regulatorios cada vez más estrictos. A medida que la industria farmacéutica sigue avanzando, es probable que surjan nuevas tecnologías y enfoques para el tratamiento y monitoreo del agua, lo que permitirá seguir mejorando la calidad y la eficiencia de los procesos.

% \subsection{Tecnologías actuales y enfoques de investigación en sistemas de tratamiento de agua para la industria farmacéutica}

\subsection{Etapas del tratamiento de agua en la industria farmacéutica}

El tratamiento de agua en la industria farmacéutica es fundamental para garantizar la calidad y seguridad de los productos finales. En este apartado, describimos en detalle las etapas principales del tratamiento de agua en la industria farmacéutica.

\begin{enumerate}
    \item \textbf{Pretratamiento:}  La etapa de pretratamiento se realiza para eliminar las impurezas más grandes y las partículas sólidas del agua.
          Esta etapa incluye procesos como la filtración, el ablandamiento y la desinfección. La filtración ayuda a eliminar partículas sólidas y
          sedimentos, mientras que el ablandamiento reduce la concentración de iones de calcio y magnesio que pueden provocar incrustaciones en las
          membranas y equipos de tratamiento posteriores. La desinfección, mediante cloración o radiación ultravioleta, elimina microorganismos,
          virus y bacterias presentes en el agua.

    \item \textbf{Tratamiento primario:}  La ósmosis inversa (RO) es el tratamiento primario más común en la industria farmacéutica. Esta tecnología
          utiliza membranas semipermeables para separar las impurezas disueltas y los iones del agua. La presión se aplica al agua para forzarla
          a través de la membrana, dejando atrás las impurezas y los iones. El resultado es un agua pura con una concentración muy baja de
          iones y contaminantes.

    \item \textbf{Tratamiento secundario:} Después del tratamiento primario, el agua se somete a un tratamiento secundario para eliminar los
          iones y contaminantes restantes. Entre los métodos más comunes de tratamiento secundario se encuentran la desionización, el
          intercambio iónico y la electrodesionización (EDI). La desionización y el intercambio iónico emplean resinas que atraen y
          retienen iones específicos, eliminándolos del agua. La EDI es una tecnología que combina intercambio iónico y electroquímica
          para eliminar iones y contaminantes del agua de manera más eficiente.

    \item \textbf{Tratamiento final:} La última etapa del tratamiento de agua en la industria farmacéutica implica procesos de esterilización y filtración.
          La esterilización garantiza la eliminación de cualquier microorganismo residual, mientras que la filtración final, que puede incluir
          filtros de membrana o filtros de profundidad, elimina partículas finas y restos de microorganismos. Este tratamiento final asegura que
          el agua cumple con los estándares de calidad requeridos en la industria farmacéutica.

\end{enumerate}





\subsection{Variantes de sistemas de purificación de agua}

En la industria farmacéutica, existen diversas variantes de sistemas de purificación de agua que se adaptan a las necesidades específicas de cada planta y a los requisitos de calidad del agua. A continuación, se presentan algunas de las variantes más comunes:

\textbf{Ósmosis inversa simple (RO):} La ósmosis inversa es una tecnología ampliamente utilizada para la purificación de agua en la industria farmacéutica. Se basa en la aplicación de presión para forzar el agua a través de una membrana semipermeable, eliminando así las impurezas disueltas y los contaminantes.

\textbf{Ósmosis inversa de doble paso (RO-RO):} Esta configuración consta de dos etapas consecutivas de ósmosis inversa. La segunda etapa de RO trata aún más el agua, eliminando impurezas adicionales y mejorando la calidad del agua. Este enfoque es especialmente útil cuando se requiere un mayor grado de purificación del agua.

\textbf{Ósmosis inversa seguida de lechos de resina de intercambio iónico (RO-IX):} Esta combinación utiliza la ósmosis inversa para eliminar la mayor parte de las impurezas disueltas, y luego pasa el agua a través de lechos de resina de intercambio iónico para eliminar los iones restantes y alcanzar una mayor pureza del agua.

\textbf{Ósmosis inversa seguida de Electrodesionización (RO-EDI):} Esta combinación es considerada una de las mejores soluciones para la industria farmacéutica. La ósmosis inversa elimina la mayor parte de las impurezas disueltas, y luego la electrodesionización (EDI) elimina los iones restantes y mejora aún más la calidad del agua. El sistema RO-EDI es altamente eficiente, confiable y requiere un mantenimiento relativamente bajo en comparación con otras configuraciones.

\textbf{Ósmosis inversa de doble paso seguida de Electrodesionización (RO-RO-EDI):} Esta configuración combina las ventajas de la ósmosis inversa de doble paso y la electrodesionización. Primero, el agua pasa a través de dos etapas de ósmosis inversa para eliminar la mayoría de las impurezas disueltas. Luego, la electrodesionización (EDI) elimina los iones restantes y mejora aún más la calidad del agua. Esta combinación proporciona una calidad de agua excepcionalmente alta, lo que la convierte en la mejor opción para aplicaciones farmacéuticas críticas.


Cabe destacar que la selección de la variante más adecuada para un sistema de purificación de agua en la industria farmacéutica dependerá de factores como la calidad del agua de entrada, los requisitos de calidad del agua de salida, las regulaciones aplicables y las consideraciones económicas.

\subsection{Innovaciones y enfoques de investigación en sistemas de tratamiento de agua}

La industria farmacéutica siempre busca mejorar la calidad y eficiencia en los sistemas de tratamiento de agua, lo que ha llevado al desarrollo de diversas innovaciones y enfoques de investigación en este campo. Algunos de estos avances incluyen:\\

\begin{itemize}

    \item \textbf{Membranas de ósmosis inversa de alto rendimiento:}  Los avances en la fabricación de membranas de ósmosis inversa han permitido el desarrollo de membranas más eficientes y selectivas. Estas membranas de alto rendimiento pueden eliminar impurezas más pequeñas y lograr una mayor pureza de agua, lo que las hace ideales para aplicaciones en la industria farmacéutica.\\

    \item \textbf{Sistemas de monitoreo y control en tiempo real:} La implementación de sensores avanzados y sistemas de control en tiempo real permite monitorear continuamente la calidad del agua y ajustar los parámetros de funcionamiento del sistema de tratamiento de agua de manera más efectiva. Esto mejora la eficiencia del proceso y garantiza que la calidad del agua se mantenga dentro de los límites establecidos por las regulaciones aplicables.\\

    \item \textbf{Tratamiento de agua sin productos químicos:} La investigación en el campo del tratamiento de agua sin productos químicos ha llevado al desarrollo de tecnologías innovadoras, como la fotocatálisis, la electrocoagulación y los sistemas de desinfección ultravioleta (UV), que eliminan la necesidad de utilizar productos químicos potencialmente dañinos en el tratamiento del agua.\\

    \item \textbf{Recuperación y reutilización del agua:} La creciente preocupación por la escasez de agua y la sostenibilidad ha llevado a la investigación en tecnologías de recuperación y reutilización del agua en la industria farmacéutica. Estas tecnologías permiten reducir la cantidad de agua fresca requerida para los procesos y minimizar la cantidad de agua residual generada, lo que reduce el impacto ambiental y los costos asociados.\\

    \item \textbf{Integración de tecnologías emergentes:} La investigación en el campo del tratamiento de agua también está explorando la integración de tecnologías emergentes, como la inteligencia artificial (IA) y el aprendizaje automático, para optimizar el funcionamiento de los sistemas de tratamiento de agua y predecir posibles problemas antes de que ocurran.\\

    \item \textbf{Tratamiento de agua a nanoescala:} La nanotecnología está siendo investigada para aplicaciones en el tratamiento de agua, como el uso de nanofiltros y nanopartículas para mejorar la eficiencia de eliminación de impurezas y la calidad del agua tratada.\\

\end{itemize}

Estas innovaciones y enfoques de investigación en sistemas de tratamiento de agua tienen el potencial de mejorar significativamente la calidad del agua, la eficiencia del proceso y la sostenibilidad en la industria farmacéutica, lo que permitirá a las plantas cumplir con los requisitos regulatorios más estrictos y garantizar la seguridad y eficacia de los productos farmacéuticos.\\


\section{Descripción del proceso actual}
La planta de AICA cuenta con un proceso integral de tratamiento y purificación de agua para abastecer
a sus instalaciones con agua de alta calidad y pureza. Este proceso es esencial para garantizar
el cumplimiento de las normativas y estándares aplicables en la industria farmacéutica y biotecnológica.
A continuación, se proporcionará una descripción detallada de las distintas etapas y componentes del
proceso actual en la planta de AICA, desde la captación del agua hasta su punto antes de la distribución y uso en las distintas áreas de producción.

Cabe destacar que en el Anexo \ref{sec:anexoA} se encuentra un diagrama
 P\&ID del sistema de ósmosis inversa en la planta de bulbos en cuestión, el cual constituye un 
 elemento clave dentro de este proceso de tratamiento y purificación de agua. 
 El P\&ID brinda una representación gráfica detallada de las distintas etapas 
 y componentes involucrados en el sistema implementado en 
 la planta de AICA.

% Sistema no tecnológico
\subsection*{Sistema Tecnológico y sus plantas de tratamiento}

El Sistema Tecnológico es el área de interés para esta investigación y se compone de dos plantas de tratamiento de agua.
La primera planta se dedica a la producción de ampolletas, mientras que la segunda planta se encarga de la producción de bulbos, esta última es en la que centra el estudio.



% Almacenamiento y bombeo del agua potable
\subsection*{Almacenamiento y bombeo del agua potable}

El Sistema de Tratamiento de Agua de Bulbos en Laboratorios AICA$^+$ se encarga de garantizar la eficiencia y calidad de los
diferentes tipos de aguas farmacéuticas, como el agua purificada y destilada, que se utilizan en la planta de producción de inyectables.
El proceso comienza con el almacenamiento del agua potable procedente del acueducto en dos cisternas con capacidades de 900 y 700 m$^3$ .
Posteriormente, el agua cruda es bombeada a través de las bombas de la estación de hidroneumáticos hacia las líneas de Servicios Generales y
al Sistema de Tratamiento de Agua, que se divide en dos partes: el Sistema No Tecnológico y el Sistema Tecnológico.


\subsection*{Dosificación de hipoclorito de sodio y filtración}

El agua proveniente de la cisterna llega al sistema de pretratamiento de aguas de Bulbo a una presión entre 4 - 5 bar. En la línea de entrada, se
dosifica hipoclorito de sodio al 3\% para desinfectar el agua y reducir la concentración de bacterias y microorganismos. El sistema de dosificación consta de un
tanque de solución de 50 L y una bomba con capacidad de 1.58 l/h, permitiendo una concentración de cloro residual cercana al 1\%.
Un contador de impulsos acoplado a la línea gobierna esta dosificación, enviando una señal a la bomba cada 100 L de agua, equivalente a 1
impulso. Posteriormente, el agua pasa por un filtro CF-60 de 50 micras, fabricado de acero inoxidable AISI 304, que cumple la función de
filtración y actúa como elemento mezclador después de la dosificación de cloro.

\subsection*{Almacenamiento y monitoreo de parámetros del agua}

Una vez filtrada, el agua sale del filtro CF-60 con un flujo que oscila entre 7-8 m$^3$/h y se almacena en el tanque de almacenamiento de agua potable,
TK-60, con capacidad de 3,000 L. Este tanque sirve como depósito de alimentación para los suavizadores. Se han instalado tomas de muestra antes y
después del filtro para monitorear el pH y el cloro residual del agua. Este monitoreo permite verificar la calidad del agua en esta etapa del proceso y
asegurar que los parámetros se encuentren dentro de los límites aceptables antes de continuar con el proceso de purificación.

% Suavizadores

\subsection*{Suavización del agua}

Los suavizadores de intercambio iónico son una parte fundamental en la planta de tratamiento de agua, ya que se encargan de eliminar la dureza del agua causada por los cationes de calcio y magnesio. Este proceso es esencial para evitar incrustaciones en las membranas de ósmosis inversa y garantizar una calidad óptima del agua tratada.

\input{tesis/estado_del_arte/descripcion_proceso/suavizacion/procesio.tex}
\input{tesis/estado_del_arte/descripcion_proceso/suavizacion/operacion.tex}
\input{tesis/estado_del_arte/descripcion_proceso/suavizacion/regeneracion.tex}
\input{tesis/estado_del_arte/descripcion_proceso/suavizacion/monitoreo.tex}
% \input{tesis/estado_del_arte/descripcion_proceso/suavizacion/componentes.tex}


\subsection*{Purificación mediante ósmosis inversa}

La purificación del agua en una planta de tratamiento es un proceso crucial para garantizar la calidad del agua que será suministrada a los usuarios finales. Uno de los métodos más eficientes y ampliamente utilizados para la purificación del agua es la ósmosis inversa (OI), que se basa en la aplicación de presión para forzar el paso del agua a través de una membrana semipermeable, reteniendo así las impurezas y contaminantes disueltos en el agua.

\input{tesis/estado_del_arte/descripcion_proceso/purificacion/descripcion_general.tex}
\input{tesis/estado_del_arte/descripcion_proceso/purificacion/dosificacion.tex}
\input{tesis/estado_del_arte/descripcion_proceso/purificacion/filtracion.tex}
\input{tesis/estado_del_arte/descripcion_proceso/purificacion/ph.tex}
\input{tesis/estado_del_arte/descripcion_proceso/purificacion/omosis1.tex}
\input{tesis/estado_del_arte/descripcion_proceso/purificacion/omosis2.tex}
% \input{tesis/estado_del_arte/descripcion_proceso/purificacion/alnacenamiento.tex}
\input{tesis/estado_del_arte/descripcion_proceso/purificacion/manejo_de_flujos.tex}
