\subsection{ Requisitos y regulaciones aplicables al agua}

La calidad del agua utilizada en la industria farmacéutica está sujeta a una serie de requisitos y regulaciones establecidos por diversas entidades y organismos a nivel nacional e internacional. Estas regulaciones aseguran que el agua cumpla con los estándares de calidad necesarios para garantizar la seguridad y eficacia de los productos farmacéuticos. Algunas de las principales regulaciones y requisitos aplicables al agua en la industria farmacéutica incluyen:

\textbf{Farmacopeas:} Las farmacopeas son documentos oficiales que contienen las especificaciones técnicas y
 requisitos de calidad para sustancias y productos farmacéuticos, incluidos los diferentes 
 tipos de agua. Entre las farmacopeas más reconocidas a nivel mundial se encuentran la 
 Farmacopea de Estados Unidos (USP), la Farmacopea Europea (EP) y la Farmacopea de Japón 
 (JP). Cada farmacopea establece parámetros específicos de calidad, como la conductividad, 
 el pH, la presencia de sustancias orgánicas, inorgánicas y microbiológicas, entre otros \cite{farm.veronicamartinezFARMACOPEAS2005}. \\

\textbf{ Buenas Prácticas de Fabricación (GMP):} Las GMP son normas que establecen los requisitos mínimos que deben cumplir
los procesos de fabricación, control de calidad y distribución de productos farmacéuticos, incluida la gestión del agua.
Estas normas son aplicables a nivel mundial y son emitidas por organismos como la Food and Drug Administration (FDA) en 
Estados Unidos, la European Medicines Agency (EMA) en Europa y la Organización Mundial de la Salud (OMS) \cite{ispeGoodManufacturingPractice}.

\textbf{ Directrices y guías técnicas:} Además de las farmacopeas y las GMP, existen directrices y guías técnicas 
emitidas por organismos internacionales y nacionales que abordan aspectos específicos relacionados con el agua en 
la industria farmacéutica. Estas directrices pueden incluir recomendaciones sobre el diseño y validación de sistemas
 de tratamiento de agua, el monitoreo de la calidad del agua y la prevención de la contaminación.

\textbf{ Normativas nacionales y locales:} Cada país puede tener sus propias normativas y requisitos legales 
aplicables al agua en la industria farmacéutica. Estas normativas pueden estar en línea con las farmacopeas y
 las GMP, pero también pueden incluir requisitos adicionales específicos para cada país o región.


El cumplimiento de estas regulaciones y requisitos garantiza la calidad y seguridad del agua utilizada en la fabricación de productos farmacéuticos y, en última instancia, protege la salud de los pacientes \cite{juanantoniodelacuerdaImportanciaAguaIndustria2021}.
