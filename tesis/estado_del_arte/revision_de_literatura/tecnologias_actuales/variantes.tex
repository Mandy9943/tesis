\subsection{Variantes de sistemas de purificación de agua}

En la industria farmacéutica, existen diversas variantes de sistemas de purificación de agua que se adaptan a las necesidades específicas de cada planta y a los requisitos de calidad del agua. A continuación, se presentan algunas de las variantes más comunes:\\

\textbf{Ósmosis inversa simple (RO):} La ósmosis inversa es una tecnología ampliamente utilizada para la purificación de agua en la industria farmacéutica. Se basa en la aplicación de presión para forzar el agua a través de una membrana semipermeable, eliminando así las impurezas disueltas y los contaminantes.\\

\textbf{Ósmosis inversa de doble paso (RO-RO):} Esta configuración consta de dos etapas consecutivas de ósmosis inversa. La segunda etapa de RO trata aún más el agua, eliminando impurezas adicionales y mejorando la calidad del agua. Este enfoque es especialmente útil cuando se requiere un mayor grado de purificación del agua.\\

\textbf{Ósmosis inversa seguida de lechos de resina de intercambio iónico (RO-IX):} Esta combinación utiliza la ósmosis inversa para eliminar la mayor parte de las impurezas disueltas, y luego pasa el agua a través de lechos de resina de intercambio iónico para eliminar los iones restantes y alcanzar una mayor pureza del agua.\\

\textbf{Ósmosis inversa seguida de Electrodesionización (RO-EDI):} Esta combinación es considerada una de las mejores soluciones para la industria farmacéutica. La ósmosis inversa elimina la mayor parte de las impurezas disueltas, y luego la electrodesionización (EDI) elimina los iones restantes y mejora aún más la calidad del agua. El sistema RO-EDI es altamente eficiente, confiable y requiere un mantenimiento relativamente bajo en comparación con otras configuraciones.\\

\textbf{Ósmosis inversa de doble paso seguida de Electrodesionización (RO-RO-EDI):} Esta configuración combina las ventajas de la ósmosis inversa de doble paso y la electrodesionización. Primero, el agua pasa a través de dos etapas de ósmosis inversa para eliminar la mayoría de las impurezas disueltas. Luego, la electrodesionización (EDI) elimina los iones restantes y mejora aún más la calidad del agua. Esta combinación proporciona una calidad de agua excepcionalmente alta, lo que la convierte en la mejor opción para aplicaciones farmacéuticas críticas.\\


Cabe destacar que la selección de la variante más adecuada para un sistema de purificación de agua en la industria farmacéutica dependerá de factores como la calidad del agua de entrada, los requisitos de calidad del agua de salida, las regulaciones aplicables y las consideraciones económicas.
