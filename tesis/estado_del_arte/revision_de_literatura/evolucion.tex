\subsection{Evolución histórica de las tecnologías de tratamiento de agua}

La historia del tratamiento de agua en la industria farmacéutica ha experimentado una evolución considerable a lo largo del tiempo. A medida que la industria ha crecido y los requisitos regulatorios han aumentado en complejidad, las tecnologías de tratamiento de agua han seguido mejorando para garantizar la calidad y la seguridad de los productos farmacéuticos.

\textbf{Pre-Siglo XX}: Antes del siglo XX, los métodos de purificación de agua eran bastante rudimentarios, 
enfocándose principalmente en la eliminación de sólidos y materia orgánica a través de procesos físicos como 
la sedimentación y la filtración a través de medios porosos como la arena. Estos procesos, aunque rudimentarios, 
establecieron la base para las técnicas modernas de tratamiento de agua \cite{higieneambientalHistoriaTratamientoAgua2018}.

\textbf{Principios del Siglo XX}: Con la introducción del uso del cloro como agente desinfectante en 1908 en Jersey City, 
Estados Unidos, las industrias empezaron a utilizar este método para garantizar la seguridad microbiológica de su agua. 
Por otro lado, la destilación, un proceso que se basa en la evaporación y condensación del agua para separarla de sus
 impurezas, también se empleaba aunque era energéticamente costoso \cite{higieneambientalHistoriaTratamientoAgua2018}.

\textbf{Mediados del Siglo XX}: A mediados del siglo XX, comenzó a ser común el uso de la filtración por membrana, 
específicamente la ósmosis inversa (RO), para la eliminación de partículas y solutos disueltos. Este proceso utiliza 
una membrana semipermeable para eliminar iones, moléculas y partículas más grandes del agua potable Además, 
la radiación ultravioleta (UV) empezó a ser utilizada como un método eficaz de esterilización del agua, 
matando o inactivando microorganismos al destruir su material genético \cite{higieneambientalHistoriaTratamientoAgua2018}.

\textbf{Finales del Siglo XX y principios del Siglo XXI}: Las técnicas de purificación de agua se volvieron más avanzadas y selectivas hacia 
finales del siglo XX y principios del XXI. La ósmosis inversa, la desionización y la electrodesionización (EDI) 
se volvieron estándares en la industria farmacéutica. La electrodesionización, en particular, es una tecnología 
que combina la desionización electroquímica y la desionización de lecho mixto para producir agua de alta pureza de 
manera eficiente y sin el uso de productos químicos peligrosos.

La evolución de las tecnologías de tratamiento de agua en la industria farmacéutica ha sido impulsada por la creciente
demanda de productos de alta calidad y la necesidad de cumplir con requisitos regulatorios cada vez más rigurosos. A
medida que la industria farmacéutica continúa avanzando, es probable que surjan nuevas tecnologías y enfoques para el
tratamiento y monitoreo del agua en el futuro. Algunas áreas de investigación y desarrollo incluyen:\\

\textbf{Nanotecnología:} La aplicación de nanomateriales y nanopartículas en el tratamiento de agua ofrece oportunidades para
mejorar la eficiencia de los procesos existentes y desarrollar nuevos enfoques para la eliminación de impurezas. Por ejemplo,
las membranas nanocompuestas y las nanopartículas funcionales pueden mejorar la selectividad y la eficiencia de las membranas de ósmosis 
inversa y EDI \cite{quxApplicationsNanotechnologyWater2013}.

\textbf{Tratamiento biológico:} Los enfoques biológicos, como la utilización de microorganismos para la degradación de contaminantes
orgánicos, pueden proporcionar alternativas sostenibles y de bajo costo a las tecnologías convencionales de tratamiento de agua \cite{b.siziriciyildizWaterWastewaterTreatment2012}.

\textbf{Sistemas avanzados de monitoreo y control:} Los avances en sensores, analítica en línea y tecnologías de control permiten
una mejor comprensión y control del proceso de tratamiento de agua en tiempo real. Esto puede llevar a una mayor eficiencia y
garantizar una calidad de agua más consistente \cite{kayayAdvancesRealtimeMonitoring2020}.


