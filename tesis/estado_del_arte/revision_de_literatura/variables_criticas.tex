\subsection{Variables críticas en la calidad del agua}

El tratamiento y monitoreo de la calidad del agua en la industria farmacéutica requieren un enfoque riguroso y sistemático para garantizar la eliminación efectiva de impurezas y el cumplimiento de los requisitos regulatorios. A continuación, se presentan algunas de las variables críticas que deben considerarse durante el tratamiento y monitoreo del agua:\\

\textbf{Conductividad eléctrica:} La conductividad eléctrica es una medida de la capacidad del agua para conducir la corriente eléctrica, y está directamente relacionada con la concentración de iones disueltos en el agua. Un mayor valor de conductividad indica una mayor concentración de impurezas inorgánicas. El monitoreo de la conductividad es fundamental para evaluar la efectividad de los procesos de purificación y para asegurar el cumplimiento de los límites establecidos por las regulaciones aplicables.\\

\textbf{Contenido de carbono orgánico total (COT):} El COT es una medida del contenido de carbono en compuestos
orgánicos disueltos en el agua. Un alto nivel de COT indica una mayor concentración de impurezas orgánicas.
El monitoreo regular del COT es esencial para garantizar que el agua cumpla con los requisitos de calidad y para evaluar la eficacia de los procesos de purificación en la eliminación de compuestos orgánicos.\\

\textbf{Conteo microbiano y endotoxinas:} El monitoreo del recuento microbiano y las endotoxinas es fundamental
para controlar la calidad microbiológica del agua y garantizar la seguridad de los productos farmacéuticos.
Los métodos de análisis microbiológico incluyen el recuento en placa, el método de filtración por membrana
y las técnicas de bioluminiscencia. Las endotoxinas, sustancias tóxicas liberadas por bacterias Gram-negativas,
se miden mediante el ensayo de lisado de amebocitos de Limulus (LAL).\\

\textbf{pH:} El pH es una medida de la acidez o alcalinidad del agua y puede afectar la solubilidad,
la estabilidad y la reactividad de los ingredientes activos y excipientes en los productos farmacéuticos.
El control del pH es esencial para mantener un ambiente adecuado en los sistemas de tratamiento de agua y
garantizar la calidad del agua producida.\\

\textbf{Turbidez:} La turbidez es una medida de la cantidad de partículas en suspensión en el agua, incluidas partículas inorgánicas, orgánicas y microbiológicas. Un nivel elevado de turbidez puede afectar la efectividad de los procesos de purificación y el rendimiento de los equipos. La turbidez se mide utilizando un turbidímetro y se expresa en unidades de turbidez nefelométrica (NTU).\\

El monitoreo y control de estas variables críticas durante el tratamiento y purificación del agua son fundamentales para garantizar la calidad, seguridad y eficacia de los productos farmacéuticos y cumplir con los requisitos regulatorios aplicables.\\
