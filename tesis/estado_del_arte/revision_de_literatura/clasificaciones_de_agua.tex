\subsection{Tipos y clasificaciones del agua}

El agua es un componente fundamental en la industria farmacéutica, y su calidad y
pureza son aspectos críticos para garantizar la seguridad y eficacia de los productos.
Dependiendo de su uso y aplicación, existen diferentes tipos y clasificaciones de agua en la industria farmacéutica.
A continuación, se presentan las categorías más comunes \cite{setaphtTratamientosAguaPara}:

\textbf{Agua purificada (PW):} Es el tipo básico de agua utilizada en la industria farmacéutica y se obtiene a través de procesos como ósmosis inversa, destilación, intercambio iónico o filtración. La calidad del agua purificada es menor que la del agua para inyección (WFI), pero es adecuada para la fabricación de productos no parenterales y para su uso en procesos de limpieza.

\textbf{Agua para inyección (WFI):} Es un tipo de agua de alta pureza que se utiliza en la fabricación de productos parenterales, es decir, aquellos que se administran por vías como intravenosa, intramuscular o subcutánea. La calidad del WFI es superior a la del agua purificada, y se obtiene mediante procesos de destilación, ósmosis inversa o por una combinación de ambos métodos.

\textbf{Agua altamente purificada (HPW):} Este tipo de agua tiene una calidad intermedia entre el agua purificada y el WFI. Se utiliza en ciertas aplicaciones farmacéuticas donde se requiere un nivel de pureza más elevado que el del agua purificada, pero no se necesita llegar al grado de pureza del WFI.

\textbf{Agua estéril:} Es agua que ha sido sometida a un proceso de esterilización, como la filtración estéril o la autoclave, para eliminar cualquier microorganismo presente. El agua estéril se utiliza en aplicaciones específicas, como en la fabricación de productos estériles o en procesos de limpieza y desinfección que requieren la eliminación de microorganismos.

Cabe destacar que las regulaciones y normativas, como las establecidas por la Farmacopea de Estados Unidos (USP), la Farmacopea Europea (EP) y la Organización Mundial de la Salud (OMS), definen los requisitos de calidad y las especificaciones para cada tipo de agua en la industria farmacéutica. Estas especificaciones incluyen parámetros como la conductividad, el pH, la presencia de sustancias orgánicas, inorgánicas y microbiológicas, entre otros.
