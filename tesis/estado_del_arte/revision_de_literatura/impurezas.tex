\subsection{Impurezas presentes en el agua y su impacto en los productos farmacéuticos}

El agua utilizada en la industria farmacéutica puede contener diversas impurezas, las cuales pueden afectar la calidad, seguridad y eficacia de los productos finales. Estas impurezas pueden clasificarse en tres categorías principales: impurezas inorgánicas, impurezas orgánicas y contaminantes microbiológicos.\\

\textbf{Impurezas inorgánicas: }Incluyen iones metálicos y no metálicos, como calcio, magnesio, sodio, cloruros, sulfatos y silicatos. Estas impurezas pueden afectar la calidad de los productos farmacéuticos al causar cambios en la solubilidad, la estabilidad y la eficacia de los ingredientes activos, así como en la formación de precipitados y la corrosión de equipos y recipientes. Además, algunos iones metálicos, como el hierro, el cobre y el cromo, pueden ser tóxicos y afectar la seguridad de los productos.\\
\textbf{Impurezas orgánicas: }Son compuestos de origen natural o sintético, como ácidos húmicos y fúlvicos, pesticidas, disolventes y productos químicos de desinfección. Las impurezas orgánicas pueden reaccionar con los ingredientes activos y otros excipientes, lo que puede alterar la estabilidad, la eficacia y la liberación de los fármacos. Además, algunos compuestos orgánicos pueden ser tóxicos y afectar la seguridad de los productos farmacéuticos.\\
\textbf{Contaminantes microbiológicos:} Incluyen bacterias, hongos, levaduras, virus y protozoos. La presencia de microorganismos en el agua puede causar la contaminación de los productos farmacéuticos, lo que puede llevar a infecciones y reacciones adversas en los pacientes. Además, algunos microorganismos pueden producir sustancias tóxicas, como endotoxinas y micotoxinas, que pueden afectar la seguridad y eficacia de los productos.\\

El tratamiento adecuado del agua es esencial para eliminar o reducir estas impurezas a niveles aceptables, de acuerdo con las regulaciones y requisitos aplicables en la industria farmacéutica. Un control riguroso de la calidad del agua, así como el uso de tecnologías de purificación adecuadas, como la ósmosis inversa, la desionización y la electrodesionización (EDI), son fundamentales para garantizar la calidad y seguridad de los productos farmacéuticos.\\
