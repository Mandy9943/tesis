
% Sistema no tecnológico
\subsection*{Sistema Tecnológico y sus plantas de tratamiento}

El Sistema Tecnológico es el área de interés para esta investigación y se compone de dos plantas de tratamiento de agua.
La primera planta se dedica a la producción de ampolletas, mientras que la segunda planta se encarga de la producción de bulbos, esta última es en la que centra el estudio.



% Almacenamiento y bombeo del agua potable
\subsection*{Almacenamiento y bombeo del agua potable}

El Sistema de Tratamiento de Agua de Bulbos en Laboratorios AICA$^+$ se encarga de garantizar la eficiencia y calidad de los
diferentes tipos de aguas farmacéuticas, como el agua purificada y destilada, que se utilizan en la planta de producción de inyectables.
El proceso comienza con el almacenamiento del agua potable procedente del acueducto en dos cisternas con capacidades de 900 y 700 m$^3$ .
Posteriormente, el agua cruda es bombeada a través de las bombas de la estación de hidroneumáticos hacia las líneas de Servicios Generales y
al Sistema de Tratamiento de Agua, que se divide en dos partes: el Sistema No Tecnológico y el Sistema Tecnológico.


\subsection*{Dosificación de hipoclorito de sodio y filtración}

El agua proveniente de la cisterna llega al sistema de pretratamiento de aguas de Bulbo a una presión entre 4 - 5 bar. En la línea de entrada, se
dosifica hipoclorito de sodio al 3\% para desinfectar el agua y reducir la concentración de bacterias y microorganismos. El sistema de dosificación consta de un
tanque de solución de 50 L y una bomba con capacidad de 1.58 l/h, permitiendo una concentración de cloro residual cercana al 1\%.
Un contador de impulsos acoplado a la línea gobierna esta dosificación, enviando una señal a la bomba cada 100 L de agua, equivalente a 1
impulso. Posteriormente, el agua pasa por un filtro CF-60 de 50 micras, fabricado de acero inoxidable AISI 304, que cumple la función de
filtración y actúa como elemento mezclador después de la dosificación de cloro.

\subsection*{Almacenamiento y monitoreo de parámetros del agua}

Una vez filtrada, el agua sale del filtro CF-60 con un flujo que oscila entre 7-8 m$^3$/h y se almacena en el tanque de almacenamiento de agua potable,
TK-60, con capacidad de 3,000 L. Este tanque sirve como depósito de alimentación para los suavizadores. Se han instalado tomas de muestra antes y
después del filtro para monitorear el pH y el cloro residual del agua. Este monitoreo permite verificar la calidad del agua en esta etapa del proceso y
asegurar que los parámetros se encuentren dentro de los límites aceptables antes de continuar con el proceso de purificación.

% Suavizadores

\subsection{Suavización del agua}

Los suavizadores de intercambio iónico son una parte fundamental en la planta de tratamiento de agua, ya que se encargan de eliminar la dureza del agua causada por los cationes de calcio y magnesio. Este proceso es esencial para evitar incrustaciones en las membranas de ósmosis inversa y garantizar una calidad óptima del agua tratada.\\


\subsubsection{Proceso de suavización y disposición de los suavizadores}

El proceso de suavización comienza cuando el agua es trasegada desde el tanque TK-60 hasta el módulo de suavizadores de intercambio iónico utilizando la bomba P-60. Antes de llegar a los suavizadores, el agua pasa a través del intercambiador de placas E60-1, que disminuye la temperatura del agua hasta valores entre 18 y 20°C, mejorando así la eficiencia del proceso de purificación.\\


En la línea de entrada y salida del intercambiador, se miden la presión y la temperatura, respectivamente. Además, se cuenta con una válvula reguladora que ajusta el flujo de agua de enfriamiento que entra al intercambiador. Luego, el agua sale del intercambiador y entra a los suavizadores a una presión aproximada de 4 bar a través de los cabezales de distribución.\\


En este proceso, los suavizadores A64-A y A64-B están dispuestos en serie. El agua que sale del suavizador A64-A entra al suavizador A64-B, que se encarga de rectificar finalmente la calidad del agua suavizada. Ambos suavizadores tienen como objetivo eliminar la dureza del agua, intercambiando los iones de calcio y magnesio por iones de sodio de la resina catiónica fuerte.\\


\subsubsection{Operaciones de producción y regeneración de los suavizadores}

Los suavizadores de intercambio iónico funcionan mediante dos operaciones principales: producción y regeneración. Estas operaciones son controladas por el Aquatimer instalado en cada suavizador.

Durante la producción, se lleva a cabo la reacción de intercambio iónico en la resina. Con el tiempo, la capacidad de intercambio iónico de la resina disminuye gradualmente y los sólidos disueltos en el agua se acumulan en ella. Cuando la resina se agota, es necesario regenerarla con una solución de cloruro de sodio al 14\% en peso.

\textbf{Proceso de regeneración de los suavizadores:}\\
La regeneración de los suavizadores consta de cuatro etapas:
\begin{enumerate}

    \item \textbf{ Contralavado:} El lavado a contraflujo tiene como objetivo remover los sólidos depositados en la resina, incluyendo las partículas de resina más pequeñas, levantando y expandiendo ligeramente la cama de resina.

    \item \textbf{ Regeneración:} Durante esta etapa, se pasa salmuera a través de la resina a una velocidad de flujo lenta, lo que aumenta el contacto entre la salmuera y la resina, favoreciendo la regeneración de la misma. La reacción de regeneración implica la liberación de los iones de calcio y magnesio, que son reemplazados por iones de sodio.

    \item \textbf{ Enjuague lento:} En este paso, se dispersa la solución de regenerante a través de todo el volumen de resina a una velocidad de flujo requerida, garantizando un contacto adecuado de la salmuera con el fondo de la cama de resina.

    \item \textbf{ Enjuague rápido:} Después de completar el desplazamiento de la salmuera a través de toda la cama de resina, este último enjuague remueve la salmuera que ha quedado remanente o en exceso en la misma.

    
\end{enumerate}

Una vez finalizado el proceso de regeneración, los suavizadores están listos para volver a funcionar en la operación de producción, garantizando la eliminación efectiva de la dureza del agua.







\subsubsection{Monitoreo de la calidad del agua suavizada}

El monitoreo de la calidad del agua suavizada es esencial para garantizar la eficiencia del proceso y la protección de las membranas de ósmosis inversa. A la salida de cada suavizador, hay un punto de toma de muestra y en la línea general de salida del agua suave, se encuentra instalado un medidor de dureza en línea (DOROMAT PROFESIONAL). Este medidor permite asegurar que la dureza del agua no supere el límite máximo establecido de 5 mg/l, evitando así la formación de incrustaciones en las membranas de ósmosis inversa.

% \subsubsection{Componentes y especificaciones de los suavizadores}

Los suavizadores de intercambio iónico están compuestos por varios componentes clave y cuentan con especificaciones particulares que son importantes para su correcto funcionamiento. Estos incluyen:\\

\textbf{Volumen de resina}: Cada suavizador tiene una capacidad de 100 litros de resina.\\

\textbf{Resina}: La resina utilizada en los suavizadores es de tipo catiónica fuerte, marca Purolite C 100. Esta resina permite la eliminación de los iones de calcio y magnesio presentes en el agua, intercambiándolos por iones de sodio.\\

\textbf{Tanque TK-64 (cuba de salmuera)}: Este tanque se utiliza para almacenar y suministrar la solución de cloruro de sodio al 14\% en peso, necesaria para la regeneración de la resina.\\

Además, el sistema cuenta con elementos como el intercambiador de placas E60-1, que disminuye la temperatura del agua antes de ingresar a los suavizadores, la bomba P-60 para trasladar el agua desde el tanque TK-60 hasta el módulo de suavizadores, y válvulas reguladoras para controlar el flujo de agua de enfriamiento que entra al intercambiador.



\subsection{Purificación mediante ósmosis inversa}

La purificación del agua en una planta de tratamiento es un proceso crucial para garantizar la calidad del agua que será suministrada a los usuarios finales. Uno de los métodos más eficientes y ampliamente utilizados para la purificación del agua es la ósmosis inversa (OI), que se basa en la aplicación de presión para forzar el paso del agua a través de una membrana semipermeable, reteniendo así las impurezas y contaminantes disueltos en el agua.

\subsubsection{Descripción general de las etapas de ósmosis inversa}

El proceso de ósmosis inversa en la planta de tratamiento de agua en estudio se compone de dos etapas o pasos de flujo. La primera etapa consta de tres porta-membranas, cada una con tres tubos colectores de 8 pulgadas de diámetro y 40 pulgadas de longitud, y membranas dispuestas en espiral en su interior. La segunda etapa, por otro lado, tiene dos porta-membranas, uno de los cuales contiene solo dos tubos colectores con membrana, mientras que el tercer tubo colector tiene una simulación de membrana. Esta configuración se estableció para lograr los parámetros de producción de agua purificada de diseño en la ósmosis inversa.

\textbf{Adición de metabisulfito de sodio:}\\
Antes de ingresar al proceso de ósmosis inversa, el agua suavizada, con un pH entre 5 y 7 y una presión entre 2 y 4 bar, debe someterse a un pretratamiento.
Este pretratamiento incluye la dosificación de metabisulfito de sodio (Na$_2$S$_2$O$_5$) mediante un conjunto de bomba dosificadora y tanque de solución.
La adición de metabisulfito de sodio es esencial para eliminar el cloro libre residual presente en el agua,
ya que este puede dañar químicamente las membranas de la ósmosis inversa.


\subsubsection{Filtración y control de calidad antes de la ósmosis inversa}

Después del pretratamiento, el agua pasa por un filtro de cartuchos de 10 micrómetros. En la entrada y salida del filtro, se instalan manómetros para monitorear la diferencia de presión y, por lo tanto, determinar el grado de ensuciamiento de los cartuchos del filtro.\\ 

A continuación, se toma una muestra del agua filtrada en el punto de muestreo del analizador de REDOX en línea, que proporciona una medida de la concentración de cloro en el agua, con un límite máximo de 400 mV. El agua filtrada y tratada se dirige al tanque de alimentación de la ósmosis inversa (TK 50-A) con una capacidad de 500 litros. \\

En esta etapa, es fundamental garantizar la calidad del agua antes de que ingrese al proceso de ósmosis inversa para evitar problemas en las membranas y garantizar una purificación eficiente. \\

\subsubsection{Ajuste del pH y eliminación del CO$_2$ disuelto }

Una vez almacenada en el tanque de alimentación de la ósmosis inversa (TK 50-A), el agua suavizada es succionada por la
bomba P50-2A para aumentar su presión hasta valores cercanos a 5 bar. Durante este proceso, se dosifica hidróxido de sodio
(NaOH) utilizando un conjunto de tanque y bomba dosificadora. La adición de NaOH tiene como objetivo eliminar el CO$_2$ disuelto en
el agua, ya que aporta conductividad, y ajustar el pH del agua de alimentación a la ósmosis inversa en un rango entre 8 y 10.
\subsubsection{Primera etapa de ósmosis inversa}

El agua tratada pasa por un filtro de cartucho de 5 micrómetros (CF50A) y luego es impulsada por la bomba P50-A hacia la primera etapa de ósmosis inversa a una presión entre 9 y 13 bar y una temperatura entre 15 y 25°C. En esta etapa, las membranas retienen sales, sustancias orgánicas y microorganismos presentes en el agua suavizada. El flujo de agua producto de la primera etapa es aproximadamente 4000 l/h, con una conductividad menor a 10 µS/cm.
\textbf{Segunda etapa de ósmosis inversa:}\\
El agua purificada de la primera etapa se bombea hacia la segunda etapa mediante la bomba P50-B, a una presión de 12 bar.
El objetivo de la segunda etapa es realizar un pulido extra del agua, tanto en términos físico-químicos como microbiológicos.
El producto de la segunda etapa, con un flujo de 3000 l/h, se almacena en el tanque TK-70, con capacidad para 6000 litros de agua purificada.
Se toman muestras de agua pura para analizar la conductividad, que debe ser menor a 1.3 µS/cm, así como otros parámetros físico-químicos y microbiológicos,
como el carbono orgánico total y la presencia de microorganismos patógenos y bacterias.


% \subsubsection{Almacenamiento y monitoreo del agua purificada}

Una vez completadas las dos etapas de ósmosis inversa, el agua purificada se almacena en el tanque TK-70, con una capacidad de 6000 litros. Durante el almacenamiento, se realizan controles de calidad para asegurar que el agua cumpla con los parámetros establecidos. Se analiza la conductividad, que debe ser inferior a 1.3 µS/cm, así como otros parámetros físico-químicos y microbiológicos, como el carbono orgánico total y la presencia de microorganismos patógenos y bacterias. Este monitoreo permite garantizar que el agua purificada sea segura para su uso posterior.
\subsubsection{Manejo del flujo de rechazo y recirculación}

Durante el proceso de ósmosis inversa, se generan flujos de rechazo que contienen las sales, sustancias orgánicas y microorganismos que han sido retenidos por las membranas. En la primera etapa de ósmosis inversa, el flujo de rechazo varía entre 3000 y 1000 l/h, mientras que en la segunda etapa, el flujo de rechazo es de aproximadamente 1000 l/h.\\

Actualmente, el rechazo proveniente de la segunda etapa se recircula al tanque de agua suave TK 50, permitiendo que el agua sea tratada nuevamente en el proceso de ósmosis inversa. A pesar de que esto aprovecha una parte del agua y reduce el volumen de agua desechada, se ha identificado que puede haber una pérdida de agua de calidad en este proceso.\\

Por otro lado, el rechazo de la primera etapa se envía al drenaje debido a su alta concentración de sales y sustancias indeseables. Este flujo de rechazo no se recircula, ya que podría afectar negativamente la calidad del agua suave y la eficiencia del proceso de ósmosis inversa.\\