
\subsubsection{Proceso de suavización y disposición de los suavizadores}

El proceso de suavización comienza cuando el agua es trasegada desde el tanque TK-60 hasta el módulo de suavizadores de intercambio iónico utilizando la bomba P-60. Antes de llegar a los suavizadores, el agua pasa a través del intercambiador de placas E60-1, que disminuye la temperatura del agua hasta valores entre 18 y 20°C, mejorando así la eficiencia del proceso de purificación.


En la línea de entrada y salida del intercambiador, se miden la presión y la temperatura, respectivamente. Además, se cuenta con una válvula reguladora que ajusta el flujo de agua de enfriamiento que entra al intercambiador. Luego, el agua sale del intercambiador y entra a los suavizadores a una presión aproximada de 4 bar a través de los cabezales de distribución.


En este proceso, los suavizadores A64-A y A64-B están dispuestos en serie. El agua que sale del suavizador A64-A entra al suavizador A64-B, que se encarga de rectificar finalmente la calidad del agua suavizada. Ambos suavizadores tienen como objetivo eliminar la dureza del agua, intercambiando los iones de calcio y magnesio por iones de sodio de la resina catiónica fuerte.
