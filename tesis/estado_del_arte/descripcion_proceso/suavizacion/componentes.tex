\subsubsection{Componentes y especificaciones de los suavizadores}

Los suavizadores de intercambio iónico están compuestos por varios componentes clave y cuentan con especificaciones particulares que son importantes para su correcto funcionamiento. Estos incluyen:

\textbf{Volumen de resina}: Cada suavizador tiene una capacidad de 100 litros de resina.

\textbf{Resina}: La resina utilizada en los suavizadores es de tipo catiónica fuerte, marca Purolite C 100. Esta resina permite la eliminación de los iones de calcio y magnesio presentes en el agua, intercambiándolos por iones de sodio.

\textbf{Tanque TK-64 (cuba de salmuera)}: Este tanque se utiliza para almacenar y suministrar la solución de cloruro de sodio al 14\% en peso, necesaria para la regeneración de la resina.

Además, el sistema cuenta con elementos como el intercambiador de placas E60-1, que disminuye la temperatura del agua antes de ingresar a los suavizadores, la bomba P-60 para trasladar el agua desde el tanque TK-60 hasta el módulo de suavizadores, y válvulas reguladoras para controlar el flujo de agua de enfriamiento que entra al intercambiador.
