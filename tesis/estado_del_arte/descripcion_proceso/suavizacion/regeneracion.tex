\textbf{Proceso de regeneración de los suavizadores:}\\
La regeneración de los suavizadores consta de cuatro etapas:
\begin{enumerate}

    \item \textbf{ Contralavado:} El lavado a contraflujo tiene como objetivo remover los sólidos depositados en la resina, incluyendo las partículas de resina más pequeñas, levantando y expandiendo ligeramente la cama de resina.

    \item \textbf{ Regeneración:} Durante esta etapa, se pasa salmuera a través de la resina a una velocidad de flujo lenta, lo que aumenta el contacto entre la salmuera y la resina, favoreciendo la regeneración de la misma. La reacción de regeneración implica la liberación de los iones de calcio y magnesio, que son reemplazados por iones de sodio.

    \item \textbf{ Enjuague lento:} En este paso, se dispersa la solución de regenerante a través de todo el volumen de resina a una velocidad de flujo requerida, garantizando un contacto adecuado de la salmuera con el fondo de la cama de resina.

    \item \textbf{ Enjuague rápido:} Después de completar el desplazamiento de la salmuera a través de toda la cama de resina, este último enjuague remueve la salmuera que ha quedado remanente o en exceso en la misma.

    
\end{enumerate}

Una vez finalizado el proceso de regeneración, los suavizadores están listos para volver a funcionar en la operación de producción, garantizando la eliminación efectiva de la dureza del agua.






