\subsubsection{Adición de metabisulfito de sodio}

Antes de ingresar al proceso de ósmosis inversa, el agua suavizada, con un pH entre 5 y 7 y una presión entre 2 y 4 bar, debe someterse a un pretratamiento.
Este pretratamiento incluye la dosificación de metabisulfito de sodio (Na$_2$S$_2$O$_5$) mediante un conjunto de bomba dosificadora y tanque de solución.
La adición de metabisulfito de sodio es esencial para eliminar el cloro libre residual presente en el agua,
ya que este puede dañar químicamente las membranas de la ósmosis inversa.