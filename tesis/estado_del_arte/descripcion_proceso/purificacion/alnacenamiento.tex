\subsubsection{Almacenamiento y monitoreo del agua purificada}

Una vez completadas las dos etapas de ósmosis inversa, el agua purificada se almacena en el tanque TK-70, con una capacidad de 6000 litros. Durante el almacenamiento, se realizan controles de calidad para asegurar que el agua cumpla con los parámetros establecidos. Se analiza la conductividad, que debe ser inferior a 1.3 µS/cm, así como otros parámetros físico-químicos y microbiológicos, como el carbono orgánico total y la presencia de microorganismos patógenos y bacterias. Este monitoreo permite garantizar que el agua purificada sea segura para su uso posterior.