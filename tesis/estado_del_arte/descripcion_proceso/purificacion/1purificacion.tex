\subsection{Purificación mediante ósmosis inversa}

La purificación del agua en una planta de tratamiento es un proceso crucial para garantizar la calidad del agua que será suministrada a los usuarios finales. Uno de los métodos más eficientes y ampliamente utilizados para la purificación del agua es la ósmosis inversa (OI), que se basa en la aplicación de presión para forzar el paso del agua a través de una membrana semipermeable, reteniendo así las impurezas y contaminantes disueltos en el agua.\\

\subsubsection{Descripción general de las etapas de ósmosis inversa}

El proceso de ósmosis inversa en la planta de tratamiento de agua en estudio se compone de dos etapas o pasos de flujo. La primera etapa consta de tres porta-membranas, cada una con tres tubos colectores de 8 pulgadas de diámetro y 40 pulgadas de longitud, y membranas dispuestas en espiral en su interior. La segunda etapa, por otro lado, tiene dos porta-membranas, uno de los cuales contiene solo dos tubos colectores con membrana, mientras que el tercer tubo colector tiene una simulación de membrana. Esta configuración se estableció para lograr los parámetros de producción de agua purificada de diseño en la ósmosis inversa.

\textbf{Adición de metabisulfito de sodio:}\\
Antes de ingresar al proceso de ósmosis inversa, el agua suavizada, con un pH entre 5 y 7 y una presión entre 2 y 4 bar, debe someterse a un pretratamiento.
Este pretratamiento incluye la dosificación de metabisulfito de sodio (Na$_2$S$_2$O$_5$) mediante un conjunto de bomba dosificadora y tanque de solución.
La adición de metabisulfito de sodio es esencial para eliminar el cloro libre residual presente en el agua,
ya que este puede dañar químicamente las membranas de la ósmosis inversa.


\subsubsection{Filtración y control de calidad antes de la ósmosis inversa}

Después del pretratamiento, el agua pasa por un filtro de cartuchos de 10 micrómetros. En la entrada y salida del filtro, se instalan manómetros para monitorear la diferencia de presión y, por lo tanto, determinar el grado de ensuciamiento de los cartuchos del filtro.\\ 

A continuación, se toma una muestra del agua filtrada en el punto de muestreo del analizador de REDOX en línea, que proporciona una medida de la concentración de cloro en el agua, con un límite máximo de 400 mV. El agua filtrada y tratada se dirige al tanque de alimentación de la ósmosis inversa (TK 50-A) con una capacidad de 500 litros. \\

En esta etapa, es fundamental garantizar la calidad del agua antes de que ingrese al proceso de ósmosis inversa para evitar problemas en las membranas y garantizar una purificación eficiente. \\

\subsubsection{Ajuste del pH y eliminación del CO$_2$ disuelto }

Una vez almacenada en el tanque de alimentación de la ósmosis inversa (TK 50-A), el agua suavizada es succionada por la
bomba P50-2A para aumentar su presión hasta valores cercanos a 5 bar. Durante este proceso, se dosifica hidróxido de sodio
(NaOH) utilizando un conjunto de tanque y bomba dosificadora. La adición de NaOH tiene como objetivo eliminar el CO$_2$ disuelto en
el agua, ya que aporta conductividad, y ajustar el pH del agua de alimentación a la ósmosis inversa en un rango entre 8 y 10.
\subsubsection{Primera etapa de ósmosis inversa}

El agua tratada pasa por un filtro de cartucho de 5 micrómetros (CF50A) y luego es impulsada por la bomba P50-A hacia la primera etapa de ósmosis inversa a una presión entre 9 y 13 bar y una temperatura entre 15 y 25°C. En esta etapa, las membranas retienen sales, sustancias orgánicas y microorganismos presentes en el agua suavizada. El flujo de agua producto de la primera etapa es aproximadamente 4000 l/h, con una conductividad menor a 10 µS/cm.
\textbf{Segunda etapa de ósmosis inversa:}\\
El agua purificada de la primera etapa se bombea hacia la segunda etapa mediante la bomba P50-B, a una presión de 12 bar.
El objetivo de la segunda etapa es realizar un pulido extra del agua, tanto en términos físico-químicos como microbiológicos.
El producto de la segunda etapa, con un flujo de 3000 l/h, se almacena en el tanque TK-70, con capacidad para 6000 litros de agua purificada.
Se toman muestras de agua pura para analizar la conductividad, que debe ser menor a 1.3 µS/cm, así como otros parámetros físico-químicos y microbiológicos,
como el carbono orgánico total y la presencia de microorganismos patógenos y bacterias.


\subsubsection{Almacenamiento y monitoreo del agua purificada}

Una vez completadas las dos etapas de ósmosis inversa, el agua purificada se almacena en el tanque TK-70, con una capacidad de 6000 litros. Durante el almacenamiento, se realizan controles de calidad para asegurar que el agua cumpla con los parámetros establecidos. Se analiza la conductividad, que debe ser inferior a 1.3 µS/cm, así como otros parámetros físico-químicos y microbiológicos, como el carbono orgánico total y la presencia de microorganismos patógenos y bacterias. Este monitoreo permite garantizar que el agua purificada sea segura para su uso posterior.
\subsubsection{Manejo del flujo de rechazo y recirculación}

Durante el proceso de ósmosis inversa, se generan flujos de rechazo que contienen las sales, sustancias orgánicas y microorganismos que han sido retenidos por las membranas. En la primera etapa de ósmosis inversa, el flujo de rechazo varía entre 3000 y 1000 l/h, mientras que en la segunda etapa, el flujo de rechazo es de aproximadamente 1000 l/h.\\

Actualmente, el rechazo proveniente de la segunda etapa se recircula al tanque de agua suave TK 50, permitiendo que el agua sea tratada nuevamente en el proceso de ósmosis inversa. A pesar de que esto aprovecha una parte del agua y reduce el volumen de agua desechada, se ha identificado que puede haber una pérdida de agua de calidad en este proceso.\\

Por otro lado, el rechazo de la primera etapa se envía al drenaje debido a su alta concentración de sales y sustancias indeseables. Este flujo de rechazo no se recircula, ya que podría afectar negativamente la calidad del agua suave y la eficiencia del proceso de ósmosis inversa.\\