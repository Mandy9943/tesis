\subsubsection{Descripción general de las etapas de ósmosis inversa}

El proceso de ósmosis inversa en la planta de tratamiento de agua en estudio se compone de dos etapas o pasos de flujo. La primera etapa consta de tres porta-membranas, cada una con tres tubos colectores de 8 pulgadas de diámetro y 40 pulgadas de longitud, y membranas dispuestas en espiral en su interior. La segunda etapa, por otro lado, tiene dos porta-membranas, uno de los cuales contiene solo dos tubos colectores con membrana, mientras que el tercer tubo colector tiene una simulación de membrana. Esta configuración se estableció para lograr los parámetros de producción de agua purificada de diseño en la ósmosis inversa.\\
