\subsubsection{Manejo del flujo de rechazo y recirculación}

Durante el proceso de ósmosis inversa, se generan flujos de rechazo que contienen las sales, sustancias orgánicas y microorganismos que han sido retenidos por las membranas. En la primera etapa de ósmosis inversa, el flujo de rechazo varía entre 3000 y 1000 l/h, mientras que en la segunda etapa, el flujo de rechazo es de aproximadamente 1000 l/h.\\

Actualmente, el rechazo proveniente de la segunda etapa se recircula al tanque de agua suave TK 50, permitiendo que el agua sea tratada nuevamente en el proceso de ósmosis inversa. A pesar de que esto aprovecha una parte del agua y reduce el volumen de agua desechada, se ha identificado que puede haber una pérdida de agua de calidad en este proceso.\\

Por otro lado, el rechazo de la primera etapa se envía al drenaje debido a su alta concentración de sales y sustancias indeseables. Este flujo de rechazo no se recircula, ya que podría afectar negativamente la calidad del agua suave y la eficiencia del proceso de ósmosis inversa.\\