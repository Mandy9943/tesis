\subsubsection{Filtración y control de calidad antes de la ósmosis inversa}

Después de la adición de metabisulfito de sodio, el agua pasa por un filtro de cartuchos de 10 micrómetros. En la entrada y salida del filtro,
se instalan manómetros para monitorear la diferencia de presión y, por lo tanto, determinar el grado de ensuciamiento de los cartuchos del filtro.

A continuación, se toma una muestra del agua filtrada en el punto de muestreo del analizador de REDOX en línea, que proporciona una medida de
la concentración de cloro en el agua, con un límite máximo de 400 mV. El agua filtrada y tratada se dirige al tanque de alimentación de la ósmosis
inversa (TK 50-A) con una capacidad de 500 litros. \\

En esta etapa, es fundamental garantizar la calidad del agua antes de que ingrese al proceso de ósmosis inversa para evitar problemas en las
membranas y garantizar una purificación eficiente. \\
