\textbf{Segunda etapa de ósmosis inversa:}\\
El agua purificada de la primera etapa se bombea hacia la segunda etapa mediante la bomba P50-B, a una presión de 12 bar.
El objetivo de la segunda etapa es realizar un pulido extra del agua, tanto en términos físico-químicos como microbiológicos.
El producto de la segunda etapa, con un flujo de 3000 l/h, se almacena en el tanque TK-70, con capacidad para 6000 litros de agua purificada.
Se toman muestras de agua pura para analizar la conductividad, que debe ser menor a 1.3 µS/cm, así como otros parámetros físico-químicos y microbiológicos,
como el carbono orgánico total y la presencia de microorganismos patógenos y bacterias.

