\subsubsection{Ajuste del pH y eliminación del CO$_2$ disuelto }

Una vez almacenada en el tanque de alimentación de la ósmosis inversa (TK 50-A), el agua suavizada es succionada por la
bomba P50-2A para aumentar su presión hasta valores cercanos a 5 bar. Durante este proceso, se dosifica hidróxido de sodio
(NaOH) utilizando un conjunto de tanque y bomba dosificadora. La adición de NaOH tiene como objetivo eliminar el CO$_2$ disuelto en
el agua, ya que aporta conductividad, y ajustar el pH del agua de alimentación a la ósmosis inversa en un rango entre 8 y 10.\\