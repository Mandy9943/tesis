\chapter*{Introducción}
\addcontentsline{toc}{chapter}{Introducción}
La calidad del agua en la industria farmacéutica es de suma importancia,
ya que influye directamente en la calidad y seguridad de los productos
farmacéuticos, como los inyectables. La presente tesis se enfoca en la
implementación de un Electrodesionizador (EDI) en una planta
de tratamiento de agua de la industria farmacéutica, con el objetivo de
mejorar la calidad del agua purificada (PW) y el agua para inyección.
A continuación, se presenta el contexto y la justificación de este
proyecto, así como el problema a resolver, la hipótesis, el objeto
de estudio, el campo de acción, los objetivos generales y específicos,
y la estructura por capítulos.\\

% secciones
\section{Contexto y justificación}
La industria farmacéutica desempeña un papel fundamental en la promoción y protección de la salud pública, ya que proporciona medicamentos y productos farmacéuticos que salvan vidas y mejoran la calidad de vida de millones de personas en todo el mundo. La producción de estos productos requiere la utilización de agua de alta calidad, especialmente en la fabricación de soluciones inyectables y otros medicamentos críticos. La calidad del agua utilizada en los procesos de fabricación de medicamentos es un factor esencial para garantizar la seguridad, eficacia y estabilidad de los productos finales.\\

La planta de tratamiento de agua de la empresa AICA, dedicada a la industria farmacéutica, actualmente utiliza un sistema de ósmosis inversa (OI) de doble etapa para la producción de agua purificada (PW). Sin embargo, la planta enfrenta desafíos en la estabilización de los parámetros de calidad del agua, lo que puede afectar negativamente la producción y la calidad de los medicamentos. Este problema se debe, en parte, a la inestabilidad de la calidad del agua potable proveniente del acueducto y otros factores externos.\\

La implementación de un equipo de Electrodesionización (EDI) como etapa posterior al proceso de OI de doble etapa tiene el potencial de mejorar significativamente la calidad del agua purificada y el agua para inyección, al estabilizar los parámetros de calidad y reducir la conductividad. El EDI es una tecnología de purificación de agua que combina procesos de intercambio iónico y electrodiálisis, eliminando efectivamente las partículas inorgánicas disueltas y reduciendo la concentración de iones en el agua.\\

La justificación para esta investigación radica en la importancia de garantizar la calidad del agua en la industria farmacéutica y la necesidad de encontrar soluciones efectivas y sostenibles para mejorar y estabilizar la calidad del agua en el proceso de producción. La implementación exitosa del EDI en la planta de tratamiento de agua de AICA podría resultar en una producción más eficiente y segura de medicamentos, reduciendo el riesgo de contaminación y garantizando el cumplimiento de los estándares regulatorios y de calidad. Además, la experiencia y el conocimiento adquiridos en este proyecto podrían ser aplicables a otras plantas de tratamiento de agua y procesos industriales, contribuyendo al avance del campo de la ingeniería automática y la optimización de procesos en la industria farmacéutica.\\

\section*{Situación problemática}
La planta de AICA enfrenta inestabilidad en los parámetros de calidad del agua purificada y el agua para inyección debido a la variabilidad en la calidad del agua potable y otros factores. Esta situación afecta la producción y calidad de los productos farmacéuticos.\\
\section*{Problema a resolver}
El problema a resolver es cómo mejorar y estabilizar la calidad del agua purificada y el agua para inyección en la planta de AICA mediante la incorporación de un equipo de Electrodesionización (EDI) y posibles modificaciones en el sistema de control e instrumentación.
\section{ Hipótesis}
La implementación del EDI como etapa posterior al proceso de OI de doble etapa mejorará significativamente la calidad y estabilidad del agua purificada y el agua para inyección en la planta de AICA.
\section{Objeto de estudio}
El objeto de estudio es el proceso de tratamiento de agua en la planta de AICA y la implementación del EDI como una solución para mejorar y estabilizar la calidad del agua.
\textbf{Campo de acción:}\\
El campo de acción se centra en la evaluación y propuesta de
modificación del sistema de tratamiento de agua en la planta de
AICA, incluyendo la implementación del EDI y propuestas
de el sistema de control e instrumentación.
\section*{Objetivo general}
El objetivo general es mejorar y estabilizar la calidad del 
agua purificada y el agua para inyección en la planta de AICA 
mediante la implementación del EDI y propuesta en el sistema de control e instrumentación.
\section{Objetivos específicos}
\begin{enumerate}
    \item Evaluar la situación actual del proceso de tratamiento de agua en la planta de AICA.
    \item Investigar y proponer la implementación del EDI como etapa posterior al proceso de OI de doble etapa.
    \item Analizar los requisitos técnicos, económicos y regulatorios para la implementación del EDI en la planta.
    \item Proponer modificaciones en el sistema de control e instrumentación existente para la integración del EDI.
\end{enumerate}
\textbf{Alcance y limitaciones:}\\
El alcance de esta tesis incluye la evaluación del proceso de tratamiento de agua en la planta de AICA, la propuesta de implementación del EDI y posibles ajustes en el sistema de control e instrumentación existente. Las limitaciones pueden incluir la disponibilidad de información técnica, económica y regulatoria específica, así como restricciones en el acceso a la planta y los equipos involucrados en el proceso.


\section*{Metodología}
Para abordar el problema planteado en esta tesis, se seguirá una metodología estructurada en diversas etapas, que permitirá una aproximación sistemática al objetivo general. Las etapas de la metodología propuesta son las siguientes:
\begin{itemize}
    \item Diagnóstico del proceso actual: En esta etapa se analizará el proceso de tratamiento de agua en la planta de AICA, identificando las variables críticas, inestabilidades y limitaciones en la calidad del agua purificada y el agua para inyección. Se recopilarán y analizarán datos de producción, calidad del agua y rendimiento de los equipos involucrados en el proceso.
    \item Revisión bibliográfica y análisis del estado del arte: Se llevará a cabo una revisión exhaustiva de la literatura científica y técnica relacionada con el tratamiento de agua en la industria farmacéutica, el proceso de OI de doble etapa y la tecnología de EDI. Se buscarán estudios de caso, investigaciones y experiencias previas en la implementación de EDI en plantas similares para identificar las mejores prácticas y lecciones aprendidas.
    \item Propuesta de implementación del EDI: Basándose en el diagnóstico del proceso actual y el análisis del estado del arte, se propondrá la implementación del EDI como etapa posterior al proceso de OI de doble etapa en la planta de AICA. Se definirán los requisitos técnicos, de instrumentación y de control para la integración del EDI en el proceso existente.
    \item Análisis de costos y beneficios: Se llevará a cabo un análisis económico para estimar los costos asociados con la implementación del EDI y las posibles modificaciones en el sistema de control e instrumentación. Además, se evaluarán los beneficios esperados en términos de mejora en la calidad y estabilidad del agua, así como posibles ahorros en el consumo de energía y recursos.
    \item Evaluación de requisitos regulatorios y de cumplimiento: Se investigarán los requisitos legales y regulatorios aplicables a la implementación del EDI en la planta de AICA, así como las normas y estándares de la industria farmacéutica relacionados con el tratamiento de agua y la calidad del agua purificada y el agua para inyección.
    \item Desarrollo de modificaciones en el sistema de control e instrumentación: Basándose en la propuesta de implementación del EDI y los requisitos identificados, se desarrollarán las modificaciones necesarias en el sistema de control e instrumentación existente, incluyendo la actualización del HMI y la programación del PLC.
\end{itemize}







% \section*{Estructura del Contenido}

\begin{itemize}

    \item \textbf{Capítulo 1: Estado del arte y descripción del proceso} \\ Este capítulo presenta un estudio exhaustivo sobre los sistemas de tratamiento de agua en la industria farmacéutica, detallando su importancia, clasificación, regulaciones aplicables, impurezas presentes, variables críticas y las distintas etapas y variantes de su tratamiento. También se discute la evolución histórica y las innovaciones actuales en este campo.

    \item \textbf{Capítulo 2: Introducción y fundamentos de la Electrodesionización (EDI)} \\ En el segundo capítulo se introducen los principios fundamentales de la Electrodesionización (EDI), sus componentes y diseño, los beneficios y desafíos que presenta, así como sus aplicaciones en la industria farmacéutica.

    \item \textbf{Capítulo 3: Análisis de la instrumentación} \\ El tercer capítulo se centra en un levantamiento instrumental completo del sistema de tratamiento de agua. Esto incluye la descripción de diversos sensores y dispositivos, como sensores de conductividad, pH, temperatura, REDOX, flujo, nivel y presión, entre otros. Además, se presenta una propuesta para la implementación de tecnologías de EDI.

    \item \textbf{Capítulo 4: Propuesta de Implementación de EDI} \\ En el cuarto capítulo, se presenta una propuesta completa para la implementación del sistema de EDI. Esta incluye una discusión del sistema de control, una propuesta de SCADA y los detalles de la instalación del EDI.

    \item \textbf{Capítulo 5: Análisis de costos y beneficios} \\ El quinto capítulo presenta un análisis económico completo del proyecto, incluyendo los costos asociados con la implementación de EDI y un análisis detallado de los beneficios esperados.

\end{itemize}
