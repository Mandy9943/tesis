\section*{Metodología}
Para abordar el problema planteado en esta tesis, se seguirá una metodología estructurada en diversas etapas, que permitirá una aproximación sistemática al objetivo general. Las etapas de la metodología propuesta son las siguientes:
\begin{itemize}
	\item Diagnóstico del proceso actual: En esta etapa se analizará el proceso de tratamiento de agua en la planta de AICA, identificando las variables críticas, inestabilidades y limitaciones en la calidad del agua purificada y el agua para inyección. Se recopilarán y analizarán datos de producción, calidad del agua y rendimiento de los equipos involucrados en el proceso.
	\item Revisión bibliográfica y análisis del estado del arte: Se llevará a cabo una revisión exhaustiva de la literatura científica y técnica relacionada con el tratamiento de agua en la industria farmacéutica, el proceso de OI de doble etapa y la tecnología de EDI. Se buscarán estudios de caso, investigaciones y experiencias previas en la implementación de un EDI en plantas similares para identificar las mejores prácticas y lecciones aprendidas.
	\item Propuesta de implementación del EDI: Basándose en el diagnóstico del proceso actual y el análisis del estado del arte, se propondrá la implementación de la tecnología EDI como etapa posterior al proceso de OI de doble etapa en la planta de AICA. Se definirán los requisitos técnicos, de instrumentación y de control para la integración de un EDI en el proceso existente.
	\item Análisis de costos y beneficios: Se llevará a cabo un análisis económico para estimar los costos asociados con la implementación del EDI y las posibles modificaciones en el sistema de control e instrumentación. Además, se evaluarán los beneficios esperados en términos de mejora en la calidad y estabilidad del agua, así como posibles ahorros en el consumo de energía y recursos.
	\item Evaluación de requisitos regulatorios y de cumplimiento: Se investigarán los requisitos legales y regulatorios aplicables a la implementación del EDI en la planta de AICA, así como las normas y estándares de la industria farmacéutica relacionados con el tratamiento de agua y la calidad del agua purificada y el agua para inyección.

	\item Desarrollo de modificaciones en el sistema de control e instrumentación: Basándose en la propuesta de
	      implementación del EDI y los requisitos identificados, se desarrollarán también las modificaciones necesarias en el
	      sistema de control e instrumentación existente, incluyendo el SCADA y la programación del PLC.
\end{itemize}






