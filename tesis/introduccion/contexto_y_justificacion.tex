\section{Contexto y justificación}
La industria farmacéutica desempeña un papel fundamental en la promoción y protección de la salud pública, ya que proporciona medicamentos y productos farmacéuticos que salvan vidas y mejoran la calidad de vida de millones de personas en todo el mundo. La producción de estos productos requiere la utilización de agua de alta calidad, especialmente en la fabricación de soluciones inyectables y otros medicamentos críticos. La calidad del agua utilizada en los procesos de fabricación de medicamentos es un factor esencial para garantizar la seguridad, eficacia y estabilidad de los productos finales.\\

La planta de tratamiento de agua de la empresa AICA, dedicada a la industria farmacéutica, actualmente utiliza un sistema de ósmosis inversa (OI) de doble etapa para la producción de agua purificada (PW). Sin embargo, la planta enfrenta desafíos en la estabilización de los parámetros de calidad del agua, lo que puede afectar negativamente la producción y la calidad de los medicamentos. Este problema se debe, en parte, a la inestabilidad de la calidad del agua potable proveniente del acueducto y otros factores externos.\\

La implementación de un equipo de Electrodesionización (EDI) como etapa posterior al proceso de OI de doble etapa tiene el potencial de mejorar significativamente la calidad del agua purificada y el agua para inyección, al estabilizar los parámetros de calidad y reducir la conductividad. El EDI es una tecnología de purificación de agua que combina procesos de intercambio iónico y electrodiálisis, eliminando efectivamente las partículas inorgánicas disueltas y reduciendo la concentración de iones en el agua.\\

La justificación para esta investigación radica en la importancia de garantizar la calidad del agua en la industria farmacéutica y la necesidad de encontrar soluciones efectivas y sostenibles para mejorar y estabilizar la calidad del agua en el proceso de producción. La implementación exitosa del EDI en la planta de tratamiento de agua de AICA podría resultar en una producción más eficiente y segura de medicamentos, reduciendo el riesgo de contaminación y garantizando el cumplimiento de los estándares regulatorios y de calidad. Además, la experiencia y el conocimiento adquiridos en este proyecto podrían ser aplicables a otras plantas de tratamiento de agua y procesos industriales, contribuyendo al avance del campo de la ingeniería automática y la optimización de procesos en la industria farmacéutica.\\
