\textbf{Contexto y justificación:}\\
La industria farmacéutica desempeña un papel fundamental en la promoción y protección de la salud pública,
ya que proporciona medicamentos y productos farmacéuticos que salvan vidas y mejoran la calidad de vida
de millones de personas en todo el mundo. La producción de estos productos requiere la utilización de
agua de alta calidad, especialmente en la fabricación de soluciones inyectables y otros medicamentos
críticos. La calidad del agua utilizada en los procesos de fabricación de medicamentos es un factor
esencial para garantizar la seguridad, eficacia y estabilidad de los productos finales.

La planta de tratamiento de agua existente, en funcionamiento durante años,
ha demostrado ser un recurso crucial en el suministro de agua pura (PW).
Sin embargo, recientemente, se han observado ciertas inestabilidades en el
proceso de tratamiento, lo que ha dificultado el logro constante de los parámetros
de calidad requeridos durante la producción de agua.

Es importante mencionar que estas inestabilidades no desacreditan mucho la eficacia
del sistema existente. Sin embargo, podrían ser indicativos de la necesidad
de mejoras o adaptaciones para hacer frente a los cambios en las condiciones
del agua de entrada o a los requisitos de calidad cada vez más exigentes.

Además, uno de los desafíos que ha surgido es la capacidad de la
planta para producir agua de la calidad necesaria para cumplir con la demanda.
La planta tiene la capacidad de producir una cantidad considerable de agua,
sin embargo, una porción de esta producción no alcanza los parámetros de
calidad requeridos. Esta agua de menor calidad debe ser desechada o
recirculada para un nuevo tratamiento, lo que resulta en un suministro
efectivo de agua de calidad inferior a la demanda.

La justificación para esta investigación radica en la importancia de garantizar la calidad del agua
en la industria farmacéutica y la necesidad de encontrar soluciones efectivas y sostenibles para
mejorar y estabilizar la calidad del agua en el proceso de producción. La implementación exitosa
del EDI en la planta de tratamiento de agua de AICA podría resultar en una producción más eficiente
y segura de medicamentos, reduciendo el riesgo de contaminación y garantizando el cumplimiento de los
estándares regulatorios y de calidad \cite{alvaradoElectrodeionizationPrinciplesStrategies2014}. Además, la experiencia y el conocimiento adquiridos en este proyecto
podrían ser aplicables a otras plantas de tratamiento de agua y procesos industriales, contribuyendo al avance
del campo de la ingeniería automática y la optimización de procesos en la industria farmacéutica.

