\section*{Estructura por capítulos}
La estructura de la tesis se presenta a continuación:\\
\textbf{Capítulo 1:}  ``Introducción``\\
En este capítulo se presenta el contexto y justificación, la situación problemática, el problema a resolver, la hipótesis, el objeto de estudio, el campo de acción, el objetivo general, los objetivos específicos, el alcance y las limitaciones, y la metodología de la investigación.



Capítulo 2: Estado del arte y descripción del proceso
Este capítulo aborda la revisión de la literatura sobre sistemas de tratamiento de agua en la industria farmacéutica, las tecnologías de purificación de agua (ósmosis inversa, EDI, etc.), la descripción del proceso actual en la planta de AICA, y la instrumentación y control en sistemas de tratamiento de agua.

Capítulo 3: Análisis de la instrumentación actual
En este capítulo se realiza la identificación y descripción de los elementos de instrumentación en el sistema de ósmosis inversa, el análisis de las señales y parámetros de control, y la evaluación del rendimiento y limitaciones de la instrumentación actual.

Capítulo 4: Propuesta de integración del EDI y nueva instrumentación
Este capítulo presenta la selección y justificación del EDI, las modificaciones necesarias en la instrumentación y control, el diseño del sistema de control e integración con el PLC existente, y el estudio de casos similares y lecciones aprendidas.

Capítulo 5: Análisis de costos y beneficios
En este capítulo se lleva a cabo la estimación de costos de adquisición e instalación del EDI e instrumentación adicional, la estimación de costos operativos y de mantenimiento, la evaluación de los beneficios en términos de mejora en la calidad del agua, eficiencia y confiabilidad del proceso, y el análisis de retorno de inversión y viabilidad económica.

Capítulo 6: Cumplimiento de normativas y regulaciones
Este capítulo aborda los requisitos regulatorios aplicables a la industria farmacéutica y sistemas de tratamiento de agua, así como la evaluación de la conformidad del sistema propuesto con las regulaciones y estándares relevantes.

Capítulo 7: Conclusiones y recomendaciones
En este último capítulo, se presentan las conclusiones generales y específicas derivadas de los resultados obtenidos en la investigación, así como las recomendaciones para la implementación del EDI en la planta de AICA y futuras investigaciones relacionadas con el tratamiento de agua en la industria farmacéutica.
