\section*{Estructura del Contenido}

\begin{itemize}

    \item \textbf{Capítulo 1: Estado del arte y descripción del proceso} \\ Este capítulo presenta un estudio exhaustivo sobre los sistemas de tratamiento de agua en la industria farmacéutica, detallando su importancia, clasificación, regulaciones aplicables, impurezas presentes, variables críticas y las distintas etapas y variantes de su tratamiento. También se discute la evolución histórica y las innovaciones actuales en este campo.

    \item \textbf{Capítulo 2: Introducción y fundamentos de la Electrodesionización (EDI)} \\ En el segundo capítulo se introducen los principios fundamentales de la Electrodesionización (EDI), sus componentes y diseño, los beneficios y desafíos que presenta, así como sus aplicaciones en la industria farmacéutica.

    \item \textbf{Capítulo 3: Análisis de la instrumentación} \\ El tercer capítulo se centra en un levantamiento instrumental completo del sistema de tratamiento de agua. Esto incluye la descripción de diversos sensores y dispositivos, como sensores de conductividad, pH, temperatura, REDOX, flujo, nivel y presión, entre otros. Además, se presenta una propuesta para la implementación de tecnologías de EDI.

    \item \textbf{Capítulo 4: Propuesta de Implementación de EDI} \\ En el cuarto capítulo, se presenta una propuesta completa para la implementación del sistema de EDI. Esta incluye una discusión del sistema de control, una propuesta de SCADA y los detalles de la instalación del EDI.

    \item \textbf{Capítulo 5: Análisis de costos y beneficios} \\ El quinto capítulo presenta un análisis económico completo del proyecto, incluyendo los costos asociados con la implementación de EDI y un análisis detallado de los beneficios esperados.

\end{itemize}
