\chapter{Introducción y fundamentos de la Electrodesionización (EDI)}\label{cap:fundamentosEDI}

La electrodesionización (EDI) es una tecnología que combina la electroquímica y la resina de intercambio iónico
para producir agua ultrapura, que es esencial en una variedad de aplicaciones industriales y de laboratorio.
Desde su invención en la década de 1950, la EDI ha evolucionado para convertirse en una opción preferida para
la purificación de agua, especialmente en industrias que requieren altos estándares de pureza, como la farmacéutica,
la de semiconductores, la de energía y la de alimentos y bebidas. La EDI es especialmente útil en aplicaciones donde se
necesita una desionización continua y sin químicos, y donde la conservación de agua es crucial \cite{alvaradoElectrodeionizationPrinciplesStrategies2014}.

El principio de la EDI se basa en la utilización de corriente eléctrica y resinas de intercambio iónico
para eliminar iones y partículas disueltas en el agua. A diferencia de los métodos tradicionales de desionización,
como la desionización por intercambio iónico, la EDI no requiere el uso de químicos para regenerar las resinas de
intercambio iónico. En su lugar, utiliza corriente eléctrica para regenerar las resinas, lo que permite un proceso
de desionización continua. Esta característica no solo elimina la necesidad de manipular y disponer de productos
químicos dañinos, sino que también mejora la eficiencia del proceso de desionización y reduce el consumo de agua \cite{condorchemUltrapureWaterElectrodeionization2019}.

Este capítulo proporciona una introducción y una descripción detallada de los fundamentos de la EDI. Incluye una
discusión sobre los principios básicos de la EDI, los componentes y el diseño de un sistema de EDI, así como los
beneficios y desafíos asociados con la implementación de la tecnología EDI. El capítulo concluye con una discusión
sobre las aplicaciones de la EDI en la industria farmacéutica, resaltando la importancia de la EDI en la producción
de agua ultrapura para aplicaciones farmacéuticas.

\section{Principios de la EDI}

La Electrodesionización (EDI) es una tecnología que combina métodos físicos y químicos para eliminar iones disueltos
del agua. En el corazón de este proceso se encuentra la resina de intercambio iónico, que actúa como un medio para la
extracción de iones y los electrodos que posibilitan el movimiento de estos \cite{condorchemUltrapureWaterElectrodeionization2019}.

En el intercambio iónico, los iones en el agua son atraídos y retenidos por una red de polímeros
cargados conocida como resina de intercambio iónico. Es un proceso dinámico en el que los iones negativos,
denominados aniones, son atraídos hacia los sitios cargados positivamente en la resina, mientras
que los iones positivos, los cationes, son atraídos hacia los sitios con carga negativa.
El resultado de este intercambio iónico es que los iones disueltos en el agua son efectivamente
atrapados y retenidos en la resina, reduciendo así su concentración en el agua \cite{lenntechElectrodeionizationEDI} \cite{condorchemUltrapureWaterElectrodeionization2019}.

En contrapartida, la electrólisis se utiliza para mover activamente los iones. Este proceso
implica la aplicación de una corriente eléctrica a través de una serie de electrodos, que se
encuentran en los extremos de la celda de EDI. Esta corriente eléctrica instiga la migración
de los iones, con los cationes moviéndose hacia el electrodo negativo y los aniones moviéndose
hacia el electrodo positivo \cite{condorchemUltrapureWaterElectrodeionization2019}.

Un dispositivo EDI típico consta de una cámara que alberga una resina de intercambio iónico,
tanto catiónica fuerte como aniónica fuerte. Esta cámara, o celda, está situada entre una membrana
de intercambio catiónico y una membrana de intercambio aniónico, lo que significa que solo
los iones pueden pasar a través de las membranas \cite{alvaradoElectrodeionizationPrinciplesStrategies2014}.

El agua de alimentación entra en este sistema y fluye a través de la resina de intercambio
iónico. Al mismo tiempo, se aplica una corriente continua externa a través de los electrodos.
Esta corriente continua impulsa a los cationes a moverse hacia el cátodo y a los aniones a
moverse hacia el ánodo \cite{alvaradoElectrodeionizationPrinciplesStrategies2014}.

Finalmente, las membranas de intercambio iónico trabajan para eliminar eléctricamente los iones del
agua de entrada y los transfieren al concentrado. De esta forma, el resultado final es un agua
de alta calidad que ha sido eficientemente desionizada.

\insertimageboxed[\label{fig:EDI}]{EDI}{scale=0.8}{0}{Funcionamiento de un electrodesionizador}

La EDI elimina los iones del agua a la vez que las resinas de intercambio iónico que se contiene entre las membranas se
regeneran con una corriente eléctrica. Esta regeneración electroquímica se sirve de un potencial eléctrico para realizar el
transporte iónico y sustituye a la regeneración química de los sistemas convencionales de intercambio iónico, que, como es conocido,
se verifica mediante ácido y sosa. Dentro del compartimento de alimentación, las resinas de intercambio iónico ayudan en el transporte de
los iones al compartimiento concentrado \cite{alvaradoElectrodeionizationPrinciplesStrategies2014}.

Como el agua va disminuyendo en su concentración de iones, se va produciendo la disociación del agua en la interfase de intercambio
catiónico y aniónico, produciéndose un flujo continuo de hidrógeno y ion hidroxilo. Estos iones actúan como regenerante para las resinas
de intercambio iónico presentes en este compartimento y mantiene las resinas a la salida de éste, en un estado de alta regeneración,
necesario para la producción del agua de alta calidad deseada \cite{alvaradoElectrodeionizationPrinciplesStrategies2014}.


\section{Componentes y diseño de la EDI}
La efectividad y eficiencia de un sistema de EDI dependen en gran medida de su diseño y de los componentes utilizados.
En este sentido, existen varios componentes clave en un sistema de EDI que son fundamentales para su operación.

\subsection{Cámara de dilución y concentración}
Las cámaras de dilución y concentración son un componente crítico en el diseño de la EDI. Estas cámaras permiten la separación
física del agua desionizada del agua concentrada con iones. La cámara de dilución es donde el agua purificada se recoge después
de que los iones disueltos son extraídos, mientras que la cámara de concentración es donde se recogen los iones extraídos.
Esta separación es crucial para mantener la pureza del agua desionizada y para garantizar que los iones disueltos no se
reintroduzcan en el agua. Las cámaras de dilución y concentración están separadas por membranas semipermeables, que permiten
el paso de iones pero restringen el flujo de agua \cite{rasSamplingPreconcentrationTechniques2009} \cite{ruiz-jimenezComparisonMultipleCalibration2020}.

\subsection{Resina de intercambio iónico}
La resina de intercambio iónico es otro componente crítico de un sistema de EDI. La resina actúa como un medio para la
atracción y retención de iones disueltos en el agua. La resina es esencialmente una red de polímeros cargados, con sitios
de intercambio iónico que atraen iones de carga opuesta. Las resinas de intercambio iónico vienen en dos tipos principales:
cationes y aniones, que atraen iones negativos y positivos respectivamente. En un sistema de EDI, se utiliza una mezcla de
resinas de intercambio de cationes y aniones para asegurar la extracción de todos los tipos de iones disueltos \cite{nogueraResinasIntercambioIonico}.

\subsection{Electrodo y membrana}
Los electrodos y las membranas son componentes esenciales en la operación de un sistema de EDI. Los electrodos, situados en
los extremos de la celda de EDI, proporcionan el campo eléctrico que impulsa la migración de iones a través de la celda.
Dependiendo de la carga del ion, los iones disueltos son atraídos hacia el electrodo positivo o negativo. Las membranas,
por otro lado, están diseñadas para permitir el paso de iones, pero no de agua. De esta manera, las membranas facilitan el
movimiento de los iones disueltos hacia la cámara de concentración, mientras que el agua purificada se recoge en la cámara
de dilución \cite{lenntechElectrodeionizationEDI}.

\subsection{Fuente de alimentación}
La fuente de alimentación es otro componente crucial de un sistema de EDI. Proporciona la corriente eléctrica necesaria
para el proceso de electrólisis, que impulsa la migración de iones a través de la celda de EDI. La fuente de alimentación
debe ser capaz de suministrar una corriente eléctrica constante y estable para garantizar una operación eficiente del sistema.


% sección

\section{Tecnologías Alternativas y sus Limitaciones}

En la búsqueda de la mejora continua y optimización de la planta de tratamiento de agua,
es importante considerar las diversas tecnologías alternativas disponibles.
Sin embargo, cada tecnología tiene sus propias limitaciones, algunas de las
cuales pueden no adaptarse a las necesidades y condiciones específicas de nuestra planta.
Las siguientes son algunas de las tecnologías que se han evaluado:

\begin{itemize}
      \item \textbf{Reforzamiento de la Ósmosis Inversa (RO):}  Nuestra planta ya cuenta con
            un sistema de ósmosis inversa de dos etapas que cumple con las necesidades
            básicas de la planta. Sin embargo, incluso con un sistema RO de dos etapas,
            todavía existen limitaciones, especialmente en términos de la eliminación de
            ciertos iones y pequeñas moléculas. Los sistemas RO también son susceptibles a
            la acumulación de sarro y biofilm, lo que puede afectar el rendimiento y
            la vida útil de la membrana.

      \item \textbf{Destilación:}  Aunque la destilación puede ofrecer altos niveles de
            purificación, la energía requerida para este proceso es considerable,
            lo que resulta en costos operativos más altos. Además, la destilación
            no elimina eficientemente algunos contaminantes volátiles que pueden ser arrastrados con el vapor.

      \item \textbf{Desionización (DI): } Los sistemas de DI pueden ser eficientes para
            la eliminación de iones de agua, pero su capacidad para eliminar
            partículas no iónicas, gases disueltos y microorganismos es limitada.
            Además, los cartuchos de DI requieren un reemplazo frecuente, lo que
            implica costos adicionales de operación y mantenimiento.

      \item \textbf{Filtración de Carbón Activado:}  Esta tecnología es efectiva para la
            eliminación de cloro y ciertos otros contaminantes, pero su eficacia es
            limitada cuando se trata de la eliminación de sales disueltas y
            algunos contaminantes orgánicos.

\end{itemize}

Teniendo en cuenta las limitaciones y desafíos presentes en estas tecnologías alternativas, y
dadas las necesidades específicas de nuestra planta de tratamiento de agua, es evidente que se
necesita una solución más eficaz y sostenible. En este contexto, la Electrodesionización (EDI)
emerge como una solución potencialmente superior, debido a su capacidad para superar muchas de las
limitaciones de las tecnologías mencionadas anteriormente.

% sección
\section{Beneficios de la EDI}
La Electrodesionización (EDI) ofrece una variedad de ventajas significativas en comparación con otras tecnologías de
purificación de agua. En este apartado, se detallarán estos beneficios de manera exhaustiva, desde la simplicidad operativa
hasta la eficacia en la eliminación de partículas inorgánicas \cite{condorchemUltrapureWaterElectrodeionization2019}.

Empezando con su operación, la EDI destaca por su simplicidad y continuidad. Dado que esta tecnología combina la electrodiálisis
y el intercambio iónico, permite una producción ininterrumpida de agua de alta pureza. Esto significa que no es necesario
interrumpir el proceso para la regeneración de las resinas, como ocurre con otros métodos de desionización. Esta característica
contribuye a mejorar la eficiencia de los procesos productivos y a minimizar el tiempo de inactividad \cite{lenntechElectrodeionizationEDI}.

Un factor crítico que diferencia a la EDI de otros métodos de purificación es la eliminación casi total del uso de productos
químicos en el proceso de regeneración. A diferencia de los sistemas tradicionales de intercambio iónico, la EDI utiliza una
corriente eléctrica para regenerar las resinas de intercambio iónico. Este enfoque no solo elimina la necesidad de manejar
y almacenar productos químicos peligrosos, sino que también reduce los costos operativos y el impacto ambiental asociado con
la eliminación de productos químicos residuales \cite{lenntechElectrodeionizationEDI}.

Desde la perspectiva operativa y de mantenimiento, la EDI también tiene ventajas económicas significativas. Gracias a su
diseño compacto y a la ausencia de partes móviles, el mantenimiento de los sistemas de EDI es relativamente simple y los
riesgos de averías son bajos. Esta característica se traduce en ahorros en los costos de mantenimiento y en la reducción
del tiempo de inactividad del sistema. Adicionalmente, la EDI se caracteriza por su eficiencia energética, lo cual se refleja
en un menor consumo de energía en comparación con otros métodos de purificación de agua, como la destilación \cite{lenntechElectrodeionizationEDI}.

En lo que respecta al impacto ambiental, la EDI se considera una tecnología ecológica. Al no producir efluentes peligrosos y
al eliminar la necesidad de manejo de productos químicos, se reduce significativamente el riesgo de contaminación ambiental.
Asimismo, no requiere la disposición de resinas de intercambio iónico agotadas, lo que minimiza aún más su huella ambiental.

Finalmente, la EDI es altamente efectiva en la eliminación de partículas inorgánicas disueltas en el agua. Con su capacidad
para eliminar hasta el 99,9\% de los iones presentes en el agua, incluyendo cationes y aniones, ofrece un grado de purificación
que supera a la mayoría de los otros métodos disponibles. Esta efectividad la convierte en una solución de purificación de
agua altamente atractiva para una amplia gama de aplicaciones \cite{lenntechElectrodeionizationEDI}.

\section{Desafíos de la EDI}
A pesar de sus numerosos beneficios, la implementación y operación de la EDI también presentan desafíos.

Uno de los principales desafíos es la necesidad de un pretratamiento del agua de alimentación. La EDI requiere agua de alimentación
de baja conductividad, por lo general proporcionada por la ósmosis inversa (RO). Además, el agua de alimentación debe estar libre
de cloro y otras sustancias oxidantes que pueden dañar las resinas de intercambio iónico y las membranas de la EDI. Por lo tanto,
el diseño del pretratamiento del agua es crucial para el rendimiento de la EDI \cite{lenntechElectrodeionizationEDI}.

Finalmente, la EDI requiere un suministro de energía eléctrica constante para su operación. Cualquier fluctuación en el suministro
de energía puede afectar el rendimiento de la EDI y resultar en una calidad de agua inconsistente. Por lo tanto, un suministro de
energía confiable es esencial para la operación de la EDI \cite{lenntechElectrodeionizationEDI}.

\section{Aplicaciones de la EDI en la industria farmacéutica}
En la industria farmacéutica, la pureza y consistencia del agua utilizada en los procesos de producción son de suma importancia.
Cualquier contaminante, ya sea orgánico, inorgánico o microbiológico, puede afectar la calidad del producto final y comprometer la
seguridad del paciente. En este contexto, la EDI ha encontrado un lugar destacado debido a su capacidad para producir agua de alta
pureza de manera confiable y continua \cite{condorchemUltrapureWaterElectrodeionization2019}.

La EDI es comúnmente utilizada en la producción de agua purificada (PW) y agua para inyección (WFI). El agua purificada es utilizada
en una amplia gama de aplicaciones en la industria farmacéutica, como la preparación de soluciones para la producción de productos
farmacéuticos y la limpieza de equipos y envases. El agua para inyección, que requiere un nivel aún mayor de pureza, es utilizada
en la producción de productos parenterales, como soluciones para inyección y productos liofilizados \cite{condorchemUltrapureWaterElectrodeionization2019}.

La EDI se utiliza a menudo en combinación con otros procesos de purificación de agua, como la ósmosis inversa (RO) y la destilación.
En un sistema típico, la RO se utiliza primero para reducir la concentración de sales y otros contaminantes en el agua. Luego, la EDI
se utiliza para eliminar los iones restantes y lograr el nivel de pureza deseado. Finalmente, si se requiere agua para inyección,
el agua producida por la EDI puede ser sometida a destilación para eliminar cualquier contaminante restante y garantizar la esterilidad \cite{condorchemUltrapureWaterElectrodeionization2019}.

