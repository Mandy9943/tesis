\chapter{Introducción y fundamentos de la Electrodeionización (EDI)}
La electrodeionización (EDI) es una tecnología que combina la electroquímica y la resina de intercambio iónico para producir agua ultrapura, que es esencial en una variedad de aplicaciones industriales y de laboratorio. Desde su invención en la década de 1950, la EDI ha evolucionado para convertirse en una opción preferida para la purificación de agua, especialmente en industrias que requieren altos estándares de pureza, como la farmacéutica, la de semiconductores, la de energía y la de alimentos y bebidas. La EDI es especialmente útil en aplicaciones donde se necesita una desionización continua y sin químicos, y donde la conservación de agua es crucial. \\

El principio de la EDI se basa en la utilización de corriente eléctrica y resinas de intercambio iónico para eliminar iones y partículas disueltas en el agua. A diferencia de los métodos tradicionales de desionización, como la desionización por intercambio iónico, la EDI no requiere el uso de químicos para regenerar las resinas de intercambio iónico. En su lugar, utiliza corriente eléctrica para regenerar las resinas, lo que permite un proceso de desionización continua. Esta característica no solo elimina la necesidad de manipular y disponer de productos químicos dañinos, sino que también mejora la eficiencia del proceso de desionización y reduce el consumo de agua. \\

Este capítulo proporciona una introducción y una descripción detallada de los fundamentos de la EDI. Incluye una discusión sobre los principios básicos de la EDI, los componentes y el diseño de un sistema de EDI, así como los beneficios y desafíos asociados con la implementación de la tecnología EDI. El capítulo concluye con una discusión sobre las aplicaciones de la EDI en la industria farmacéutica, resaltando la importancia de la EDI en la producción de agua ultrapura para aplicaciones farmacéuticas. \\

\section{Principios de la EDI}
La Electrodeionización (EDI) es una tecnología que combina métodos físicos y químicos para eliminar iones disueltos del agua. En el corazón de este proceso se encuentra la resina de intercambio iónico, que actúa como un medio para la extracción de iones y partículas disueltas en el agua. \\

El principio de funcionamiento de la EDI se basa en dos procesos fundamentales: el intercambio iónico y la electrólisis. En el intercambio iónico, los iones en el agua son atraídos y retenidos por la resina de intercambio iónico, que es esencialmente una red de polímeros cargados. Los iones negativos (aniones) son atraídos por los sitios cargados positivamente en la resina, mientras que los iones positivos (cationes) son atraídos por los sitios cargados negativamente. Este proceso de intercambio de iones efectivamente atrapa y retiene los iones disueltos en el agua. \\

Por otro lado, la electrólisis implica el uso de una corriente eléctrica para estimular la migración de iones. En un sistema de EDI, la corriente eléctrica es aplicada a través de una serie de electrodos, que están situados en los extremos de la celda de EDI. Bajo la influencia de este campo eléctrico, los iones en el agua son instigados a moverse hacia los electrodos correspondientes - los cationes hacia el electrodo negativo y los aniones hacia el electrodo positivo. \\

Combinando estos dos procesos, la EDI logra desionizar el agua de manera efectiva. La resina de intercambio iónico atrae y retiene los iones disueltos, mientras que la corriente eléctrica impulsa a estos iones a través de la celda de EDI y hacia los electrodos. Durante este proceso, el agua es efectivamente despojada de sus impurezas iónicas. \\

La EDI también se caracteriza por un proceso de regeneración continua de las resinas de intercambio iónico. Tradicionalmente, las resinas de intercambio iónico deben ser regeneradas con frecuencia mediante el uso de productos químicos. Sin embargo, en la EDI, la corriente eléctrica proporciona la energía necesaria para la regeneración. Esto significa que la resina de intercambio iónico nunca se agota realmente, lo que permite un proceso de desionización continua y sin interrupciones. \\

Además, la EDI se beneficia de la separación de las corrientes de agua desionizada y concentrada. Esto se logra mediante el uso de membranas semipermeables, que están dispuestas en la celda de tal manera que separan la celda en compartimentos de concentración y dilución. Los iones disueltos, que son atraídos por los electrodos, son llevados a través de las membranas hacia el compartimento de concentración, mientras que el agua desionizada se recoge en el compartimento de dilución. \\

\section{Componentes y diseño de la EDI}
La efectividad y eficiencia de un sistema de EDI dependen en gran medida de su diseño y de los componentes utilizados. En este sentido, existen varios componentes clave en un sistema de EDI que son fundamentales para su operación. \\

\subsection{Cámara de dilución y concentración}
Las cámaras de dilución y concentración son un componente crítico en el diseño de la EDI. Estas cámaras permiten la separación física del agua desionizada del agua concentrada con iones. La cámara de dilución es donde el agua purificada se recoge después de que los iones disueltos son extraídos, mientras que la cámara de concentración es donde se recogen los iones extraídos. Esta separación es crucial para mantener la pureza del agua desionizada y para garantizar que los iones disueltos no se reintroduzcan en el agua. Las cámaras de dilución y concentración están separadas por membranas semipermeables, que permiten el paso de iones pero restringen el flujo de agua. \\

\subsection{Resina de intercambio iónico}
La resina de intercambio iónico es otro componente crítico de un sistema de EDI. La resina actúa como un medio para la atracción y retención de iones disueltos en el agua. La resina es esencialmente una red de polímeros cargados, con sitios de intercambio iónico que atraen iones de carga opuesta. Las resinas de intercambio iónico vienen en dos tipos principales: cationes y aniones, que atraen iones negativos y positivos respectivamente. En un sistema de EDI, se utiliza una mezcla de resinas de intercambio de cationes y aniones para asegurar la extracción de todos los tipos de iones disueltos. \\

\subsection{Electrodo y membrana}
Los electrodos y las membranas son componentes esenciales en la operación de un sistema de EDI. Los electrodos, situados en los extremos de la celda de EDI, proporcionan el campo eléctrico que impulsa la migración de iones a través de la celda. Dependiendo de la carga del ión, los iones disueltos son atraídos hacia el electrodo positivo o negativo. Las membranas, por otro lado, están diseñadas para permitir el paso de iones, pero no de agua. De esta manera, las membranas facilitan el movimiento de los iones disueltos hacia la cámara de concentración, mientras que el agua purificada se recoge en la cámara de dilución. \\

\subsection{Fuente de alimentación}
La fuente de alimentación es otro componente crucial de un sistema de EDI. Proporciona la corriente eléctrica necesaria para el proceso de electrólisis, que impulsa la migración de iones a través de la celda de EDI. La fuente de alimentación debe ser capaz de suministrar una corriente eléctrica constante y estable para garantizar una operación eficiente del sistema. \\


\section{Beneficios y desafíos de la EDI}
La tecnología de la EDI tiene numerosos beneficios, pero también presenta ciertos desafíos que deben ser reconocidos y superados para su efectiva implementación y operación. \\

\subsection{Beneficios de la EDI}
La Electrodeionización (EDI) ofrece una variedad de ventajas significativas en comparación con otras tecnologías de purificación de agua. En este apartado, se detallarán estos beneficios de manera exhaustiva, desde la simplicidad operativa hasta la eficacia en la eliminación de partículas inorgánicas.\\

Empezando con su operación, la EDI destaca por su simplicidad y continuidad. Dado que esta tecnología combina la electrodiálisis y el intercambio iónico, permite una producción ininterrumpida de agua de alta pureza. Esto significa que no es necesario interrumpir el proceso para la regeneración de las resinas, como ocurre con otros métodos de desionización. Esta característica contribuye a mejorar la eficiencia de los procesos productivos y a minimizar el tiempo de inactividad.\\

Un factor crítico que diferencia a la EDI de otros métodos de purificación es la eliminación casi total del uso de productos químicos en el proceso de regeneración. A diferencia de los sistemas tradicionales de intercambio iónico, la EDI utiliza una corriente eléctrica para regenerar las resinas de intercambio iónico. Este enfoque no solo elimina la necesidad de manejar y almacenar productos químicos peligrosos, sino que también reduce los costos operativos y el impacto ambiental asociado con la eliminación de productos químicos residuales.\\

Desde la perspectiva operativa y de mantenimiento, la EDI también tiene ventajas económicas significativas. Gracias a su diseño compacto y a la ausencia de partes móviles, el mantenimiento de los sistemas de EDI es relativamente simple y los riesgos de averías son bajos. Esta característica se traduce en ahorros en los costos de mantenimiento y en la reducción del tiempo de inactividad del sistema. Adicionalmente, la EDI se caracteriza por su eficiencia energética, lo cual se refleja en un menor consumo de energía en comparación con otros métodos de purificación de agua, como la destilación.\\

En lo que respecta al impacto ambiental, la EDI se considera una tecnología ecológica. Al no producir efluentes peligrosos y al eliminar la necesidad de manejo de productos químicos, se reduce significativamente el riesgo de contaminación ambiental. Asimismo, no requiere la disposición de resinas de intercambio iónico agotadas, lo que minimiza aún más su huella ambiental.\\

Finalmente, la EDI es altamente efectiva en la eliminación de partículas inorgánicas disueltas en el agua. Con su capacidad para eliminar hasta el 99,9\% de los iones presentes en el agua, incluyendo cationes y aniones, ofrece un grado de purificación que supera a la mayoría de los otros métodos disponibles. Esta efectividad la convierte en una solución de purificación de agua altamente atractiva para una amplia gama de aplicaciones.\\

\subsection{Desafíos de la EDI}
A pesar de sus numerosos beneficios, la implementación y operación de la EDI también presentan desafíos. \\

Uno de los principales desafíos es la necesidad de una pretratamiento del agua de alimentación. La EDI requiere agua de alimentación de baja conductividad, por lo general proporcionada por la ósmosis inversa (RO). Además, el agua de alimentación debe estar libre de cloro y otras sustancias oxidantes que pueden dañar las resinas de intercambio iónico y las membranas de la EDI. Por lo tanto, el diseño del pretratamiento del agua es crucial para el rendimiento de la EDI. \\

Otro desafío es el mantenimiento de los sistemas de EDI. Aunque la EDI reduce la necesidad de químicos regenerantes, todavía requiere limpieza periódica y reemplazo de componentes para mantener su rendimiento. La membrana de EDI y las resinas de intercambio iónico pueden necesitar ser reemplazadas después de un cierto período de tiempo, dependiendo de la calidad del agua de alimentación y de las condiciones operativas. \\

Finalmente, la EDI requiere un suministro de energía eléctrica constante para su operación. Cualquier fluctuación en el suministro de energía puede afectar el rendimiento de la EDI y resultar en una calidad de agua inconsistente. Por lo tanto, un suministro de energía confiable es esencial para la operación de la EDI. \\

\section{Aplicaciones de la EDI en la industria farmacéutica}
En la industria farmacéutica, la pureza y consistencia del agua utilizada en los procesos de producción son de suma importancia. Cualquier contaminante, ya sea orgánico, inorgánico o microbiológico, puede afectar la calidad del producto final y comprometer la seguridad del paciente. En este contexto, la EDI ha encontrado un lugar destacado debido a su capacidad para producir agua de alta pureza de manera confiable y continua.\\

La EDI es comúnmente utilizada en la producción de agua purificada (PW) y agua para inyección (WFI). El agua purificada es utilizada en una amplia gama de aplicaciones en la industria farmacéutica, como la preparación de soluciones para la producción de productos farmacéuticos y la limpieza de equipos y envases. El agua para inyección, que requiere un nivel aún mayor de pureza, es utilizada en la producción de productos parenterales, como soluciones para inyección y productos liofilizados.\\

La EDI se utiliza a menudo en combinación con otros procesos de purificación de agua, como la ósmosis inversa (RO) y la destilación. En un sistema típico, la RO se utiliza primero para reducir la concentración de sales y otros contaminantes en el agua. Luego, la EDI se utiliza para eliminar los iones restantes y lograr el nivel de pureza deseado. Finalmente, si se requiere agua para inyección, el agua producida por la EDI puede ser sometida a destilación para eliminar cualquier contaminante restante y garantizar la esterilidad.\\

