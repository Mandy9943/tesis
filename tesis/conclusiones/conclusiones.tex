\chapter*{Conclusiones}
\phantomsection
\addcontentsline{toc}{chapter}{Conclusiones}
\vspace{-2cm} 
El presente trabajo ha demostrado el éxito en el cumplimiento de los objetivos propuestos para la propuesta de instrumentación y control en la planta de tratamiento de agua de los laboratorios farmacéuticos AICA. Se realizó un exhaustivo estudio de los métodos de purificación de agua, centrándose en la ósmosis inversa y sus diferentes configuraciones, y se destacó la amplia aplicabilidad de esta técnica en la industria biofarmacéutica, especialmente al complementarla con un electrodeionizador.

Además, se llevó a cabo un detallado levantamiento instrumental que permitió el análisis técnico y funcional de los instrumentos y medios técnicos instalados en la ósmosis, así como la evaluación de los dispositivos e instrumentos propuestos. La lógica de programación desarrollada para el autómata aseguró la correcta puesta en marcha del electrodeionizador, garantizando que no se produjeran alteraciones en el funcionamiento actual de la planta.

Para la monitorización y control de la planta, se diseñó una aplicación visual basada en un sistema SCADA, que incorpora el nuevo dispositivo y sus correspondientes sensores. Esto proporcionará a los operadores una interfaz intuitiva y eficiente para supervisar y controlar el proceso de producción de agua purificada.

Por último, se realizó un exhaustivo cálculo y análisis económico que contempló el costo total del proyecto, brindando información relevante para su futura implementación. Estas conclusiones destacan la viabilidad técnica y económica de la propuesta, así como el potencial impacto positivo que tendrá en la producción de agua purificada para la elaboración de medicamentos e inyectables en los laboratorios farmacéuticos AICA.

% \section*{Recomendaciones para futuras investigaciones}


% Para futuros trabajos en esta área, se recomienda realizar estudios prácticos y
% experimentales para validar los resultados obtenidos teóricamente en este estudio.
% Implementar pruebas piloto del sistema EDI en una planta de AICA proporcionaría
% datos valiosos y confirmaría o refutaría los hallazgos actuales.

% Además, sería beneficioso investigar cómo la tecnología EDI podría integrarse
% con otras tecnologías emergentes de tratamiento de agua. Por ejemplo,
% la nanofiltración podría
% trabajar en conjunto con la tecnología EDI para optimizar aún más el
% proceso de purificación de agua.

% Finalmente, es esencial continuar buscando y recopilando más información
% sobre la implementación y el funcionamiento del EDI en la industria farmacéutica.
% A medida que la tecnología continúa avanzando y más empresas comienzan a adoptarla,
% es probable que la información y los estudios de caso disponibles aumenten.
% Mantenerse al día con esta literatura será vital para cualquier trabajo futuro en esta área.