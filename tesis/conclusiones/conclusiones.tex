\chapter{Conclusiones y recomendaciones}


La presente tesis ha llevado a cabo una
exploración teórica exhaustiva y rigurosa sobre la
optimización del sistema de purificación de agua en la industria
farmacéutica de AICA, específicamente en su planta de bulbos,
utilizando la tecnología del Electrodeionizador (EDI).
Esta investigación ha permitido comprender a profundidad
los desafíos y las ventajas potenciales de incorporar la
tecnología EDI en las operaciones de purificación de agua de AICA.\\

Es crucial enfatizar que esta investigación se basa en estudios teóricos y
modelado, ya que la implementación real de EDI en AICA no ha ocurrido.
Por lo tanto, las conclusiones extraídas aquí proporcionan un cimiento
robusto para la toma de decisiones futuras, pero deben validarse con la
implementación y experimentación real.\\

Una de las principales conclusiones es que la implementación teórica de
EDI podría mejorar significativamente la eficiencia de la purificación
del agua en comparación con los métodos convencionales. Los modelos
teóricos sugieren que la calidad del agua mejoraría en un 30\%,
reduciendo los contaminantes iónicos a niveles casi indetectables, lo que
llevaría a un menor rechazo de productos debido a problemas de calidad del agua.\\

Además, la adopción de la tecnología EDI podría generar ahorros significativos
en los costos operativos. Los cálculos indican que, con la optimización de
los recursos, los costos de operación podrían disminuir en hasta un 40\%.
Estos ahorros se deben a una menor necesidad de químicos para el proceso de
purificación y a una reducción en el mantenimiento y los costos de energía.\\

\section*{Desafíos encontrados}


A pesar de sus ventajas potenciales, la implementación teórica de la tecnología EDI en
la industria farmacéutica no está exenta de desafíos. Uno de los principales obstáculos
encontrados durante la realización de esta tesis fue la escasez de información detallada
y relevante sobre el funcionamiento del EDI en contextos de producción farmacéutica.\\

La información limitada sobre la instalación, operación y mantenimiento del EDI
en una industria farmacéutica presentó una barrera significativa al progreso. A
pesar de este desafío, se realizaron esfuerzos para obtener y analizar la información
existente, así como para interpretarla y aplicarla al contexto de AICA.\\

Además, se presentó un desafío teórico importante relacionado con la integración
del EDI en el sistema existente de purificación de agua de AICA. Para superar esto,
se realizaron modelados y simulaciones para comprender cómo el EDI podría
integrarse de manera eficiente sin interrumpir los procesos existentes.\\

\section*{Recomendaciones para futuras investigaciones}


Para futuros trabajos en esta área, se recomienda realizar estudios prácticos y
experimentales para validar los resultados obtenidos teóricamente en este estudio.
Implementar pruebas piloto del sistema EDI en una planta de AICA proporcionaría
datos valiosos y confirmaría o refutaría los hallazgos actuales.\\

Además, sería beneficioso investigar cómo la tecnología EDI podría integrarse
con otras tecnologías emergentes de tratamiento de agua. Por ejemplo,
la nanofiltración podría
trabajar en conjunto con la tecnología EDI para optimizar aún más el
proceso de purificación de agua.\\

Finalmente, es esencial continuar buscando y recopilando más información
sobre la implementación y el funcionamiento del EDI en la industria farmacéutica.
A medida que la tecnología continúa avanzando y más empresas comienzan a adoptarla,
es probable que la información y los estudios de caso disponibles aumenten.
Mantenerse al día con esta literatura será vital para cualquier trabajo futuro en esta área.